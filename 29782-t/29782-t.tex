% %%%%%%%%%%%%%%%%%%%%%%%%%%%%%%%%%%%%%%%%%%%%%%%%%%%%%%%%%%%%%%%%%%%%%%% %
%                                                                         %
% Project Gutenberg's Space, Time and Gravitation, by A. S. Eddington     %
%                                                                         %
% This eBook is for the use of anyone anywhere at no cost and with        %
% almost no restrictions whatsoever.  You may copy it, give it away or    %
% re-use it under the terms of the Project Gutenberg License included     %
% with this eBook or online at www.gutenberg.org                          %
%                                                                         %
%                                                                         %
% Title: Space, Time and Gravitation                                      %
%        An Outline of the General Relativity Theory                      %
%                                                                         %
% Author: A. S. Eddington                                                 %
%                                                                         %
% Release Date: August 24, 2009 [EBook #29782]                            %
%                                                                         %
% Language: English                                                       %
%                                                                         %
% Character set encoding: ISO-8859-1                                      %
%                                                                         %
% *** START OF THIS PROJECT GUTENBERG EBOOK SPACE, TIME AND GRAVITATION ***
%                                                                         %
% %%%%%%%%%%%%%%%%%%%%%%%%%%%%%%%%%%%%%%%%%%%%%%%%%%%%%%%%%%%%%%%%%%%%%%% %

\def\ebook{29782}
%%%%%%%%%%%%%%%%%%%%%%%%%%%%%%%%%%%%%%%%%%%%%%%%%%%%%%%%%%%%%%%%%%%%%%
%%                                                                  %%
%% Packages and substitutions:                                      %%
%%                                                                  %%
%% book:     Required.                                              %%
%% inputenc: Standard DP encoding. Required.                        %%
%%                                                                  %%
%% textcomp: Better ditto marks. Optional.                          %%
%% fix-cm:   For larger title page fonts. Optional.                 %%
%% ifthen:   Logical conditionals. Required.                        %%
%%                                                                  %%
%% amsmath:  AMS mathematics enhancements. Required.                %%
%% amssymb:  Additional mathematical symbols. Required.             %%
%%                                                                  %%
%% alltt:    Fixed-width font environment. Required.                %%
%% array:    Enhanced tabular features. Required.                   %%
%%                                                                  %%
%% footmisc: Extended footnote capabilities. Required.              %%
%% multicol: Multi-column environment for index. Required.          %%
%% makeidx:  Indexing capabilities. Required.                       %%
%%                                                                  %%
%% fancyhdr: Enhanced running headers and footers. Required.        %%
%%                                                                  %%
%% graphicx: Standard interface for graphics inclusion. Required.   %%
%% wrapfig:  Illustrations surrounded by text. Required.            %%
%%                                                                  %%
%% geometry: Enhanced page layout package. Required.                %%
%% hyperref: Hypertext embellishments for pdf output. Required.     %%
%%                                                                  %%
%%                                                                  %%
%% Producer's Comments:                                             %%
%%                                                                  %%
%%   Minor spelling/punctuation changes, etc. are [** PP: noted]    %%
%%   in this file. On occasions where a word is hyphenated across a %%
%%   line in the original and the word occurs only once, comparison %%
%%   with similar words was made to decide whether to hyphenate.    %%
%%   Such instances are noted.                                      %%
%%                                                                  %%
%%   The following spellings are retained:                          %%
%%   debateable, idiosyncracies, and unbiassed.                     %%
%%                                                                  %%
%%                                                                  %%
%% Compilation Flags:                                               %%
%%                                                                  %%
%%   The following behaviors may be controlled by boolean flags.    %%
%%                                                                  %%
%%   ForPrinting (true by default):                                 %%
%%   Compile a print-optimized PDF file. Set to false for screen-   %%
%%   optimized file (pages cropped, one-sided, blue hyperlinks).    %%
%%                                                                  %%
%%   IndexExtras (false by default):                                %%
%%   Add a few entries to the index, e.g. Geometry, Euclidean.      %%
%%                                                                  %%
%%                                                                  %%
%% Things to Check:                                                 %%
%%                                                                  %%
%%   Aviator's time table (PDF p. 22) is not separated from the     %%
%%     preceding sentence or broken across pages                    %%
%%                                                                  %%
%%   Wrapped images do not fall near a page bottom, lest indented   %%
%%     lines continue conspicuously past the break                  %%
%%     Fig. 1, p. 15;   Fig. 2, p. 42;   Fig. 8, p. 50              %%
%%                                                                  %%
%%                                                                  %%
%% Spellcheck: .................................. OK                %%
%% Smoothreading pool: ......................... yes                %%
%%                                                                  %%
%% lacheck: ..................................... OK                %%
%%   Numerous false positives from commented code                   %%
%%                                                                  %%
%% PDF pages: 219 (if ForPrinting set to true)                      %%
%% PDF page size: US Letter                                         %%
%% PDF bookmarks: created, point to ToC entries                     %%
%% PDF document info: filled in                                     %%
%% Images: 18 pdf diagrams, 1 jpg (frontispiece)                    %%
%%                                                                  %%
%% Summary of log file:                                             %%
%% * One overfull hbox (0.7pt too wide).                            %%
%% * Three underfull vboxes (but good illo placement overall).      %%
%%                                                                  %%
%%                                                                  %%
%% Compile History:                                                 %%
%%                                                                  %%
%% March, 2009: adhere (Andrew D. Hwang)                            %%
%%              texlive2007, GNU/Linux                              %%
%% July, 2009:  dcwilson                                            %%
%%              MiKTeX 2.7, WinXP Pro                               %%
%%                                                                  %%
%% Command block:                                                   %%
%%                                                                  %%
%%     pdflatex x3 (Run pdflatex three times)                       %%
%%     makeindex                                                    %%
%%     pdflatex                                                     %%
%%                                                                  %%
%%                                                                  %%
%% August 2009: pglatex.                                            %%
%%   Compile this project with:                                     %%
%%   pdflatex 29782-t.tex ..... THREE times                         %%
%%   makeindex 29782-t.idx                                          %%
%%   pdflatex 29782-t.tex                                           %%
%%                                                                  %%
%%   pdfTeXk, Version 3.141592-1.40.3 (Web2C 7.5.6)                 %%
%%                                                                  %%
%%%%%%%%%%%%%%%%%%%%%%%%%%%%%%%%%%%%%%%%%%%%%%%%%%%%%%%%%%%%%%%%%%%%%%
\listfiles
\documentclass[12pt]{book}[2005/09/16]

%%%%%%%%%%%%%%%%%%%%%%%%%%%%% PACKAGES %%%%%%%%%%%%%%%%%%%%%%%%%%%%%%%

\usepackage[latin1]{inputenc}[2006/05/05] %% DP standard encoding

\IfFileExists{textcomp.sty}{%    %% For ditto marks
\usepackage{textcomp}[2005/09/27]%
}{}

\newlength{\MySkip}
\IfFileExists{fix-cm.sty}{%      %% For larger title page fonts
\usepackage{fix-cm}[2006/03/24]%
\newcommand{\MyHuge}{\fontsize{38}{48}\selectfont}%
\setlength{\MySkip}{0.375in}}% else
{\newcommand{\MyHuge}{\Huge}%
\setlength{\MySkip}{0.25in}}


\usepackage{ifthen}[2001/05/26]  %% Logical conditionals

\usepackage{amsmath}[2000/07/18] %% Displayed equations
\usepackage{amssymb}[2002/01/22] %% and additional symbols

\usepackage{alltt}[1997/06/16]   %% boilerplate, credits, license

\usepackage{array}[2005/08/23]   %% extended array/tabular features

                                 %% extended footnote capabilities
\usepackage[symbol,perpage]{footmisc}[2005/03/17]

\usepackage{multicol}[2006/05/18]
\usepackage{makeidx}[2000/03/29]

\usepackage{graphicx}[1999/02/16]%% For diagrams
\usepackage{wrapfig}[2003/01/31] %% and wrapping text around them


% for running heads; no package date available
\usepackage{fancyhdr}
\renewcommand{\headrulewidth}{0pt}

%%%%%%%%%%%%%%%%%%%%%%%%%%%%%%%%%%%%%%%%%%%%%%%%%%%%%%%%%%%%%%%%%
%%%%             Conditional compilation switches            %%%%
%%%%%%%%%%%%%%%%%%%%%%%%%%%%%%%%%%%%%%%%%%%%%%%%%%%%%%%%%%%%%%%%%

% Sets up a handful of additional index entries
\newboolean{IndexExtras}
% UNCOMMENT the next line for extra entries
%\setboolean{IndexExtras}{true}

%%%%%%%%%%%%%%%%%%%%%%%%%%%%%%%%%%%%%%%%%%%%%%%%%%%%%%%%%%%%%%%%%
%%%%        Set up PRINTING (default) or SCREEN VIEWING      %%%%
%%%%%%%%%%%%%%%%%%%%%%%%%%%%%%%%%%%%%%%%%%%%%%%%%%%%%%%%%%%%%%%%%

% ForPrinting=true (default)           false
% Letterpaper                          Cropped pages
% Asymmetric margins                   Symmetric margins
% Black hyperlinks                     Blue hyperlinks
\newboolean{ForPrinting}

%% COMMENT the next line for a SCREEN-OPTIMIZED VERSION of the text %%
\setboolean{ForPrinting}{true}

%% Initialize values to ForPrinting=false
\newcommand{\Margins}{hmarginratio=1:1}     % Symmetric margins
\newcommand{\HLinkColor}{blue}              % Hyperlink color
\newcommand{\PDFPageLayout}{SinglePage}
\newcommand{\TransNote}{Transcriber's Note}
\newcommand{\TransNoteCommon}
{
  Figures may have been moved with respect to the surrounding text.
  Minor typographical corrections and presentational changes have
  been made without comment.
}

\newcommand{\TransNoteText}
{
  \TransNoteCommon

  This PDF file is formatted for screen viewing, but may be easily
  formatted for printing. Please consult the preamble of the \LaTeX\
  source file for instructions.
}

%% Re-set if ForPrinting=true
\ifthenelse{\boolean{ForPrinting}}{%
  \renewcommand{\Margins}{hmarginratio=2:3} % Asymmetric margins
  \renewcommand{\HLinkColor}{black}         % Hyperlink color
  \renewcommand{\PDFPageLayout}{TwoPageRight}
  \renewcommand{\TransNote}{Transcriber's Note}
  \renewcommand{\TransNoteText}{%
  \TransNoteCommon

  This PDF file is formatted for printing, but may be easily formatted
  for screen viewing. Please see the preamble of the \LaTeX\ source
  file for instructions.
  }
}{}
%%%%%%%%%%%%%%%%%%%%%%%%%%%%%%%%%%%%%%%%%%%%%%%%%%%%%%%%%%%%%%%%%
%%%%  End of PRINTING/SCREEN VIEWING code; back to packages  %%%%
%%%%%%%%%%%%%%%%%%%%%%%%%%%%%%%%%%%%%%%%%%%%%%%%%%%%%%%%%%%%%%%%%

% Text block size carefully chosen to accommodate tall illos
\usepackage[body={5.2in,8.125in},\Margins]{geometry}[2002/07/08]

\providecommand{\ebook}{00000}    % Overridden during white-washing
\usepackage[pdftex,
  hyperref,
  hyperfootnotes=false,
  pdftitle={The Project Gutenberg eBook \#\ebook: Space, Time and Gravitation},
  pdfauthor={Arthur Stanley Eddington},
  pdfkeywords={David Clarke, Andrew D. Hwang,
               Project Gutenberg Online Distributed Proofreading Team,
               The Internet Archive/American Libraries},
  pdfstartview=Fit,    % default value
  pdfstartpage=1,      % default value
  pdfpagemode=UseNone, % default value
  bookmarks=true,      % default value
  linktocpage=false,   % default value
  pdfpagelayout=\PDFPageLayout,
  pdfdisplaydoctitle,
  pdfpagelabels=true,
  bookmarksopen=true,
  bookmarksopenlevel=1,
  colorlinks=true,
  linkcolor=\HLinkColor]{hyperref}[2007/02/07]

%%%% Re-crop screen-formatted version, omit blank verso pages %%%%
\ifthenelse{\boolean{ForPrinting}}
  {}
  {\hypersetup{pdfpagescrop = 100 90 512 780}
  % If ForPrinting=false, don't skip to recto
  \renewcommand{\cleardoublepage}{\clearpage}
}


%%%%%%%%%%%%%%%%%%%%%%%%%%%%% COMMANDS %%%%%%%%%%%%%%%%%%%%%%%%%%%%%%%

% For various ad hoc alignment needs
\newlength{\TmpLen}

%%%% Fixed-width environment to format PG boilerplate %%%%
% 9.2pt leaves no overfull hbox at 80 char line width
\newenvironment{PGtext}{%
\begin{alltt}
\fontsize{9.2}{10.5}\ttfamily\selectfont}%
{\end{alltt}}

% Extra index entries, e.g., ``Geometry, Euclidean''
\newcommand{\IndexExtra}[1]%
{%
  \ifthenelse{\boolean{IndexExtras}}%
  {\index{#1}}%
  {}%
}

% Copyright page formatting
\newcommand{\Publine}[2]{\makebox[\TmpLen][s]{\textsc{#1} : \textsc{#2}}}


% Cross-referencing: anchors
\newcommand{\Pagelabel}[1]
  {\phantomsection\label{page:#1}}

\newcommand{\Figlabel}[1]
  {\phantomsection\label{fig:#1}}

\newcommand{\Tag}[1]%
  {\tag{#1}\phantomsection\label{eqn:#1}}

% and links
\newcommand{\Pageref}[1]
  {\hyperref[page:#1]{p.~\pageref{page:#1}}}

\newcommand{\Figref}[1]%
  {\hyperref[fig:#1]{\texorpdfstring{Fig.~#1}{Fig #1}}}

\newcommand{\Noteref}[1]%
  {\hyperref[appnote:#1]{\texorpdfstring{Note~#1}{Note #1}}}


% \Eqref{equation}{5}, \Eqref{formula}{2}, etc.
\newcommand{\Eqref}[2]{\hyperref[eqn:#2]{#1~(#2)}}

\newcommand{\Chapref}[1]{\hyperref[chapter:#1]%
  {Chapter~\textsc{\MakeLowercase{#1}}}}


%%%% Table of contents %%%%
% Dot leader for chapter-level entries
\makeatletter
\renewcommand{\l@chapter}{\@dottedtocline{0}{0pt}{0pt}}
\makeatother

% No page numbers
\newcommand{\TableofContents}{{\let\thepage=\empty\tableofcontents}}

% Centered chapter-level titles
\newlength{\ToCBox}
\settowidth{\ToCBox}{\scshape chapter~viii}% Widest title

\newcommand{\ToCCenter}[2]%
{\normalfont\scshape\null\hfill%
  \makebox[\ToCBox][l]{\scshape #1~\MakeLowercase{#2}}\hfill}

% The table of contents may be set across two pages; we'll ensure the
% word ``page'' is printed at the top of the page number column,
% centered on the first page (to match the scan) and right-justified
% on subsequent pages (better visual appearance).
%
% Each chapter-like unit (Prologue, Chapter, Appendix) puts a \ToCLine
% into the toc file. The Prologue command defines a \ToCAnchor macro,
% which expands to the current page of the table of contents.
%
% Subsequent units put down a label and get the \pageref. If this has
% changed, the word ``page'' is written at the right margin, and the
% \ToCAnchor command is updated.
\newcommand{\ToCLine}[2]%
{\label{#1-toc:#2}%
\ifthenelse{\equal{#1}{prologue}}%
  {\ToCCenter{#1}{#2}\makebox[0pt][c]{\footnotesize page}}% else...
  {\ifthenelse{\not\equal{\pageref{#1-toc:#2}}{\ToCAnchor}}%
  {\renewcommand{\ToCAnchor}{\pageref{#1-toc:#2}}%
   \ToCCenter{#1}{#2}\makebox[0pt][r]{\footnotesize page}}% else...
  {\ToCCenter{#1}{#2}}}}


% [** PP: ``Frontispiece'' heading in ToC would overlap dot leaders]
% To prevent this, write the code below into the toc file:
% \newlength{\FPlen}
% \settowidth{\FPlen}{\small\textit{Frontis}}
% \makeatletter
% \renewcommand{\@pnumwidth}{\FPlen}
% \makeatother
\newcommand{\FrontispieceToCEntry}
{
\addtocontents{toc}{\protect\newlength{\protect\FPlen}}

\ifthenelse{\boolean{ForPrinting}}
{\addtocontents{toc}{\protect\settowidth{\protect\FPlen}%
  {\protect\small\protect\textit{Frontis}}}} % About half the width
{\addtocontents{toc}{\protect\settowidth{\protect\FPlen}%
  {\protect\small\protect\textit{Frontispie}}}} % A bit wider

\addtocontents{toc}{\protect\makeatletter}
\addtocontents{toc}{\protect\renewcommand{\protect\@pnumwidth}{\protect\FPlen}}
\addtocontents{toc}{\protect\makeatother}

% The actual contents
\ifthenelse{\boolean{ForPrinting}}
{\addtocontents{toc}%
{\protect\contentsline{chapter}%
  {\hyperref[frontispiece]{\protect\scshape Eclipse Instruments at Sobral}}%
  {\protect\textit{\protect\makebox[12pt][c]{\protect\small Frontispiece}}}{}}}
% else move ``frontispiece'' heading slightly farther to the left
{\addtocontents{toc}%
{\protect\contentsline{chapter}%
  {\hyperref[frontispiece]{\protect\scshape Eclipse Instruments at Sobral}}%
  {\protect\textit{\protect\makebox[36pt][c]{\protect\small Frontispiece}}}{}}}

% And restore \@pnumwidth
\addtocontents{toc}{\protect\settowidth{\protect\FPlen}{100}}
\addtocontents{toc}{\protect\makeatletter}
\addtocontents{toc}{\protect\renewcommand{\protect\@pnumwidth}{\protect\FPlen}}
\addtocontents{toc}{\protect\makeatother}
}

% redefine hyperref's re-definition
% so that chapter anchor is above chapter title
\makeatletter
\AtBeginDocument{% in case hyperref clobbers this
\def\@schapter#1{%
  \begingroup
    \let\@mkboth\@gobbletwo
    \Hy@GlobalStepCount\Hy@linkcounter
    \xdef\@currentHref{\Hy@chapapp*.\the\Hy@linkcounter}%
    \Hy@raisedlink{%
      \hyper@anchorstart{\@currentHref}\hyper@anchorend
    }%
  \endgroup
  \H@old@schapter{#1}%
}}
\makeatother



% Sectioning: Paragraph, Chapter, Preface, Prologue, Appendix

% Each chapter starts unindented, with smallcaps
\newcommand{\First}[1]{\noindent\textsc{#1}}

% Appendix Notes are indented
\newcommand{\Indent}{\hspace*{\parindent}}

\newcommand{\Paragraph}[1]{#1} % Null semantic markup

\newcommand{\ChapterHead}[1]%
{\protect\centering\normalfont\upshape\textsc{\Large \MakeUppercase{#1}}}

% \Chapter[title for ToC]{N}{Title} -- for numbered chapters
\newcommand{\Chapter}[3][]{%
  % Clear stale heading on previous page
  \ifthenelse{\boolean{ForPrinting}}%
    {\fancyhead[RE]{}}{}%

  % page formatting
  \chapter*{\ChapterHead{Chapter~#2}\\ %
            \protect\centering\textsc{\large \MakeUppercase{#3}}}

  \label{chapter:#2}

  % ToC entry
  \addtocontents{toc}{\protect\filbreak}
  \addtocontents{toc}{\protect\ToCLine{chapter}{#2}}

  \ifthenelse{\equal{#1}{}}
  {\addcontentsline{toc}{chapter}
   {\texorpdfstring{\protect\scshape\protect{#3}}{#3}}
    % Running heads
    \fancyhead{}
    \fancyhead[C]{\textsc{\MakeUppercase{#3}}}}
  {\addcontentsline{toc}{chapter}
   {\texorpdfstring{\protect\scshape\protect{#1}}{#1}}
    \fancyhead{}
    \fancyhead[C]{\textsc{\MakeUppercase{#1}}}}

  \thispagestyle{empty}

  \ifthenelse{\boolean{ForPrinting}}%
    {\fancyhead[RO,LE]{\thepage}
     \fancyhead[RE]{\textsc{[ch.}}
     \fancyhead[LO]{\textsc{\MakeLowercase{#2}]}}}% End of ForPrinting
    {\fancyhead[R]{\thepage}
     \fancyhead[L]{\textsc{[ch.~\MakeLowercase{#2}]}}}
}

% Prologue
\newcommand{\Prologue}{%
  % page formatting
  \chapter*{\ChapterHead{Prologue}\\ %
            \centering\textsc{\large WHAT IS GEOMETRY?}}

  \label{prologue}% Location in document

  \addtocontents{toc}%
    {\protect\newcommand{\protect\ToCAnchor}{\protect\pageref{prolog-toc}}}

  % ToC entry
  \addtocontents{toc}{\protect\ToCLine{prologue}{}}
  \addtocontents{toc}{\protect\label{prolog-toc}}% Location in ToC
  \addcontentsline{toc}{chapter}%
    {\texorpdfstring{\protect\scshape What is Geometry?}{What is Geometry?}}

  % Running heads
  \fancyhead{}
  \fancyhead[CE]{\textsc{PROLOGUE}}
  \fancyhead[CO]{\textsc{WHAT IS GEOMETRY?}}

  \thispagestyle{empty}

  \ifthenelse{\boolean{ForPrinting}}%
    {\fancyhead[RO,LE]{\thepage}}% End of ForPrinting
    {\fancyhead[R]{\thepage}}
}


% Preface
\makeatletter
\newcommand{\Preface}{%
  % page formatting
  \chapter*{\ChapterHead{Preface}}

  % Bookmark; No ToC entry
  % access the anchor created by the \chapter* command
  \xdef\foo{chapter*.\the\Hy@linkcounter}%
  \Hy@writebookmark{}{Preface}{\foo}{0}{toc}%

  \label{preface}

  % Running heads
  \setlength{\headheight}{14.5pt}

  % add some stretch to paragraph breaks to reduce underfull pages
  \setlength\parskip{0pt plus 3pt}

  \pagestyle{fancy}
  \fancyhead{}
  \fancyfoot{}
  \fancyhead[C]{\textsc{PREFACE}}
  \thispagestyle{empty}

  \ifthenelse{\boolean{ForPrinting}}%
    {\fancyhead[RO,LE]{\thepage}}% End of ForPrinting
    {\fancyhead[R]{\thepage}}
}


% Appendix
\newcommand{\Appendix}{%
  % Clear stale heading on previous page
  \ifthenelse{\boolean{ForPrinting}}%
    {\fancyhead[RE]{}}{}%

  % page formatting
  % the preliminary \pdfbookmark ensures the anchor is above the heading
  % can't re-use the \chapter* anchor because a parent and child bookmark
  % must have distinct anchors
  \chapter*{\pdfbookmark[-1]{Appendix}{Appendix}\ChapterHead{Appendix}\\ %
            \centering\textsc{\large MATHEMATICAL NOTES}}

  \label{appendix}

  % ToC entry -- PDF bookmarks auto-generated
  \addtocontents{toc}{\protect\filbreak}
  \addtocontents{toc}{\protect\ToCLine{appendix}{}}
  \addcontentsline{toc}{chapter}
  {\texorpdfstring{\protect\scshape Mathematical Notes}{Mathematical Notes}}

  % Running heads
  \fancyhead{}
  \fancyhead[CE]{\textsc{APPENDIX}}
  \fancyhead[CO]{\textsc{MATHEMATICAL NOTES}}

  \thispagestyle{empty}

  \ifthenelse{\boolean{ForPrinting}}%
    {\fancyhead[RO,LE]{\thepage}}%
    {\fancyhead[R]{\thepage}}
}
\makeatother


% Appendix Notes \AppNote{2}{(p.~20)}
\newcommand{\AppNote}[2]{%
  % page formatting
  \section*{\centering\textbf{Note~#1} \textrm{\normalsize #2}}
  \label{appnote:#1}
}


% Define custom index format
\makeatletter
\renewcommand{\@idxitem}{\par\hangindent 30\p@\global\let\idxbrk\nobreak}
\renewcommand\subitem{\idxbrk\@idxitem \hspace*{15\p@}\let\idxbrk\relax}
\renewcommand{\indexspace}{\par\penalty-3000 \vskip 10pt plus5pt minus3pt\relax}

% raw TeX manipulations are to position the bookmark anchor above the heading
\renewenvironment{theindex}
  {\setlength\columnseprule{0.5pt}\setlength\columnsep{18pt}%
  \begin{multicols}{2}[%
    \begin{center}\Large IN\setbox0=\hbox{\phantomsection
       \vbox to40pt{\hsize=20pt\IndexBookmark\vss}\hss}%
       \ht0=0pt\dp0=0pt\wd0=0pt\box0DEX\vspace*{12pt}\end{center}]%
  \setlength\parindent{0pt}\setlength\parskip{0pt plus 0.3pt}%
  \thispagestyle{empty}\let\item\@idxitem\raggedright }
  {\end{multicols}\clearpage\fancyhead{}\cleardoublepage}
\makeatother
\newcommand\IndexBookmark{\pdfbookmark[-1]{Index}{Index}}


% Contents heading
\AtBeginDocument{\renewcommand{\contentsname}%
  {\protect\centering\normalfont\large\scshape CONTENTS\protect\\[0pt]}}

% Illustrations
\newcommand{\Graphic}[3][]
  {\includegraphics[width=#2]{./images/#3.pdf}%
  \ifthenelse{\not\equal{#1}{}}{\Figlabel{#1}}{}}

% %%%%% GLOBAL STYLE PARAMETERS %%%%%
\setlength{\parindent}{1em}

\newlength{\QIndent}
\setlength{\QIndent}{0.6\parindent}

% Chapter quotations
% \Quote[break]{Author (Date)}{Text}, etc.
\newcommand{\Signature}[1]{\allowbreak\null\nobreak%
  \hfill\nobreak\raisebox{-2ex}{#1}}

\newcommand{\Quote}[3][]%
{{\par\noindent\hspace*{\QIndent}\small#3%
\ifthenelse{\equal{#1}{break}}{\hfill\break\null}{}%
\hfill\textsc{\footnotesize #2}%
\ifthenelse{\equal{#1}{break}}{}{\hspace*{\QIndent}}\medskip}}


% Macros used in only one or two locations
\newcommand{\Actor}[1]{\par\hangindent 4.5\parindent \hangafter 1\qquad #1}

\newcommand{\Ditto}{\normalfont\ttfamily\textquotesingle$\!$\textquotesingle}

\newcommand{\Magnitude}{{}^\text{m}}

\newcommand{\Neg}{\phantom{-}}

\DeclareInputMath{176}{\mbox{\textdegree}}
\DeclareInputMath{183}{\cdot}


\newcommand{\HalfTitleBlock}%
{\settowidth{\TmpLen}{\textbf{\MyHuge GRAVITATION}}
\begin{minipage}{\TmpLen}
\noindent{\textbf{\MyHuge SPACE \hfill TIME}}\\[\MySkip]
\textbf{\LARGE\null\hfill AND\hfill\null}\\[\MySkip]
\textbf{\MyHuge GRAVITATION}
\end{minipage}}


\makeindex

%%%%%%%%%%%%%%%%%%%%%%%% START OF DPALIGN %%%%%%%%%%%%%%%%%%%%%%%%%%
\makeatletter
\providecommand\shortintertext\intertext
\newcount\DP@lign@no
\newtoks\DP@lignb@dy
\newif\ifDP@cr
\newif\ifbr@ce
\def\f@@zl@bar{\null}
\def\addto@DPbody#1{\global\DP@lignb@dy\@xp{\the\DP@lignb@dy#1}}
\def\parseb@dy#1{\ifx\f@@zl@bar#1\f@@zl@bar
    \addto@DPbody{{}}\let\@next\parseb@dy
  \else\ifx\end#1
    \let\@next\process@DPb@dy
    \ifDP@cr\else\addto@DPbody{\DPh@@kr&\DP@rint}\@xp\addto@DPbody\@xp{\@xp{\the\DP@lign@no}&}\fi
    \addto@DPbody{\end}
  \else\ifx\intertext#1
    \def\@next{\eat@command0}%
  \else\ifx\shortintertext#1
    \def\@next{\eat@command1}%
  \else\ifDP@cr\addto@DPbody{&\DP@lint}\@xp\addto@DPbody\@xp{\@xp{\the\DP@lign@no}&\DPh@@kl}
          \DP@crfalse\fi
    \ifx\begin#1\def\begin@stack{b}
      \let\@next\eat@environment
  \else\ifx\lintertext#1
    \let\@next\linter@text
  \else\ifx\rintertext#1
    \let\@next\rinter@text
  \else\ifx\\#1
    \addto@DPbody{\DPh@@kr&\DP@rint}\@xp\addto@DPbody\@xp{\@xp{\the\DP@lign@no}&\\}\DP@crtrue
    \global\advance\DP@lign@no\@ne
    \let\@next\parse@cr
  \else\check@braces#1!Q!Q!Q!\ifbr@ce\addto@DPbody{{#1}}\else
    \addto@DPbody{#1}\fi
    \let\@next\parseb@dy
  \fi\fi\fi\fi\fi\fi\fi\fi\@next}
\def\process@DPb@dy{\let\lintertext\@gobble\let\rintertext\@gobble
  \@xp\start@align\@xp\tw@\@xp\st@rredtrue\@xp\m@ne\the\DP@lignb@dy}
\def\linter@text#1{\@xp\DPlint\@xp{\the\DP@lign@no}{#1}\parseb@dy}
\def\rinter@text#1{\@xp\DPrint\@xp{\the\DP@lign@no}{#1}\parseb@dy}
\def\DPlint#1#2{\@xp\def\csname DP@lint:#1\endcsname{\text{#2}}}
\def\DPrint#1#2{\@xp\def\csname DP@rint:#1\endcsname{\text{#2}}}
\def\DP@lint#1{\ifbalancedlrint\@xp\ifx\csname DP@lint:#1\endcsname\relax\phantom
  {\csname DP@rint:#1\endcsname}\else\csname DP@lint:#1\endcsname\fi
  \else\csname DP@lint:#1\endcsname\fi}
\def\DP@rint#1{\ifbalancedlrint\@xp\ifx\csname DP@rint:#1\endcsname\relax\phantom
  {\csname DP@lint:#1\endcsname}\else\csname DP@rint:#1\endcsname\fi
  \else\csname DP@rint:#1\endcsname\fi}
\def\eat@command#1#2{\ifcase#1\addto@DPbody{\intertext{#2}}\or
  \addto@DPbody{\shortintertext{#2}}\fi\DP@crtrue
  \global\advance\DP@lign@no\@ne\parseb@dy}
\def\parse@cr{\new@ifnextchar*{\parse@crst}{\parse@crst{}}}
\def\parse@crst#1{\addto@DPbody{#1}\new@ifnextchar[{\parse@crb}{\parseb@dy}}
\def\parse@crb[#1]{\addto@DPbody{[#1]}\parseb@dy}
{\catcode`\$=13\gdef\check@braces#1#2!Q!Q!Q!{\ifx#2$$\br@cefalse\else\br@cetrue\fi
  }\gdef${\textbf{\huge ERROR}\GenericError{\space\space\space\@spaces\@spaces\@spaces}%
  {!!! DPalign/gather brace-parsing problem}%
  {Likely nested argument beginning with doubled character}%
  {Try putting an empty group at the start of the argument}\let$\relax}}
\def\eat@environment#1{\addto@DPbody{\begin{#1}}\begingroup
  \def\@currenvir{#1}\let\@next\digest@env\@next}
\def\digest@env#1\end#2{%
  \edef\begin@stack{\push@begins#1\begin\end \@xp\@gobble\begin@stack}%
  \ifx\@empty\begin@stack
    \@checkend{#2}
    \endgroup\let\@next\parseb@dy\fi
    \addto@DPbody{#1\end{#2}}
    \@next}
\def\lintertext{lint}\def\rintertext{rint}
\newif\ifbalancedlrint
\let\DPh@@kl\empty\let\DPh@@kr\empty
\def\DPg@therl{&\omit\hfil$\displaystyle}
\def\DPg@therr{$\hfil}

\newenvironment{DPalign*}[1][a]{%
  \if m#1\balancedlrintfalse\else\balancedlrinttrue\fi
  \global\DP@lign@no\z@\DP@crfalse
  \DP@lignb@dy{&\DP@lint0&}\parseb@dy
}{%
  \endalign
}
\newenvironment{DPgather*}[1][a]{%
  \if m#1\balancedlrintfalse\else\balancedlrinttrue\fi
  \global\DP@lign@no\z@\DP@crfalse
  \let\DPh@@kl\DPg@therl
  \let\DPh@@kr\DPg@therr
  \DP@lignb@dy{&\DP@lint0&\DPh@@kl}\parseb@dy
}{%
  \endalign
}
\makeatother

%%%%%%%%%%%%%%%%%%%%%%%% END  OF  DPALIGN %%%%%%%%%%%%%%%%%%%%%%%%%%

% to avoid over/underfull boxes without using explicit linebreaks
\def\stretchyspace{\spaceskip0.5em plus 0.5em minus 0.25em}

%%%%%%%%%%%%%%%%%%%%%%%% START OF DOCUMENT %%%%%%%%%%%%%%%%%%%%%%%%%%

\begin{document}

\pagestyle{empty}
\pagenumbering{Alph}
\phantomsection
\pdfbookmark[-1]{Front Matter}{Front Matter}

%%%% PG BOILERPLATE %%%%
\Pagelabel{PGBoilerplate}
\phantomsection
\pdfbookmark[0]{PG Boilerplate}{Project Gutenberg Boilerplate}

\begin{center}
\begin{minipage}{\textwidth}
\small
\begin{PGtext}
Project Gutenberg's Space, Time and Gravitation, by A. S. Eddington

This eBook is for the use of anyone anywhere at no cost and with
almost no restrictions whatsoever.  You may copy it, give it away or
re-use it under the terms of the Project Gutenberg License included
with this eBook or online at www.gutenberg.org


Title: Space, Time and Gravitation
       An Outline of the General Relativity Theory

Author: A. S. Eddington

Release Date: August 24, 2009 [EBook #29782]

Language: English

Character set encoding: ISO-8859-1

*** START OF THIS PROJECT GUTENBERG EBOOK SPACE, TIME AND GRAVITATION ***
\end{PGtext}
\end{minipage}
\end{center}

\clearpage


%%%% Credits and transcriber's note %%%%
\begin{center}
\begin{minipage}{\textwidth}
\begin{PGtext}
Produced by David Clarke, Andrew D. Hwang and the Online
Distributed Proofreading Team at http://www.pgdp.net (This
file was produced from images generously made available
by The Internet Archive/American Libraries.)
\end{PGtext}
\end{minipage}
\end{center}
\vfill

\begin{minipage}{0.85\textwidth}
\small
\pdfbookmark[0]{Transcriber's Note}{Transcriber's Note}
\subsection*{\centering\normalfont\scshape%
\normalsize\MakeLowercase{\TransNote}}%

\raggedright
\TransNoteText
\end{minipage}


%%%%%%%%%%%%%%%%%%%%%%%%%%% FRONT MATTER %%%%%%%%%%%%%%%%%%%%%%%%%%

\frontmatter

\pagenumbering{roman}
\pagestyle{empty}

\normalsize

%% -----File: 001.png---Folio -9-------

%Title Page
\cleardoublepage

\null\vfil
\begin{center}
\HalfTitleBlock
\end{center}
\vfil

%% -----File: 002.png---Folio -8-------

%Copyright Page

\clearpage
\settowidth{\TmpLen}{\textsc{CAMBRIDGE UNIVERSITY PRESS}}
\addtolength{\TmpLen}{0.25in} % [** PP: Hard-coded padding]
\null
\vfill
\begin{center}\stretchyspace
\makebox[\TmpLen][s]{\textsc{CAMBRIDGE UNIVERSITY PRESS}} \\[0.15in]
\addtolength{\TmpLen}{-0.125in}
\begin{minipage}{\TmpLen}
\centering
\footnotesize
\textsc{C. F. CLAY, Manager} \\[0.125in]
\Publine{\small LONDON}{FETTER LANE, E.C.~4}\\[0.1in]

\Graphic{1in}{device} %Publisher's device

\Publine{NEW YORK}{THE MACMILLAN CO.} \\
\textsc{BOMBAY}\hfill\break
\textsc{CALCUTTA}\smash{$\!\left.\rule[-12pt]{0pt}{12pt}\right\}$}\hfill
\textsc{MACMILLAN AND CO., Ltd.} \\
\textsc{MADRAS}\hfill\break
\Publine{TORONTO}{THE MACMILLAN CO. OF} \\
\textsc{CANADA, Ltd.} \\
\Publine{TOKYO}{MARUZEN-KABUSHIKI-KAISHA}
\end{minipage}

\vspace*{0.5in}
\textsc{\scriptsize ALL RIGHTS RESERVED}
\end{center}
\vfill

%% -----File: 003.png---Folio -7-------
%[Blank Page]
%% -----File: 004.png---Folio -6-------

% [** PP: Force frontispiece to verso if ForPrinting]
\ifthenelse{\boolean{ForPrinting}}
{\cleardoublepage
\null\vfill
\newpage}{}

\begin{center}
\pdfbookmark[0]{Frontispiece}{Frontispiece}
\rotatebox[origin=c]{90}{%
\begin{minipage}{8in} % [** PP: hard-coded size]
\centering
\includegraphics[width=7.6in]{./images/frontis_bw.jpg}\\
\hspace*{1em}{\scriptsize
\textit{C.~Davidson\hfill Frontispiece\hfill
See \hyperref[page:117]{page~\upshape{\pageref{page:117}}}}}\hspace*{1em}\null\\
{\textsc{\small eclipse instruments at sobral}}
\end{minipage}}
\end{center}
\phantomsection
\label{frontispiece}

\FrontispieceToCEntry

%% -----File: 005.png---Folio -5-------

%Title Page

\cleardoublepage
\begin{center}
\HalfTitleBlock\\[\MySkip]

\textbf{\Large AN OUTLINE OF THE GENERAL}\\[0.125in]
\textbf{\Large RELATIVITY THEORY}\\[0.5in]
{\large BY}\\[0.2in]
\textbf{\Large A.~S. EDDINGTON, M.A., M.Sc., F.R.S.}\\[0.15in]
{\footnotesize PLUMIAN PROFESSOR OF ASTRONOMY AND EXPERIMENTAL\\
PHILOSOPHY, CAMBRIDGE}

\vfill

\textbf{\Large CAMBRIDGE}\\[0.15in]
\textbf{\Large AT THE UNIVERSITY PRESS}\\[0.15in]
\textbf{\large 1920}
\end{center}

%% -----File: 006.png---Folio -4-------

\clearpage
\null\vfil
\begin{center}
\begin{minipage}{4.5in}
\begin{verse}
\hfil\qquad\qquad Perhaps to move \\
His laughter at their quaint opinions wide \\
Hereafter, when they come to model heaven \\
And calculate the stars: how they will wield \\
The mighty frame: how build, unbuild, contrive \\
To save appearances.\Signature{\textit{Paradise Lost.}}
\end{verse}
\end{minipage}
\end{center}
\vfil

%% -----File: 007.png---Folio -3-------


\Preface

\First{By} his theory of relativity Albert Einstein has provoked a
revolution of thought in physical science.

The achievement consists essentially in this:---Einstein has
succeeded in separating far more completely than hitherto the
share of the observer and the share of external nature in the
things we see happen. The perception of an object by an observer
depends on his own situation and circumstances; for example,
distance will make it appear smaller and dimmer. We make
allowance for this almost unconsciously in interpreting what we
see. But it now appears that the allowance made for the \textit{motion}
of the observer has hitherto been too crude---a fact overlooked
because in practice all observers share nearly the same motion,
that of the earth. Physical space and time are found to be
closely bound up with this motion of the observer; and only an
amorphous combination of the two is left inherent in the external
world. When space and time are relegated to their proper source---the
observer---the world of nature which remains appears
strangely unfamiliar; but it is in reality simplified, and the
underlying unity of the principal phenomena is now clearly
revealed. The deductions from this new outlook have, with one
doubtful exception, been confirmed when tested by experiment.

It is my aim to give an account of this work without introducing
anything very technical in the way of mathematics,
physics, or philosophy. The new view of space and time, so
opposed to our habits of thought, must in any case demand
unusual mental exercise. The results appear strange; and the
incongruity is not without a humorous side. For the first nine
chapters the task is one of interpreting a clear-cut theory,
accepted in all its essentials by a large and growing school of
physicists---although perhaps not everyone would accept the
author's views of its meaning.
% [** PP: Reads ``Chapters X and XI'' in original; modify for screen version]
\ifthenelse{\boolean{ForPrinting}}%
{Chapters~\hyperref[chapter:X]{\textsc{x}} and~\hyperref[chapter:XI]{\textsc{xi}}}%
{\Chapref{X} and~\Chapref{XI}}
deal with
very recent advances, with regard to which opinion is more
fluid. As for the last chapter, containing the author's speculations
on the meaning of nature, since it touches on the rudiments
of a philosophical system, it is perhaps too sanguine to hope that
it can ever be other than controversial.
%% -----File: 008.png---Folio -2-------

A non-mathematical presentation has necessary limitations;
and the reader who wishes to learn how certain exact results
follow from Einstein's, or even Newton's, law of gravitation is
bound to seek the reasons in a mathematical treatise. But this
limitation of range is perhaps less serious than the limitation of
intrinsic truth. There is a relativity of truth, as there is a
relativity of space.---%
\begin{center}
``For \textsc{is} and \textsc{is-not} though \textit{with} Rule and Line\\[-3ex]
\phantom{``}And \textsc{up-and-down} \textit{without}, I could define.''
\end{center}
Alas! It is not so simple. We abstract from the phenomena that
which is peculiar to the position and motion of the observer;
but can we abstract that which is peculiar to the limited imagination
of the human brain? We think we can, but only in the
symbolism of mathematics. As the language of a poet rings with
a truth that eludes the clumsy explanations of his commentators,
so the geometry of relativity in its perfect harmony expresses a
truth of form and type in nature, which my bowdlerised version
misses.

But the mind is not content to leave scientific Truth in a dry
husk of mathematical symbols, and demands that it shall be
alloyed with familiar images. The mathematician, who handles~$x$
so lightly, may fairly be asked to state, not indeed the inscrutable
meaning of~$x$ in nature, but the meaning which~$x$
conveys to \textit{him}.

Although primarily designed for readers without technical
knowledge of the subject, it is hoped that the book may also
appeal to those who have gone into the subject more deeply.
A few notes have been added in the Appendix mainly to bridge
the gap between this and more mathematical treatises, and to
indicate the points of contact between the argument in the text
and the parallel analytical investigation.

It is impossible adequately to express my debt to contemporary
literature and discussion. The writings of Einstein,
Minkowski, Hilbert, Lorentz, Weyl, Robb, and others, have
provided the groundwork; in the give and take of debate with
friends and correspondents, the extensive ramifications have
gradually appeared. \Signature{A.~S.~E.\qquad}
\medskip

\qquad{\small 1~\textit{May}, 1920.}

%% -----File: 009.png---Folio -1-------

%[** Table of Contents]

\cleardoublepage
\phantomsection
\pdfbookmark[0]{Contents}{Contents}
\TableofContents % Arrange for empty pagestyle

\fancyhead{}

\iffalse %%%%%%%%%% BEGIN DEAD CODE %%%%%%%%%%
CONTENTS

ECLIPSE INSTRUMENTS AT SOBRAL .  Frontispiece

PROLOGUE                            PAGE

WHAT IS GEOMETRY?   .  .  .  .  .  .  1

CHAPTER I

THE FITZGERALD CONTRACTION  .  .  .  17

CHAPTER II

RELATIVITY   .  .  .  .  .  .  .  .  30

CHAPTER III

THE WORLD OF FOUR DIMENSIONS   .  .  45

CHAPTER IV

FIELDS OF FORCE .  .  .  .  .  .  .  63

CHAPTER V

KINDS OF SPACE  .  .  .  .  .  .  .  77

CHAPTER VI

THE NEW LAW OF GRAVITATION AND THE
OLD LAW   .  .  .  .  .  .  .  .  .  93

CHAPTER VII

WEIGHING LIGHT .  .  .  .  .  .  .  110

CHAPTER VIII

OTHER TESTS OF THE THEORY  .  .  .  123

CHAPTER IX

MOMENTUM AND ENERGY  .  .  .  .  .  136

CHAPTER X

TOWARDS INFINITY  .  .  .  .  .  .  152

CHAPTER XI

ELECTRICITY AND GRAVITATION   .  .  167

CHAPTER XII

ON THE NATURE OF THINGS .  .  .  .  180

APPENDIX

MATHEMATICAL NOTES   .  .  .  .  .  202

HISTORICAL NOTE   .  .  .  .  .  .  210
\fi %%%%%%%%%% END OF DEAD CODE %%%%%%%%%%

%% -----File: 010.png---Folio 0-------
%[Blank Page]
%% -----File: 011.png---Folio 1-------

\mainmatter
\pagenumbering{arabic}

\phantomsection
\pdfbookmark[-1]{Main Matter}{Main Matter}

\Prologue

{\small
\qquad A conversation between---%
\Actor{An experimental \textsc{Physicist}.}
\Actor{A pure \textsc{Mathematician}.}
\Actor{A \textsc{Relativist}, who advocates the newer conceptions of time
and space in physics.}

}\medskip
\index{Euclidean geometry}%
\index{Geometry!Euclidean}%


\textit{Rel}. There is a well-known proposition of Euclid which states
that ``Any two sides of a triangle are together greater than the
third side.'' Can either of you tell me whether nowadays there
is good reason to believe that this proposition is true?

\textit{Math}. For my part, I am quite unable to say whether the
proposition is true or not. I can deduce it by trustworthy
reasoning from certain other propositions or axioms, which are
supposed to be still more elementary. If these axioms are true,
the proposition is true; if the axioms are not true, the proposition
is not true universally. Whether the axioms are true or not
I cannot say, and it is outside my province to consider.

\textit{Phys}. But is it not claimed that the truth of these axioms is
self-evident?

\textit{Math}. They are by no means self-evident to me; and I think
the claim has been generally abandoned.

\textit{Phys}. Yet since on these axioms you have been able to found
a logical and self-consistent system of geometry, is not this
indirect evidence that they are true?

\textit{Math}. No. Euclid's geometry is not the only self-consistent
system of geometry. By choosing a different set of axioms I can,
for example, arrive at Lobatchewsky's geometry, in which many
of the propositions of Euclid are not in general true. From my
point of view there is nothing to choose between these different
geometries.%
\index{Geometry!Lobatchewskian}%
\index{Lobatchewsky}%

\textit{Rel}. How is it then that Euclid's geometry is so much the
most important system?

\textit{Math}. I am scarcely prepared to admit that it is the most
important. But for reasons which I do not profess to understand,
my friend the Physicist is more interested in Euclidean geometry
%% -----File: 012.png---Folio 2-------
than in any other, and is continually setting us problems in it.
Consequently we have tended to give an undue share of attention
to the Euclidean system. There have, however, been great
geometers like Riemann who have done something to restore
a proper perspective.%
\index{Riemann}%

\textit{Rel}. (to Physicist). Why are you specially interested in
Euclidean geometry? Do you believe it to be the true geometry?

\textit{Phys}. Yes. Our experimental work proves it true.

\textit{Rel}. How, for example, do you prove that any two sides of
a triangle are together greater than the third side?

\textit{Phys}. I can, of course, only prove it by taking a very large
number of typical cases, and I am limited by the inevitable
inaccuracies of experiment. My proofs are not so general or so
perfect as those of the pure mathematician. But it is a recognised
principle in physical science that it is permissible to generalise
from a reasonably wide range of experiment; and this kind of
proof satisfies me.

\textit{Rel}. It will satisfy me also. I need only trouble you with
a special case. Here is a triangle $ABC$; how will you prove that
$AB + BC$ is greater than~$AC$?

\textit{Phys}. I shall take a scale and measure the three sides.

\textit{Rel}. But we seem to be talking about different things. I was
speaking of a proposition of geometry---properties of space, not
of matter. Your experimental proof only shows how a material
scale behaves when you turn it into different positions.

\textit{Phys}. I might arrange to make the measures with an optical
device.

\textit{Rel}. That is worse and worse. Now you are speaking of
properties of light.

\textit{Phys}. I really cannot tell you anything about it, if you will
not let me make measurements of any kind. Measurement is
my only means of finding out about nature. I am not a metaphysicist.

\textit{Rel}. Let us then agree that by \textit{length} and \textit{distance} you always
mean a quantity arrived at by measurements with material or
optical appliances. You have studied experimentally the laws
obeyed by these \textit{measured lengths}, and have found the geometry
to which they conform. We will call this geometry ``Natural
Geometry'';
\index{Geometry!natural}%
\index{Natural geometry}%
\index{Length!definition of}%
and it evidently has much greater importance for
%% -----File: 013.png---Folio 3-------
you than any other of the systems which the brain of the
mathematician has invented. But we must remember that its
subject matter involves the behaviour of material scales---the
properties of matter. Its laws are just as much laws of physics
as, for example, the laws of electromagnetism.

\textit{Phys}. Do you mean to compare space to a kind of magnetic
field? I scarcely understand.%
\index{Space!meaning of}%

\textit{Rel}. You say that you cannot explore the world without
some kind of apparatus. If you explore with a scale, you find
out the natural geometry; if you explore with a magnetic needle,
you find out the magnetic field. What we may call the field of
extension, or space-field, is just as much a physical quality as
the magnetic field. You can think of them both existing together
in the aether, if you like. The laws of both must be determined
by experiment. Of course, certain approximate laws of the space-field
(Euclidean geometry) have been familiar to us from childhood;
but we must get rid of the idea that there is anything
inevitable about these laws, and that it would be impossible to
find in other parts of the universe space-fields where these laws
do not apply. As to how far space really resembles a magnetic
field, I do not wish to dogmatise; my point is that they present
themselves to experimental investigation in very much the same
way.

Let us proceed to examine the laws of natural geometry.
I have a tape-measure, and here is the triangle. $AB = 39\frac{1}{2}$~in.,
$BC = \frac{1}{8}$~in., $CA = 39\frac{7}{8}$~in. Why, your proposition does not hold!

\textit{Phys}. You know very well what is wrong. You gave the
tape-measure a big stretch when you measured~$AB$.

\textit{Rel}. Why shouldn't I?

\textit{Phys}. Of course, a length must be measured with a rigid
scale.

\textit{Rel}. That is an important addition to our definition of length.
But what is a rigid scale?%
\index{Rigid scale, definition of}%

\textit{Phys}. A scale which always keeps the same length.

\textit{Rel}. But we have just defined length as the quantity arrived
at by measures with a rigid scale; so you will want another rigid
scale to test whether the first one changes length; and a third
to test the second; and so \textit{ad infinitum}. You remind me of the
incident of the clock and time-gun in Egypt. The man in charge
%% -----File: 014.png---Folio 4-------
of the time-gun fired it by the clock; and the man in charge of
the clock set it right by the time-gun. No, you must not define
length by means of a rigid scale, and define a rigid scale by
means of length.

\textit{Phys}. I admit I am hazy about strict definitions. There is
not time for everything; and there are so many interesting
things to find out in physics, which take up my attention. Are
you so sure that you are prepared with a logical definition of all
the terms you use?

\textit{Rel}. Heaven forbid! I am not naturally inclined to be
rigorous about these things. Although I appreciate the value of
the work of those who are digging at the foundations of science,
my own interests are mainly in the upper structure. But sometimes,
if we wish to add another storey, it is necessary to deepen
the foundations. I have a definite object in trying to arrive at
the exact meaning of length. A strange theory is floating round,
to which you may feel initial objections; and you probably
would not wish to let your views go by default. And after all,
when you claim to determine lengths to eight significant figures,
you must have a pretty definite standard of right and wrong
measurements.

\textit{Phys}. It is difficult to define what we mean by rigid; but in
practice we can tell if a scale is likely to change length appreciably
in different circumstances.

\textit{Rel}. No. Do not bring in the idea of change of length in
describing the apparatus for defining length. Obviously the
adopted standard of length cannot change length, whatever it
is made of. If a metre is defined as the length of a certain bar,
that bar can never be anything but a metre long; and if we
assert that this bar changes length, it is clear that we must have
changed our minds as to the definition of length. You recognised
that my tape-measure was a defective standard---that it was
not rigid. That was not because it changed length, because, if
it was the standard of length, it could not change length. It
was lacking in some other quality.

You know an approximately rigid scale when you see one.
What you are comparing it with is not some non-measurable
ideal of length, but some attainable, or at least approachable,
ideal of material constitution. Ordinary scales have defects---%
%% -----File: 015.png---Folio 5-------
flexure, expansion with temperature, etc.---which can be reduced
by suitable precautions; and the limit, to which you approach
as you reduce them, is your rigid scale. You can define these
defects without appealing to any extraneous definition of length;
for example, if you have two rods of the same material whose
extremities are just in contact with one another, and when one
of them is heated the extremities no longer can be adjusted to
coincide, then the material has a temperature-coefficient of
expansion. Thus you can compare experimentally the temperature-coefficients
of different metals and arrange them in
diminishing sequence. In this sort of way you can specify the
nature of your ideal rigid rod, before you introduce the term
length.

\textit{Phys}. No doubt that is the way it should be defined.

\textit{Rel}. We must recognise then that all our knowledge of space
rests on the behaviour of material measuring-scales free from
certain definable defects of constitution.

\textit{Phys}. I am not sure that I agree. Surely there is a sense in
which the statement $AB = 2CD$ is true or false, even if we had
no conception of a material measuring-rod. For instance, there
is, so to speak, twice as much paper between $A$ and~$B$, as between
$C$ and~$D$.

\textit{Rel}. Provided the paper is uniform. But then, what does
uniformity of the paper mean? That the amount in given length
is constant. We come back at once to the need of defining length.

If you say instead that the amount of ``space'' between
$A$ and~$B$ is twice that between $C$ and~$D$, the same thing applies.
You imagine the intervals filled with uniform space; but the
uniformity simply means that the same amount of space corresponds
to each inch of your rigid measuring-rod. You have
arbitrarily used your rod to divide space into so-called equal
lumps. It all comes back to the rigid rod.

I think you were right at first when you said that you could
not find out anything without measurement; and measurement
involves some specified material appliance.

Now you admit that your measures cannot go beyond a
certain close approximation, and that you have not tried all
possible conditions. Supposing that one corner of your triangle
was in a very intense gravitational field---far stronger than any
%% -----File: 016.png---Folio 6-------
we have had experience of---I have good ground for believing
that under those conditions you might find the sum of two sides
of a triangle, as measured with a rigid rod, appreciably less than
the third side. In that case would you be prepared to give up
Euclidean geometry?

\textit{Phys}. I think it would be risky to assume that the strong
force of gravitation made no difference to the experiment.

\textit{Rel}. On my supposition it makes an important difference.

\textit{Phys}. I mean that we might have to make corrections to the
measures, because the action of the strong force might possibly
distort the measuring-rod.

\textit{Rel}. In a rigid rod we have eliminated any special response
to strain.

\textit{Phys}. But this is rather different. The extension of the rod
is determined by the positions taken up by the molecules under
the forces to which they are subjected; and there might be a
response to the gravitational force which all kinds of matter
would share. This could scarcely be regarded as a defect; and
our so-called rigid rod would not be free from it any more than
any other kind of matter.

\textit{Rel}. True; but what do you expect to obtain by correcting
the measures? You correct measures, when they are untrue to
standard. Thus you correct the readings of a hydrogen-thermometer
to obtain the readings of a perfect gas-thermometer,
because the hydrogen molecules have finite size, and exert special
attractions on one another, and you prefer to take as standard
an ideal gas with infinitely small molecules. But in the present
case, what is the standard you are aiming at when you propose
to correct measures made with the rigid rod?

\textit{Phys}. I see the difficulty. I have no knowledge of space
apart from my measures, and I have no better standard than
the rigid rod. So it is difficult to see what the corrected measures
would mean. And yet it would seem to me more natural to
suppose that the failure of the proposition was due to the
measures going wrong rather than to an alteration in the character
of space.

\textit{Rel}. Is not that because you are still a bit of a metaphysicist?
You keep some notion of a space which is superior to measurement,
and are ready to throw over the measures rather than let
%% -----File: 017.png---Folio 7-------
this space be distorted. Even if there were reason for believing
in such a space, what possible reason could there be for assuming
it to be Euclidean? Your sole reason for believing space to be
Euclidean is that hitherto your measures have made it appear so;
if now measures of certain parts of space prefer non-Euclidean
geometry, all reason for assuming Euclidean space disappears.
Mathematically and conceptually Euclidean and non-Euclidean
space are on the same footing; our preference for Euclidean
space was based on measures, and must stand or fall by
measures.%
\index{Geometry!non-Euclidean, or Riemannian}%
\index{Non-Euclidean geometry}%
\index{Riemannian, or non-Euclidean, geometry}%

\textit{Phys}. Let me put it this way. I believe that I am trying to
measure something called length, which has an absolute meaning
in nature, and is of importance in connection with the laws of
nature. This length obeys Euclidean geometry. I believe my
measures with a rigid rod determine it accurately when no
disturbance like gravitation is present; but in a gravitational
field it is not unreasonable to expect that the uncorrected
measures may not give it exactly.

\textit{Rel}. You have three hypotheses there:---(1)~there is an
absolute thing in nature corresponding to length, (2)~the
geometry of these absolute lengths is Euclidean, and (3)~practical
measures determine this length accurately when there is no
gravitational force. I see no necessity for these hypotheses, and
propose to do without them. \textit{Hypotheses non fingo.} The second
hypothesis seems to me particularly objectionable. You assume
that this absolute thing in nature obeys the laws of Euclidean
geometry. Surely it is contrary to scientific principles to lay
down arbitrary laws for nature to obey; we must find out her
laws by experiment. In this case the only experimental evidence
is that measured lengths (which by your own admission are not
necessarily the same as this absolute thing) sometimes obey
Euclidean geometry and sometimes do not. Again it would
seem reasonable to doubt your third hypothesis beyond, say,
the sixth decimal place; and that would play havoc with your
more delicate measures. But where I fundamentally differ from
you is the first hypothesis. Is there some absolute quantity in
nature that we try to determine when we measure length?
When we try to determine the number of molecules in a given
piece of matter, we have to use indirect methods, and different
%% -----File: 018.png---Folio 8-------
methods may give systematically different results; but no one
doubts that there is a definite number of molecules, so that there
is some meaning in saying that certain methods are theoretically
good and others inaccurate. Counting appears to be an absolute
operation. But it seems to me that other physical measures are
on a different footing. Any physical quantity, such as length,
mass, force, etc., which is not a pure number, can only be defined
as the result arrived at by conducting a physical experiment
according to specified rules.%
\index{Space!meaning of}%

So I cannot conceive of any ``length'' in nature independent
of a definition of the way of measuring length. And, if there is,
we may disregard it in physics, because it is beyond the range
of experiment. Of course, it is always possible that we may
come across some quantity, not given directly by experiment,
which plays a fundamental part in theory. If so, it will turn up
in due course in our theoretical formulae. But it is no good
assuming such a quantity, and laying down \textit{a~priori} laws for it
to obey, on the off-chance of its proving useful.

\textit{Phys}. Then you will not let me blame the measuring-rod
when the proposition fails?

\textit{Rel}. By all means put the responsibility on the measuring-rod.
Natural geometry is the theory of the behaviour of material
scales. Any proposition in natural geometry is an assertion as
to the behaviour of rigid scales, which must accordingly take
the blame or credit. But do not say that the rigid scale is
wrong, because that implies a standard of right which does not
exist.

\textit{Phys}. The space which you are speaking of must be a sort of
abstraction of the extensional relations of matter.%
\index{Matter!extensional relations of}%

\textit{Rel}. Exactly so. And when I ask you to believe that space
can be non-Euclidean, or, in popular phrase, warped, I am not
asking you for any violent effort of the imagination;
\index{Warping of space}%
I only
mean that the extensional relations of matter obey somewhat
modified laws. Whenever we investigate the properties of space
experimentally, it is these extensional relations that we are
finding. Therefore it seems logical to conclude that space as
known to us must be the abstraction of these material relations,
and not something more transcendental. The reformed methods
of teaching geometry in schools would be utterly condemned,
%% -----File: 019.png---Folio 9-------
and it would be misleading to set schoolboys to verify propositions
of geometry by measurement, if the space they are supposed to
be studying had not this meaning.

I suspect that you are doubtful whether this abstraction of
extensional relations quite fulfils your general idea of space; and,
as a necessity of thought, you require something beyond. I do
not think I need disturb that impression, provided you realise
that it is not the properties of this more transcendental thing
we are speaking of when we describe geometry as Euclidean or
non-Euclidean.

\textit{Math}. The view has been widely held that space is neither
physical nor metaphysical, but conventional. Here is a passage
from Poincar�'s \textit{Science and Hypothesis}, which describes this
alternative idea of space:%
\index{Space!conventional}%

``If Lobatchewsky's geometry is true, the parallax of a very
distant star will be finite.
\index{Geometry!Lobatchewskian}%
\index{Lobatchewsky}%
If Riemann's is true, it will be negative.
These are the results which seem within the reach of experiment,
and it is hoped that astronomical observations may enable us
to decide between the two geometries. But what we call a
straight line in astronomy is simply the path of a ray of light.
If, therefore, we were to discover negative parallaxes, or to
prove that all parallaxes are higher than a certain limit, we
should have a choice between two conclusions: we could give
up Euclidean geometry, or modify the laws of optics, and
suppose that light is not rigorously propagated in a straight
line. It is needless to add that everyone would look upon this
solution as the more advantageous. Euclidean geometry,
therefore, has nothing to fear from fresh experiments.''

\textit{Rel}. Poincar�'s brilliant exposition is a great help in understanding
the problem now confronting us.
\index{Poincar�}%
He brings out the
interdependence between geometrical laws and physical laws,
which we have to bear in mind continually. We can add on to
one set of laws that which we subtract from the other set.
I admit that space is conventional---for that matter, the meaning
of every word in the language is conventional. Moreover, we
have actually arrived at the parting of the ways imagined by
Poincar�, though the crucial experiment is not precisely the
one he mentions. But I deliberately adopt the alternative,
which, he takes for granted, everyone would consider less
%% -----File: 020.png---Folio 10-------
advantageous. I call the space thus chosen \textit{physical space}, and
its geometry \textit{natural geometry}, thus admitting that other conventional
meanings of space and geometry are possible. If it
were only a question of the meaning of space---a rather vague
term---these other possibilities might have some advantages.
But the meaning assigned to length and distance has to go
along with the meaning assigned to space. Now these are
quantities which the physicist has been accustomed to measure
with great accuracy; and they enter fundamentally into the
whole of our experimental knowledge of the world. We have a
knowledge of the so-called extent of the stellar universe, which,
whatever it may amount to in terms of ultimate reality, is not
a mere description of location in a conventional and arbitrary
mathematical space. Are we to be robbed of the terms in which
we are accustomed to describe that knowledge?

The law of Boyle states that the pressure of a gas is proportional
to its density. It is found by experiment that this law is
only approximately true. A certain mathematical simplicity
would be gained by conventionally redefining \textit{pressure} in such
a way that Boyle's law would be rigorously obeyed. But it
would be high-handed to appropriate the word pressure in this
way, unless it had been ascertained that the physicist had no
further use for it in its original meaning.

\textit{Phys}. I have one other objection. Apart from measures, we
have a general perception of space, and the space we perceive
is at least approximately Euclidean.

\textit{Rel}. Our perceptions are crude measures.
\index{Perceptions, as crude measures}%
It is true that our
perception of space is very largely a matter of optical measures
with the eyes. If in a strong gravitational field optical and
mechanical measures diverged, we should have to make up our
minds which was the preferable standard, and afterwards abide
by it. So far as we can ascertain, however, they agree in all
circumstances, and no such difficulty arises. So, if physical
measures give us a non-Euclidean space, the space of perception
will be non-Euclidean. If you were transplanted into an extremely
intense gravitational field, you would directly perceive
the non-Euclidean properties of space.

\textit{Phys}. Non-Euclidean space seems contrary to reason.

\textit{Math}. It is not contrary to reason, but contrary to common
%% -----File: 021.png---Folio 11-------
experience, which is a very different thing, since experience is
very limited.

\textit{Phys}. I cannot imagine myself perceiving non-Euclidean space!

\textit{Math}. Look at the reflection of the room in a polished doorknob, %[** PP: Hyphenated across a line in original]
and imagine yourself one of the actors in what you see
going on there.

\textit{Rel}. I have another point to raise. The distance between
two points is to be the length measured with a rigid scale. Let
us mark the two points by particles of matter, because we must
somehow identify them by reference to material objects. For
simplicity we shall suppose that the two particles have no
relative motion, so that the distance---whatever it is---remains
constant. Now you will probably agree that there is no such
thing as absolute motion; consequently there is no standard
condition of the scale which we can call ``at rest.'' We may
measure with the scale moving in any way we choose, and if
results for different motions disagree, there is no criterion for
selecting the true one. Further, if the particles are sliding past
the scale, it makes all the difference what instants we choose
for making the two readings.

\textit{Phys}. You can avoid that by defining distance as the measurement
made with a scale which has the same velocity as the two
points. Then they will always be in contact with two particular
divisions of the scale.

\textit{Rel}. A very sound definition; but unfortunately it does not
agree with the meaning of distance in general use. When the
relativist wishes to refer to this length, he calls it the \textit{proper-length};
\index{Proper-length}%
in non-relativity physics it does not seem to have been
used at all. You see it is not convenient to send your apparatus
hurling through the laboratory---after a pair of $\alpha$ particles, for
example. And you could scarcely measure the length of a wave
of light by this convention\footnote%
{The proper-length of a light-wave is actually infinite.}. So the physicist refers his lengths
to apparatus at rest on the earth; and the mathematician starts
with the words ``Choose unaccelerated rectangular axes~$Ox$, $Oy$,
$Oz$,~$\dotsc$'' and assumes that the measuring-scales are at rest
relatively to these axes. So when the term length is used some
arbitrary standard motion of the measuring apparatus must
always be implied.

%% -----File: 022.png---Folio 12-------

\textit{Phys}. Then if you have fixed your standard motion of the
measuring-rod, there will be no ambiguity if you take the
readings of both particles at the same moment.

\textit{Rel}. What is the same moment at different places? The
conception of simultaneity in different places is a difficult one.
Is there a particular instant in the progress of time on another
world, Arcturus, which is the same as the present instant on the
Earth?%
\index{Simultaneity}%

\textit{Phys}. I think so, if there is any connecting link. We can
observe an event, say a change of brightness, on Arcturus, and,
allowing for the time taken by light to travel the distance,
determine the corresponding instant on the earth.

\textit{Rel}. But then you must know the speed of the earth through
the aether. It may have shortened the light-time by going some
way to meet the light coming from Arcturus.

\textit{Phys}. Is not that a small matter?

\textit{Rel}. At a very modest reckoning the motion of the earth in
the interval might alter the light-time by several days. Actually,
however, any speed of the earth through the aether up to the
velocity of light is admissible, without affecting anything observable.
At least, nothing has been discovered which contradicts
this. So the error may be months or years.

\textit{Phys}. What you have shown is that we have not sufficient
knowledge to determine in practice which are simultaneous
events on the Earth and Arcturus. It does not follow that there
is no definite simultaneity.%
\index{Absolute simultaneity}%

\textit{Rel}. That is true, but it is at least possible that the reason
why we are unable to determine simultaneity in practice (or,
what comes to pretty much the same thing, our motion through
the aether) in spite of many brilliant attempts, is that there is
no such thing as absolute simultaneity of distant events. It is
better therefore not to base our physics on this notion of absolute
simultaneity, which may turn out not to exist, and is in any
case out of reach at present.

But what all this comes to is that time as well as space is
implied in all our measures. The fundamental measurement is
not the interval between two points of space, but between two
points of space associated with instants of time.

Our natural geometry is incomplete at present. We must
%% -----File: 023.png---Folio 13-------
supplement it by bringing in time as well as space. We shall
need a perfect clock as well as a rigid scale for our measures.
\index{Clock!perfect}%
It may be difficult to choose an ideal standard clock; but whatever
definition we decide on must be a physical definition. We
must not dodge it by saying that a perfect clock is one which
keeps perfect time. Perhaps the best theoretical clock would be
a pulse of light travelling in vacuum to and fro between mirrors
at the ends of a rigid scale. The instants of arrival at one end
would define equal intervals of time.%
\index{Time!measurement of}%

\textit{Phys}. I think your unit of time would change according to
the motion of your ``clock'' through the aether.

\textit{Rel}. Then you are comparing it with some notion of absolute
time. I have no notion of time except as the result of measurement
with some kind of clock. (Our immediate perception of
the flight of time is presumably associated with molecular
processes in the brain which play the part of a material clock.)
If you know a better clock, let us adopt it; but, having once
fixed on our ideal clock there can be no appeal from its judgments.
You must remember too that if you wish to measure
a second \textit{at one place}, you must keep your clock fixed at what
you consider to be one place; so its motion is defined. The
necessity of defining the motion of the clock emphasises that
one cannot consider time apart from space; there is one geometry
comprising both.

\textit{Phys}. Is it right to call this study \textit{geometry}? %[** PP: Changed . to ?]
Geometry deals
with space alone.

\textit{Math}. I have no objection. It is only necessary to consider
time as a fourth dimension. Your complete natural geometry
will be a geometry of four dimensions.%
\index{Fourth dimension}%

\textit{Phys}. Have we then found the long-sought fourth dimension?

\textit{Math}. It depends what kind of a fourth dimension you were
seeking. Probably not in the sense you intend. For me it only
means adding a fourth variable,~$t$, to my three space-variables
$x$, $y$,~$z$. It is no concern of mine what these variables really
represent. You give me a few fundamental laws that they
satisfy, and I proceed to deduce other consequences that may
be of interest to you. The four variables may for all I know be
the pressure, density, temperature and entropy of a gas; that
is of no importance to me. But you would not say that a gas
%% -----File: 024.png---Folio 14-------
had four dimensions because four mathematical variables were
used to describe it. Your use of the term ``dimensions'' is
probably more restricted than mine.

\textit{Phys}. I know that it is often a help to represent pressure
and volume as height and width on paper; and so geometry
may have applications to the theory of gases. But is it not going
rather far to say that geometry can deal directly with these
things and is not necessarily concerned with lengths in
space?

\textit{Math}. No. Geometry is nowadays largely analytical, so that
in form as well as in effect, it deals with variables of an unknown
nature. It is true that I can often see results more easily by
taking my $x$ and~$y$ as lengths on a sheet of paper. Perhaps it
would be helpful in seeing other results if I took them as pressure
and density in a steam-engine; but a steam-engine is not so
handy as a pencil. It is literally true that I do not want to
know the significance of the variables $x$, $y$, $z$, $t$ that I am discussing.
That is lucky for the Relativist, because although he has defined
carefully how they are to be measured, he has certainly not
conveyed to me any notion of how I am to picture them, if my
picture of absolute space is an illusion.

\textit{Phys}. Yours is a strange subject. You told us at the beginning
that you are not concerned as to whether your propositions are
true, and now you tell us you do not even care to know what
you are talking about.

\textit{Math}. That is an excellent description of Pure Mathematics,
which has already been given by an eminent mathematician\footnotemark.
\index{Russell}%
  \footnotetext{``Pure mathematics consists entirely of such asseverations as that, if such
  and such a proposition is true of \textit{anything}, then such and such a proposition
  is true of that thing. It is essential not to discuss whether the first proposition
  is really true, and not to mention what the anything is of which it is supposed
  to be true\ldots. Thus mathematics may be defined as the subject in which we
  never know what we are talking about, nor whether what we are saying is true.''
  \Signature{\textsc{Bertrand Russell}.}}

\textit{Rel}. I think there is a real sense in which time is a fourth
dimension---as distinct from a fourth variable. The term
dimension seems to be associated with relations of \textit{order}.
\index{Order and dimensions}%
I believe that the order of events in nature is one indissoluble
four-dimensional order. We may split it arbitrarily into space
and time, just as we can split the order of space into length,
%% -----File: 025.png---Folio 15-------
breadth and thickness. But space without time is as incomplete
as a surface without thickness.

\textit{Math}. Do you argue that the real world behind the phenomena
is four-dimensional?

\textit{Rel}. I think that in the real world there must be a set of
entities related to one another in a four-dimensional order, and
that these are the basis of the perceptual world so far as it is
yet explored by physics. But it is possible to pick out a four-dimensional
set of entities from a basal world of five dimensions,
or even of three dimensions. The straight lines in three-dimensional
space form a four-dimensional set of entities, i.e.\
they have a four-fold % [** PP: Regularized fourfold]
order. So one cannot predict the ultimate
number of dimensions in the world---if indeed the expression
\textit{dimensions} is applicable.

\textit{Phys}. What would a philosopher think of these conceptions?
Or is he solely concerned with a metaphysical space and time
which is not within reach of measurement.

\textit{Rel}. In so far as he is a psychologist our results must concern
him. Perception is a kind of crude physical measurement;
\index{Perceptions, as crude measures}%
and
perceptual space and time is the same as the measured space
and time, which is the subject-matter of natural geometry. In
other respects he may not be so immediately concerned.
Physicists and philosophers have long agreed that motion
through absolute space can have no meaning;
\index{Space!meaning of}%
but in physics
the question is whether motion through aether has any meaning.
I consider that it has no meaning; but that answer, though it
brings philosophy and physics into closer relation, has no bearing
on the philosophic question of absolute motion. I think,
however, we are entitled to expect a benevolent interest from
philosophers, in that we are giving to their ideas a perhaps
unexpected practical application.
\bigskip

%[** Thought break]

Let me now try to sum up my conclusions from this conversation.
We have been trying to give a precise meaning to the
term \textit{space}, so that we may be able to determine exactly the
properties of the space we live in. There is no means of determining
the properties of our space by \textit{a~priori} reasoning, because
there are many possible kinds of space to choose from, no one
of which can be considered more likely than any other. For
%% -----File: 026.png---Folio 16-------
more than 2000 years we have believed in a Euclidean space,
because certain experiments favoured it; but there is now reason
to believe that these same experiments when pushed to greater
accuracy decide in favour of a slightly different space (in the
neighbourhood of massive bodies). The relativist sees no reason
to change the rules of the game because the result does not
agree with previous anticipations. Accordingly when he speaks
of space, he means the space revealed by measurement, whatever
its geometry. He points out that this is the space with which
physics is concerned; and, moreover, it is the space of everyday
perception. If his right to appropriate the term space in this
way is challenged, he would urge that this is the sense in which
the term has always been used in physics hitherto; it is only
recently that conservative physicists, frightened by the revolutionary
consequences of modern experiments, have begun to
play with the idea of a pre-existing space whose properties
cannot be ascertained by experiment---a metaphysical space, to
which they arbitrarily assign Euclidean properties, although it
is obvious that its geometry can never be ascertained by experiment.
But the relativist, in defining space as \textit{measured space},
clearly recognises that all measurement involves the use of
material apparatus; the resulting geometry is specifically a study
of the extensional relations of matter. He declines to consider
anything more transcendental.

My second point is that since natural geometry is the study
of extensional relations of natural objects, and since it is found
that their space-order cannot be discussed without reference to
their time-order as well, it has become necessary to extend our
geometry to four dimensions in order to include time.
%% -----File: 027.png---Folio 17-------

\Chapter{I}{The FitzGerald Contraction}

\Quote{Descartes.}
{In order to reach the Truth, it is necessary, once in one's life, to put every
thing in doubt---so far as possible.}


\First{Will} it take longer to swim to a point $100$~yards up-stream
and back, or to a point $100$~yards across-stream and back?

In the first case there is a long toil up against the current,
and then a quick return helped by the current, which is all too
short to compensate. In the second case the current also hinders,
because part of the effort is devoted to overcoming the drift
down-stream. But no swimmer will hesitate to say that the
hindrance is the greater in the first case.

Let us take a numerical example. Suppose the swimmer's
speed is $50$~yards a minute in still water, and the current is
$30$~yards a minute. Thus the speed against the current is~$20$,
and with the current $80$~yards a minute. The up journey then
takes $5$~minutes and the down journey $1\frac{1}{4}$ minutes. Total time,
$6\frac{1}{4}$ minutes.

%[Illustration: Fig. 1.]
\begin{wrapfigure}{r}{1.5in}
\Graphic[1]{1.375in}{027a}% \Figlabel{1}
\end{wrapfigure}
Going across-stream the swimmer must aim at a point~$E$ above
the point~$B$ where he wishes to arrive, so
that $OE$ represents his distance travelled
in still water, and $EB$ the amount he has
drifted down. These must be in the ratio
$50$ to~$30$, and we then know from the right-angled
triangle $OBE$ that $OB$ will correspond
to~$40$. Since $OB$ is $100$~yards, $OE$
is $125$~yards, and the time taken is $2\frac{1}{2}$
minutes. Another $2\frac{1}{2}$ minutes will be
needed for the return journey. Total time,
$5$~minutes.

In still water the time would have been $4$~minutes.

The up-and-down swim is thus longer than the transverse
swim in the ratio $6\frac{1}{4}:5$ minutes. Or we may write the ratio
\[
\dfrac{1}{\surd \bigl(1 - (\tfrac{30}{50})^2\bigr)}
\]
%% -----File: 028.png---Folio 18-------
which shows how the result depends on the ratio of the speed
of the current to the speed of the swimmer, viz.~$\frac{30}{50}$.

A very famous experiment on these lines was tried in America
in the year~1887. The swimmer was a wave of light, which we
know swims through the aether with a speed of $186,330$ miles
a second. The aether was flowing through the laboratory like
a river past its banks. The light-wave was divided, by partial
reflection at a thinly silvered surface, into two parts, one of
which was set to perform the up-and-down stream journey and
the other the across-stream journey. When the two waves
reached their proper turning-points they were sent back to the
starting-point by mirrors. To judge the result of the race, there
was an optical device for studying interference fringes; because
the recomposition of the two waves after the journey would
reveal if one had been delayed more than the other, so that, for
example, the crest of one instead of fitting on to the crest of
the other coincided with its trough.

To the surprise of Michelson and Morley, who conducted the
experiment, the result was a dead-heat.
\index{Michelson-Morley experiment}%
It is true that the
direction of the current of aether was not known---they hoped
to find it out by the experiment. That, however, was got over
by trying a number of different orientations. Also it was
possible that there might actually be no current at a particular
moment. But the earth has a velocity of $18\frac{1}{2}$ miles a second,
continually changing direction as it goes round the sun; so that
at some time during the year the motion of a terrestrial laboratory
through the aether must be at least $18\frac{1}{2}$ miles a second.
The experiment should have detected the delay by a much
smaller current; in a repetition of it by Morley and Miller
in~1905, a current of $2$~miles a second would have been
sufficient.

If we have two competitors, one of whom is known to be
slower than the other, and yet they both arrive at the winning-post
at the same time, it is clear that they cannot have travelled
equal courses. To test this, the whole apparatus was rotated
through a right angle, so that what had been the up-and-down
course became the transverse course, and \textit{vice versa}. Our two
competitors interchanged courses, but still the result was a
dead-heat.

%% -----File: 029.png---Folio 19-------

The surprising character of this result can be appreciated by
contrasting it with a similar experiment on sound-waves.
Sound consists of waves in air or other material, as light consists
of waves in aether. It would be possible to make a precisely
similar experiment on sound, with a current of air past the
apparatus instead of a current of aether. In that case the greater
delay of the wave along the direction of the current would
certainly show itself experimentally. Why does light seem to
behave differently?

The straightforward interpretation of this remarkable result
is that each course undergoes an automatic contraction when it
is swung from the transverse to the longitudinal position, so
that whichever arm of the apparatus is placed up-stream it
straightway becomes the shorter. The course is marked out in
the rigid material apparatus, and we have to suppose that the
length of any part of the apparatus changes as it is turned in
different directions with respect to the aether-current. It is
found that the kind of material---metal, stone or wood---makes
no difference to the experiment. The contraction must be the
same for all kinds of matter; the expected delay depends only
on the ratio of the speed of the aether current to the speed of
light, and the contraction which compensates it must be equally
definite.%
\index{Length!effect of motion on}%

This explanation was proposed by FitzGerald, and at first
sight it seems a strange and arbitrary hypothesis. But it has
been rendered very plausible by subsequent theoretical researches
of Larmor and Lorentz.
\index{Larmor}%
\index{Lorentz}%
Under ordinary circumstances the form
and size of a solid body is maintained by the forces of cohesion
between its particles. What is the nature of cohesion? We guess
that it is made up of electric forces between the molecules. But
the aether is the medium in which electric force has its seat;
hence it will not be a matter of indifference to these forces how
the electric medium is flowing with respect to the molecules.
When the flow changes there will be a readjustment of cohesive
forces, and we must expect the body to take a new shape and
size.

The theory of Larmor and Lorentz enables us to trace in
detail the readjustment. Taking the accepted formulae of
electromagnetic theory, they showed that the new form of
%% -----File: 030.png---Folio 20-------
equilibrium would be contracted in just such a way and by
just such an amount as FitzGerald's explanation requires\footnote%
{Appendix, \Noteref{1}.}.%
\Pagelabel{note1}%
\index{Contraction, FitzGerald}%
\index{FitzGerald Contraction}%

The contraction in most cases is extremely minute. We have
seen that when the ratio of the speed of the current to that
of the swimmer is $\frac{3}{5}$, a contraction in the ratio
$\surd \bigl(1 - (\frac{3}{5})^2\bigr)$
is needed to compensate for the delay. The earth's orbital
velocity is $\frac{1}{10000}$ of the velocity of light, so that it will give a
contraction of $\surd \bigl(1 - (\frac{1}{10000})^2\bigr)$, or $1$~part in~$200,000,000$. This
would mean that the earth's diameter in the direction of its
motion is shortened by $2\frac{1}{2}$ inches.

The Michelson-Morley experiment has thus failed to detect
our motion through the aether, because the effect looked for---the
delay of one of the light waves---is exactly compensated by
an automatic contraction of the matter forming the apparatus.
Other ingenious experiments have been tried, electrical and
optical experiments of a more technical nature. They likewise
have failed, because there is always an automatic compensation
somewhere. We now believe there is something in the nature
of things which inevitably makes these compensations, so that
it will never be possible to determine our motion through the
aether. Whether we are at rest in it, or whether we are rushing
through it with a speed not much less than that of light, will
make no difference to anything that can possibly be observed.

This may seem a rash generalization from the few experiments
actually performed; more particularly, since we can only experiment
with the small range of velocity caused by the earth's
orbital motion. With a larger range residual differences might
be disclosed. But there is another reason for believing that the
compensation is not merely approximate but exact. The compensation
has been traced theoretically to its source in the
well-known laws of electromagnetic force; and here it is mathematically
exact. Thus the generalization is justified, at least in
so far as the observed phenomena depend on electromagnetic
causes, and in so far as the universally accepted laws of electromagnetism
are accurate.

The generalization here laid down is called the restricted
Principle of Relativity:---\textit{It is impossible by any experiment to
detect uniform motion relative to the aether.}\Pagelabel{20}%
\index{Principle of Relativity (restricted)}%
\index{Relativity!restricted Principle of}%

%% -----File: 031.png---Folio 21-------

There are other natural forces which have not as yet been
recognised as coming within the electromagnetic scheme---gravitation,
for example---and for these other tests are required.
Indeed we were scarcely justified in stating above that the
diameter of the earth would contract $2\frac{1}{2}$~inches, because the
figure of the earth is determined mainly by gravitation, whereas
the Michelson-Morley experiment relates to bodies held together
by cohesion. There is fair evidence of a rather technical kind
that the compensation exists also for phenomena in which
gravitation is concerned; and we shall assume that the principle
covers all the forces of nature.

Suppose for a moment it were not so, and that it were possible
to determine a kind of absolute motion of the earth by experiments
or observations involving gravitation. Would this throw
light on our motion through the aether? I think not. It would
show that there is some standard of rest with respect to which
the law of gravitation takes a symmetrical and simple form;
presumably this standard corresponds to some gravitational
medium, and the motion determined would be motion with
respect to that medium. Similarly if the motion were revealed
by vital or psychical phenomena, it would be motion relative
to some vital or psychical medium. The aether, defined as the
seat of electric forces, must be revealed, if at all, by electric
phenomena.

It is well to remember that there is reasonable justification
for adopting the principle of relativity even if the evidence is
insufficient to prove it. In Newtonian dynamics the phenomena
are independent of uniform motion of the system; no explanation
is asked for, because it is difficult to see any reason why there
should be an effect.
\index{Gravitation!relativity for uniform motion}%
If in other phenomena the principle fails,
then we must seek for an explanation of its failure---and no
doubt a plausible explanation can be devised; but so long as
experiment gives no indication of a failure, it is idle to anticipate
such a complication. Clearly physics cannot concern itself with
all the possible complexities which \textit{may} exist in nature, but have
not hitherto betrayed themselves in any experiment.

The principle of relativity has implications of a most revolutionary
kind. Let us consider what is perhaps an exaggerated
case---or perhaps the actual case, for we cannot tell. Let the
%% -----File: 032.png---Folio 22-------
reader suppose that he is travelling through the aether at
$161,000$~miles a second vertically upwards; if he likes to make
the positive assertion that this is his velocity, no one will be
able to find any evidence to contradict him. For this speed the
FitzGerald contraction is just $\frac{1}{2}$, so that every object contracts
to half its original length when turned into the vertical position.%
\index{FitzGerald Contraction!consequences of}%

As you lie in bed, you are, say, $6$~feet long. Now stand upright;
you are $3$~feet. You are incredulous? Well, let us prove it!
Take a yard-measure; when turned vertically it must undergo
the FitzGerald contraction, and become only half a yard. If you
measure yourself with it, you will find you are just two---\textit{half-yards}.
``But I can see that the yard-measure does not change
length when I turn it.'' What you perceive is an image of the
rod on the retina of your eye; you imagine that the image
occupies the same space in both positions; but your retina has
contracted in the vertical direction without your knowing it, so
that your visual estimates of vertical length are double what
they should be. And so on with every test you can devise.
Because everything is altered in the same way, nothing appears
to be altered at all.

It is possible to devise electrical and optical tests; in that
case the argument is more complicated, because we must consider
the effect of the rapid current of aether on the electric
forces and on waves of light. But the final conclusion is always
the same; the tests will reveal nothing. Here is one illustration.
To avoid distortion of the retina, lie on your back on the floor,
and watch in a suitably inclined mirror someone turn the rod
from the horizontal to the vertical position. You will, of course,
see no change of length, and it is not possible to blame the
retina this time. But is the appearance in the mirror a faithful
reproduction of what is actually occurring?
\index{Mirror, distortion by moving}%
In a plane mirror
at rest the appearance is correct; the rays of light come off the
mirror at the same angle as they fall on to it, like billiard balls
rebounding from an elastic cushion. But if the cushion is in
rapid motion the angle of the billiard-ball will be altered; and
similarly the rapid motion of the mirror through the aether
alters the law of reflection.
\index{Reflection by moving mirror}%
Precise calculation shows that the
moving mirror will distort the image, so as to conceal exactly
the changes of length which occur.

%% -----File: 033.png---Folio 23-------

The mathematician does not need to go through all the
possible tests in detail; he knows that the complete compensation
is inherent in the fundamental laws of nature, and so must
occur in every case. So if any suggestion is made of a device
for detecting these effects, he starts at once to look for the
fallacy which must surely be there. Our motion through the
aether may be very much less than the value here adopted, and
the changes of length may be very small; but the essential point
is that they escape notice, not because they are small (if they
are small), but because from their very nature they are undetectable.

There is a remarkable reciprocity about the effects of motion
on length, which can best be illustrated by another example.
Suppose that by development in the powers of aviation, a man
flies past us at the rate of $161,000$ miles a second.
\index{Aviator, space and time-reckoning of|(}% [** PP: Using range]
We shall
suppose that he is in a comfortable travelling conveyance in
which he can move about, and act normally and that his length
is in the direction of the flight. If we could catch an instantaneous
glimpse as he passed, we should see a figure about three feet
high, but with the breadth and girth of a normal human being.
And the strange thing is that he would be sublimely unconscious
of his own undignified appearance. If he looks in a mirror in
his conveyance, he sees his usual proportions; this is because of
the contraction of his retina, or the distortion by the moving
mirror, as already explained. But when he looks down on us,
he sees a strange race of men who have apparently gone through
some flattening-out process; one man looks barely $10$~inches
across the shoulders, another standing at right angles is almost
``length and breadth, without thickness.'' As they turn about
they change appearance like the figures seen in the old-fashioned
convex-mirrors. If the reader has watched a cricket-match
through a pair of prismatic binoculars, he will have seen this
effect exactly.

It is the reciprocity of these appearances--that each party
should think the other has contracted---that is so difficult to
realise. Here is a paradox beyond even the imagination of
Dean Swift. Gulliver regarded the Lilliputians as a race of
dwarfs; and the Lilliputians regarded Gulliver as a giant. That
is natural. If the Lilliputians had appeared dwarfs to Gulliver,
%% -----File: 034.png---Folio 24-------
and Gulliver had appeared a dwarf to the Lilliputians---but no!
that is too absurd for fiction, and is an idea only to be found in
the sober pages of science.

This reciprocity is easily seen to be a necessary consequence
of the Principle of Relativity. The aviator must detect a FitzGerald
contraction of objects moving rapidly relatively to him,
just as we detect the contraction of objects moving relatively to us,
and as an observer at rest in the aether detects the contraction
of objects moving relatively to the aether. Any other result
would indicate an observable effect due to his own motion
through the aether.

Which is right? Are we or the aviator? Or are both the
victims of illusion? It is not illusion in the ordinary sense,
because the impressions of both would be confirmed by every
physical test or scientific calculation suggested. No one knows
which is right. No one will ever know, because we can never
find out which, if either, is truly at rest in the aether.

It is not only in space but in time that these strange variations
occur. If we observed the aviator carefully we should infer that
he was unusually slow in his movements; and events in the
conveyance moving with him would be similarly retarded---as
though time had forgotten to go on.
\index{Retardation of time}%
\index{Time!for moving observer}%
His cigar lasts twice as
long as one of ours. I said ``infer'' deliberately; we should \textit{see}
a still more extravagant slowing down of time; but that is easily
explained, because the aviator is rapidly increasing his distance
from us and the light-impressions take longer and longer to
reach us. The more moderate retardation referred to remains
after we have allowed for the time of transmission of light.

But here again reciprocity comes in, because in the aviator's
opinion it is we who are travelling at $161,000$ miles a second
past him; and when he has made all allowances, he finds that
it is we who are sluggish. Our cigar lasts twice as long as his.

Let us examine more closely how the two views are to be
reconciled. Suppose we both light similar cigars at the instant
he passes us. At the end of $30$~minutes our cigar is finished.
This signal, borne on the waves of light, hurries out at the rate
of $186,000$ miles a second to overtake the aviator travelling at
$161,000$ miles a second, who has had $30$~minutes start. It will
take nearly $194$~minutes to overtake him, giving a total time of
%% -----File: 035.png---Folio 25-------
$224$~minutes after lighting the cigar. His watch like everything
else about him (including his cigar) is going at half-speed; so
it records only $112$~minutes elapsed when our signal arrives.
The aviator knows, of course, that this is not the true time when
our cigar was finished, and that he must correct for the time of
transmission of the light-signal. He sets himself this problem---that
man has travelled away from me at $161,000$~miles a second
for an unknown time $x$~minutes; he has then sent a signal which
travels the same distance back at $186,000$~miles a second; the
total time is $112$~minutes; problem, find~$x$. Answer, $x = 60$
minutes. He therefore judges that our cigar lasted $60$~minutes,
or twice as long as his own. His cigar lasted $30$~minutes by his
watch (because the same retardation affects both watch and
cigar); and that was in our opinion twice as long as ours, because
his watch was going at half-speed.

Here is the full time-table.
\begin{center}
{\footnotesize
\begin{tabular}{r@{}c% @{} signifies no inter-column separation
  >{\centering\hspace{0pt}}m{1.5in}@{}%
  >{\centering\hspace{0pt}}m{1.75in}r@{}c} % end of alignment preamble
% Two multicolumn headers and extra vertical space
\multicolumn{2}{c}{\parbox[c]{0.625in}% [** PP: 0.7pt overfull]
  {\centering Stationary watch}}
      & Stationary Observer      & Aviator
      & \multicolumn{2}{c}{\parbox[c]{0.625in}%
          {\centering Aviator's watch}} \\[3ex]
%
  $0$ & min.   & Lights cigar   & Lights cigar   &   $0$ &   min. \\
 $30$ & \Ditto & Finishes cigar &    \ldots      &  $15$ & \Ditto \\
 $60$ & \Ditto & Inferred time aviator's cigar finished
                                & Finishes cigar &  $30$ & \Ditto \\
%
$112$ & \Ditto & Receives signal aviator's cigar finished
                                &    \ldots      &  $56$ & \Ditto \\
%
$120$ & \Ditto &     \ldots     & Inferred time stationary cigar finished
                                                 &  $60$ & \Ditto \\
%
$224$ & \Ditto &     \ldots     & Receives signal stationary cigar finished
                                                 & $112$ & \Ditto \\
\end{tabular}
}% End of \footnotesize
\end{center}

This is analysed from our point of view, not the aviator's;
because it makes out that he was wrong in his inference and we
were right. But no one can tell which was really right.

The argument will repay a careful examination, and it will
be recognised that the chief cause of the paradox is that we
assume that we are at rest in the aether, whereas the aviator
assumes that he is at rest. Consequently whereas in our opinion
the light-signal is overtaking him at merely the difference
between $186,000$ and $161,000$ miles a second, he considers that
it is coming to him through the relatively stationary aether at
the normal speed of light. It must be remembered that each
observer is furnished with complete experimental evidence in
support of his own assumption. If we suggest to the aviator
%% -----File: 036.png---Folio 26-------
that owing to his high velocity the relative speed of the wave
overtaking him can only be $25,000$ miles a second, he will reply
``I have determined the velocity of the wave relatively to me
by timing it as it passes two points in my conveyance; and it
turns out to be $186,000$ miles a second. So I know my correction
for light-time is right\footnotemark.''
  \footnotetext{We need not stop to prove this directly. If the aviator could detect anything
  in his measurements inconsistent with the hypothesis that he was at rest
  in the aether (e.g.\ a difference of velocity of overtaking waves of light and
  waves meeting him) it would contradict the restricted principle of relativity.}%
His clocks and scales are all behaving
in an extraordinary way from our point of view, so it is not
surprising that he should arrive at a measure of the velocity of
the overtaking wave which differs from ours; but there is no
way of convincing him that our reckoning is preferable.

Although not a very practical problem, it is of interest to
inquire what happens when the aviator's speed is still further
increased and approximates to the velocity of light.
\index{Light, velocity of!system moving with}%
\index{Time!``standing still''}%
\index{Velocity of light!system moving with}%
Lengths
in the direction of flight become smaller and smaller, until for
the speed of light they shrink to zero. The aviator and the
objects accompanying him shrink to two dimensions. We are
saved the difficulty of imagining how the processes of life can
go on in two dimensions, because nothing goes on. Time is
arrested altogether. This is the description according to the
terrestrial observer. The aviator himself detects nothing unusual;
he does not perceive that he has stopped moving. He is
merely waiting for the next instant to come before making the
next movement; and the mere fact that time is arrested means
that he does not perceive that the next instant is a long time
coming.

It is a favourite device for bringing home the vast distances
of the stars to imagine a voyage through space with the velocity
of light. The youthful adventurer steps on to his magic carpet
loaded with provisions for a century. He reaches his journey's
end, say Arcturus, a decrepit centenarian. This is wrong. It is
quite true that the journey would last something like a hundred
years by terrestrial chronology; but the adventurer would arrive
at his destination no more aged than when he started, and he
would not have had time to think of eating. So long as he travels
with the speed of light he has immortality and eternal youth.
%% -----File: 037.png---Folio 27-------
If in some way his motion were reversed so that he returned to
the earth again, he would find that centuries had elapsed here,
whilst he himself did not feel a day older---for him the voyage
had lasted only an instant\footnotemark.
  \footnotetext{Since the earth is moving relatively to our adventurer with the velocity
  of light, we might be tempted to argue that from this point of view the terrestrial
  observer would have perpetual youth whilst the voyager grew older. Evidently,
  if they met again, they could disprove one or other of the two arguments. But
  in order to meet again the velocity of one of them must be reversed by supernatural % [** PP: Hyphenated across a line in original]
  means or by an intense gravitational force so that the conditions are
  not symmetrical and reciprocity does not apply. The argument given in the
  text appears to be the correct one.}

Our reason for discussing at length the effects of these
improbably high velocities is simply in order that we may speak
of the results in terms of common experience; otherwise it
would be necessary to use the terms of refined technical measurement.
The relativist is sometimes suspected of an inordinate
fondness for paradox; but that is rather a misunderstanding of
his argument. The paradoxes exist when the new experimental
discoveries are woven into the scheme of physics hitherto
current, and the relativist is ready enough to point this out.
But the conclusion he draws is that a revised scheme of physics
is needed in which the new experimental results will find a natural
place without paradox.

To sum up---on any planet moving with a great velocity
through the aether, extraordinary changes of length of objects
are continually occurring as they move about, and there is a
slowing down of all natural processes as though time were
retarded. These things cannot be perceived by anyone on the
planet; but similar effects would be detected by any observer
having a great velocity relative to the planet (who makes all
allowances for the effect of the motion on the observations, but
takes it %[** PP: Typo ``if'']
for granted that he himself is at rest in the aether\footnotemark).
  \footnotetext{The last clause is perhaps unnecessary. The correction applied for light
  transmission will naturally be based on the observer's own experimental determination
  of the velocity of light. According to experiment the velocity of light
  relatively to him is \textit{apparently} the same in all directions, and he will apply
  the corrections accordingly. This is equivalent to assuming that he is at rest
  in the aether; but he need not, and probably would not, make the assumption
  explicitly.}%
There is complete reciprocity so that each of two observers in
relative motion will find the same strange phenomena occurring
%% -----File: 038.png---Folio 28-------
to the other; and there is nothing to help us to decide which is
right.%
\index{Aviator, space and time-reckoning of|)} % [** PP: Using range]

I think that no one can contemplate these results without
feeling that the whole strangeness must arise from something
perverse and inappropriate in our ordinary point of view.
Changes go on on a planet, all nicely balanced by adjustments
of natural forces, in such a way that no one on the planet can
possibly detect what is taking place. Can we seriously imagine
that there is anything in the reality behind the phenomena,
which reflects these changes? Is it not more probable that we
ourselves introduce the complexity, because our method of
description is not well-adapted to give a simple and natural
statement of what is really occurring?

The search for a more appropriate apparatus of description
leads us to the standpoint of relativity described in the next
chapter.
\index{Relativity, standpoint of}%
I draw a distinction between the principle and the
standpoint of relativity. The principle of relativity is a statement
of experimental fact, which may be right or wrong; the
first part of it---the restricted principle---has already been
enunciated. Its consequences can be deduced by mathematical
reasoning, as in the case of any other scientific generalization.
It postulates no particular mechanism of nature, \textit{and no particular
view as to the meaning of time and space}, though it may suggest
theories on the subject. The only question is whether it is
experimentally true or not.

The standpoint of relativity is of a different character. It
asserts first that certain unproved hypotheses as to time and
space have insensibly crept into current physical theories, and
that these are the source of the difficulties described above.
Now the most dangerous hypotheses are those which are tacit
and unconscious. So the standpoint of relativity proposes
tentatively to do without these hypotheses (not making any
others in their place); and it discovers that they are quite
unnecessary and are not supported by any known fact. This in
itself appears to be sufficient justification for the standpoint.
Even if at some future time facts should be discovered which
confirm the rejected hypotheses, the relativist is not wrong in
reserving them until they are required.

It is not our policy to take shelter in impregnable positions;
%% -----File: 039.png---Folio 29-------
and we shall not hesitate to draw reasonable conclusions as well
as absolutely proved conclusions from the knowledge available.
But to those who think that the relativity theory is a passing
phase of scientific thought, which may be reversed in the light
of future experimental discoveries, we would point out that,
though like other theories it may be developed and corrected,
there is a certain minimum statement possible which represents
irreversible progress. Certain hypotheses enter into all physical
descriptions and theories hitherto current, dating back in some
cases for 2000 years, in other cases for 200 years. It can now
be proved that these hypotheses have nothing to do with any
phenomena yet observed, and do not afford explanations of any
known fact. This is surely a discovery of the greatest importance---quite
apart from any question as to whether the hypotheses
are actually wrong.

I am not satisfied with the view so often expressed that the
sole aim of scientific theory is ``economy of thought.'' I cannot
reject the hope that theory is by slow stages leading us nearer
to the truth of things. But unless science is to degenerate into
idle guessing, the test of value of any theory must be whether
it expresses with as little redundancy as possible the facts
which it is intended to cover. Accidental truth of a conclusion
is no compensation for erroneous deduction.

The relativity standpoint is then a discarding of certain
hypotheses, which are uncalled for by any known facts, and
stand in the way of an understanding of the simplicity of nature.
%% -----File: 040.png---Folio 30-------


\Chapter{II}{Relativity}

\Quote{H.~Minkowski (1908).}
{The views of time and space, which I have to set forth, have their foundation
in experimental physics. Therein is their strength. Their tendency is revolutionary.
From henceforth space in itself and time in itself sink to mere shadows,
and only a kind of union of the two preserves an independent existence.}%
\index{Minkowski}%

\First{There} are two parties to every observation---the observed and
the observer.%
\index{Observer and observed}%

What we see depends not only on the object looked at, but
on our own circumstances---position, motion, or more personal
idiosyncracies. Sometimes by instinctive habit, sometimes by
design, we attempt to eliminate our own share in the observation,
and so form a general picture of the world outside us,
which shall be common to all observers. A small speck on the
horizon of the sea is interpreted as a giant steamer. From the
window of our railway carriage we see a cow glide past at fifty
miles an hour, and remark that the creature is enjoying a rest.
We see the starry heavens revolve round the earth, but decide
that it is really the earth that is revolving, and so picture the
state of the universe in a way which would be acceptable to an
astronomer on any other planet.

The first step in throwing our knowledge into a common
stock must be the elimination of the various individual standpoints
and the reduction to some specified standard observer.
The picture of the world so obtained is none the less relative.
We have not eliminated the observer's share; we have only
fixed it definitely.

To obtain a conception of the world from the point of view
of no one in particular is a much more difficult task. The
position of the observer can be eliminated; we are able to grasp
the conception of a chair as an object in nature---looked at all
round, and not from any particular angle or distance. We can
think of it without mentally assigning ourselves some position
with respect to it. This is a remarkable faculty, which has
evidently been greatly assisted by the perception of solid relief
%% -----File: 041.png---Folio 31-------
with our two eyes. But the motion of the observer is not
eliminated so simply. We had thought that it was accomplished;
but the discovery in the last chapter that observers with
different motions use different space- and time-reckoning shows
that the matter is more complicated than was supposed. It may
well require a complete change in our apparatus of description,
because all the familiar terms of physics refer primarily to the
relations of the world to an observer in some specified circumstances.

\Pagelabel{31}%
Whether we are able to go still further and obtain a knowledge
of the world, which not merely does not particularise the
observer, but does not postulate an observer at all; whether if
such knowledge could be obtained, it would convey any intelligible
meaning; and whether it could be of any conceivable
interest to anybody if it could be understood---these questions
need not detain us now. The answers are not necessarily
negative, but they lie outside the normal scope of physics.

The circumstances of an observer which affect his observations
are his position, motion and gauge of magnitude. More personal
idiosyncracies disappear if, instead of relying on his crude
senses, he employs scientific measuring apparatus. But scientific
apparatus has position, motion and size, so that these are still
involved in the results of any observation. There is no essential
distinction between scientific measures and the measures of the
senses. In either case our acquaintance with the external world
comes to us through material channels; the observer's body can
be regarded as part of his laboratory equipment, and, so far as
we know, it obeys the same laws. We therefore group together
perceptions and scientific measures, and in speaking of ``a
particular observer'' we include all his measuring appliances.%
\index{Perceptions, as crude measures}%

Position, motion, magnitude-scale---these factors have a profound
influence on the aspect of the world to us. Can we form
a picture of the world which shall be a synthesis of what is seen
by observers in all sorts of positions, having all sorts of velocities,
and all sorts of sizes? %[** PP: Changed . to ?]
\index{Synthesis of appearances}%
As already stated we have accomplished
the synthesis of positions. We have two eyes, which have
dinned into our minds from babyhood that the world has to be
looked at from more than one position. Our brains have so far
responded as to give us the idea of solid relief, which enables us
%% -----File: 042.png---Folio 32-------
to appreciate the three-dimensional world in a vivid way that
would be scarcely possible if we were only acquainted with
strictly two-dimensional pictures. We not merely deduce the
three-dimensional world; we see it. But we have no such aid
in synthesising different motions. Perhaps if we had been
endowed with two eyes moving with different velocities our
brains would have developed the necessary faculty; we should
have perceived a kind of relief in a fourth dimension so as to
combine into one picture the aspect of things seen with different
motions. Finally, if we had had two eyes of different sizes, we
might have evolved a faculty for combining the points of view
of the mammoth and the microbe.

It will be seen that we are not fully equipped by our senses
for forming an impersonal picture of the world. And it is
because the deficiency is manifest that we do not hesitate to
advocate a conception of the world which transcends the images
familiar to the senses. Such a world can perhaps be grasped,
but not pictured by the brain. It would be unreasonable to
limit our thought of nature to what can be comprised in sense-pictures.
As Lodge has said, our senses were developed by the
struggle for existence, not for the purpose of philosophising on
the world.%
\index{Lodge}%

Let us compare two well-known books, which might be
described as elementary treatises on relativity, \textit{Alice in Wonderland}
and \textit{Gulliver's Travels}. Alice was continually changing size,
sometimes growing, sometimes on the point of vanishing altogether.
Gulliver remained the same size, but on one occasion
he encountered a race of men of minute size with everything in
proportion, and on another voyage a land where everything was
gigantic. It does not require much reflection to see that both
authors are describing the same phenomenon---a relative change
of scale of observer and observed. Lewis Carroll took what is
probably the ordinary scientific view, that the observer had
changed, rather than that a simultaneous change had occurred
to all her surroundings. But it would never have appeared like
that to Alice; she could not have ``stepped outside and looked
at herself,'' picturing herself as a giant filling the room. She
would have said that the room had unaccountably shrunk.
Dean Swift took the truer view of the human mind when he
%% -----File: 043.png---Folio 33-------
made Gulliver attribute his own changes to the things around
him; it never occurred to Gulliver that his own size had altered;
and, if he had thought of the explanation, he could scarcely
have accustomed himself to that way of thinking. But both
points of view are legitimate. The size of a thing can only be
imagined as relative to something else; and there is no means of
assigning the change to one end of the relation rather than the
other.%
\index{Relativity of size}%

We have seen in the theory of the Michelson-Morley experiment
that, according to current physical views, our standard of
size---the rigid meas\-uring-rod---must change according to the
circumstances of its motion; and the aviator's adventures
illustrated a similar change in the standard of duration of time.
Certain rather puzzling irregularities have been discovered in
the apparent motions of the Sun, Mercury, Venus and the Moon;
but there is a strong family resemblance between these, which
leads us to believe that the real phenomenon is a failure of the
time-keeping of our standard clock, the Earth. Instances could
be multiplied where a change of the observer or his standards
produces or conceals changes in the world around him.

The object of the relativity theory, however, is not to attempt
the hopeless task of apportioning responsibility between the
observer and the external world, but to emphasise that in our
ordinary description and in our scientific description of natural
phenomena the two factors are indissolubly united. All the
familiar terms of physics---length, duration of time, motion,
force, mass, energy, and so on---refer primarily to this relative
knowledge of the world; and it remains to be seen whether any
of them can be retained in a description of the world which is
not relative to a particular observer.

Our first task is a description of the world independent of
the motion of the observer. The question of the elimination of
his gauge of magnitude belongs to a later development of the
theory discussed in \Chapref{XI}.
\index{Gauge!effect on observations}%
Let us draw a square $ABCD$ on
a sheet of paper, making the sides equal, to the best of our
knowledge. We have seen that an aviator flying at $161,000$
miles a second in the direction $AB$, would judge that the sides
$AB$, $DC$ had contracted to half their length, so that for him
the figure would be an oblong. If it were turned through a right
%% -----File: 044.png---Folio 34-------
angle $AB$ and $DC$ would expand and the other two sides contract---in
his judgment. For us, the lengths of $AB$ and $AC$ are
equal; for him, one length is twice the other. Clearly length
cannot be a property inherent in our drawing; it needs the
specification of some observer.

We have seen further that duration of time also requires that
an observer should be specified. The stationary observer and
the aviator disagreed as to whose cigar lasted the longer time.

Thus \textit{length} and \textit{duration} are not things inherent in the
external world; they are relations of things in the external
world to some specified observer.
\index{Duration, not inherent in external world}%
\index{Length!relativity of}%
\index{Relativity!of length and duration}%
\index{Space!relativity of}%
If we grasp this all the mystery
disappears from the phenomena described in \Chapref{I}. When
the rod in the Michelson-Morley experiment is turned through
a right angle it contracts; that naturally gives the impression
that something has happened to the rod itself. Nothing whatever
has happened to the rod---the object in the external world.
Its length has altered, but length is not an intrinsic property of
the rod, since it is quite indeterminate until some observer is
specified. Turning the rod through a right angle has altered the
relation to the observer (implied in the discussion of the experiment);
but the rod itself, or the relation of a molecule at one
end to a molecule at the other, is unchanged. Measurement of
length and duration is a comparison with partitions of space
and time drawn by the observer concerned, with the help of
apparatus which shares his motion. Nature is not concerned
with these partitions; it has, as we shall see later, a geometry
of its own which is of a different type.

Current physics has hitherto assumed that all observers are
not to be regarded as on the same footing, and that there is
some absolute observer whose judgments of length and duration
are to be treated with respect, because nature pays attention to
\textit{his} space-time partitions. He is supposed to be at rest in the
aether, and the aether materialises his space-partitions so that
they have a real significance in the external world. This is
sheer hypothesis, and we shall find it is unsupported by any
facts. Evidently our proper course is to pursue our investigations,
and call in this hypothetical observer only if we find there
is something which he can help to explain.

We have been leading up from the older physics to the new
%% -----File: 045.png---Folio 35-------
outlook of relativity, and the reader may feel some doubt as to
whether the strange phenomena of contraction and time-retardation,
that were described in the last chapter, are to be
taken seriously, or are part of a \textit{reductio ad absurdum} argument.
The answer is that we believe that the phenomena do occur as
described; only the description (like that of all observed phenomena)
concerns the relations of the external world to some
observer, and not the external world itself. The startling
character of the phenomena arises from the natural but fallacious
inference that they involve intrinsic changes in the objects
themselves.

We have been considering chiefly the observer's end of the
observation; we must now turn to the other end---the thing
observed. Although length and duration have no exact counterparts
in the external world, it is clear that there is a certain
ordering of things and events outside us which we must now
find more appropriate terms to describe. The order of events is
a four-fold order; we can arrange them as right-and-left, backwards-and-forwards,
up-and-down, sooner-and-later.
\index{Ordering of events in external world}%
An individual
may at first consider these as four independent orders,
but he will soon attempt to combine some of them. It is
recognised at once that there is no essential distinction between
right-and-left and backwards-and-forwards. The observer has
merely to turn through a right angle and the two are interchanged.
If he turns through a smaller angle, he has first to
combine them, and then to redivide them in a different way.
Clearly it would be a nuisance to continually combine and redivide;
so we get accustomed to the thought of leaving them
combined in a two-fold or two-dimensional order. The amalgamation
of up-and-down is less simple. There are obvious reasons
for considering this dimension of the world as fundamentally
distinct from the other two. Yet it would have been a great
stumbling-block to science if the mind had refused to combine
space into a three-dimensional whole. The combination has not
concealed the real distinction of horizontal and vertical, but has
enabled us to understand more clearly its nature---for what
phenomena it is relevant, and for what irrelevant. We can
understand how an observer in another country redivides the
combination into a different vertical and horizontal. We must
%% -----File: 046.png---Folio 36-------
now go further and amalgamate the fourth order, sooner-and-later.
This is still harder for the mind. It does not imply that
there is no distinction between space and time; but it gives a
fresh unbiassed start by which to determine what the nature of
the distinction is.

The idea of putting together space and time, so that time is
regarded as a fourth dimension, is not new. But until recently
it was regarded as merely a picturesque way of looking at things
without any deep significance. We can put together time and
temperature in a thermometer chart, or pressure and volume
on an indicator-diagram. It is quite non-committal. But our
theory is going to lead much further than that. We can lay
two dimensional surfaces---sheets of paper---on one another till
we build up a three-dimensional block; but there is a difference
between a block which is a pile of sheets and a solid block of
paper. The solid block is the true analogy for the four-dimensional
combination of space-time; it does not separate naturally
into a particular set of three-dimensional spaces piled in time-order.
It can be redivided into such a pile; \textit{but it can be redivided
in any direction we please}.

Just as the observer by changing his orientation makes a new
division of the two-dimensional plane into right-and-left, backwards-and-forwards---just
as the observer by changing his
longitude makes a new division of three-dimensional space into
vertical and horizontal---so the observer by \textit{changing his motion}
makes a new division of the four-dimensional order into time
and space.%
\index{Four-dimensional order}%

This will be justified in detail later; it indicates that observers
with different motions will have different time and space-reckoning---a
conclusion we have already reached from another
point of view.

Although different observers separate the four orders differently,
they all agree that the order of events is four-fold; and
it appears that this undivided four-fold order is the same for
all observers. We therefore believe that it is inherent in the
external world; it is in fact the synthesis, which we have been
seeking, of the appearances seen by observers having all sorts of
positions and all sorts of (uniform) motions. It is therefore to
be regarded as a conception of the real world not relative to any
particularly circumstanced observer.
%% -----File: 047.png---Folio 37-------

The term ``real world'' is used in the ordinary sense of physics,
without any intention of prejudging philosophical questions as
to reality.
\index{Real world of physics}%
It has the same degree of reality as was formerly
attributed to the three-dimensional world of scientific theory or
everyday conception, which by the advance of knowledge it
replaces. As I have already indicated, it is merely the accident
that we are not furnished with a pair of eyes in rapid relative
motion, which has allowed our brains to neglect to develop a
faculty for visualising this four-dimensional world as directly
as we visualise its three-dimensional section.

It is now easy to see that length and duration must be the
components of a single entity in the four-dimensional world of
space-time. Just as we resolve a structure into plan and elevation,
so we resolve extension in the four-dimensional world into
length and duration. The structure has a size and shape
independent of our choice of vertical. Similarly with things in
space-time. Whereas length and duration are relative, the
single ``extension'' of which they are components has an absolute
significance in nature, independent of the particular decomposition
into space and time separately adopted by the observer.%
\index{Extension in four dimensions}%

Consider two events; for example, the stroke of one o'clock
and the stroke of two o'clock by Big Ben. These occupy two
points in space-time, and there is a definite separation between
them. An observer at Westminster considers that they occur at
the same place, and that they are separated by an hour in time;
thus he resolves their four-dimensional separation into zero
distance in space and one hour distance in time. An observer
on the sun considers that they do not occur at the same place;
they are separated by about $70,000$ miles, that being the distance
travelled by the earth in its orbital motion with respect to the
sun. It is clear that he is not resolving in quite the same directions
as the terrestrial observer, since he finds the space-component
to be $70,000$ miles instead of zero. But if he alters one
component he must necessarily alter the other; so he will make
the time-component differ slightly from an hour. By analogy
with resolution into components in three-dimensions, we should
expect him to make it less than an hour---having, as it were,
borrowed from time to make space; but as a matter of fact he
makes it longer. This is because space-time has a different
%% -----File: 048.png---Folio 38-------
geometry, which will be described later. Our present point is
that there is but one separation of two events in four dimensions,
which can be resolved in any number of ways into the components
length and duration.

We see further how motion must be purely relative.
\index{Relativity!of motion}%
Take
two events $A$ and~$B$ in the history of one particle. We can choose
any direction as the time-direction; let us choose it along~$AB$.
Then $A$ and~$B$ are separated only in time and not in space, so
the particle is at rest. If we choose a slightly inclined time-direction,
the separation $AB$ will have a component in space;
the two events then do not occur at the same place, that is to
say, the particle has moved. The negation of absolute motion
is thus associated with the possibility of choosing the time-direction
in any way we please. What determines the separation
of space and time for any particular observer can now be seen.
Let the observer place himself so that he is, to the best of his
knowledge, at rest. If he is a normal human being, he will
seat himself in an arm-chair; if he is an astronomer, he will
place himself on the sun or at the centre of the stellar universe.
Then all the events happening directly to him will in his opinion
occur at the same place. Their separation will have no space-component,
and they will accordingly be ranged solely in the
time-direction. This chain of events, marking his track through
the four-dimensional world, will be his time-direction. Each
observer bases his separation of space and time on his own track
through the world.%
\index{Time!depends on observer's track}%

Since any separation of space and time is admissible, it is
possible for the astronomer to base his space and time on the
track of a solar observer instead of that of a terrestrial observer;
but it must be remembered that in practice the space and time
of the solar observer have to be inferred indirectly from those
of the terrestrial observer; and, if the corrections are made
according to the crude methods hitherto employed, they may
be inferred wrongly (if extreme accuracy is needed).

The most formidable objection to this relativist view of the
world is the aether difficulty. We have seen that uniform motion
through the aether cannot be detected by experiment, and
therefore it is entirely in accordance with experiment that such
motion should have no counterpart in the four-dimensional
%% -----File: 049.png---Folio 39-------
world. Nevertheless, it would almost seem that such motion
must logically exist, if the aether exists;
\index{Aether!non-material nature of}%
and, even at the
expense of formal simplicity, it ought to be exhibited in any
theory which pretends to give a complete account of what is
going on in nature. If a substantial aether analogous to a
material ocean exists, it must rigidify, as it were, a definite
space; and whether the observer or whether nature pays any
attention to that space or not, a fundamental separation of
space and time must be there. Some would cut the knot by
denying the aether altogether. We do not consider that desirable,
or, so far as we can see, possible; but we do deny that the aether
need have such properties as to separate space and time in the
way supposed. It seems an abuse of language to speak of a
division existing, when nothing has ever been found to pay any
attention to the division.

Mathematicians of the nineteenth century devoted much time
to theories of elastic solid and other material aethers. Waves of
light were supposed to be actual oscillations of this substance;
it was thought to have the familiar properties of rigidity and
density; it was sometimes even assigned a place in the table of
the elements. The real death-blow to this materialistic conception
of the aether was given when attempts were made to explain
matter as some state in the aether. For if matter is vortex-motion
or beknottedness in aether, the aether cannot be matter--some
state in itself. If any property of matter comes to be
regarded as a thing to be explained by a theory of its structure,
clearly that property need not be attributed to the aether.
If physics evolves a theory of matter which explains some
property, it stultifies itself when it postulates that the same
property exists unexplained in the primitive basis of matter.

Moreover the aether has ceased to take any very active part
in physical theory and has, as it were, gone into reserve. A
modern writer on electromagnetic theory will generally start
with the postulate of an aether pervading all space; he will then
explain that at any point in it there is an electromagnetic vector
whose intensity can be measured; henceforth his sole dealings
are with this vector, and probably nothing more will be heard
of the aether itself. In a vague way it is supposed that this
vector represents some condition of the aether, and we need not
%% -----File: 050.png---Folio 40-------
dispute that without some such background the vector would
scarcely be intelligible---but the aether is now only a background
and not an active participant in the theory.

There is accordingly no reason to transfer to this vague background
of aether the properties of a material ocean. Its properties
must be determined by experiment, not by analogy. In particular
there is no reason to suppose that it can partition out space in
a definite way, as a material ocean would do. We have seen in
the Prologue that natural geometry depends on laws of matter;
therefore it need not apply to the aether. Permanent identity
of particles is a property of matter, which Lord Kelvin sought
to explain in his vortex-ring hypothesis.
\index{Identity, permanent} % [** PP: Added comma]
\index{Permanent identity}%
This abandoned
hypothesis at least teaches us that permanence should not be
regarded as axiomatic, but may be the result of elaborate constitution.
There need not be anything corresponding to
permanent identity in the constituent portions of the aether;
we cannot lay our finger at one spot and say ``this piece of
aether was a few seconds ago over there.'' Without any continuity
of identity of the aether motion through the aether
becomes meaningless; and it seems likely that this is the true
reason why no experiment ever reveals it.

This modern theory of the relativity of all uniform motion is
essentially a return to the original Newtonian view, temporarily
disturbed by the introduction of aether problems; for in Newton's
dynamics uniform motion of the whole system has not---and no
one would expect it to have---any effect. But there are considerable
difficulties in the limitation to uniform motion. Newton
himself seems to have appreciated the difficulty; but the experimental
evidence appeared to him to be against any extension
of the principle. Accordingly Newton's laws of mechanics are
not of the general type in which it is unnecessary to particularise
the observer; they hold only for observers with a special kind
of motion which is described as ``unaccelerated.''
\index{Newton!relativity for uniform motion}%
\index{Relativity, Newtonian}%
The only
definition of this epithet that can be given is that an ``unaccelerated''
observer is one for whom Newton's laws of motion
hold. On this theory, the phenomena are not indifferent to an
acceleration or non-uniform motion of the whole system. Yet
an absolute non-uniform motion through space is just as impossible
to imagine as an absolute uniform motion. The partial
%% -----File: 051.png---Folio 41-------
relativity of phenomena makes the difficulty all the greater.
If we deny a fundamental medium with continuous identity of
its parts, motion uniform or non-uniform should have no
significance; if we admit such a medium, motion uniform or
non-uniform should be detectable; but it is much more difficult
to devise a plan of the world according to which uniform motion
has no significance and non-uniform motion is significant.\Pagelabel{41}

It is through experiment that we have been led back to the
principle of relativity for uniform motion. In seeking some kind
of extension of this principle to accelerated motion, we are led
by the feeling that, having got so far, it is difficult and arbitrary
to stop at this point. We now try to conceive a system of nature
for which all kinds of motion of the observer are indifferent.
It will be a completion of our synthesis of what is perceived by
observers having all kinds of motions with respect to one
another, removing the restriction to uniform motion. The
experimental tests must follow after the consequences of this
generalisation have been deduced.

The task of formulating such a theory long appeared impossible.
It was pointed out by Newton that, whereas there is no criterion
for detecting whether a body is at rest or in uniform motion, it
is easy to detect whether it is in rotation. For example the
bulge of the earth's equator is a sign that the earth is rotating,
since a plastic body at rest would be spherical.%
\index{Newton!absolute rotation}%

This problem of rotation affords a hint as to the cause of the
incomplete relativity of Newtonian mechanics. The laws of
motion are formulated with respect to an unaccelerated observer,
and do not apply to a frame of reference rotating with the earth.
Yet mathematicians frequently do use a rotating frame. Some
modification of the laws is then necessary; and the modification
is made by introducing a centrifugal force---not regarded as a
real force like gravitation, but as a mathematical fiction employed
to correct for the improper choice of a frame of reference.
\index{Centrifugal Force!compared with gravitation}%
The bulge of the earth's equator may be attributed indifferently
to the earth's rotation or to the outward pull of the centrifugal
force introduced when the earth is regarded as non-rotating.

Now it is generally assumed that the centrifugal force is
something \textit{sui generis}, which could always be distinguished
experimentally from any other natural phenomenon. If then
%% -----File: 052.png---Folio 42-------
on choosing a frame of reference we find that a centrifugal force
is detected, we can at once infer that the frame of reference is
a ``wrong'' one; rotating and non-rotating frames can be distinguished
by experiment, and rotation is thus strictly absolute.
But this assumes that the observed effects of centrifugal force
cannot be produced in any other way than by rotation of the
observer's frame of reference. If once it is admitted that centrifugal
force may not be completely distinguishable by experiment
from another kind of force---gravitation---perceived even by
Newton's unaccelerated observer, the argument ceases to apply.
We can never determine exactly how much of the observed field
of force is centrifugal force and how much is gravitation; and
we cannot find experimentally any definite standard that is to
be considered absolutely non-rotating.

The question then, whether there exists a distinction between
``right'' frames of reference and ``wrong'' frames, turns on
whether the use of a ``wrong'' frame produces effects experimentally
distinguishable from any natural effects which can be
perceived when a ``right'' frame is used.
\index{Frames@Frames of reference, ``right'' and ``wrong''}%
If there is no such
difference, all frames may be regarded as on the same footing
and equally right. In that case we can have a complete relativity
of natural phenomena. Since the effect of departing from
Newton's standard frame is the introduction of a field of force,
this generalised relativity theory must be largely occupied with
the nature of fields of force.

The precise meaning of the statement that all frames of
reference are on the same footing is rather difficult to grasp.
We believe that there are absolute things in the world---not only
matter, but certain characteristics in empty space or aether.
In the atmosphere a frame of reference which moves with the
air is differentiated from other frames moving in a different
manner; this is because, besides discharging the normal functions
of a frame of reference, the air-frame embodies certain of the
absolute properties of the matter existing in the region.
Similarly, if in empty space we choose a frame of reference
which more or less follows the lines of the absolute structure in
the region, the frame will usurp some of the absolute qualities
of that structure. What we mean by the equivalence of all
frames is that they are not differentiated by any qualities
%% -----File: 053.png---Folio 43-------
formerly supposed to be intrinsic in the frames themselves---rest,
rectangularity, acceleration---independent of the absolute
structure of the world that is referred to them. Accordingly the
objection to attributing absolute properties to Newton's frame
of reference is not that it is impossible for a frame of reference
to acquire absolute properties, but that the Newtonian frame
has been laid down on the basis of relative knowledge without
any attempt to follow the lines of absolute structure.

Force, as known to us observationally, is like the other
quantities of physics, a relation. The force, measured with a
spring-balance, for example, depends on the acceleration of the
observer holding the balance; and the term may, like length
and duration, have no exact counterpart in a description of
nature independent of the observer. Newton's view assumes
that there is such a counterpart, an active cause in nature
which is identical with the force perceived by his standard
unaccelerated observer. Although any other observer perceives
this force with additions of his own, it is implied that the
original force in nature and the observer's additions can in some
way be separated without ambiguity. There is no experimental
foundation for this separation, and the relativity view is that
a field of force can, like length and duration, be nothing but
a link between nature and the observer. There is, of course,
something at the far end of the link, just as we found an
extension in four dimensions at the far end of the relations
denoted by length and duration. We shall have to study the
nature of this unknown whose relation to us appears as force.
Meanwhile we shall realise that the alteration of perception of
force by non-uniform motion of the observer, as well as the
alteration of the perception of length by his uniform motion, is
what might be expected from the nature of these quantities as
relations solely.%
\index{Force!relativity of}%
\index{Relativity of Force}%

We proceed now to a more detailed study of the four-dimensional
world, of the things which occur in it, and of the laws by
which they are regulated. It is necessary to dive into this
absolute world to seek the truth about nature; but the physicist's
object is always to obtain knowledge which can be applied to
the relative and familiar aspect of the world. The absolute
world is of so different a nature, that the relative world, with
%% -----File: 054.png---Folio 44-------
which we are acquainted, seems almost like a dream. But if
indeed we are dreaming, our concern is with the baseless fabric
of our vision. We do not suggest that physicists ought to
translate their results into terms of four-dimensional space for
the empty satisfaction of working in the realm of reality. It is
rather the opposite. They explore the new field and bring back
their spoils---a few simple generalisations---to apply them to the
practical world of three-dimensions. Some guiding light will be
given to the attempts to build a scheme of things entire. For
the rest, physics will continue undisturbedly to explore the
relative world, and to employ the terms applicable to relative
knowledge, but with a fuller appreciation of its relativity.
%% -----File: 055.png---Folio 45-------


\Chapter{III}{The World of Four Dimensions}

\Quote{H.~G. Wells, \textit{The Time Machine.}}
{Here is a portrait of a man at eight years old, another at fifteen, another at
seventeen, another at twenty-three, and so on. All these are evidently sections,
as it were, Three-Dimensional representations of his Four-Dimensional being,
which is a fixed and unalterable thing.}%
\index{Four-dimensional space-time!geometry of}%

\First{The} distinction between horizontal and vertical is not an
illusion; and the man who thinks it can be disregarded is likely
to come to an untimely end. Yet we cannot arrive at a comprehensive
view of nature unless we combine horizontal and
vertical dimensions into a three-dimensional space. By doing
this we obtain a better idea of what the distinction of horizontal
and vertical really is in those cases where it is relevant, e.g.\ the
phenomena of motion of a projectile. We recognise also that
vertical is not a universally differentiated direction in space, as
the flat-earth philosophers might have imagined.

Similarly by combining the time-ordering and space-ordering
of the events of nature into a single order of four dimensions,
we shall not only obtain greater simplicity for the phenomena
in which the separation of time and space is irrelevant, but we
shall understand better the nature of the differentiation when it
is relevant.

A point in this space-time, that is to say a given instant at
a given place, is called an ``event.''
\index{Event, definition of}%
\index{Point-event}%
\index{Space-time}%
An event in its customary
meaning would be the physical happening which occurs at and
identifies a particular place and time. However, we shall use
the word in both senses, because it is scarcely possible to think
of a point in space-time without imagining some identifying
occurrence.

In the ordinary geometry of two or three dimensions, the
distance between two points is something which can be measured,
usually with a rigid scale; it is supposed to be the same for all
observers, and there is no need to specify horizontal and vertical
directions or a particular system of coordinates. In four-dimensional
space-time there is likewise a certain extension or
%% -----File: 056.png---Folio 46-------
generalised distance between two events, of which the distance
in space and the separation in time are particular components.
This extension in space and time combined is called the
``interval'' between the two events;
\index{Extension in four dimensions}%
\index{Interval}%
it is the same for all
observers, however they resolve it into space and time separately.
We may think of the interval as something intrinsic in external
nature---an absolute relation of the two events, which postulates
no particular observer. Its practical measurement is suggested
by analogy with the distance of two points in space.

%[Illustration: \textsc{Fig}. 2.]
\begin{wrapfigure}{l}{1.75in}
\Graphic[2]{1.75in}{056a}
\end{wrapfigure}
In two dimensions on a plane, two points $P_1$, $P_2$ (\Figref{2}) can
be specified by their rectangular
coordinates $(x_1, y_1)$ and $(x_2, y_2)$,
when arbitrary axes have been selected.
In the figure, $OX_1 = x_1$,
$OY_1 = y_1$, etc. We have
\begin{align*}
P_1 P_2{}^2
  &= P_1 M^2 + M P_2{}^2 \\
  &= X_1 X_2{}^2 + Y_1 Y_2{}^2 \\
  &= (x_2 - x_1)^2 + (y_2 - y_1)^2,
\end{align*}
so that if $s$ is the distance between
$P_1$ and~$P_2$
\[
s^2 = (x_2 - x_1)^2 + (y_2 - y_1)^2.
\]

The extension to three dimensions is, as we should expect,
\[
s^2 = (x_2 - x_1)^2 + (y_2 - y_1)^2 + (z_2 - z_1)^2.
\]
Introducing the times of the events $t_1$, $t_2$, we should naturally
expect that the interval in the four-dimensional world would
be given by
\[
s^2 = (x_2 - x_1)^2 + (y_2 - y_1)^2 + (z_2 - z_1)^2 + (t_2 - t_1)^2.
\]

An important point arises here. It was, of course, assumed
that the same scale was used for measuring $x$ and $y$ and~$z$. But
how are we to use the same scale for measuring~$t$? We cannot
use a scale at all; some kind of clock is needed. The most
natural connection between the measure of time and length is
given by the fact that light travels $300,000$ kilometres in
$1$~second. For the four-dimensional world we shall accordingly
regard $1$~second as the equivalent of $300,000$ kilometres, and
measure lengths and times in seconds or kilometres indiscriminately;
in other words we make the velocity of light the unit of
%% -----File: 057.png---Folio 47-------
velocity. It is not essential to do this, but it greatly simplifies
the discussion.

Secondly, the formulae here given for $s^2$ are the characteristic
formulae of Euclidean geometry. So far as three-dimensional
space is concerned the applicability of Euclidean geometry is
very closely confirmed by experiment. But space-time is not
Euclidean; it does, however, conform (at least approximately)
to a very simple modification of Euclidean geometry indicated
by the corrected formula
\index{Euclidean geometry}%
\IndexExtra{Geometry!Euclidean}%
\[
s^2 = (x_2-x_1)^2 + (y_2-y_1)^2 + (z_2-z_1)^2 - (t_2-t_1)^2.
\]
There is only a sign altered; but that minus sign is the secret
of the differences of the manifestations of time and space in
nature.

This change of sign is often found puzzling at the start. We
could not define~$s$ by the expression originally proposed (with
the positive sign), because the expression does not define anything
objective. Using the space and time of one observer, one
value is obtained; for another observer, another value is
obtained. But if $s$ is defined by the expression now given, it is
found that the same result is obtained by all observers\footnote%
  {Appendix, \Noteref{2}.}.
\Pagelabel{note2}%
The
quantity~$s$ is thus something which concerns solely the two
events chosen; we give it a name---the interval between the two
events. In ordinary space the distance between two points is
the corresponding property, which concerns only the two points
and not the extraneous coordinate system of location which is
used. Hence interval, as here defined, is the analogue of distance;
and the analogy is strengthened by the evident resemblance
of the formula for~$s$ in both cases. Moreover, when the difference
of time vanishes, the interval reduces to the distance. But the
discrepancy of sign introduces certain important differences.
These differences are summed up in the statement that the
geometry of space is Euclidean, but the geometry of space-time
is semi-Euclidean or ``hyperbolic.''
\index{Geometry!hyperbolic}%
\index{Geometry!semi-Euclidean}%
\index{Hyperbolic geometry}%
The association of a geometry
with any continuum always implies the existence of some
uniquely measurable quantity like interval or distance; in
ordinary space, geometry without the idea of distance would be
meaningless.

%% -----File: 058.png---Folio 48-------

For the moment the difficulty of thinking in terms of an
unfamiliar geometry may be evaded by a dodge. Instead of
real time~$t$, consider imaginary time~$\tau$; that is to say, let
\index{Imaginary time}%
\index{Time!imaginary}%
\begin{DPgather*}
t = \tau \sqrt{-1}. \\
\lintertext{Then}
(t_2 - t_1)^2 = -(\tau_2 - \tau_1)^2, \\
\intertext{so that}
s^2 = (x_2 - x_1)^2 + (y_2 - y_1)^2 + (z_2 - z_1)^2 + (\tau_2 - \tau_1)^2.
\end{DPgather*}
Everything is now symmetrical and there is no distinction
between~$\tau$ and the other variables. The continuum formed of
space and imaginary time is completely isotropic for all measurements;
no direction can be picked out in it as fundamentally
distinct from any other.

The observer's separation of this continuum into space and
time consists in slicing it in some direction, viz.\ that perpendicular
to the path along which he is himself travelling. The
section gives three-dimensional space at some moment, and the
perpendicular dimension is (imaginary) time. Clearly the slice
may be taken in any direction; there is no question of a true
separation and a fictitious separation. There is no conspiracy
of the forces of nature to conceal our absolute motion---because,
looked at from this broader point of view, there is nothing to
conceal. The observer is at liberty to orient his rectangular axes
of $x$, $y$, $z$ and~$\tau$ arbitrarily, just as in three-dimensions he can
orient his axes of $x$, $y$, $z$ arbitrarily.

It can be shown that the different space and time used by
the aviator in \Chapref{I} correspond to an orientation of the
time-axis along his own course in the four-dimensional world,
whereas the ordinary time and space are given when the time-axis
is oriented along the course of a terrestrial observer. The
FitzGerald contraction and the change of time-measurement
are given exactly by the usual formulae for rotation of rectangular
axes\footnote{Appendix, \Noteref{3}.}.
\Pagelabel{note3}

It is not very profitable to speculate on the implication of the
mysterious factor~$\sqrt{-1}$, which seems to have the property of
turning time into space. It can scarcely be regarded as more
than an analytical device. To follow out the theory of the four-%
%% -----File: 059.png---Folio 49-------
dimensional world in more detail, it is necessary to return to
real time, and face the difficulties of a strange geometry.

Consider a particular observer, $S$, and represent time according
to his reckoning by distance up the page parallel to~$OT$. One
dimension of his space will be represented by horizontal distance
parallel to~$OX$; another will stand out at right angles from the
page; and the reader must imagine the third as best he can.
Fortunately it will be sufficient for us to consider only the one
dimension of space $OX$ and deal with the phenomena of ``line-land,''
i.e.\ we limit ourselves to motion to and fro in one straight
line in space.
\begin{figure*}[hbt]
\begin{center}
\Graphic[3]{3.5in}{059a}
\end{center}
\end{figure*}
%[Illustration: \textsc{Fig}. 3.]

The two lines $U'OU$, $V'OV$, at $45�$ to the axes, represent the
tracks of points which progress $1$~unit horizontally (in space)
for $1$~unit vertically (in time); thus they represent points moving
with unit velocity. We have chosen the velocity of light as unit
velocity; hence $U'OU$, $V'OV$ will be the tracks of pulses of light
in opposite directions along the straight line.

Any event $P$ within the sector $UOV$ is indubitably after the
event~$O$, whatever system of time-reckoning is adopted. For it
would be possible for a material particle to travel from~$O$ to~$P$,
the necessary velocity being less than that of light; and no
%% -----File: 060.png---Folio 50-------
rational observer would venture to state that the particle had
completed its journey before it had begun it. It would, in fact,
be possible for an observer travelling along $NP$ to receive a
light-signal or wireless telegram announcing the event~$O$, just
as he reached~$N$, since $ON$ is the track of such a message; and
then after the time $NP$ he would have direct experience of the
event~$P$. To have actual evidence of the occurrence of one
event before experiencing the second is a clear proof of their
absolute order in nature, which should convince not merely
the observer concerned but any other observer with whom he
can communicate.

% [** PP: 8.75pt overfull at 5in]
Similarly events in the sector $U'OV'$ are indubitably before
the event~$O$.

With regard to an event~$P'$ in the sector $UOV'$ or~$VOU'$ we
cannot assert that it is absolutely before or after~$O$. According
to the time-reckoning of our chosen observer~$S$, $P'$ is after~$O$,
because it lies above the line~$OX$; but there is nothing absolute
about this. The track~$OP'$ corresponds to a velocity greater
than that of light, so that we know of no particle or physical
impulse which could follow the track. An observer experiencing
the event $P'$ could not get news of the event~$O$ by any known
means until after~$P'$ had happened. The order of the two events
can therefore only be inferred by estimating the delay of the
message and this estimate will depend on the observer's mode
of reckoning space and time.

% [** PP: 4.3pt overfull at 5in]
Space-time is thus divided into three zones with respect to
the event~$O$. $U'OV'$ belongs to the indubitable past. $UOV$ is
the indubitable future. $UOV'$ and $VOU'$ are (absolutely) neither
past nor future, but simply ``elsewhere.''
\index{Elsewhere@``Elsewhere''}%
It may be remarked
that, as we have no means of identifying points in space as ``the
same point,'' and as the events $O$ and~$P$ might quite well happen
to the same particle of matter, the events are not necessarily to
be regarded as in different places, though the observer~$S$ will
judge them so; but the events $O$ and~$P'$ cannot happen to the
same particle, and no observer could regard them as happening
at the same place. The main interest of this analysis is that it
shows that the arbitrariness of time-direction is not inconsistent
with the existence of regions of absolute past and future.%
\index{Absolute past and future}%
\index{Future, absolute}%
\index{Past, absolute}%

Although there is an absolute past and future, there is between
%% -----File: 061.png---Folio 51-------
them an extended neutral zone; and simultaneity of events at
different places has no absolute meaning.
\index{Absolute simultaneity}%
\index{Simultaneity}%
For our selected
observer all events along~$OX$ are simultaneous with one another;
for another observer the line of events simultaneous with~$O$
would lie in a different direction. The denial of absolute
simultaneity is a natural complement to the denial of absolute
motion. The latter asserts that we cannot find out what is the
same place at two different times; the former that we cannot
find out what is the same time at two different places. It is
curious that the philosophical denial of absolute motion is
readily accepted, whilst the denial of absolute simultaneity
appears to many people revolutionary.

The division into past and future (a feature of time-order
which has no analogy in space-order) is closely associated with
our ideas of causation and free will. % [** PP: Removed hyphen]
\index{Causation and free will}%
\index{Free will}% [** PP: Added space]
\index{Time!past and future}%
In a perfectly determinate
scheme the past and future may be regarded as lying mapped
out---as much available to present exploration as the distant
parts of space. Events do not happen; they are just there, and
we come across them. ``The formality of taking place'' is merely
the indication that the observer has on his voyage of exploration
passed into the absolute future of the event in question; and it
has no important significance. We can be aware of an eclipse
in the year 1999, very much as we are aware of an unseen
companion to Algol. Our knowledge of things \textit{where} we are not,
and of things \textit{when} we are not, is essentially the same---an
inference (sometimes a mistaken inference) from brain impressions,
including memory, \textit{here} and \textit{now}.

So, if events are determinate, there is nothing to prevent a
person from being \textit{aware} of an event before it happens; and an
event may cause other events previous to it. Thus the eclipse
of the Sun in May 1919 caused observers to embark in March.
It may be said that it was not the eclipse, but the calculations
of the eclipse, which caused the embarkation; but I do not
think any such distinction is possible, having regard to the
indirect character of our acquaintance with all events except
those at the precise point of space where we stand. A detached
observer contemplating our world would see some events
apparently causing events in their future, others apparently
causing events in their past---the truth being that all are linked
%% -----File: 062.png---Folio 52-------
by determinate laws, the so-called causal events being merely
conspicuous foci from which the links radiate.

The recognition of an absolute past and future seems to
depend on the possibility of events which are not governed by
a determinate scheme. If, say, the event~$O$ is an ultimatum,
and the person describing the path~$NP$ is a ruler of the country
affected, then it may be manifest to all observers that it is his
knowledge of the actual occurrence of the event~$O$ which has
caused him to create the event~$P$. $P$~must then be in the absolute
%[Illustration: \textsc{Fig}. 4.]
\begin{figure*}[hbt]%
\begin{center}%
\Graphic[4]{3.5in}{062a}
\end{center}%
\end{figure*}%
future of~$O$, and, as we have seen, must lie in the sector~$UOV$.
But the inference is only permissible, if the event~$P$ could be
determined by the event~$O$, and was not predetermined by
causes anterior to both---if it was possible for it to happen or
not, consistently with the laws of nature. Since physics does
not attempt to cover indeterminate events of this kind, the
distinction of absolute past and future is not directly important
for physics; but it is of interest to show that the theory of four-dimensional
space-time provides an absolute past and future, in
accordance with common requirements, although this can
usually be ignored in applications to physics.

%% -----File: 063.png---Folio 53-------

Consider now all the events which are at an interval of one
unit from~$O$, according to the definition of the interval~$s$
\Pagelabel{53}
\[
s^2 = - (x_2 - x_1)^2 - (y_2 - y_1)^2 - (z_2 - z_1)^2 + (t_2 - t_1)^2.
\Tag{1}
\]
We have changed the sign of $s^{2}$, because usually (though not
always) the original $s^{2}$ would have come out negative. In
Euclidean space points distant a unit interval lie on a circle;
but, owing to the change in geometry due to the altered sign
of $(t_{2} - t_{1})^{2}$, they now lie on a rectangular hyperbola with two
branches $KLM$, $K'L'M'$. Since the interval is an absolute
quantity, all observers will agree that these points are at unit
interval from~$O$.

Now make the following construction:---draw a straight line
$OFT_{1}$ to meet the hyperbola in~$F$; draw the tangent~$FG$ at~$F$,
meeting the light-line $U'OU$ in~$G$; complete the parallelogram
$OFGH$; produce~$OH$ to~$X_{1}$. We now assert that an observer
$S_{1}$ who chooses $OT_{1}$ for his time-direction will regard $OX_{1}$ as
his space direction and will consider $OF$ and~$OH$ to be the units
of time and space.

The two observers make their partitions of space and time
in different ways, as illustrated in Figs.~5 and~6, where in each
case the partitions are at unit distance (in space and time)
according to the observers' own reckoning. The same diagram
of events in the world will serve for both observers; $S_{1}$ merely
removes $S$'s partitions and overlays his own, locating the events
in his space and time accordingly. It will be seen at once that
the lines of unit velocity---progress of one unit of space for one
unit of time---agree, so that the velocity of a pulse of light is
unity for both observers. It can be shown from the properties
of the hyperbola that the locus of points at any interval~$s$ from~$O$,
given by \Eqref{equation}{1}, viz.
\[
s^2 = (t - t_0)^2 - (x - x_0)^2,
\]
is the same locus (a hyperbola) for both systems of reckoning
$x$ and~$t$. The two observers will always agree on the measures
of intervals, though they will disagree about lengths, durations,
and the velocities of everything except light. This rather complex
transformation is mathematically equivalent to the simple
rotation of the axes required when imaginary time is used.

It must not be supposed that there is any natural distinction
%% -----File: 064.png---Folio 54-------
corresponding to the difference between the square-partitions
of observer $S$ and the diamond-shaped partitions of observer~$S_1$.
We might say that $S_1$ transplants the space-time world unchanged
from \Figref{5} to \Figref{6}, and then distorts it until the
diamonds shown become squares; or we might equally well start
with this distorted space-time, partitioned by $S_1$ into squares,
and then $S$'s partitions would be represented by diamonds.
It cannot be said that either observer's space-time is distorted
absolutely, but one is distorted relatively to the other. It is the
relation of \textit{order} which is intrinsic in nature, and is the same
both for the squares and diamonds; \textit{shape} is put into nature by
the observer when he has chosen his partitions.
\index{Ordering of events in external world}%
\index{Space-time!partitions of}%
\begin{figure*}[htb]
\begin{center}
\Graphic[5]{4.5in}{064a}
\Figlabel{6}%
\end{center}
\end{figure*}
% [** PP: Moved up one paragraph]
%[Illustration: \textsc{Fig}. 5.]
%[Illustration: \textsc{Fig}. 6.]

We can now deduce the FitzGerald contraction.
\index{Contraction, FitzGerald}%
Consider
a rod of unit length at rest relatively to the observer~$S$. The
two extremities are at rest in his space, and consequently remain
on the same space-partitions; hence their tracks in four dimensions
$PP'$, $QQ'$ (\Figref{7}) are entirely in the time-direction. The
real rod in nature is the four-dimensional object shown in section
as $P'PQQ'$. Overlay the same figure with $S_1$'s space and time
partitions, shown by the dotted lines. Taking a section at any
one ``time,'' the instantaneous rod is $P_1Q_1$, viz.\ the section of
$P'PQQ'$ by $S_1$'s time-line. Although on paper $P_1Q_1$ is actually
longer than $PQ$, it is seen that it is a little shorter than one of
$S_1$'s space-partitions; and accordingly $S_1$ judges that it is less
%% -----File: 065.png---Folio 55-------
than one unit long---it has contracted on account of its motion
relative to him.
%[Illustration: Fig. 7.]
\begin{figure*}[htb]
\begin{center}
\Graphic[7]{4in}{065a}
\end{center}
\end{figure*}

Similarly $RR'-SS'$ is a rod of unit length at rest relatively
to $S_1$. Overlaying $S$'s partitions we see that it occupies $R_1S_1$ at
a particular instant for $S$; and this is less than one of $S$'s
partitions. Thus $S$ judges it to have contracted on account of
its motion relative to him.%
\index{Retardation of time}%

%[Illustration: Fig. 8.]
\begin{wrapfigure}{r}{2.5in}
\Graphic[8]{2.5in}{065b}
\end{wrapfigure}
In the same way we can illustrate the problem of the duration
of the cigar; each observer
believed the other's cigar to
last the longer time. Taking
$LM$ (\Figref{8}) to represent the
duration of $S$'s cigar (two
units), we see that in $S_1$'s
reckoning it reaches over a
little more than two time-partitions.
Moreover it has not
kept to one space-partition,
i.e.\ it has moved. Similarly $L'N'$ is the duration of $S_1$'s cigar
(two time-units for him); and it lasts a little beyond two unit-partitions
in $S$'s time-reckoning. (Note, in comparing the two
diagrams, $L', M', N'$ are the same points as $L, M, N$.)

If in \Figref{4} we had taken the line $OT_1$ very near to $OU$, our
%% -----File: 066.png---Folio 56-------
diamonds would have been very elongated, and the unit-divisions
$OF, OH$ very large. This kind of partition would be
made by an observer whose course through the world is $OT_1$,
and who is accordingly travelling with a velocity approaching
that of light relative to $S$.
\index{Light, velocity of!system moving with}%
\index{Velocity of light!system moving with}%
In the limit, when the velocity
reaches that of light, both space-unit and time-unit become
infinite, so that in the natural units for an observer travelling
with the speed of light, all the events in the finite experience of
$S$ take place ``in no time'' and the size of every object is zero.
This applies, however, only to the two dimensions $x$ and $t$; the
space-partitions parallel to the plane of the paper are not
affected by this motion along $x$. Consequently for an observer
travelling with the speed of light all ordinary objects become
two-dimensional, preserving their lateral dimensions, but infinitely
thin longitudinally. The fact that events take place ``in
no time'' is usually explained by saying that the inertia of any
particle moving with the velocity of light becomes infinite so
that all molecular processes in the observer must stop; many
things may happen in $S$'s world in a twinkling of an eye---of
$S_1$'s eye.%
\index{Inertia!infinite}%

However successful the theory of a four-dimensional world
may be, it is difficult to ignore a voice inside us which whispers
``At the back of your mind, you know that a fourth dimension
is all nonsense.''
\index{Four-dimensional order}%
I fancy that that voice must often have had a
busy time in the past history of physics. What nonsense to
say that this solid table on which I am writing is a collection
of electrons moving with prodigious speeds in empty spaces,
which relatively to electronic dimensions are as wide as the
spaces between the planets in the solar system! What nonsense
to say that the thin air is trying to crush my body with a load
of 14~lbs.\ to the square inch! What nonsense that the star-cluster,
which I see through the telescope obviously there now,
is a glimpse into a past age $50,000$ years ago! Let us not be
beguiled by this voice. It is discredited.

But the statement that time is a fourth dimension may
suggest unnecessary difficulties which a more precise definition
avoids. It is in the external world that the four dimensions are
united---not in the relations of the external world to the
individual which constitute his direct acquaintance with space
%% -----File: 067.png---Folio 57-------
and time. Just in that process of relation to an individual, the
order falls apart into the distinct manifestations of space and time.
An individual is a four-dimensional object of greatly elongated
form; in ordinary language we say that he has considerable extension
in time and insignificant extension in space. Practically he
is represented by a line---his track through the world. When the
world is related to such an individual, his own asymmetry is
introduced into the relation; and that order of events which is
parallel with his track, that is to say with \textit{himself}, appears in
his experience to be differentiated from all other orders of events.%
\index{Observer, an unsymmetrical object}%
\index{Time!depends on observer's track}%

Probably the best known exposition of the fourth dimension
is that given in E.~Abbott's popular book \textit{Flatland}.
\index{Flatland}%
It may be of
interest to see how far the four-dimensional world of space-time
conforms with his anticipations. He lays stress on three points.

(1)~As a four-dimensional body moves, its section by the
three-dim\-en\-sional world may vary; thus a rigid body can alter
size and shape.

(2)~It should be possible for a body to enter a completely
closed room, by travelling into it in the direction of the fourth
dimension, just as we can bring our pencil down on to any point
within a square without crossing its sides.

(3)~It should be possible to see the inside of a solid, just as
we can see the inside of a square by viewing it from a point
outside its plane.

The first phenomenon is manifested by the FitzGerald contraction.%
\index{FitzGerald contraction!relativity explanation of}%

If quantity of matter is to be identified with its mass, the
second phenomenon does not happen. It could easily be conceived
of as happening, but it is provided against by a special
law of nature---the conservation of mass. It could happen,
but it does not happen.

The third phenomenon does not happen for two reasons.
A natural body extends in time as well as in space, and is
therefore four-dimensional; but for the analogy to hold, the
object must have one dimension less than the world, like the
square seen from the third dimension. If the solid suddenly
went out of existence so as to present a plane section towards
time, we should still fail to see the interior of it; because light-tracks
in four-dimensions are restricted to certain lines like
%% -----File: 068.png---Folio 58-------
$UOV, U'OV'$ in \Figref{3}, whereas in three-dimensions light can
traverse any straight line. This could be remedied by interposing
some kind of dispersive medium, so that light of some wave-length
could be found travelling with every velocity and following
every track in space-time; then, looking at a solid which suddenly
went out of existence, we should receive at the same moment
light-impressions from every particle in its interior (supposing
them self-luminous). We actually should see the inside of it.

How our poor eyes are to disentangle this overwhelming
experience is quite another question.

The interval is a quantity so fundamental for us that we may
consider its measurement in some detail. Suppose we have a
scale $AB$ divided into kilometres, say, and at each division is
placed a clock also registering kilometres. (It will be remembered
that time can be measured in seconds or kilometres indifferently.)
%[Illustration: Fig. 9.]
\begin{figure*}[hbt]
\begin{center}
\Graphic[9]{\textwidth}{068a}
\end{center}
\end{figure*}%
\index{Clock-scale}% [** PP: Presumed location of entry]
\index{Interval!practical measurement of}%
When the clocks are correctly set and viewed from $A$ the sum
of the readings of any clock and the division beside it is the
same for all, since the scale-reading gives the correction for the
time taken by light, travelling with unit velocity, to reach~$A$.
This is shown in \Figref{9} where the clock-readings are given as
though they were being viewed from~$A$.

Now lay the scale in line with the two events; note the clock
and scale-readings $t_1, x_1$, of the first event, and the corresponding
readings $t_2, x_2$, of the second event. Then by the formula
already given
\index{Clock!affected by velocity}%
\[
s^2 = (t_2 - t_1)^2 - (x_2 - x_1)^2.
\]
But suppose we took a different standard of rest, and set the
scale moving uniformly in the direction~$AB$. Then the divisions
would have advanced to meet the second event, and $(x_2 - x_1)$
would be smaller. This is compensated, because $t_2-t_1$ also
becomes altered. $A$~is now advancing to meet the light coming
from any of the clocks along the rod; the light arrives too
%% -----File: 069.png---Folio 59-------
quickly, and in the initial adjustment described above the clock
must be set back a little. The clock-reading of the event is thus
smaller. There are other small corrections arising from the
FitzGerald contraction, etc.; and the net result is that, it does
not matter what uniform motion is given to the scale, the final
result for $s$ is always the same.

In elementary mechanics we are taught that velocities can be
compounded by adding.
\index{Addition of velocities}%
\index{Velocity, addition-law}%
If $B$'s velocity relative to $A$ (as observed
by either of them) is $100$~km.\ per sec., and $C$'s velocity relative
to $B$ is $100$~km.\ per sec.\ in the same direction, then $C$'s velocity
relative to~$A$ should be $200$~km.\ per sec. This is not quite
accurate; the true answer is $199.999978$~km.\ per sec. The discrepancy
is not difficult to explain. The two velocities and their
resultant are not all reckoned with respect to the same partitions
of space and time. When $B$ measures $C$'s velocity relative to
him he uses his own space and time, and it must be corrected
to reduce to $A$'s space and time units, before it can be added
on to a velocity measured by $A$.

If we continue the chain, introducing $D$ whose velocity
relative to $C$, and measured by $C$, is $100$~km.\ per sec., and so on
\textit{ad infinitum}, we never obtain an infinite velocity with respect
to~$A$, but gradually approach the limiting velocity of $300,000$
km.\ per sec., the speed of light. This speed has the remarkable
property of being absolute, whereas every other speed is relative.
\index{Light, velocity of!an absolute velocity}%
If a speed of $100$~km.\ per sec.\ or of $100,000$~km.\ per sec.\ is
mentioned, we have to ask---speed relative to what? But if
a speed of $300,000$~km.\ per sec.\ is mentioned, there is no need
to ask the question; the answer is---relative to any and every
piece of matter. A $\beta$ particle shot off from radium can move at
more than $200,000$~km.\ per sec.;
\index{Beta particles}%
but the speed of light relative
to an observer travelling with it is still $300,000$~km.\ per sec. It
reminds us of the mathematicians' transfinite number Aleph;
you can subtract any number you like from it, and it still
remains the same.

The velocity of light plays a conspicuous part in the relativity
theory, and it is of importance to understand what is the
property associated with it which makes it fundamental.
\index{Light, velocity of!importance of}%
\index{Velocity of light!importance of}%
The
fact that the velocity of light is the same for all observers is a
consequence rather than a cause of its pre-eminent character.
%% -----File: 070.png---Folio 60-------
Our first introduction of it, for the purpose of coordinating
units of length and time, was merely conventional with a view
to simplifying the algebraic expressions. Subsequently, considerable
use has been made of the fact that nothing is known
in physics which travels with greater speed, so that in practice
our determinations of simultaneity depend on signals transmitted
with this speed. If some new kind of ray with a higher
speed were discovered, it would perhaps tend to displace light-signals
and light-velocity in this part of the work, time-reckoning
being modified to correspond; on the other hand, this would
lead to greater complexity in the formulae, because the FitzGerald
contraction which affects space-measurement depends
on light-velocity. But the chief importance of the velocity of
light is that no material body can exceed this velocity. This
gives a general physical distinction between paths which are time-like
and space-like, respectively---those which can be traversed
by matter, and those which cannot. The material structure of
the four-dimensional world is fibrous, with the threads all running
along time-like tracks; it is a tangled warp without a woof.
Hence, even if the discovery of a new ray led us to modify the
reckoning of time and space, it would still be necessary in the
study of material systems to preserve the \textit{present} absolute
distinction of time-like and space-like intervals, under a new
name if necessary.%
\index{Space-like intervals}%
\index{Time-like intervals}%

It may be asked whether it is possible for anything to have
a speed greater than the velocity of light. Certainly matter
cannot attain a greater speed; but there might be other things
in nature which could. ``Mr Speaker,'' said Sir Boyle Roche,
``not being a bird, I could not be in two places at the same time.''
Any entity with a speed greater than light would have the
peculiarity of Sir Boyle Roche's bird. It can scarcely be said to
be a self-contradictory property to be in two places at the same
time any more than for an object to be at two times in the same
place. The perplexities of the quantum theory of energy sometimes
seem to suggest that the possibility ought not to be
overlooked; but, on the whole, the evidence seems to be against
the existence of anything moving with a speed beyond that of
light.%
\index{Quanta}%

The standpoint of relativity and the principle of relativity
%% -----File: 071.png---Folio 61-------
are quite independent of any views as to the constitution of
matter or light. Hitherto our only reference to electrical theory
has been in connection with Larmor and Lorentz's explanation
of the FitzGerald contraction; but now from the discussion of
the four-dimensional world, we have found a more general
explanation of the change of length. The case for the electrical
theory of matter is actually weakened, because many experimental
effects formerly thought to depend on the peculiar
properties of electrical forces are now found to be perfectly
general consequences of the relativity of observational knowledge.

Whilst the evidence for the electrical theory of matter is not
so conclusive, as at one time appeared, the theory may be
accepted without serious misgivings. To postulate two entities,
matter and electric charges, when one will suffice is an arbitrary
hypothesis, unjustifiable in our present state of knowledge. The
great contribution of the electrical theory to this subject is a
precise explanation of the property of inertia. It was shown
theoretically by J.~J. Thomson that if a charged conductor is
to be moved or stopped, additional effort will be necessary
simply on account of the charge.
\index{Thomson, J.~J.}%
The conductor has to carry
its electric field with it, and force is needed to set the field
moving. This property is called inertia, and it is measured by
\textit{mass}. If, keeping the charge constant, the size of the conductor
is diminished, this inertia increases. Since the smallest separable
particles of matter are found by experiment to be very minute
and to carry charges, the suggestion arises that these charges
may be responsible for the whole of the inertia detected in
matter. The explanation is sufficient; and there seems no reason
to doubt that all inertia is of this electrical kind.%
\index{Electrical theory of inertia}%
\index{Energy!inertia of}%
\index{Inertia!electrical theory of}%
\index{Mass!electrical theory of}%

When the calculations are extended to charges moving with
high velocities, it is found that the electrical inertia is not
strictly constant but depends on the speed; in all cases the
variation is summed up in the statement that the inertia is
simply proportional to the total energy of the electromagnetic
field. We can say if we like that the mass of a charged particle
at rest belongs to its electrostatic energy; when the charge is
set in motion, kinetic energy is added, and this kinetic energy
also has mass. Hence it appears that mass (inertia) and energy
%% -----File: 072.png---Folio 62-------
are essentially the same thing, or, at the most, two aspects of
the same thing. It must be remembered that on this view the
greater part of the mass of matter is due to concealed energy,
which is not as yet releasable.

The question whether electrical energy not bound to electric
charges has mass, is answered in the affirmative in the case of
light. Light has mass.
\index{Light!mass of}%
\index{Mass of light}%
Presumably also gravitational energy
has mass; or, if not, mass will be created when, as often happens,
gravitational energy is converted into kinetic energy. The mass
of the whole (negative) gravitational energy of the earth is of
the order \textit{minus} a billion tons.

The theoretical increase of the mass of an electron with speed
has been confirmed experimentally, the agreement with calculation
being perfect if the electron undergoes the FitzGerald
contraction by its motion.
\index{Electron!Kaufmann's experiment on}%
\index{Kaufmann's experiment}%
This has been held to indicate that
the electron cannot have any inertia other than that due to the
electromagnetic field carried with it.
\index{Electron!inertia of}%
But the conclusion (though
probable enough) is not a fair inference; because these results,
obtained by special calculation for electrical inertia, are found
to be predicted by the theory of relativity for any kind of
inertia. This will be shown in \Chapref{IX}. The factor giving
the increase of mass with speed is the same as that which affects
length and time. Thus if a rod moves at such a speed that its
length is halved, its mass will be doubled. Its density will be
increased four-fold, since it is both heavier and less in volume.%
\index{Density, effect of motion on}%

We have thought it necessary to include this brief summary
of the electrical theory of matter and mass, because, although
it is not required by the relativity theory, it is so universally
accepted in physics that we can scarcely ignore it. Later on we
shall reach in a more general way the identification of mass with
energy and the variation of mass with speed; but, since the
experimental measurement of inertia involves the study of a
body in non-uniform motion, it is not possible to enter on a
satisfactory discussion of mass until the more general theory of
relativity for non-uniform motion has been developed.
%% -----File: 073.png---Folio 63-------


\Chapter{IV}{Fields of Force}

\Quote{Lucretius, \textit{De Natura Rerum.}}
{For whenever bodies fall through water and thin air, they must quicken their
descents in proportion to their weights, because the body of water and subtle
nature of air cannot retard everything in equal degree, but more readily give
way overpowered by the heavier; on the other hand empty void cannot offer
resistance to anything in any direction at any time, but must, as its nature
craves, continually give way; and for this reason all things must be moved and
borne along with equal velocities though of unequal weights through the
unresisting void.}%
\index{Force!elementary conception of}%


\First{The} primary conception of force is associated with the muscular
sensation felt when we make an effort to cause or prevent the
motion of matter. Similar effects on the motion of matter can
be caused by non-living agency, and these also are regarded as
due to forces. As is well known, the scientific measure of a force
is the momentum that it communicates to a body in given time.
There is nothing very abstract about a force transmitted by
material contact; modern physics shows that the momentum is
communicated by a process of molecular bombardment. We can
visualise the mechanism, and see the molecules carrying the
motion in small parcels across the boundary into the body that
is being acted on. Force is no mysterious agency; it is merely
a convenient summary of this flow of motion, which we can
trace continuously if we take the trouble. It is true that the
difficulties are only set back a stage, and the exact mode by
which the momentum is redistributed during a molecular
collision is not yet understood; but, so far as it goes, this analysis
gives a clear idea of the transmission of motion by ordinary
forces.

But even in elementary mechanics an important natural force
appears, which does not seem to operate in this manner. Gravitation
is not resolvable into a succession of molecular blows.
A massive body, such as the earth, seems to be surrounded by
a field of latent force, ready, if another body enters the field, to
become active, and transmit motion. One usually thinks of this
influence as existing in the space round the earth even when
%% -----File: 074.png---Folio 64-------
there is no test-body to be affected, and in a rather vague way
it is suspected to be some state of strain or other condition of
an unperceived medium.

Although gravitation has been recognised for thousands of
years, and its laws were formulated with sufficient accuracy for
almost all purposes more than 200 years ago, it cannot be said
that much progress has been made in explaining the nature or
mechanism of this influence. It is said that more than 200
theories of gravitation have been put forward; but the most
plausible of these have all had the defect that they lead nowhere
and admit of no experimental test. Many of them would nowadays
be dismissed as too materialistic for our taste---filling space
with the hum of machinery---a procedure curiously popular in
the nineteenth century. Few would survive the recent discovery
that gravitation acts not only on the molecules of matter, but
on the undulations of light.

The nature of gravitation has seemed very mysterious, yet it
is a remarkable fact that in a limited region it is possible to
create an artificial field of force which imitates a natural
gravitational field so exactly that, so far as experiments have
yet gone, no one can tell the difference.
\index{Artificial fields of force}%
\index{Fields of force!artificial}%
\index{Force!fields of}%
Those who seek for an
explanation of gravitation naturally aim to find a model which
will reproduce its effects; but no one before Einstein seems to
have thought of finding the clue in these artificial fields, familiar
as they are.

When a lift starts to move upwards the occupants feel a
characteristic sensation, which is actually identical with a
sensation of increased weight.
\index{Lift, accelerated}%
The feeling disappears as soon
as the motion becomes uniform; it is associated only with the
change of motion of the lift, that is to say, the acceleration.
Increased weight is not only a matter of sensation; it is shown
by any physical experiments that can be performed. The usual
laboratory determination of the value of gravity by Atwood's
machine would, if carried out inside the accelerated lift, give
a higher value. A spring-balance would record higher weights.
Projectiles would follow the usual laws of motion but with a
higher value of gravity. In fact, the upward acceleration of
the lift is in its mechanical effects exactly similar to an additional
gravitational field superimposed on that normally present.

%% -----File: 075.png---Folio 65-------

Perhaps the equivalence is most easily seen when we produce
in this manner an artificial field which just neutralises the earth's
field of gravitation. Jules Verne's book \textit{Round the Moon} tells
the story of three men in a projectile shot from a cannon into
space.
\index{Projectile, Jules Verne's}%
The author enlarges on their amusing experiences when
their weight vanished altogether at the neutral point, where the
attraction of the earth and moon balance one another. As a
matter of fact they would not have had any feeling of weight
at any time during their journey after they left the earth's
atmosphere.
\index{Weight!vanishes inside free projectile}%
The projectile was responding freely to the pull of
gravity, and so were its occupants. When an occupant let go
of a plate, the plate could not ``fall'' any more than it was
doing already, and so it must remain poised.

It will be seen that the sensation of weight is not felt when
we are free to respond to the force of gravitation; it is only
felt when something interferes to prevent our falling. It is
primarily the floor or the chair which causes the sensation of
weight by checking the fall. It seems literally true to say that
we never feel the force of the earth's gravitation; what we do
feel is the bombardment of the soles of our boots by the molecules
of the ground, and the consequent impulses spreading upwards
through the body. This point is of some importance, since the
idea of the force of gravitation as something which can be felt,
predisposes us to a materialistic view of its nature.

Another example of an artificial field of force is the centrifugal
force of the earth's rotation. In most books of Physical Constants
will be found a table of the values of~``$g$,'' the acceleration
due to gravity, at different latitudes.
\index{Centrifugal Force!compared with gravitation}%
But the numbers given
do not relate to gravity alone; they are the resultant of gravity
and the centrifugal force of the earth's rotation. These are so
much alike in their effects that for practical purposes physicists
have not thought it worth while to distinguish them.

Similar artificial fields are produced when an aeroplane
changes its course or speed; and one of the difficulties of navigation
is the impossibility of discriminating between these and the
true gravitation of the earth with which they combine. One
usually finds that the practical aviator requires little persuasion
of the relativity of force.

To find a unifying idea as to the origin of these artificial
%% -----File: 076.png---Folio 66-------
fields of force, we must return to the four-dimensional world of
space-time. The observer is progressing along a certain track
in this world. Now his course need not necessarily be straight.
It must be remembered that straight in the four-dimensional
world means something more than straight in space; it implies
also uniform velocity, since the velocity determines the inclination
of the track to the time-axis.

The observer in the accelerated lift travels upwards in a
straight line, say $1$~foot in the first second, $4$~feet in two seconds,
$9$~feet in three seconds, and so on. If we plot these points as
$x$ and~$t$ on a diagram we obtain a curved track. Presently the
speed of the lift becomes uniform and the track in the diagram
becomes straight. So long as the track is curved (accelerated
motion) a field of force is perceived; it disappears when the
track becomes straight (uniform motion).

Again the observer on the earth is carried round in a circle
once a day by the earth's rotation; allowing for steady progress
through time, the track in four dimensions is a spiral. For an
observer at the north pole the track is straight, and there the
centrifugal force is zero.

Clearly the artificial field of force is associated with curvature
of track, and we can lay down the following rule:---

Whenever the observer's track through the four-dimensional
world is curved he perceives an artificial field of force.

The field of force is not only perceived by the observer in his
sensations, but reveals itself in his physical measures. It should
be understood, however, that the curvature of track must not
have been otherwise allowed for. Naturally if the observer in
the lift recognises that his measures are affected by his own
acceleration and applies the appropriate corrections, the artificial
force will be removed by the process. It only exists if he is
unaware of, or does not choose to consider, his acceleration.

The centrifugal force is often called ``unreal.'' From the point
of view of an observer who does not rotate with the earth, there
is no centrifugal force; it only arises for the terrestrial observer
who is too lazy to make other allowance for the effects of the
earth's rotation. It is commonly thought that this ``unreality''
quite differentiates it from a ``real'' force like gravity; but if
we try to find the grounds of this distinction they evade us.
%% -----File: 077.png---Folio 67-------
The centrifugal force is made to disappear if we choose a suitable
standard observer not rotating with the earth; the gravitational
force was made to disappear when we chose as standard observer
an occupant of Jules Verne's falling projectile. If the possibility
of annulling a field of force by choosing a suitable standard
observer is a test of unreality, then gravitation is equally unreal
with centrifugal force.%
\index{Fields of force!relativity of}%
\index{Force!relativity of}%
\IndexExtra{Relativity of Force}%

It may be urged that we have not stated the case quite
fairly. When we choose the non-rotating observer the centrifugal
force disappears completely and everywhere. When we choose
the occupant of the falling projectile, gravitation disappears in
his immediate neighbourhood; but he would notice that,
although unsupported objects round him experienced no acceleration
relative to him, objects on the other side of the earth would
fall towards him. So far from getting rid of the field of force,
he has merely removed it from his own surroundings, and piled
it up elsewhere. Thus gravitation is removable locally, but
centrifugal force can be removed everywhere. The fallacy of
this argument is that it speaks as though gravitation and
centrifugal force were distinguishable experimentally. It presupposes
the distinction that we are challenging. Looking simply
at the resultant of gravitation and centrifugal force, which is all
that can be observed, neither observer can get rid of the resultant
force at all parts of space. Each has to be content with leaving
a residuum. The non-rotating observer claims that he has got
rid of all the unreal part, leaving a remainder (the usual gravitational
field) which he regards as really existing. We see no
justification for this claim, which might equally well be made
by Jules Verne's observer.

It is not denied that the separation of centrifugal and gravitational
force generally adopted has many advantages for
mathematical calculation. If it were not so, it could not have
endured so long. But it is a mathematical separation only,
without physical basis; and it often happens that the separation
of a mathematical expression into two terms of distinct nature,
though useful for elementary work, becomes vitiated for more
accurate work by the occurrence of minute cross-terms which
have to be taken into account.

Newtonian mechanics proceeds on the supposition that there
%% -----File: 078.png---Folio 68-------
is some super-observer. If \textit{he} feels a field of force, then that
force really exists. Lesser beings, such as the occupants of the
falling projectile, have other ideas, but they are the victims of
illusion. It is to this super-observer that the mathematician
appeals when he starts a dynamical investigation with the words
``Take unaccelerated rectangular axes, $Ox$, $Oy$, $Oz\dotsc$.''
Unaccelerated rectangular axes are the measuring-appliances of the
super-observer.%
\index{Newton!super-observer}%
\index{Super-observer, Newton's}%

It is quite possible that there might be a super-observer,
whose views have a natural right to be regarded as the truest,
or at least the simplest. A society of learned fishes would probably
agree that phenomena were best described from the point
of view of a fish at rest in the ocean. But relativity mechanics
finds that there is no evidence that the circumstances of any
observer can be such as to make his views pre-eminent. All are
on an equality. Consider an observer $A$ in a projectile falling
freely to the earth, and an observer $B$ in space out of range of
any gravitational attraction. Neither $A$ nor~$B$ feel any field of
force in their neighbourhood. Yet in Newtonian mechanics an
artificial distinction is drawn between their circumstances; $B$ is
in no field of force at all, but $A$ is really in a field of force, only
its effects are neutralised by his acceleration. But what is this
acceleration of~$A$? Primarily it is an acceleration relative to the
earth; but then that can equally well be described as an acceleration
of the earth relative to~$A$, and it is not fair to regard it as
something located with $A$. Its importance in Newtonian
philosophy is that it is an acceleration relative to what we have
called the super-observer. This potentate has drawn planes and
lines partitioning space, as space appears to him. I fear that
the time has come for his abdication.

Suppose the whole system of the stars were falling freely
under the uniform gravitation of some vast external mass, like
a drop of rain falling to the ground. Would this make any
difference to phenomena? None at all. There would be a
gravitational field; but the consequent acceleration of the
observer and his landmarks would produce a field of force
annulling it. Who then shall say what is absolute acceleration?%
\index{Absolute acceleration}%

We shall accordingly give up the attempt to separate artificial
fields of force and natural gravitational fields; and call the whole
%% -----File: 079.png---Folio 69-------
measured field of force the gravitational field, generalising the
expression. This field is not absolute, but always requires that
some observer should be specified.

It may avoid some mystification if we state at once that there
are certain intricacies in the gravitational influence radiating
from heavy matter which are distinctive. A theory which did
not admit this would run counter to common sense. What our
argument has shown is that the characteristic symptom in a
region in the neighbourhood of matter is not the field of force;
it must be something more intricate. In due course we shall
have to explain the nature of this more complex effect of matter
on the condition of the world.

Our previous rule, that the observer perceives an \textit{artificial}
field of force when he deviates from a straight track, must now
be superseded. We need rather a rule determining when he
perceives a field of force of any kind.
\index{Fields of force!due to disturbance of observer}%
Indeed the original rule
has become meaningless, because a straight track is no longer
an absolute conception. Uniform motion in a straight line is
not the same for an observer rotating with the earth as for a
non-rotating observer who takes into account the sinuosity of
the rotation. We have decided that these two observers are on
the same footing and their judgments merit the same respect.
A straight-line in space-time is accordingly not an absolute
conception, but is only defined relative to some observer.

Now we have seen that so long as the observer and his
measuring-appliances are unconstrained (falling freely) the field
of force immediately round him vanishes. It is only when he is
deflected from his proper track that he finds himself in the
midst of a field of force. Leaving on one side the question of
the motion of electrically charged bodies, which must be reserved
for more profound treatment, the observer can only leave his
proper track if he is being disturbed by material impacts, e.g.\
the molecules of the ground bombarding the soles of his boots.
We may say then that a body does not leave its natural track
without visible cause; and any field of force round an observer
is the result of his leaving his natural track by such cause.
There is nothing mysterious about this field of force; it is merely
the reflection in the phenomena of the observer's disturbance;
just as the flight of the houses and hedgerows past our railway-carriage
is the reflection of our motion with the train.

%% -----File: 080.png---Folio 70-------

Our attention is thus directed to the natural tracks of unconstrained
bodies, which appear to be marked out in some
absolute way in the four-dimensional world. There is no
question of an observer here; the body takes the same course
in the world whoever is watching it. Different observers will
describe the track as straight, parabolical, or sinuous, but it is
the same absolute locus.%
\index{Geodesic!absolute significance of}%
\index{Tracks, natural}%
\index{Natural tracks}%

Now we cannot pretend to predict without reference to
experiment the laws determining the nature of these tracks;
but we can examine whether our knowledge of the four-dimensional
world is already sufficient to specify definite tracks of this
kind, or whether it will be necessary to introduce new hypothetical
factors. It will be found that it is already sufficient. So far we
have had to deal with only one quantity which is independent
of the observer and has therefore an absolute significance in the
world, namely the \textit{interval} between two events in space and time.
Let us choose two fairly distant events $P_1$ and~$P_2$. These can
be joined by a variety of tracks, and the interval-length from
$P_1$ to~$P_2$ along any track can be measured. In order to make
sure that the interval-length is actually being measured along
the selected track, the method is to take a large number of
intermediate points on the track, measure the interval corresponding
to each subdivision, and take the sum. It is virtually
the same process as measuring the length of a twisty road on
a map with a piece of cotton. The interval-length along a
particular track is thus something which can be measured
absolutely, since all observers agree as to the measurement of
the interval for each subdivision. It follows that all observers
will agree as to which track (if any) is the shortest track between
the two points, judged in terms of interval-length.%
\index{Interval-length!tracks of maximum}%
\index{Longest tracks}%

This gives a means of defining certain tracks in space-time as
having an absolute significance, and we proceed tentatively to
identify them with the natural tracks taken by freely moving
particles.

In one respect we have been caught napping. Dr A. A. Robb
has pointed out the curious fact that it is not the shortest track,
but the longest track, which is unique\footnotemark.
  \footnotetext{It is here assumed that $P_2$ is in the future of $P_1$ so that it is possible for
  a particle to travel from $P_1$ to~$P_2$. If $P_1$ and~$P_2$ are situated like $O$ and~$P'$ in
  \Figref{3}, the interval-length is imaginary, and the \textit{shortest} track is unique.}%
There are any number
%% -----File: 081.png---Folio 71-------
of tracks from $P_1$ to~$P_2$ of zero interval-length; there is just one
which has maximum length. This is because of the peculiar
geometry which the minus sign of $(t_2-t_1)^2$ introduces. For
instance, it will be seen from \Eqref{equation}{1}, \Pageref{53}, that when
\index{Interval-length!zero for velocity of light}%
\[
(x_2 - x_1)^2 + (y_2 - y_1)^2 + (z_2 - z_1)^2 = (t_2 - t_1)^2,
\]
that is to say when the resultant distance travelled in space is
equal to the distance travelled in time, then $s$ is zero. This
happens when the velocity is unity---the velocity of light. To
get from $P_1$ to $P_2$ by a path of no interval-length, we must
simply keep on travelling with the velocity of light, cruising
round if necessary, until the moment comes to turn up at~$P_2$.
On the other hand there is evidently an upper limit to the interval-length
of the track, because each portion of $s$ is always less than
the corresponding portion of $(t_2-t_1)$, and $s$ can never exceed
$t_2-t_1$.

There is a physical interpretation of interval-length along the
path of a particle which helps to give a more tangible idea of
its meaning. It is the time as perceived by an observer, or
measured by a clock, carried on the particle. This is called the
proper-time;
\index{Clock!recording proper-time}%
\index{Interval-length!identified with proper time}%
\index{Proper-time}%
and, of course, it will not in general agree with the
time-reckoning of the independent onlooker who is supposed to
be watching the whole proceedings. To prove this, we notice
from \Eqref{equation}{1} that if $x_2 = x_1, y_2 = y_1$ and $z_2 = z_1$, then
$s = t_2-t_1$. The condition $x_2 = x_1$, etc.\ means that the particle
must remain stationary relative to the observer who is measuring
$x$, $y$, $z$,~$t$. To secure this we mount our observer on the particle
and then the interval-length $s$ will be $t_2-t_1$, which is the time
elapsed according to his clock.

We can use proper-time as generally equivalent to interval-length;
but it must be admitted that the term is not very
logical unless the track in question is a natural track. For any
other track, the drawback to defining the interval-length as the
time measured by a clock which follows the track, is that no
clock could follow the track without violating the laws of nature.
We may force it into the track by continually hitting it; but
that treatment may not be good for its time-keeping qualities.
The original definition by \Eqref{equation}{1} is the more general
definition.

%% -----File: 082.png---Folio 72-------

We are now able to state formally our proposed law of motion---Every
particle moves so as to take the track of greatest interval-length
between two events, except in so far as it is disturbed by
impacts of other particles or electrical forces.

This cannot be construed into a truism like Newton's first
law of motion. The reservation is not an undefined agency like
force, whose meaning can be extended to cover any breakdown
of the law. We reserve only direct material impacts and electromagnetic
causes, the latter being outside our present field of
discussion.

Consider, for example, two events in space-time, viz.\ the
position of the earth at the present moment, and its position a
hundred years ago. Call these events $P_2$ and~$P_1$. In the interim
the earth (being undisturbed by impacts) has moved so as to
take the longest track from $P_1$ to $P_2$---or, if we prefer, so as to
take the longest possible proper-time over the journey. In the
weird geometry of the part of space-time through which it
passes (a geometry which is no doubt associated in some way
with our perception of the existence of a massive body, the sun)
this longest track is a spiral---a circle in space, drawn out into
a spiral by continuous displacement in time. Any other course
would have had shorter interval-length.

In this way the study of fields of force is reduced to a study
of geometry. To a certain extent this is a retrograde step; we
adopt Kepler's description of the sun's gravitational field instead
of Newton's. The field of force is completely described if the
tracks through space and time of particles projected in every
possible way are prescribed. But we go back in order to go
forward in a new direction. To express this unmanageable mass
of detail in a unified way, a world-geometry is found in which
the tracks of greatest length are the actual tracks of the particles.
It only remains to express the laws of this geometry in a concise
form. The change from a mechanical to a geometrical theory of
fields of force is not so fundamental a change as might be
supposed. If we are now reducing mechanics to a branch of
natural geometry, we have to remember that natural geometry
is equally a branch of mechanics, since it is concerned with the
behaviour of material measuring-appliances.

Reference has been made to weird geometry. There is no
%% -----File: 083.png---Folio 73-------
help for it, if the longest track can be a spiral like that known
to be described by the earth. Non-Euclidean geometry is
necessary. In Euclidean geometry the shortest track is always
a straight line; and the slight modification of Euclidean geometry
described in \Chapref{III} is found to give a straight line as the
longest track. The status of non-Euclidean geometry has already
been thrashed out in the Prologue; and there seems to be no
reason whatever for preferring Euclid's geometry unless observations
decide in its favour.
\index{Euclidean geometry}%
\index{Geometry!non-Euclidean, or Riemannian}%
\index{Non-Euclidean geometry}%
\index{Riemannian, or non-Euclidean, geometry}%
\index{Semi-Euclidean geometry}%
\IndexExtra{Geometry!Euclidean}%
\Eqref{Equation}{1}, \Pageref{53}, is the expression
of the Euclidean (or semi-Euclidean) geometry we have hitherto
adopted; we shall have to modify it, if we adopt non-Euclidean
geometry.

But the point arises that the geometry arrived at in \Chapref{III}
was not arbitrary. It was the synthesis of measures made with
clocks and scales, by observers with all kinds of uniform motion
relative to one another; we cannot modify it arbitrarily to fit
the behaviour of moving particles like the earth. Now, if the
worst came to the worst, and we could not reconcile a geometry
based on measures with clocks and scales and a geometry based
on the natural tracks of moving particles---if we had to select
one or the other and keep to it---I think we ought to prefer to
use the geometry based on the tracks of moving particles. The
free motion of a particle is an example of the simplest possible
kind of phenomenon; it is unanalysable; whereas, what the
readings of any kind of clock record, what the extension of a
material rod denotes, may evidently be complicated phenomena
involving the secrets of molecular constitution. Each geometry
would be right in its own sphere; but the geometry of moving
particles would be the more fundamental study. But it turns
out that there is probably no need to make the choice; clocks,
scales, moving particles, light-pulses, give the same geometry.
This might perhaps be expected since a clock must comprise
moving particles of some kind.%
\index{Clock-scale geometry, not fundamental}%

A formula, such as \Eqref{equation}{1}, based on experiment can
only be verified to a certain degree of approximation. Within
certain limits it will be possible to introduce modifications. Now
it turns out that the free motion of a particle is a much more
sensitive way of exploring space-time, than any practicable
measures with scales and clocks. If then we employ our accurate
%% -----File: 084.png---Folio 74-------
knowledge of the motion of particles to correct the formula, we
shall find that the changes introduced are so small that they are
inappreciable in any practical measures with scales and clocks.
There is only one case where a possible detection of the modification
is indicated; this refers to the behaviour of a clock on the
surface of the sun, but the experiment is one of great difficulty
and no conclusive answer has been given.
\index{Clock!on sun}%
We conclude then
that the geometry of space and time based on the motions of
particles is accordant with the geometry based on the cruder
observations with clocks and scales; but if subsequent experiment
should reveal a discrepancy, we shall adhere to the moving
particle on account of its greater simplicity.

The proposed modification can be regarded from another
point of view. \Eqref{Equation}{1} is the synthesis of the experiences
of all observers in uniform motion. But uniform motion means
that their four-dimensional tracks are straight lines. We must
suppose that the observers were moving in their natural tracks;
for, if not, they experienced fields of force, and presumably
allowed for these in their calculations, so that reduction was
made to the natural tracks. If then \Eqref{equation}{1} shows that
the natural tracks are straight lines, we are merely getting out
of the equation that which we originally put into it.

The formula needs generalising in another way. Suppose there
is a region of space-time where, for some observer, the natural
tracks are all straight lines and \Eqref{equation}{1} holds rigorously.
For another (accelerated) observer the tracks will be curved,
and the equation will not hold. At the best it is of a form which
can only hold good for specially selected observers.

Although it has become necessary to throw our formula into
the melting-pot, that does not create any difficulty in measuring
the interval. Without going into technical details, it may be
pointed out that the innovations arise solely from the introduction
of gravitational fields of force into our scheme. When
there is no force, the tracks of all particles are straight lines as
our previous geometry requires. In any small region we can
choose an observer (falling freely) for whom the force vanishes,
and accordingly the original formula holds good. Thus it is only
necessary to modify our rule for determining the interval by
two provisos (1)~that the interval measured must be small,
%% -----File: 085.png---Folio 75-------
(2)~that the scales and clocks used for measuring it must be
falling freely. The second proviso is natural, because, if we do
not leave our apparatus to fall freely, we must allow for the
strain that it undergoes. The first is not a serious disadvantage,
because a larger interval can be split up into a number of small
intervals and the parts measured separately. In mathematical
problems the same device is met with under the name of integration.
To emphasise that the formula is strictly true only for
infinitesimal intervals, it is written with a new notation
\Pagelabel{75}
\index{Interval!practical measurement of}%
\[
ds^2 = - dx^2 - dy^2 - dz^2 + dt^2
\Tag{2}
\]
where $dx$ stands for the small difference $x_2-x_1$, etc.

The condition that the measuring appliances must not be
subjected to a field of force is illustrated by Ehrenfest's paradox.
\index{Ehrenfest's paradox}%
Consider a wheel revolving rapidly. Each portion of the
circumference is moving in the direction of its length, and
might be expected to undergo the FitzGerald contraction due
to its velocity; each portion of a radius is moving transversely
and would therefore have no longitudinal contraction.
\index{Acceleration!modifies FitzGerald contraction}%
\index{FitzGerald Contraction!modified by acceleration}%
It looks
as though the rim of the wheel should contract and the spokes
remain the same length, when the wheel is set revolving. The
conclusion is absurd, for a revolving wheel has no tendency to
buckle---which would be the only way of reconciling these
conditions. The point which the argument has overlooked is
that the results here appealed to apply to unconstrained bodies,
which have no acceleration relative to the natural tracks in
space. Each portion of the rim of the wheel has a radial acceleration,
and this affects its extensional properties. When accelerations
as well as velocities occur a more far-reaching theory is
needed to determine the changes of length.

To sum up---the interval between two (near) events is something
quantitative which has an absolute significance in nature.
The track between two (distant) events which has the longest
interval-length must therefore have an absolute significance.
Such tracks are called \textit{geodesics}.
\index{Geodesic!definition of}%
Geodesics can be traced practically,
because they are the tracks of particles undisturbed by
material impacts. By the practical tracing of these geodesics
we have the best means of studying the character of the natural
geometry of the world. An auxiliary method is by scales and
%% -----File: 086.png---Folio 76-------
clocks, which, it is believed, when unconstrained, measure a
small interval according to \Eqref{formula}{2}.

The identity of the two methods of exploring the geometry
of the world is connected with a principle which must now be
enunciated definitely. We have said that no experiments have
been able to detect a difference between a gravitational field
and an artificial field of force such as the centrifugal force.
\index{Force!relativity of}%
\index{Relativity of Force}%
This
is not quite the same thing as saying that it has been proved
that there is no difference. It is well to be explicit when a
positive generalisation is made from negative experimental
evidence. The generalisation which it is proposed to adopt is
known as the Principle of Equivalence.%
\index{Equivalence!Principle of}%
\index{Principle of Equivalence}%

\textit{A gravitational field of force is precisely equivalent to an artificial
field of force, so that in any small region it is impossible by any
conceivable experiment to distinguish between them.}

In other words, force is purely relative.
%% -----File: 087.png---Folio 77-------


\Chapter{V}{Kinds of Space}

\Quote[break]{W.~K. Clifford ({\upshape and} K.~Pearson), \textit{Common Sense of the Exact Sciences.}}
{The danger of asserting dogmatically that an axiom based on the experience
of a limited region holds universally will now be to some extent apparent to
the reader. It may lead us to entirely overlook, or when suggested at once
reject, a possible explanation of phenomena. The hypothesis that space is not
homaloidal [flat], and again that its geometrical character may change with
the time, may or may not be destined to play a great part in the physics of the
future; yet we cannot refuse to consider them as possible explanations of
physical phenomena, because they may be opposed to the popular dogmatic
belief in the universality of certain geometrical axioms---a belief which has
risen from centuries of indiscriminating worship of the genius of Euclid.}%
\index{Clifford}%

\First{On} any surface it requires two independent numbers or ``coordinates''
to specify the position of a point. For this reason
a surface, whether flat or curved, is called a two-dimensional
space. Points in three-dimensional space require three, and in
four-dimensional space-time four numbers or coordinates.%
\index{Coordinates}%

To locate a point on a surface by two numbers, we divide the
surface into meshes by any two systems of lines which cross one
another. Attaching consecutive numbers to the lines, or better
to the channels between them, one number from each system
will identify a particular mesh; and if the subdivision is sufficiently
fine any point can be specified in this way with all the accuracy
needed. This method is used, for example, in the Post Office
Directory of London for giving the location of streets on the
map. The point $(4, 2)$ will be a point in the mesh where channel
No.~$4$ of the first system crosses channel No.~$2$ of the second.
If this indication is not sufficiently accurate, we must divide
channel No.~$4$ into ten parts numbered $4.0$, $4.1$ etc. The subdivision
must be continued until the meshes are so small that
all points in one mesh can be considered identical within the
limits of experimental detection.

The diagrams, Figs.~10, 11, 12, illustrate three of the many
kinds of mesh-systems commonly used on a flat surface.%
\index{Mesh-systems}%

If we speak of the properties of the triangle formed by the
points $(1, 2)$, $(3, 0)$, $(4, 4)$, we shall be at once asked, What mesh-%
%% -----File: 088.png---Folio 78-------
system are you using? No one can form a picture of the triangle
until that information has been given. But if we speak of the
properties of a triangle whose sides are of lengths $2$, $3$, $4$~inches,
anyone with a graduated scale can draw the triangle, and follow
our discussion of its properties. The distance between two points
can be stated without referring to any mesh-system. For this
reason, if we use a mesh-system, it is important to find formulae
connecting the absolute distance with the particular system that
is being used.

In the more complicated kinds of mesh-systems it makes a
great simplification if we content ourselves with the formulae for
very short distances. The mathematician then finds no difficulty
in extending the results to long distances by the process called
integration. We write $ds$ for the distance between two points
\begin{figure*}[hbt]%
\begin{center}
\Graphic[10]{\textwidth}{088a}
\Figlabel{11}%
\Figlabel{12}%
\end{center}
\end{figure*}%
close together, $x_1$ and $x_2$ for the two numbers specifying the
location of one of them, $dx_1$ and $dx_2$ for the small differences of
these numbers in passing from the first point to the second.
But in using one of the particular mesh-systems illustrated in
the diagrams, we usually replace $x_1$, $x_2$ by particular symbols
sanctioned by custom, viz.\ $(x_1, x_2)$ becomes $(x, y)$, $(r, \theta)$, $(\xi, \eta)$
for Figs.~10, 11, 12, respectively.
%[Illustration: Fig. 10.]
%[Illustration: Fig. 11.]
%[Illustration: Fig. 12.]

The formulae, found by geometry, are:

For rectangular coordinates $(x, y)$, \Figref{10},
\[
ds^2 = dx^2 + dy^2.
\]

For polar coordinates $(r, \theta)$, \Figref{11},
\[
ds^2 = dr^2 + r^2\, d\theta^2.
\]

For oblique coordinates $(\xi, \eta)$, \Figref{12},
\[
ds^2 = d\xi^2 - 2\kappa\, d\xi d\eta + d\eta^2,
\]
where $\kappa$ is the cosine of the angle between the lines of partition.

%% -----File: 089.png---Folio 79-------

As an example of a mesh-system on a curved surface, we may
take the lines of latitude and longitude on a sphere.

For latitude and longitude $(\beta, \lambda)$
\Pagelabel{79}
\[
ds^2 = d\beta^2 + \cos^2 \beta\, d\lambda^2.
\]

These expressions form a test, and in fact the only possible
test, of the kind of coordinates we are using. It may perhaps
seem inconceivable that an observer should for an instant be in
doubt whether he was using the mesh-system of \Figref{10} or
\Figref{11}. He sees at a glance that \Figref{11} is not what he would
call a rectangular mesh-system. But in that glance, he makes
measures with his eye, that is to say he determines $ds$ for pairs
of points, and he notices how these values are related to the
number of intervening channels. In fact he is testing which
formula for $ds$ will fit. For centuries man was in doubt whether
the earth was flat or round---whether he was using plane rectangular
coordinates or some kind of spherical coordinates. In
some cases an observer adopts his mesh-system blindly and long
afterwards discovers by accurate measures that $ds$ does not fit
the formula he assumed---that his mesh-system is not exactly of
the nature he supposed it was. In other cases he deliberately
sets himself to plan out a mesh-system of a particular variety,
say rectangular coordinates; he constructs right angles and rules
parallel lines; but these constructions are all measurements of
the way the $x$-channels and $y$-channels ought to go, and the
rules of construction reduce to a formula connecting his measures
$ds$ with $x$ and~$y$.

The use of special symbols for the coordinates, varying
according to the kind of mesh-system used, thus anticipates a
knowledge which is really derived from the form of the formulae.
In order not to give away the secret prematurely, it will be
better to use the symbols $x_1$, $x_2$ in all cases. The four kinds of
coordinates already considered then give respectively the relations,
\begin{align*}
ds^2 &= dx_1{}^2 + dx_2{}^2              && \text{(rectangular)}, \\
ds^2 &= dx_1{}^2 +  x_1{}^2\, dx_2{}^2   && \text{(polar)}, \\
ds^2 &= dx_1{}^2 - 2 \kappa\, dx_1 dx_2 + dx_2{}^2
                                         && \text{(oblique)}, \\
ds^2 &= dx_1{}^2 + \cos^2 x_1\, dx_2{}^2 && \text{(latitude and longitude)}.
\end{align*}
If we have any mesh-system and want to know its nature, we
%% -----File: 090.png---Folio 80-------
must make a number of measures of the length $ds$ between
adjacent points $(x_1, x_2)$ and $(x_1 + dx_1, x_2 + dx_2)$ and test which
formula fits. If, for example, we then find that $ds^2$ is always
equal to $dx_1{}^2 + x_1{}^2\, dx_2{}^2$, we know that our mesh-system is like
that in \Figref{11}, $x_1$ and $x_2$ being the numbers usually denoted by
the polar coordinates $r$, $\theta$. The statement that polar coordinates
are being used is unnecessary, because it adds nothing to our
knowledge which is not already contained in the formula. It is
merely a matter of giving a name; but, of course, the name calls
to our minds a number of familiar properties which otherwise
might not occur to us.

For instance, it is characteristic of the polar coordinate system
that there is only one point for which $x_1$ (or~$r$) is equal to~$0$,
whereas in the other systems $x_1 = 0$ gives a line of points. This
is at once apparent from the formula; for if we have two points
for which $x_1 = 0$ and $x_1 + dx_1 = 0$, respectively, then
\[
dx_1{}^2 + x_1{}^2\, dx_2{}^2 = 0.
\]
The distance $ds$ between the two points vanishes, and accordingly
they must be the same point.

The examples given can all be summed up in one general
expression
\[
ds^2 = g_{11}\, dx_1{}^2 + 2g_{12}\, dx_1 dx_2 + g_{22}\, dx_2{}^2,
\]
where $g_{11}$, $g_{12}$, $g_{22}$ may be constants or functions of $x_1$ and~$x_2$.
For instance, in the fourth example their values are $1$, $0$, $\cos^2 x_1$.
It is found that all possible mesh-systems lead to values of $ds^2$
which can be included in an expression of this general form; so
that mesh-systems are distinguished by three functions of
position $g_{11}$, $g_{12}$, $g_{22}$ which can be determined by making physical
measurements. These three quantities are sometimes called
potentials.%
\index{Potentials}%

We now come to a point of far-reaching importance. The
formula for $ds^2$ teaches us not only the character of the mesh-system,
but the nature of our two-dimensional space, which is
independent of any mesh-system. If $ds^2$ satisfies any one of the
first three formulae, then the space is like a flat surface;
\index{Flat space in two dimensions}%
if it
satisfies the last formula, then the space is a surface curved like
a sphere. Try how you will, you cannot draw a mesh-system on
a flat (Euclidean) surface which agrees with the fourth formula.

%% -----File: 091.png---Folio 81-------

If a being limited to a two-dimensional world finds that his
measures agree with the first formula, he can make them agree
with the second or third formulae by drawing the meshes
differently. But to obtain the fourth formula he must be translated
to a different world altogether.

We thus see that there are different kinds of two-dimensional
space, betrayed by different metrical properties.
\index{Kinds of space}%
\index{Space!kinds of}%
They are
naturally visualised as different surfaces in Euclidean space of
three dimensions. This picture is helpful in some ways, but
perhaps misleading in others. The metrical relations on a plane
sheet of paper are not altered when the paper is rolled into a
cylinder---the measures being, of course, confined to the two-dimensional
world represented by the paper, and not allowed to
take a short cut through space. The formulae apply equally
well to a plane surface or a cylindrical surface; and in so far as
our picture draws a distinction between a plane and a cylinder,
it is misleading.
\index{Cylinder and plane, indistinguishable in two dimensions}%
But they do not apply to a sphere, because
a plane sheet of paper cannot be wrapped round a sphere.
A genuinely two-dimensional being could not be cognisant of
the difference between a cylinder\footnote%
  {One should perhaps rather say a roll, to avoid any question of joining the
  two edges.} and a plane; but a sphere
would appear as a different kind of space, and he would recognise
the difference by measurement.

Of course there are many kinds of mesh-systems, and many
kinds of two-dimensional spaces, besides those illustrated in the
four examples. Clearly it is not going to be a simple matter to
discriminate the different kinds of spaces by the values of the~$g$'s.
There is no characteristic, visible to cursory inspection,
which suggests why the first three formulae should all belong to
the same kind of space, and the fourth to a different one.
Mathematical investigation has discovered what is the common
link between the first three formulae. The $g_{11}$, $g_{12}$, $g_{22}$ satisfy in
all three cases a certain differential equation\footnotemark; and whenever
this differential equation is satisfied, the same kind of space
occurs.
\footnotetext{Appendix, \Noteref{4}.}
\Pagelabel{note4}

No doubt it seems a very clumsy way of approaching these
intrinsic differences of kinds of space---to introduce potentials
%% -----File: 092.png---Folio 82-------
which specifically refer to a particular mesh-system, although
the mesh-system can have nothing to do with the matter. It is
worrying not to be able to express the differences of space in a
purer form without mixing them up with irrelevant differences
of potential. But we have neither the vocabulary nor the
imagination for a description of absolute properties as such.
All physical knowledge is relative to space and time partitions;
and to gain an understanding of the absolute it is necessary to
approach it through the relative.
\index{Absolute@Absolute, approached through the relative}%
The absolute may be defined
as a relative which is always the same no matter what it is
relative to\footnote{Cf.\ \Pageref{31}, where a distinction was drawn between knowledge which does
not particularise the observer and knowledge which does not postulate an
observer at all.}. Although we think of it as self-existing, we cannot
give it a place in our knowledge without setting up some dummy
to relate it to. And similarly the absolute differences of space
always appear as related to some mesh-system, although the
mesh-system is only a dummy and has nothing to do with the
problem.

The results for two dimensions can be generalised, and applied
to four-dimensional space-time.
\index{Four-dimensional space-time!geometry of}%
Distance must be replaced by
interval, which it will be remembered, is an absolute quantity,
and therefore independent of the mesh-system used. Partitioning
space-time by any system of meshes, a mesh being given by the
crossing of four channels, we must specify a point in space-time
by four coordinate numbers, $x_1$, $x_2$, $x_3$, $x_4$. By analogy the
general formula will be
\index{Interval!general expression for}%
\iffalse %%%%%%%%%% DEAD CODE (matches original) %%%%%%%%%%
\begin{multline*}
ds^2 = g_{11}\, dx_1{}^2 + g_{22}\, dx_2{}^2
     + g_{33}\, dx_3{}^2 + g_{44}\, dx_4{}^2
  + 2g_{12}\, dx_1 dx_2 \\
  + 2g_{13}\, dx_1 dx_3
  + 2g_{14}\, dx_1 dx_4
  + 2g_{23}\, dx_2 dx_3 \\
  + 2g_{24}\, dx_2 dx_4
  + 2g_{34}\, dx_3 dx_4.
\Tag{3}
\end{multline*}
\fi %%%%%%%%%% END OF DEAD CODE %%%%%%%%%%
\begin{align*} % [** PP: Re-breaking]
ds^2 &= g_{11}\, dx_1{}^2 + g_{22}\, dx_2{}^2
      + g_{33}\, dx_3{}^2 + g_{44}\, dx_4{}^2 \\
  &\quad+ 2g_{12}\, dx_1 dx_2  + 2g_{13}\, dx_1 dx_3 + 2g_{14}\, dx_1 dx_4 \\
  &\quad+ 2g_{23}\, dx_2 dx_3  + 2g_{24}\, dx_2 dx_4 + 2g_{34}\, dx_3 dx_4.
\Tag{3}
\end{align*}
The only difference is that there are now ten $g$'s, or potentials,
instead of three, to summarise the metrical properties of the
mesh-system. It is convenient in specifying special values of
the potentials to arrange them in the standard form
\[
\begin{matrix}
g_{11} & g_{12} & g_{13} & g_{14} \\
       & g_{22} & g_{23} & g_{24} \\
       &        & g_{33} & g_{34} \\
       &        &        & g_{44}
\end{matrix}
\]
%% -----File: 093.png---Folio 83-------
The space-time already discussed at length in \Chapref{III}
corresponded to the \Eqref{formula}{2}, \Pageref{75},
\[
ds^2 = - dx^2 - dy^2 - dz^2 + dt^2.
\]
Here $(x, y, z, t)$ are the conventional symbols for $(x_1, x_2, x_3, x_4)$
when this special mesh-system is used, viz.\ rectangular coordinates
and time. Comparing with~(3) the potentials have the
special values
\index{Flat space-time}%
\index{Galilean potentials}%
\index{Potentials!Galilean values}%
\[
\begin{matrix}
-1 & \Neg0 & \Neg0 & \Neg0 \\
   &    -1 & \Neg0 & \Neg0 \\
   &       &    -1 & \Neg0 \\
   &       &       &    +1
\end{matrix}
\]
These are called the ``Galilean values.'' If the potentials have
these values everywhere, space-time may be called ``flat,''
because the geometry is that of a plane surface drawn in
Euclidean space of five dimensions.
\index{Euclidean space of five dimensions}%
Recollecting what we found
for two dimensions, we shall realise that a quite different set
of values of the potentials may also belong to flat space-time,
because the meshes may be drawn in different ways. We must
clearly understand that

(1)~The only way of discovering what kind of space-time is
being dealt with is from the values of the potentials, which are
determined practically by measurements of intervals,

(2)~Different values of the potentials do not necessarily
indicate different kinds of space-time,

(3)~There is some complicated mathematical property
common to all values of the potentials which belong to the
same space-time, which is not shared by those which belong to
a different kind of space-time. This property is expressed by
a set of differential equations.

It can now be deduced that the space-time in which we live
is not quite flat. If it were, a mesh-system could be drawn for
which the $g$'s have the Galilean values, and the geometry with
respect to these partitions of space and time would be that
discussed in \Chapref{III}. For that geometry the geodesics, giving
the natural tracks of particles, are straight lines.

Thus in flat space-time the law of motion is that (with
suitably chosen coordinates) every particle moves uniformly in
a straight line except when it is disturbed by the impacts of
%% -----File: 094.png---Folio 84-------
other particles. Clearly this is not true of our world; for example,
the planets do not move in straight lines although they do not
suffer any impacts. It is true that if we confine attention to a
small region like the interior of Jules Verne's projectile, all the
tracks become straight lines for an appropriate observer, or,
as we generally say, he detects no field of force. It needs a
large region to bring out the differences of geometry. That is
not surprising, because we cannot expect to tell whether a
surface is flat or curved unless we consider a reasonably large
portion of it.%
\index{Flat space-time!at infinity}%

According to Newtonian ideas, at a great distance from all
matter beyond the reach of any gravitation, particles would all
move uniformly in straight lines. Thus at a great distance from
all matter space-time tends to become perfectly flat. This can
only be checked by experiment to a certain degree of accuracy,
and there is some doubt as to whether it is rigorously true. We
shall leave this afterthought to \Chapref{X}, meanwhile assuming
with Newton that space-time far enough away from everything
is flat, although near matter it is curved. It is this puckering
near matter which accounts for its gravitational effects.

Just as we picture different kinds of two-dimensional space
as differently curved surfaces in our ordinary space of three-dimensions,
so we are now picturing different kinds of four-dimensional
space-time as differently curved surfaces in a
Euclidean space of \textit{five} dimensions. This is a picture only\footnotemark.
  \footnotetext{A fifth dimension suffices for illustrating the property here considered;
  \index{Curvature!merely illustrative}%
  but for an exact representation of the geometry of the world, Euclidean space
  of \textit{ten} dimensions is required. We may well ask whether there is merit in
  Euclidean geometry sufficient to justify going to such extremes.}%
The fifth dimension is neither space nor time nor anything that
can be perceived; so far as we know, it is nonsense. I should not
describe it as a mathematical fiction, because it is of no great
advantage in a mathematical treatment. It is even liable to
mislead because it draws distinctions, like the distinction between
a plane and a roll, which have no meaning. It is, like
the notion of a field of force acting in space and time, merely
introduced to bolster up Euclidean geometry, when Euclidean
geometry has been found inappropriate.
\index{Geometry!non-Euclidean, or Riemannian}%
\index{Non-Euclidean geometry}%
\index{Riemannian, or non-Euclidean, geometry}%
The real difference
between the various kinds of space-time is that they have
%% -----File: 095.png---Folio 85-------
different kinds of geometry, involving different properties of the~$g$'s.
It is no explanation to say that this is because the surfaces
are differently curved in a real Euclidean space of five dimensions.
We should naturally ask for an explanation why the space of
five dimensions is Euclidean; and presumably the answer would
be, because it is a plane in a real Euclidean space of six dimensions,
and so on \textit{ad infinitum}.

The value of the picture to us is that it enables us to describe
important properties with common terms like ``pucker'' and
``curvature'' instead of technical terms like ``differential
invariant.''
\index{Pucker in space-time}%
We have, however, to be on our guard, because
analogies based on three-dimensional space do not always apply
immediately to many-dimensional space. The writer has keen
recollections of a period of much perplexity, when he had not
realised that a four-dimensional space with ``no curvature'' is
not the same as a ``flat'' space! Three-dimensional geometry
does not prepare us for these surprises.

Picturing the space-time in the gravitational field round the
earth as a pucker, we notice that we cannot locate the pucker
at a point; it is ``somewhere round'' the point. At any special
point the pucker can be pressed out flat, and the irregularity
runs off somewhere else. That is what the inhabitants of Jules
Verne's projectile did; they flattened out the pucker inside the
projectile so that they could not detect any field of force there;
but this only made things worse somewhere else, and they
would find an increased field of force (relative to them) on the
other side of the earth.

What determines the existence of the pucker is not the values
of the $g$'s at any point, or, what comes to the same thing, the
field of force there. It is the way these values link on to those
at other points---the gradient of the~$g$'s, and more particularly
the gradient of the gradient. Or, as has already been said, the
kind of space-time is fixed by differential equations.

Thus, although a gravitational field of force is not an absolute
thing, and can be imitated or annulled at any point by an
acceleration of the observer or a change of his mesh-system,
nevertheless the presence of a heavy particle does modify the
world around it in an absolute way which cannot be imitated
artificially. Gravitational force is relative; but there is this
%% -----File: 096.png---Folio 86-------
more complex character of gravitational influence which is
absolute.

The question must now be put, Can every possible kind of
space-time occur in an empty region in nature? Suppose we
give the ten potentials perfectly arbitrary values at every point;
that will specify the geometry of some mathematically possible
space-time. But could that kind of space-time actually occur---by
any arrangement of the matter round the region?

The answer is that only certain kinds of space-time can occur
in an empty region in nature. The law which determines what
kinds can occur is the law of gravitation.

It is indeed clear that, since we have reduced the theory of
fields of force to a theory of the geometry of the world, if there
is any law governing fields of force (including the gravitational
field), that law must be of the nature of a restriction on the
possible geometries of the world.

The choice of $g$'s in any special problem is thus arrived at by
a three-fold sorting out: (1)~many sets of values can be dismissed
because they can never occur in nature, (2)~others, while possible,
do not relate to the kind of space-time present in the problem
considered, (3)~of those which remain, one set of values relates
to the particular mesh-system that has been chosen. We have
now to find the law governing the first discrimination. What is
the criterion that decides what values of the $g$'s give a kind of
space-time possible in nature?

In solving this problem Einstein had only two clues to guide
him.

(1)~Since it is a question of whether the \textit{kind of space-time} is
possible, the criterion must refer to those properties of the $g$'s
which distinguish different kinds of space-time, not to those
which distinguish different kinds of mesh-system in the same
space-time. The formulae must therefore not be altered in any
way, if we change the mesh-system.

(2)~We know that flat space-time \textit{can} occur in nature (at
great distances from all gravitating matter). Hence the criterion
must be satisfied by any values of the $g$'s belonging to flat
space-time.

It is remarkable that these slender clues are sufficient to
indicate almost uniquely a particular law. Afterwards the
%% -----File: 097.png---Folio 87-------
further test must be applied---whether the law is confirmed by
observation.

\index{Coincidences|(}%
The irrelevance of the mesh-system to the laws of nature is
sometimes expressed in a slightly different way.
\index{Mesh-systems!irrelevance to laws of nature}%
There is one
type of observation which, we can scarcely doubt, must be
independent of any possible circumstances of the observer,
namely a complete coincidence in space and time. The track of
a particle through four-dimensional space-time is called its
world-line.
\index{World-line}%
Now, the world-lines of two particles either intersect
or they do not intersect; the standpoint of the observer is not
involved. In so far as our knowledge of nature is a knowledge
of intersections of world-lines, it is absolute knowledge independent
of the observer. If we examine the nature of our
observations, distinguishing what is actually seen from what is
merely inferred, we find that, at least in all exact measurements,
our knowledge is primarily built up of intersections of world-lines
of two or more entities, that is to say their coincidences.
For example, an electrician states that he has observed a current
of $5$~milliamperes. This is his inference: his actual observation
was a \textit{coincidence} of the image of a wire in his galvanometer
with a division of a scale. A meteorologist finds that the temperature
of the air is~$75�$; his observation was the \textit{coincidence} of
the top of the mercury-thread with division $75$ on the scale of
his thermometer. It would be extremely clumsy to describe the
results of the simplest physical experiment entirely in terms of
coincidence. The absolute observation is, whether or not the
coincidence exists, not when or where or under what circumstances
the coincidence exists; unless we are to resort to relative
knowledge, the place, time and other circumstances must in
their turn be described by reference to other coincidences. But
it seems clear that if we could draw all the world-lines so as to
show all the intersections in their proper order, but otherwise
arbitrary, this would contain a complete history of the world,
and nothing within reach of observation would be omitted.%
\index{Coincidences|)}%

Let us draw such a picture, and imagine it embedded in a
jelly. If we deform the jelly in any way, the intersections will
still occur in the same order along each world-line and no
additional intersections will be created. The deformed jelly will
represent a history of the world, just as accurate as the one
%% -----File: 098.png---Folio 88-------
originally drawn; there can be no criterion for distinguishing
which is the best representation.

Suppose now we introduce space and time-partitions, which
we might do by drawing rectangular meshes in both jellies.
We have now two ways of locating the world-lines and events
in space and time, both on the same absolute footing. But
clearly it makes no difference in the result of the location whether
we first deform the jelly and then introduce regular meshes, or
whether we introduce irregular meshes in the undeformed jelly.
And so all mesh-systems are on the same footing.

This account of our observational knowledge of nature shows
that there is no \textit{shape} inherent in the absolute world, so that
when we insert a mesh-system, it has no shape initially, and a
rectangular mesh-system is intrinsically no different from any
other mesh-system.

Returning to our two clues, condition (1)~makes an extraordinarily
clean sweep of laws that might be suggested; among
them Newton's law is swept away. The mode of rejection can
be seen by an example; it will be sufficient to consider two
dimensions. If in one mesh-system $(x, y)$
\begin{align*}
ds^2 &= g_{11} dx^2 + 2g_{12} dxdy + g_{22} dy^2, \\
\intertext{and in another system $(x', y')$}
ds^2 &= g_{11}'\, {dx'}^2 + 2g_{12}'\, dx'dy' + g_{22}'\, dy'^2,
\end{align*}
the same law must be satisfied if the unaccented letters are
throughout replaced by accented letters. Suppose the law
$g_{11} = g_{22}$ is suggested. Change the mesh-system by spacing the
$y$-lines twice as far apart, that is to say take $y' = \frac{1}{2}y$, with
$x' = x$. Then
\begin{DPalign*}
ds^2 &= g_{11}\, dx^2    + 2g_{12}\, dx dy  +  g_{22}\, dy^2 \\
     &= g_{11}\, {dx'}^2 + 4g_{12}\, dx'dy' + 4g_{22}\, {dy'}^2, \\
\lintertext{so that}
     &\quad {g_{11}}' = g_{11},\qquad {g_{22}}' = 4g_{22}.
\end{DPalign*}
And if $g_{11}$ is equal to $g_{22}$, $g_{11}'$ cannot be equal to $g_{22}'$.

After a few trials the reader will begin to be surprised that
any possible law could survive the test. It seems so easy to
defeat any formula that is set up by a simple change of mesh-system.
Certainly it is unlikely that anyone would hit on such
a law by trial. But there are such laws, composed of exceedingly
complicated mathematical expressions. The theory of these is
%% -----File: 099.png---Folio 89-------
called the ``theory of tensors,'' and had already been worked
out by the pure mathematicians Riemann, Christoffel, Ricci,
Levi-Civita who, it may be presumed, never dreamt of a physical
application for it.%
\index{Christoffel}%
\index{Levi-Civita}%
\index{Ricci}%
\index{Riemann}%

One law of this kind is the condition for flat space-time,
which is generally written in the simple, but not very illuminating,
form
\index{Flat space-time!conditions for}%
\[
B ^\rho _{\mu\nu\sigma} = 0.
\Tag{4}
\]
The quantity on the left is called the Riemann-Christoffel
tensor, and it is written out in a less abbreviated form in the
Appendix\footnote{Appendix, \Noteref{5}.}. % [** PP: Added ``Appendix'']
\Pagelabel{note5}%
\index{Riemann-Christoffel tensor}%
\index{Tensors}%
It must be explained that the letters $\mu$, $\nu$, $\sigma$, $\rho$
indicate \textit{gaps}, which are to be filled up by any of the numbers
$1$, $2$, $3$, $4$, chosen at pleasure. (When the expression is written
out at length, the gaps are in the suffixes of the $x$'s and $g$'s.)
Filling the gaps in different ways, a large number of expressions,
$B^1_{111}$, $B^4_{123}$, $B^1_{432}$, etc., are obtained. The \Eqref{equation}{4} states that
all of these are zero. There are $4^4$, or~$256$, of these expressions
altogether, but many of them are repetitions. Only 20 of the
equations are really necessary; the others merely say the same
thing over again.

It is clear that the \Eqref{law}{4} is not the law of gravitation for
which we are seeking, because it is much too drastic. If it were
a law of nature, then only flat space-time could exist in nature,
and there would be no such thing as gravitation. It is not the
general condition, but a special case---when all attracting
matter is infinitely remote.

But in finding a general condition, it may be a great help to
know a special case. Would it do to select a certain number of
the 20~equations to be satisfied generally, leaving the rest to
be satisfied only in the special case? Unfortunately the equations
hang together; and, unless we take them all, it is found that
the condition is not independent of the mesh-system. But there
happens to be one way of building up out of the 20~conditions
a less stringent set of conditions independent of the mesh-system.
Let
\begin{align*}
G_{11}
  &= B^1_{111} + B^2_{112} + B^3_{113} + B^4_{114}, \\
\intertext{and, generally}
G_{\mu\nu}
  &= B^1_{\mu\nu 1}  + B^2_{\mu\nu 2} + B^3_{\mu\nu 3} + B^4_{\mu\nu 4},
\end{align*}
%% -----File: 100.png---Folio 90-------
then the conditions
\[
G_{\mu\nu} = 0
\Tag{5}
\]
will satisfy our requirements for a general law of nature.

This law is independent of the mesh-system, though this can
only be proved by elaborate mathematical analysis. Evidently,
when all the $B$'s vanish, \Eqref{equation}{5} is satisfied; so, when flat
space-time occurs, this law of nature is not violated. Further
it is not so stringent as the condition for flatness, and admits
of the occurrence of a limited variety of non-Euclidean geometries.
\index{Geometry!non-Euclidean, or Riemannian}%
\index{Non-Euclidean geometry}%
\index{Riemannian, or non-Euclidean, geometry}%
Rejecting duplicates, it comprises 10~equations; but four
of these can be derived from the other six, so that it gives
six conditions, which happens to be the number required for a
law of gravitation\footnotemark.
  \footnotetext{Isolate a region of empty space-time; and suppose that everywhere outside
  the region the potentials are known. It should then be possible by the law of
  gravitation to determine the nature of space-time in the region. Ten differential
  equations together with the boundary-values would suffice to determine the
  ten potentials throughout the region; but that would determine not only the
  kind of space-time but the mesh-system, whereas the partitions of the mesh-system
  can be continued across the region in any arbitrary way. The four
  sets of partitions give a four-fold arbitrariness; and to admit of this, the number
  of equations required is reduced to six.}

The suggestion is thus reached that
\index{Gravitation, Einstein's law of!differential formula}%
\[
G_{\mu\nu} = 0
\]
may be the general law of gravitation. Whether it is so or not
can only be settled by experiment. In particular, it must in
ordinary cases reduce to something so near the Newtonian law,
that the remarkable confirmation of the latter by observation
is accounted for. Further it is necessary to examine whether
there are any exceptional cases in which the difference between
it and Newton's law can be tested. We shall see that these
tests are satisfied.

What would have been the position if this suggested law had
failed? We might continue the search for other laws satisfying
the two conditions laid down; but these would certainly be far
more complicated mathematically. I believe too that they would
not help much, because practically they would be indistinguishable
from the simpler law here suggested---though this has not
been demonstrated rigorously. The other alternative is that
there is something causing force in nature not comprised in the
%% -----File: 101.png---Folio 91-------
geometrical scheme hitherto considered, so that force is not
purely relative, and Newton's super-observer exists.

Perhaps the best survey of the meaning of our theory can be
obtained from the standpoint of a ten-dimensional Euclidean
continuum, in which space-time is conceived as a particular
four-dimensional surface. It has to be remarked that in ten
dimensions there are gradations intermediate between a flat
surface and a fully curved surface, which we shall speak of as
curved in the ``first degree'' or ``second degree\footnote{This is not a recognised nomenclature.}.'' The distinction
is something like that of curves in ordinary space,
which may be \textit{curved} like a circle, or \textit{twisted} like a helix; but the
analogy is not very close. The full ``curvature'' of a surface is a
single quantity called~$G$, built up out of the various terms $G_{\mu\nu}$ in
somewhat the same way as these are built up out of $B^\rho_{\mu\nu\sigma}$.
The following conclusions can be stated.%
\index{Curvature!degrees of}%

\begin{DPgather*}
\lintertext{\indent If}
  B^\rho_{\mu\nu\sigma} = 0
\rintertext{(20 conditions)\quad} \\
\intertext{space-time is flat. This is the state of the world at an infinite
distance from all matter and all forms of energy.}
%
\lintertext{\indent If}
  G_{\mu\nu} = 0
\rintertext{(6 conditions)\quad} \\
\intertext{space-time is curved in the first degree. This is the state of the
world in an empty region---not containing matter, light or
electromagnetic fields, but in the neighbourhood of these forms
of energy.}
%
\lintertext{\indent If}
  G = 0
\rintertext{(1 condition)\quad} \\
\intertext{space-time is curved in the second degree. This is the state of
the world in a region not containing matter or electrons (bound
energy), but containing light or electromagnetic fields (free
energy).}
%
\lintertext{\indent If}
G~\text{is not zero}
\end{DPgather*}
space-time is fully curved. This is the state of the world in a
region containing continuous matter.%
\index{Continuous matter}%
\index{Matter!continuous}%

According to current physical theory continuous matter does
not exist, so that strictly speaking the last case never arises.
Matter is built of electrons or other nuclei. The regions lying
between the electrons are not fully curved, whilst the regions
inside the electrons must be cut out of space-time altogether.
\index{Electron!geometry inside}%
We cannot imagine ourselves exploring the inside of an electron
%% -----File: 102.png---Folio 92-------
with moving particles, light-waves, or material clocks and
measuring-rods; hence, without further definition, any geometry
of the interior, or any statement about space and time in the
interior, is meaningless. But in common life, and frequently in
physics, we are not concerned with this \textit{microscopic} structure of
matter. We need to know, not the actual values of the $g$'s at
a point, but their average values through a region, small from
the ordinary standpoint but large compared with the molecular
structure of matter. In this \textit{macroscopic} treatment molecular
matter is replaced by continuous matter, and uncurved space-time
studded with holes is replaced by an equivalent fully
curved space-time without holes.%
\index{Macroscopic!equations}%

It is natural that our senses should have developed faculties
for perceiving some of these intrinsic distinctions of the possible
states of the world around us. I prefer to think of matter and
energy, not as agents causing the degrees of curvature of the
world, but as parts of our perceptions of the existence of the
curvature.

It will be seen that the law of gravitation can be summed up
in the statement that in an empty region space-time can be
curved only in the first degree.
%% -----File: 103.png---Folio 93-------

% [** PP: Re-breaking title to match running heads]
\Chapter[The New Law of Gravitation and the Old Law]{VI}%
{The New Law of Gravitation \break and the Old Law}

% [** PP: Special headers]
\fancyhead[CE]{\textsc{THE NEW LAW OF GRAVITATION}}
\fancyhead[CO]{\textsc{AND THE OLD LAW}}

\Quote{Sir Isaac Newton.}
{I don't know what I may seem to the world, but, as to myself, I seem to have
been only as a boy playing on the sea-shore, and diverting myself in now and
then finding a smoother pebble or a prettier shell than ordinary, whilst the
great ocean of truth lay all undiscovered before me.}


\First{Was} there any reason to feel dissatisfied with Newton's law of
gravitation?

Observationally it had been subjected to the most stringent
tests, and had come to be regarded as the perfect model of an
exact law of nature. The cases, where a possible failure could
be alleged, were almost insignificant. There are certain unexplained
irregularities in the moon's motion; but astronomers
generally looked---and must still look---in other directions for
the cause of these discrepancies. One failure only had led to
a serious questioning of the law; this was the discordance of
motion of the perihelion of Mercury. How small was this discrepancy
may be judged from the fact that, to meet it, it was
proposed to amend \textit{square} of the distance to the $2.00000016$
power of the distance. Further it seemed possible, though
unlikely, that the matter causing the zodiacal light might be of
sufficient mass to be responsible for this effect.%
\index{Gravitation, Newton's law of!ambiguity of}%
\index{Moon, motion of}%
\index{Newton!law of gravitation}%

The most serious objection against the Newtonian law as an
exact law was that it had become ambiguous. The law refers
to the product of the masses of the two bodies; but the mass
depends on the velocity---a fact unknown in Newton's day.
Are we to take the variable mass, or the mass reduced to rest?
Perhaps a learned judge, interpreting Newton's statement like
a last will and testament, could give a decision; but that is
scarcely the way to settle an important point in scientific
theory.

Further \textit{distance}, also referred to in the law, is something
relative to an observer. Are we to take the observer travelling
with the sun or with the other body concerned, or at rest in the
aether or in some gravitational medium?

%% -----File: 104.png---Folio 94-------

Finally is the force of gravitation propagated instantaneously,
or with the velocity of light, or some other velocity?
\index{Gravitation!propagation with velocity of light}%
\index{Propagation of Gravitation}%
\index{Velocity of gravitation}%
Until
comparatively recently it was thought that conclusive proof
had been given that the speed of gravitation must be far higher
than that of light. The argument was something like this. If
the Sun attracts Jupiter towards its present position~$S$, and
Jupiter attracts the Sun towards its present position~$J$, the two
forces are in the same line and balance. But if the Sun attracts
Jupiter towards its previous position~$S'$, and Jupiter attracts
the Sun towards its previous position~$J'$, when the force of
attraction started out to cross the gulf, then the two forces
%[Illustration: Fig. 13.]
\begin{figure*}[hbt]
\begin{center}
\Graphic[13]{3.5in}{104a}
\end{center}
\end{figure*}%
give a couple. This couple will tend to increase the angular
momentum of the system, and, acting cumulatively, will soon
cause an appreciable change of period, disagreeing with observation
if the speed is at all comparable with that of light. The
argument is fallacious, because the effect of propagation will not
necessarily be that $S$ is attracted in the direction towards~$J'$.
Indeed it is found that if $S$ and~$J$ are two electric charges, $S$ will
be attracted very approximately towards~$J$ (not~$J'$) in spite of
the electric influence being propagated with the velocity of
light\footnotemark. In the theory given in this book, gravitation is propagated
with the speed of light, and there is no discordance with
observation.
\footnotetext{Appendix, \Noteref{6}.}
\Pagelabel{note6}

It is often urged that Newton's law of gravitation is much
%% -----File: 105.png---Folio 95-------
simpler than Einstein's new law. That depends on the point of
view; and from the point of view of the four-dimensional world
Newton's law is far more complicated. Moreover, it will be seen
that if the ambiguities are to be cleared up, the statement of
Newton's law must be greatly expanded.

Some attempts have been made to expand Newton's law on
the basis of the restricted principle of relativity (\Pageref{20}) alone.
This was insufficient to determine a definite amendment. Using
the principle of equivalence, or relativity of force, we have
arrived at a definite law proposed in the last chapter. Probably
the question has arisen in the reader's mind, why should it be
called the law of gravitation? It may be plausible as a law of
nature; but what has the degree of curvature of space-time to
do with attractive forces, whether real or apparent?

A race of flat-fish once lived in an ocean in which there were
only two dimensions.
\index{Flatfish, analogy of} % [** PP: No hyphen in index]
It was noticed that in general fishes swam
in straight lines, unless there was something obviously interfering
with their free courses. This seemed a very natural behaviour.
But there was a certain region where all the fish seemed to be
bewitched; some passed through the region but changed the
direction of their swim, others swam round and round indefinitely.
One fish invented a theory of vortices, and said that
there were whirlpools in that region which carried everything
round in curves. By-and-by a far better theory was proposed;
it was said that the fishes were all attracted towards a particularly
large fish---a sun-fish---which was lying asleep in the middle
of the region; and that was what caused the deviation of their
paths. The theory might not have sounded particularly plausible
at first; but it was confirmed with marvellous exactitude by all
kinds of experimental tests. All fish were found to possess this
attractive power in proportion to their sizes; the law of attraction
was extremely simple, and yet it was found to explain all the
motions with an accuracy never approached before in any
scientific investigations. Some fish grumbled that they did not
see how there could be such an influence at a distance; but it
was generally agreed that the influence was communicated
through the ocean and might be better understood when more
was known about the nature of water. Accordingly, nearly
every fish who wanted to explain the attraction started by
%% -----File: 106.png---Folio 96-------
proposing some kind of mechanism for transmitting it through
the water.

But there was one fish who thought of quite another plan.
He was impressed by the fact that whether the fish were big
or little they always took the same course, although it would
naturally take a bigger force to deflect the bigger fish. He therefore
concentrated attention on the courses rather than on the
forces. And then he arrived at a striking explanation of the
whole thing. There was a mound in the world round about
where the sun-fish lay. Flat-fish could not appreciate it directly
because they were two-dimensional; but whenever a fish went
swimming over the slopes of the mound, although he did his
best to swim straight on, he got turned round a bit. (If a traveller
goes over the left slope of a mountain, he must consciously
keep bearing away to the left if he wishes to keep to his original
direction relative to the points of the compass.) This was the
secret of the mysterious attraction, or bending of the paths,
which was experienced in the region.

The parable is not perfect, because it refers to a hummock in
space alone, whereas we have to deal with hummocks in space-time.
But it illustrates how a curvature of the world we live
in may give an illusion of attractive force, and indeed can only
be discovered through some such effect. How this works out in
detail must now be considered.

In the form $G_{\mu\nu} =0$, Einstein's law expresses conditions to be
satisfied in a gravitational field produced by any arbitrary
distribution of attracting matter. An analogous form of Newton's
law was given by Laplace in his celebrated expression $\nabla^{2} V = 0$.
\index{Laplace's equation}%
A more illuminating form of the law is obtained if, instead of
putting the question what kinds of space-time can exist under
the most general conditions in an empty region, we ask what
kind of space-time exists in the region round a single attracting
particle? We separate out the effect of a single particle, just as
Newton did. We can further simplify matters by introducing
some definite mesh-system, which, of course, must be of a type
which is not inconsistent with the kind of space-time found.

We need only consider space of two dimensions---sufficient
for the so-called plane orbit of a planet---time being added as
the third dimension. The remaining dimension of space can
%% -----File: 107.png---Folio 97-------
always be added, if desired, by conditions of symmetry. The
result of long algebraic calculations\footnote{Appendix, \Noteref{7}.}
\Pagelabel{note7}%
is that, round a particle
\index{Gravitation, Einstein's law of!integrated formula for a particle}%
\[
ds^{2}
  = - \frac{1}{\gamma}\, dr^{2}
    - r^{2}\, d\theta^{2}
    + \gamma\, dt^{2}
\Tag{6}
\]
where $\gamma = 1 - \dfrac{2m} {r}$.

The quantity $m$ is the gravitational mass of the particle---but
we are not supposed to know that at present. $r$~and $\theta$ are
polar coordinates, the mesh-system being as in \Figref{11}; or rather
they are the nearest thing to polar coordinates that can be
found in space which is not truly flat.

The fact is that this expression for $ds^{2}$ is found in the first
place simply as a particular solution of Einstein's equations of
the gravitational field; it is a variety of hummock (apparently
the simplest variety) which is not curved beyond the first degree.
\index{Hummock in space-time}%
There \textit{could} be such a state of the world under suitable circumstances.
To find out what those circumstances are, we have to
trace some of the consequences, find out how any particle
moves when $ds^{2}$ is of this form, and then examine whether we
know of any case in which these consequences are found
observationally. It is only after having ascertained that this
form of $ds^{2}$ does correspond to the leading observed effects
attributable to a particle of mass $m$ at the origin that we have
the right to identify this particular solution with the one we
hoped to find.

It will be a sufficient illustration of this procedure, if we
indicate how the position of the matter causing this particular
solution is located. Wherever the \Eqref{formula}{6} holds good there
can be no matter, because the law which applies to empty space
is satisfied. But if we try to approach the origin ($r = 0$), a
curious thing happens. Suppose we take a measuring-rod, and,
laying it radially, start marking off equal lengths with it along
a radius, gradually approaching the origin. Keeping the time
$t$ constant, and $d\theta$ being zero for radial measurements, the
\Eqref{formula}{6} reduces to
\begin{DPalign*}
ds^{2} &= - \frac{1}{\gamma}\, dr^{2} \\
\lintertext{or}
dr^{2} &= - \gamma\, ds^{2}.
\end{DPalign*}
%% -----File: 108.png---Folio 98-------
We start with $r$ large. By-and-by we approach the point
where $r = 2m$. But here, from its definition, $\gamma$ is equal to~$0$.
So that, however large the measured interval $ds$ may be, $dr = 0$.
We can go on shifting the measuring-rod through its own length
time after time, but $dr$ is zero; that is to say, we do not reduce~$r$.
There is a magic circle which no measurement can bring us
inside. It is not unnatural that we should picture something
obstructing our closer approach, and say that a particle of
matter is filling up the interior.%
\index{Matter!definition of a particle}%

The fact is that so long as we keep to space-time curved only
in the first degree, we can never round off the summit of the
hummock. It must end in an infinite chimney. In place of the
chimney, however, we round it off with a small region of greater
curvature. This region cannot be empty because the law applying
to empty space does not hold. We describe it therefore as containing
matter---a procedure which practically amounts to a
definition of matter. Those familiar with hydrodynamics may
be reminded of the problem of the irrotational rotation of a
fluid; the conditions cannot be satisfied at the origin, and it is
necessary to cut out a region which is filled by a vortex-filament.

A word must also be said as to the coordinates $r$ and $t$ used
in~(6). They correspond to our ordinary notion of radial distance
and time---as well as any variables in a non-Euclidean world
can correspond to words which, as ordinarily used, presuppose
a Euclidean world. We shall thus call $r$ and~$t$, distance and time. % [** PP: Retaining comma]
But to give names to coordinates does not give more information---and
in this case gives considerably less information---than is
already contained in the formula for $ds^{2}$. If any question arises
as to the exact significance of $r$ and $t$ it must always be settled
by reference to \Eqref{equation}{6}.

The want of flatness in the gravitational field is indicated by
the deviation of the coefficient $\gamma$ from unity. If the mass $m = 0$,
$\gamma = 1$, and space-time is perfectly flat. Even in the most intense
gravitational fields known, the deviation is extremely small.
For the sun, the quantity $m$, called the gravitational mass, is
only $1.47$~kilometres\footnote{Appendix, \Noteref{8}.},
\Pagelabel{note8}%
for the earth it is $5$~millimetres.
\index{Gravitational field of Sun}%
\index{Mass!gravitational}%
In any
practical problem the ratio $2m/r$ must be exceedingly small.
%% -----File: 109.png---Folio 99-------
Yet it is on the small corresponding difference in $\gamma$ that the
whole of the phenomena of gravitation depend.

The coefficient $\gamma$ appears twice in the formula, and so modifies
the flatness of space-time in two ways. But as a rule these two
ways are by no means equally important. Its appearance as a
coefficient of $dt^{2}$ produces much the most striking effects.
Suppose that it is wished to measure the interval between two
events in the history of a planet. If the events are, say $1$~second
apart in time, $dt = 1~\text{second} = 300,000~\text{kilometres}$. Thus
$dt^{2}= 90,000,000,000~\text{sq.\ km}$. Now no planet moves more than
$50$~kilometres in a second, so that the change $dr$ associated with
the lapse of $1$~second in the history of the planet will not be
more than $50$~km. Thus $dr^{2}$ is not more than $2500$ sq.~km.
Evidently the small term $2m/r$ has a much greater chance of
making an impression where it is multiplied by $dt^{2}$ than where
it is multiplied by~$dr^{2}$.

Accordingly as a first approximation, we ignore the coefficient
of~$dr^{2}$, and consider only the meaning of
\[
ds^{2} = - dr^{2} - r^{2}\, d\theta^{2} + (1 - 2m/r)\, dt^{2}.
\Tag{7}
\]
We shall now show that particles situated in this kind of space-time
will appear to be under the influence of an attractive force
directed towards the origin.

Let us consider the problem of mapping a small portion of this
kind of world on a plane.

It is first necessary to define carefully the distinction which is
here drawn between a ``picture'' and a ``map.'' If we are given
the latitudes and longitudes of a number of places on the earth,
we can make a picture by taking latitude and longitude as
vertical and horizontal distances, so that the lines of latitude
and longitude form a mesh-system of squares; but that does not
give a true map. In an ordinary map of Europe the lines of
longitude run obliquely and the lines of latitude are curved.
Why is this? Because the map aims at showing as accurately
as possible all distances in their true proportions\footnotemark.
  \footnotetext{This is usually the object, though maps are sometimes made for a different
  purpose, e.g.\ Mercator's Chart.}%
\index{Map of sun's gravitational field}%
Distance is
the important thing which it is desired to represent correctly.
In four dimensions interval is the analogue of distance, and a
map of the four-dimensional world will aim at showing all the
%% -----File: 110.png---Folio 100-------
intervals in their correct proportions. Our natural \textit{picture} of
space-time takes $r$ and $t$ as horizontal and vertical distances,
e.g.\ when we plot the graph of the motion of a particle; but in
a true \textit{map}, representing the intervals in their proper proportions,
the $r$ and $t$ lines run obliquely or in curves across the map.

The instructions for drawing latitude and longitude lines $(\beta, \lambda)$
on a map, are summed up in the formula for $ds$, \Pageref{79},
\[
ds^{2} = d\beta^{2} + \cos^{2}\beta\, d\lambda^{2},
\]
%[Illustration: \textsc{Fig}. 14.]
\begin{figure*}[hbt]
\begin{center}
\Graphic[14]{3in}{110a}
\end{center}
\end{figure*}%
and similarly the instructions for drawing the $r$ and $t$ lines are
given by the \Eqref{formula}{7}.

The map is shown in \Figref{14}. It is not difficult to see why the
$t$-lines converge to the left of the diagram. The factor $1 - 2m/r$
decreases towards the left where $r$ is small; and consequently
any change of $t$ corresponds to a shorter interval, and must be
represented in the map by a shorter distance on the left. It is
less easy to see why the $r$-lines take the courses shown; by
analogy with latitude and longitude we might expect them to
be curved the other way. But we discussed in \Chapref{III} how
%% -----File: 111.png---Folio 101-------
the slope of the time-direction is connected with the slope of
the space-direction; and it will be seen that the map gives
approximately diamond-shaped partitions of the kind represented
in \Figref{6}\footnotemark.
  \footnotetext{The substitution $x = r + \frac{1}{2} t^{2} m/r^{2}$,
  $y=t(1-m/r)$, gives $ds^{2}= -dx^{2} + dy^{2}$, if
  squares of~$m$ are negligible. The map is drawn with $x$ and $y$ as rectangular
  coordinates.}

Like all maps of curved surfaces, the diagram is only accurate
in the limit when the area covered is very small.

It is important to understand clearly the meaning of this map.
When we speak in the ordinary way of distance from the sun
and the time at a point in the solar system, we mean the two
variables $r$ and~$t$. These are not the result of any precise measures
with scales and clocks made at a point, but are mathematical
variables most appropriate for describing the whole solar system.
%[Illustration: \textsc{Fig}. 15.]
\begin{figure*}[hbt]
\begin{center}
\Graphic[15]{3.5in}{111a}
\end{center}
\end{figure*}%
They represent a compromise, because it is necessary to deal
with a region too large for accurate representation on a plane
map. We should naturally picture them as rectangular coordinates
partitioning space-time into square meshes, as in
\Figref{15}; but such a picture is not a true map, because it does
not represent in their true proportions the intervals between the
various points in the picture. It is not possible to draw any
map of the whole curved region without distortion; but a small
enough portion can be represented without distortion if the
partitions of equal $r$ and $t$ are drawn as in \Figref{14}. To get back
%% -----File: 112.png---Folio 102-------
from the true map to the customary picture of $r$ and $t$ as perpendicular
space and time, we must strain \Figref{14} until all the
meshes become squares as in \Figref{15}.

Now in the map the geometry is Euclidean and the tracks of
all material particles will be straight lines. Take such a straight
track~$PQ$, which will necessarily be nearly vertical, unless the
velocity is very large. Strain the figure so as to obtain the
customary representation of $r$ and~$t$ (in \Figref{15}), and the track
$PQ$ will become curved---curved towards the left, where the sun
lies. In each successive vertical interval (time), a successively
greater progress is made to the left horizontally (space). Thus
the velocity towards the sun increases. We say that the particle
is attracted to the sun.

The mathematical reader should find no difficulty in proving
from the diagram that for a particle with small velocity the
acceleration towards the sun is approximately $m/r^{2}$, agreeing
with the Newtonian law.%
\index{Gravitational field of Sun!Newtonian attraction}%

Tracks for very high speeds may be affected rather differently.
The track corresponding to a wave of light is represented by
a straight line at~$45�$ to the horizontal in \Figref{14}. It would
require very careful drawing to trace what happens to it when
the strain is made transforming to \Figref{15}; but actually, whilst
becoming more nearly vertical, it receives a curvature in the
opposite direction. The effect of the gravitation of the sun on
a light-wave, or very fast particle, proceeding radially is actually
a \textit{repulsion}!%
\index{Repulsion of light proceeding radially}

The track of a transverse light-wave, coming out from the
plane of the paper, will be affected like that of a particle of
zero velocity in distorting from \Figref{14} to \Figref{15}. Hence the
sun's influence on a transverse light-wave is always an attraction.
The acceleration is simply $m/r^{2}$ as for a particle at rest.

The result that the expression found for the geometry of the
gravitational field of a particle leads to Newton's law of attraction
is of great importance. It shows that the law, $G_{\mu\nu}= 0$,
proposed on theoretical grounds, agrees with observation at
least approximately. It is no drawback that the Newtonian
law applies only when the speed is small; all planetary speeds
are small compared with the velocity of light, and the considerations
mentioned at the beginning of this chapter suggest that
%% -----File: 113.png---Folio 103-------
some modification may be needed for speeds comparable with
that of light.

Another important point to notice is that the attraction of
gravitation is simply a geometrical deformation of the straight
tracks. It makes no difference what body or influence is pursuing
the track, the deformation is a general discrepancy between the
``mental picture'' and the ``true map'' of the portion of space-time
considered. Hence light is subject to the same disturbance
of path as matter. This is involved in the Principle of Equivalence;
otherwise we could distinguish between the acceleration
of a lift and a true increase of gravitation by optical experiments;
in that case the observer for whom light-rays appear to
take straight tracks might be described as absolutely unaccelerated
and there could be no relativity theory. Physicists in
general have been prepared to admit the likelihood of an
influence of gravitation on light similar to that exerted on
matter; and the problem whether or not light has ``weight''
has often been considered.

The appearance of $\gamma$ as the coefficient of $dt^{2}$ is responsible for
the main features of Newtonian gravitation; the appearance of
$1/\gamma$ as the coefficient of $dr^{2}$ is responsible for the principal
deviations of the new law from the old. This classification seems
to be correct; but the Newtonian law is ambiguous and it is
difficult to say exactly what are to be regarded as discrepancies
from it. Leaving aside now the time-term as sufficiently discussed,
we consider the space-terms alone\footnote
  {We change the sign of $ds^{2}$, so that $ds$, when real, means measured space
  instead of measured time.}
\index{Gravitation, Newton's law of!approximation to Einstein's law}%
\[
ds^{2} = \frac{1}{\gamma}\, dr^{2} + r^{2}\, d\theta^{2}.
\]

The expression shows that space considered alone is non-Euclidean
in the neighbourhood of an attracting particle. This
is something entirely outside the scope of the old law of gravitation.
Time can only be explored by something moving, whether
a free particle or the parts of a clock, so that the non-Euclidean
character of space-time can be covered up by introducing a field
of force, suitably modifying the motion, as a convenient fiction.
But space can be explored by static methods; and theoretically
%% -----File: 114.png---Folio 104-------
its non-Euclidean character could be ascertained by sufficiently
precise measures with rigid scales.

If we lay our measuring scale transversely and proceed to
measure the circumference of a circle of nominal radius~$r$, we
see from the formula that the measured length~$ds$ is equal to
$r\,d\theta$, so that, when we have gone right round the circle, $\theta$~has
increased by~$2\pi$ and the measured circumference is $2\pi r$.
\index{Circle in non-Euclidean space}%
But
when we lay the scale radially the measured length~$ds$ is equal
to $dr/\sqrt{\gamma}$, which is always greater than~$dr$. Thus, in measuring
a diameter, we obtain a result greater than~$2r$, each portion being
greater than the corresponding change of~$r$.

Thus if we draw a circle, placing a massive particle near the
centre so as to produce a gravitational field, and measure with
a rigid scale the circumference and the diameter, the ratio of
the measured circumference to the measured diameter will not be
the famous number $\pi = 3.141592653589793238462643383279\ldots$
but a little smaller. Or if we inscribe a regular hexagon in this
circle its sides will not be exactly equal to the radius of the
circle. Placing the particle near, instead of at, the centre,
avoids measuring the diameter \textit{through} the particle, and so
makes the experiment a practical one. But though practical,
it is not practicable to determine the non-Euclidean character
of space in this way. Sufficient refinement of measures is not
attainable. If the mass of a ton were placed inside a circle of
$5$~yards radius, the defect in the value of $\pi$ would only appear
in the twenty-fourth or twenty-fifth place of decimals.

It is of value to put the result in this way, because it shows
that the relativist is not talking metaphysics when he says that
space in the gravitational field is non-Euclidean. His statement
has a plain physical meaning, which we may some day learn how
to test experimentally. Meanwhile we can test it by indirect
methods.

Suppose that a plane field is uniformly studded with hurdles.
\index{Hurdles, analogy of counts of}%
The distance between any two points will be proportional to
the number of hurdles that must be passed over in getting from
one point to the other by the straight route---in fact the minimum
number of hurdles. We can use counts of hurdles as the equivalent
of distance, and map the field by these counts. The map
can be drawn on a plane sheet of paper without any inconsistency,
%% -----File: 115.png---Folio 105-------
since the field is plane. Let us now dismiss from our
minds all idea of distances in the field or straight lines in the
field, and assume that distances on the map merely represent
the minimum number of hurdles between two points; straight
lines on the map will represent the corresponding routes. This
has the advantage that if an earthquake occurs, deforming the
field, the map will still be correct. The path of fewest hurdles
will still cross the same hurdles as before the earthquake; it
will be twisted out of the straight line in the field; but we should
gain nothing by taking a straighter course, since that would
lead through a region where the hurdles are more crowded.
We do not alter the number of hurdles in any path by deforming
it.

This can be illustrated by Figs.~14 and~15. \Figref{14} represents
the original undistorted field with the hurdles uniformly placed.
The straight line $PQ$ represents the path of fewest hurdles from
$P$ to~$Q$, and its length is proportional to the number of hurdles.
\Figref{15} represents the distorted field, with $PQ$ distorted into
a curve; but $PQ$ is still the path of fewest hurdles from $P$ to~$Q$,
and the number of hurdles in the path is the same as before.
If therefore we map according to hurdle-counts we arrive at
\Figref{14} again, just as though no deformation had taken place.

To make any difference in the hurdle-counts, the hurdles
must be taken up and replanted. Starting from a given point
as centre, let us arrange them so that they gradually thin out
towards the boundaries of the field. Now choose a circle with
this point as centre;---but first, what is a circle? It has to be
defined in terms of hurdle-counts; and clearly it must be a
curve such that the minimum number of hurdles between any
point on it and the centre is a constant (the radius). With this
definition we can defy earthquakes. The number of hurdles in
the circumference of such a circle will not bear the same proportion
to the number in the radius as in the field of uniform
hurdles; owing to the crowding near the centre, the ratio will
be less. Thus we have a suitable analogy for a circle whose
circumference is less than $\pi$ times its diameter.

This analogy enables us to picture the condition of space
round a heavy particle, where the ratio of the circumference of
a circle to the diameter is less than~$\pi$. Hurdle-counts will no
%% -----File: 116.png---Folio 106-------
longer be accurately mappable on a plane sheet of paper,
because they do not conform to Euclidean geometry.

Now suppose a heavy particle wishes to cross this field,
passing near but not through the centre. In Euclidean space,
with the hurdles uniformly distributed, it travels in a straight
line, i.e.\ it goes between any two points by a path giving the
fewest hurdle jumps. We may assume that in the non-Euclidean
field with rearranged hurdles, the particle still goes by the path
of least effort. In fact, in any small portion we cannot distinguish
between the rearrangement and a distortion; so we may imagine
that the particle takes each portion as it comes according to the
rule, and is not troubled by the rearrangement which is only
visible to a general survey of the whole field\footnotemark.
  \footnotetext{There must be some absolute track, and if absolute significance can only
  be associated with hurdle-counts and not with distances in the field, the path
  of fewest hurdles is the only track capable of absolute definition.}

Now clearly it will pay not to go straight through the dense
portion, but to keep a little to the outside where the hurdles
are sparser---not too much, or the path will be unduly lengthened.
The particle's track will thus be a little concave to the centre,
and an onlooker will say that it has been attracted to the centre.
It is rather curious that we should call it attraction, when the
track has rather been avoiding the central region; but it is clear
that the direction of motion has been bent round in the way
attributable to an attractive force.

This bending of the path is additional to that due to the
Newtonian force of gravitation which depends on the second
appearance of~$\gamma$ in the formula. As already explained it is in
general a far smaller effect and will appear only as a minute
correction to Newton's law. The only case where the two rise
to equal importance is when the track is that of a light-wave,
or of a particle moving with a speed approaching that of light;
for then $dr^{2}$ rises to the same order of magnitude as~$dt^{2}$.

To sum up, a ray of light passing near a heavy particle will
be bent, firstly, owing to the non-Euclidean character of the
combination of time with space. This bending is equivalent to
that due to Newtonian gravitation, and may be calculated in
the ordinary way on the assumption that light has weight like
a material body. Secondly, it will be bent owing to the non-%
%% -----File: 117.png---Folio 107-------
Euclidean character of space alone, and this curvature is
additional to that predicted by Newton's law. If then we can
observe the amount of curvature of a ray of light, we can make
a crucial test of whether Einstein's or Newton's theory is
obeyed.

This separation of the attraction into two parts is useful in
a comparison of the new theory with the old; but from the
point of view of relativity it is artificial. Our view is that light
is bent just in the same way as the track of a material particle
moving with the same velocity would be bent.
\index{Bending of light!theory of}%
\index{Light, bending of}%
\index{Weight!of light}%
Both causes of
bending may be ascribed either to weight or to non-Euclidean
space-time, according to the nomenclature preferred. The only
difference between the predictions of the old and new theories
is that in one case the weight is calculated according to Newton's
law of gravitation, in the other case according to Einstein's.

There is an alternative way of viewing this effect on light
according to Einstein's theory, which, for many reasons is to
be preferred. This depends on the fact that the velocity of
light in the gravitational field is not a constant (unity) but
becomes smaller as we approach the sun. This does not mean
that an observer determining the velocity of light experimentally
at a spot near the sun would detect the decrease; if he performed
Fizeau's experiment, his result in kilometres per second would
be exactly the same as that of a terrestrial observer. It is the
coordinate velocity that is here referred to, described in terms
of the quantities $r$, $\theta$, $t$, introduced by the observer who is
contemplating the whole solar system at the same time.%
\index{Coordinate velocity}%
\index{Light!coordinate velocity of}%

It will be remembered that in discussing the approximate
geometry of space-time in \Figref{3}, we found that certain events
like $P$ were in the absolute past or future of~$O$, and others like
$P'$ were neither before nor after~$O$, but elsewhere. Analytically
the distinction is that for the interval $OP$, $ds^{2}$ is positive; for
$OP'$, $ds^{2}$ is negative. In the first case the interval is real or
``time-like''; in the second it is imaginary or ``space-like.'' The
two regions are separated by lines (or strictly, cones) in crossing
which $ds^{2}$ changes from positive to negative; and along the lines
themselves $ds$ is zero. It is clear that these lines must have
important absolute significance in the geometry of the world.
Physically their most important property is that pulses of light
%% -----File: 118.png---Folio 108-------
travel along these tracks, and the motion of a light-pulse is
always given by the equation $ds = 0$.

Using the expression for $ds^{2}$ in a gravitational field, we
accordingly have for light
\[
0 = -\frac{1}{\gamma}\, dr^{2}
    -r^{2}\, d\theta^{2} + \gamma\, dt^{2}.
\]
For radial motion, $d\theta = 0$, and therefore
\[
\left(\frac{dr}{dt}\right)^{2} = \gamma^{2}.
\]
For transverse motion, $dr = 0$, and therefore
\[
\left(\frac{r\, d\theta}{dt}\right)^{2} = \gamma.
\]
Thus the coordinate velocity of light travelling radially is $\gamma$,
and of light travelling transversely is $\surd\gamma$, in the coordinates
chosen.%
\index{Velocity of light!in gravitational field}%

The coordinate velocity must depend on the coordinates
chosen; and it is more convenient to use a slightly different
system in which the velocity of light is the same in all directions\footnotemark,
  \footnotetext{This is obtained by writing $r + m$ instead of~$r$, or diminishing the nominal
  distance of the sun by $1\frac{1}{2}$~kilometres. This change of coordinates simplifies
  the problem, but can, of course, make no difference to anything observable.
  After we have traced the course of the light ray in the coordinates chosen, we
  have to connect the results with experimental measures, using the corresponding
  formula for~$ds^{2}$. This final connection of mathematical and experimental results
  is, however, comparatively simple, because it relates to measuring operations
  performed in a terrestrial observatory where the difference of $\gamma$ from unity is
  negligible.}%
viz.\ $\gamma$ or~$1 - 2m/r$. This diminishes as we approach the sun---an
illustration of our previous remark that a pulse of light
proceeding radially is repelled by the sun.%
\index{Repulsion of light proceeding radially}%

The wave-motion in a ray of light can be compared to a
succession of long straight waves rolling onward in the sea. If
the motion of the waves is slower at one end than the other, the
whole wave-front must gradually slew round, and the direction
in which it is rolling must change.
\index{Wave-front, slewing of}%
In the sea this happens when
one end of the wave reaches shallow water before the other,
because the speed in shallow water is slower. It is well known
that this causes waves proceeding diagonally across a bay to
slew round and come in parallel to the shore; the advanced end
%% -----File: 119.png---Folio 109-------
is delayed in the shallow water and waits for the other. In the
same way when the light waves pass near the sun, the end nearest
the sun has the smaller velocity and the wave-front slews round;
thus the course of the waves is bent.

Light moves more slowly in a material medium than in
vacuum, the velocity being inversely proportional to the refractive
index of the medium. The phenomenon of refraction
is in fact caused by a slewing of the wave-front in passing into
a region of smaller velocity. We can thus imitate the gravitational
effect on light precisely, if we imagine the space round
the sun filled with a refracting medium which gives the
appropriate velocity of light.
\index{Refracting medium equivalent to gravitational field}%
To give the velocity $1-2m/r$, the
refractive index must be $1/(1-2m/r)$, or, very approximately,
$1 + 2m/r$. At the surface of the sun, $r= 697,000~\text{km.}$, $m = 1.47~\text{km.}$,
hence the necessary refractive index is~$1.00000424$. At a
height above the sun equal to the radius it is~$1.00000212$.

Any problem on the paths of rays near the sun can now be
solved by the methods of geometrical optics applied to the
equivalent refracting medium. It is not difficult to show that
the total deflection of a ray of light passing at a distance $r$ from
the centre of the sun is (in circular measure)
\index{Deflection of light!theory of}%
\index{Gravitational field of Sun!deflection of light}%
\[
\frac{4m}{r},
\]
whereas the deflection of the same ray calculated on the
Newtonian theory would be
\index{Gravitation, Newton's law of!deflection of light}%
\[
\frac{2m}{r}.
\]

For a ray grazing the surface of the sun the numerical value
of this deflection is
\begin{align*}
&1''.75 \quad \text{(Einstein's theory)}, \\
&0''.87 \quad \text{(Newtonian theory)}.
\end{align*}
%% -----File: 120.png---Folio 110-------


\Chapter{VII}{Weighing Light}

\Quote{Newton, \textit{Opticks}.}
{Query~1. Do not Bodies act upon Light at a distance, and by their action
bend its Rays, and is not this action (\textit{caeteris paribus}) strongest at the least
distance?}%
\index{Newton!bending of light}%

\First{We} come now to the experimental test of the influence of
gravitation on light discussed theoretically in the last chapter.
It is not the general purpose of this book to enter into details
of experiments; and if we followed this plan consistently, we
should, as hitherto, summarise the results of the observations
in a few lines. But it is this particular test which has turned
public attention towards the relativity theory, and there appears
to be widespread desire for information. We shall therefore tell
the story of the eclipse expeditions in some detail. It will make
a break in the long theoretical arguments, and will illustrate
the important applications of this theory to practical observations.

It must be understood that there were two questions to
answer: firstly, whether light has weight (as suggested by
Newton), or is indifferent to gravitation; secondly, if it has
weight, is the amount of the deflection in accordance with
Einstein's or Newton's laws?

It was already known that light possesses mass or inertia like
other forms of electromagnetic energy.
\index{Inertia!of light}%
\index{Light!mass of}%
\index{Mass of light}%
This is manifested in
the phenomena of radiation-pressure.
\index{Radiation-pressure}%
Some force is required to
stop a beam of light by holding an obstacle in its path; a searchlight % [** PP: Hyphenated across a line in original]
experiences a minute force of recoil just as if it were a
machine-gun firing material projectiles. The force, which is
predicted by orthodox electromagnetic theory, is exceedingly
minute; but delicate experiments have detected it. Probably
this inertia of radiation is of great cosmical importance, playing
a great part in the equilibrium of the more diffuse stars. Indeed
it is probably the agent which has carved the material of the
universe into stars of roughly uniform mass. Possibly the tails
of comets are a witness to the power of the momentum of sunlight,
%% -----File: 121.png---Folio 111-------
which drives outwards the smaller or the more absorptive
particles.%
\index{Comets!radiation-pressure in}%

It is legitimate to speak of a pound of light as we speak of
a pound of any other substance. The mass of ordinary quantities
of light is however extremely small, and I have calculated that
at the low charge of~3\textit{d}.\ a unit, an Electric Light Company
would have to sell light at the rate of �140,000,000 a pound.
All the sunlight falling on the earth amounts to $160$~tons daily.

It is perhaps not easy to realise how a wave-motion can have
inertia, and it is still more difficult to understand what is meant
by its having weight. Perhaps this will be better understood if
we put the problem in a concrete form. Imagine a hollow body,
with radiant heat or light-waves traversing the hollow; the
mass of the body will be the sum of the masses of the material
and of the radiant energy in the hollow; a greater force will be
required to shift it because of the light-waves contained in it.
Now let us weigh it with scales or a spring-balance. Will it also
weigh heavier on account of the radiation contained, or will the
weight be that of the solid material alone? If the former, then
clearly from this aspect light has weight; and it is not difficult
to deduce the effect of this weight on a freely moving light-beam
not enclosed within a hollow.%
\index{Light!weight of}%
\index{Momentum!of light}%
\index{Weight!of light}%
\index{Weight!of radio-active energy}%

The effect of weight is that the radiation in the hollow body
acquires each second a downward momentum proportional to
its mass. This in the long run is transmitted to the material
enclosing it. For a free light-wave in space, the added momentum
combines with the original momentum, and the total
momentum determines the direction of the ray, which is
accordingly bent. Newton's theory suggests no means for
bringing about the bending, but contents itself with predicting
it on general principles. Einstein's theory provides a means,
viz.\ the variation of velocity of the waves.

Hitherto mass and weight have always been found associated
in strict proportionality. One very important test had already
shown that this proportionality is not confined to material
energy. The substance uranium contains a great deal of radio-active
energy, presumably of an electromagnetic nature, which
it slowly liberates. The mass of this energy must be an appreciable
fraction of the whole mass of the substance. But it was shown
%% -----File: 122.png---Folio 112-------
by experiments with the E�tv�s torsion-balance that the ratio
of weight to mass for uranium is the same as for all other
substances; so the energy of radio-activity has weight.
\index{Energy!weight of radio-active}%
\index{Eotvos@E�tv�s torsion-balance}%
Still
even this experiment deals only with bound electromagnetic
energy, and we are not justified in deducing the properties of
the free energy of light.

It is easy to see that a terrestrial experiment has at present
no chance of success. If the mass and weight of light are in the
same proportion as for matter, the ray of light will be bent
just like the trajectory of a material particle. On the earth a
rifle bullet, like everything else, drops $16$~feet in the first second,
$64$~feet in two seconds, and so on, below its original line of flight;
the rifle must thus be aimed above the target. Light would also
drop $16$~feet in the first second\footnotemark;
  \footnotetext{Or $32$~feet according to Einstein's law. The fall increases with the speed of
  the motion.}%
but, since it has travelled $186,000$
miles along its course in that time, the bend is inappreciable.
% [Illustration: Fig. 16.]
\begin{figure*}[hbt]
\begin{center}
\Graphic[16]{4.5in}{122a}
\end{center}
\end{figure*}%
In fact any terrestrial course is described so quickly that
gravitation has scarcely had time to accomplish anything.

The experiment is therefore transferred to the neighbourhood
of the sun. There we get a pull of gravitation $27$~times more
intense than on the earth; and---what is more important---the
greater size of the sun permits a much longer trajectory throughout
which the gravitation is reasonably powerful. The deflection
in this case may amount to something of the order of a second
of arc, which for the astronomer is a fairly large quantity.%
\index{Deflection of light!effect on star's position}%
\index{Gravitation, Newton's law of!deflection of light}% [** PP: Index refers to p. 111]

In \Figref{16} the line $EFQP$ shows the track of a ray of light
from a distant star~$P$ which reaches the earth~$E$. The main
part of the bending of the ray occurs as it passes the sun~$S$;
\index{Bending of light!effect on star's position}%
\index{Light, bending of}%
and the initial course~$PQ$ and the final course~$FE$ are practically
straight. Since the light rays enter the observer's eye or telescope
in the direction~$FE$, this will be the direction in which the star
appears. But its true direction from the earth is~$QP$, the initial
%% -----File: 123.png---Folio 113-------
course. So the star appears displaced outwards from its true
position by an angle equal to the total deflection of the light.%
\index{Displacement of star-images}%

\Pagelabel{113}%
It must be noticed that this is only true because a star is so
remote that its true direction with respect to the earth~$E$ is
indistinguishable from its direction with respect to the point~$Q$.
For a source of light within the solar system, the apparent
displacement of the source is by no means equal to the deflection
of the light-ray. It is perhaps curious that the attraction of
light by the sun should produce an apparent displacement of
the star away from the sun; but the necessity for this is
clear.

The bending affects stars seen near the sun, and accordingly
the only chance of making the observation is during a total
eclipse when the moon cuts off the dazzling light.
\index{Eclipse, observations during}%
Even then
there is a great deal of light from the sun's corona which stretches
far above the disc. It is thus necessary to have rather bright
stars near the sun, which will not be lost in the glare of the
corona. Further the displacements of these stars can only be
measured relatively to other stars, preferably more distant from
the sun and less displaced; we need therefore a reasonable
number of outer bright stars to serve as reference points.

In a superstitious age a natural philosopher wishing to perform
an important experiment would consult an astrologer to ascertain
an auspicious moment for the trial. With better reason, an
astronomer to-day consulting the stars would announce that the
most favourable day of the year for weighing light is May 29.
The reason is that the sun in its annual journey round the
ecliptic goes through fields of stars of varying richness, but on
May~29 it is in the midst of a quite exceptional patch of bright
stars---part of the Hyades---by far the best star-field encountered.
Now if this problem had been put forward at some other period
of history, it might have been necessary to wait some thousands
of years for a total eclipse of the sun to happen on the lucky
date. But by strange good fortune an eclipse did happen on
May~29, 1919. Owing to the curious sequence of eclipses a
similar opportunity will recur in~1938; we are in the midst of
the most favourable cycle. It is not suggested that it is impossible
to make the test at other eclipses; but the work will
necessarily be more difficult.

%% -----File: 124.png---Folio 114-------
Attention was called to this remarkable opportunity by the
Astron\-o\-mer Royal in March, 1917; and preparations were begun
by a Committee of the Royal Society and Royal Astronomical
Society for making the observations. Two expeditions were sent
to different places on the line of totality to minimise the risk
of failure by bad weather. Dr A.~C.~D. Crommelin
\index{Crommelin}%
and Mr~C.
Davidson went to Sobral in North Brazil; Mr E.~T. Cottingham
\index{Cottingham}%
\index{Davidson}%
and the writer went to the Isle of Principe in the Gulf of Guinea,
West Africa.
\index{Principe, eclipse expedition to}%
The instrumental equipment for both expeditions
was prepared at Greenwich Observatory under the care of the
Astronomer Royal; and here Mr Davidson made the arrangements
which were the main factor in the success of both
parties.%
\index{Greenwich, Royal Observatory}%

The circumstances of the two expeditions were somewhat
different and it is scarcely possible to treat them together. We
shall at first follow the fortunes of the Principe observers. They
had a telescope of focal length $11$~feet $4$~inches. On their
photographs $1$~second of arc (which was about the largest displacement
to be measured) corresponds to about $\frac{1}{1500}$ inch---by
no means an inappreciable quantity. The aperture of the
object-glass was $13$~inches, but as used it was stopped down to
$8$~inches to give sharper images. It is necessary, even when the
exposure is only a few seconds, to allow for the diurnal motion
of the stars across the sky, making the telescope move so as to
follow them. But since it is difficult to mount a long and heavy
telescope in the necessary manner in a temporary installation
in a remote part of the globe, the usual practice at eclipses is
to keep the telescope rigid and reflect the stars into it by a
coelostat---a plane mirror kept revolving at the right rate by
clock-work. This arrangement was adopted by both expeditions.

The observers had rather more than a month on the island
to make their preparations. On the day of the eclipse the
weather was unfavourable. When totality began the dark disc
of the moon surrounded by the corona was visible through cloud,
much as the moon often appears through cloud on a night when
no stars can be seen. There was nothing for it but to carry out
the arranged programme and hope for the best. One observer
was kept occupied changing the plates in rapid succession, whilst
the other gave the exposures of the required length with a screen
%% -----File: 125.png---Folio 115-------
held in front of the object-glass to avoid shaking the telescope in
any way.
\begin{verse}
For in and out, above, about, below \\
'Tis nothing but a Magic \textit{Shadow}-show \\
Played in a Box whose candle is the Sun \\
Round which we Phantom Figures come and go.
\end{verse}
Our shadow-box takes up all our attention. There is a marvellous
spectacle above, and, as the photographs afterwards revealed,
a wonderful prominence-flame is poised a hundred thousand
miles above the surface of the sun. We have no time to snatch
a glance at it. We are conscious only of the weird half-light of
the landscape and the hush of nature, broken by the calls of the
observers, and beat of the metronome ticking out the $302$~seconds
of totality.

Sixteen photographs were obtained, with exposures ranging
from $2$ to $20$~seconds. The earlier photographs showed no stars,
though they portrayed the remarkable prominence; but apparently
the cloud lightened somewhat towards the end of totality,
and a few images appeared on the later plates. In many cases
one or other of the most essential stars was missing through
cloud, and no use could be made of them; but one plate was
found showing fairly good images of five stars, which were
suitable for a determination. This was measured on the spot
a few days after the eclipse in a micrometric measuring-machine.
The problem was to determine how the apparent positions of
the stars, affected by the sun's gravitational field, compared
with the normal positions on a photograph taken when the sun
was out of the way. Normal photographs for comparison had
been taken with the same telescope in England in January.
The eclipse photograph and a comparison photograph were
placed film to film in the measuring-machine so that corresponding
images fell close together\footnotemark,
  \footnotetext{This was possible because at Principe the field of stars was reflected in
  the coelostat mirror, whereas in England it was photographed direct.}%
and the small distances
were measured in two rectangular directions. From these the
relative displacements of the stars could be ascertained.
\index{Displacement of star-images}%
In
comparing two plates, various allowances have to be made for
refraction, aberration, plate-orientation, etc.; but since these
occur equally in determinations of stellar parallax, for which
%% -----File: 126.png---Folio 116-------
much greater accuracy is required, the necessary procedure is
well-known to astronomers.

The results from this plate gave a definite displacement, in
good accordance with Einstein's theory and disagreeing with
the Newtonian prediction. Although the material was very
meagre compared with what had been hoped for, the writer
(who it must be admitted was not altogether unbiassed) believed
it convincing.

It was not until after the return to England that any further
confirmation was forthcoming. Four plates were brought home
undeveloped, as they were of a brand which would not stand
development in the hot climate. One of these was found to
show sufficient stars; and on measurement it also showed the
deflection predicted by Einstein, confirming the other plate.

The bugbear of possible systematic error affects all investigations
of this kind. How do you know that there is not something
in your apparatus responsible for this apparent deflection?
Your object-glass has been shaken up by travelling; you have
introduced a mirror into your optical system; perhaps the $50�$
rise of temperature between the climate at the equator and
England in winter has done some kind of mischief. To meet
this criticism, a different field of stars was photographed at
night in Principe and also in England at the same altitude as
the eclipse field. If the deflection were really instrumental, stars
on these plates should show relative displacements of a similar
kind to those on the eclipse plates. But on measuring these
check-plates no appreciable displacements were found. That
seems to be satisfactory evidence that the displacement observed
during the eclipse is really due to the sun being in the region,
and is not due to differences in instrumental conditions between
England and Principe. Indeed the only possible loophole is a
difference between the night conditions at Principe when the
check-plates were taken, and the day, or rather eclipse, conditions
when the eclipse photographs were taken. That seems
impossible since the temperature at Principe did not vary more
than $1�$ between day and night.

The problem appeared to be settled almost beyond doubt;
and it was with some confidence that we awaited the return of
the other expedition from Brazil.
\index{Brazil, eclipse expedition to}%
The Brazil party had had
%% -----File: 127.png---Folio 117-------
fine weather and had gained far more extensive material on
their plates. They had remained two months after the eclipse
to photograph the same region before dawn, when clear of the
sun, in order that they might have comparison photographs
taken under exactly the same circumstances. One set of
photographs was secured with a telescope similar to that used
at Principe. In addition they used a longer telescope of 4 inches
aperture and $19$~feet focal length\footnotemark.
  \footnotetext{See \hyperref[frontispiece]{Frontispiece}. The two telescopes are shown and the backs of the two
  coelostat-mirrors which reflect the sky into them. The clock driving the larger
  mirror is seen on the pedestal on the left.}%
The photographs obtained
with the former were disappointing. Although the full number
of stars expected (about~12) were shown, and numerous plates
had been obtained, the definition of the images had been spoiled
by some cause, probably distortion of the coelostat-mirror by
the heat of the sunshine falling on it. The observers were
pessimistic as to the value of these photographs; but they were
the first to be measured on return to England, and the results
came as a great surprise after the indications of the Principe
plates. The measures pointed with all too good agreement to
the ``half-deflection,'' that is to say, the Newtonian value which
is one-half the amount required by Einstein's theory. It seemed
difficult to pit the meagre material of Principe against the wealth
of data secured from the clear sky of Sobral.
\index{Sobral, eclipse expedition to}%
\Pagelabel{117}%
It is true the
Sobral images were condemned, but whether so far as to
invalidate their testimony on this point was not at first clear;
besides the Principe images were not particularly well-defined,
and were much enfeebled by cloud. Certain compensating
advantages of the latter were better appreciated later. Their
strong point was the satisfactory check against systematic error
afforded by the photographs of the check-field; there were
no check-plates taken at Sobral, and, since it was obvious
that the discordance of the two results depended on systematic
error and not on the wealth of material, this distinctly
favoured the Principe results. Further, at Principe there could
be no evil effects from the sun's rays on the mirror, for the
sun had withdrawn all too shyly behind the veil of cloud.
A further advantage was provided by the check-plates at
Principe, which gave an independent determination of the
%% -----File: 128.png---Folio 118-------
difference of scale of the telescope as used in England and at
the eclipse; for the Sobral plates this scale-difference was
eliminated by the method of reduction, with the consequence
that the results depended on the measurement of a much smaller
relative displacement.

There remained a set of seven plates taken at Sobral with the
$4$-inch lens;
\index{Bending of light!observational results}%
\index{Light, bending of}%
their measurement had been delayed by the necessity
of modifying a micrometer to hold them, since they were of
unusual size. From the first no one entertained any doubt that
the final decision must rest with them, since the images were
almost ideal, and they were on a larger scale than the other
photographs. The use of this instrument must have presented
considerable difficulties---the unwieldy length of the telescope,
the slower speed of the lens necessitating longer exposures and
more accurate driving of the clock-work, the larger scale rendering
the focus more sensitive to disturbances---but the observers
achieved success, and the perfection of the negatives surpassed
anything that could have been hoped for.

These plates were now measured and they gave a final verdict
definitely confirming Einstein's value of the deflection, in agreement
with the results obtained at Principe.

It will be remembered that Einstein's theory predicts a
deflection of $1''.74$ at the edge of the sun\footnotemark,
  \footnotetext{The predicted deflection of light from infinity to infinity is just over~$1''.745$,
  from infinity to the earth it is just under.}%
the amount falling
off inversely as the distance from the sun's centre. The simple
Newtonian deflection is half this,~$0''.87$. The final results
(reduced to the edge of the sun) obtained at Sobral and Principe
with their ``probable accidental errors'' were
\index{Deflection of light!observational results}%
\index{Gravitational field of Sun!deflection of light}%
\begin{center}
\begin{tabular}{lc}
Sobral   & $1''.98 � 0''.12$, \\
Principe & $1''.61 � 0''.30$.
\end{tabular}
\end{center}
It is usual to allow a margin of safety of about twice the probable
error on either side of the mean. The evidence of the Principe
plates is thus just about sufficient to rule out the possibility of
the ``half-deflection,'' and the Sobral plates exclude it with
practical certainty. The value of the material found at Principe
cannot be put higher than about one-sixth of that at Sobral;
but it certainly makes it less easy to bring criticism against this
confirmation of Einstein's theory seeing that it was obtained
%% -----File: 129.png---Folio 119-------
independently with two different instruments at different places
and with different kinds of checks.

The best check on the results obtained with the $4$-inch lens
at Sobral is the striking internal accordance of the measures for
different stars. The theoretical deflection should vary inversely
as the distance from the sun's centre; hence, if we plot the mean
radial displacement found for each star separately against the
inverse distance, the points should lie on a straight line. This
%[Illustration: \textsc{Fig}. 17.]
\begin{figure*}[hbt]
\begin{center}
\Graphic[17]{4.5in}{129a}
\end{center}
\end{figure*}%
is shown in \Figref{17} where the broken line shows the theoretical
prediction of Einstein, the deviations being within the accidental
errors of the determinations. A line of half the slope representing
the half-deflection would clearly be inadmissible.

Moreover, values of the deflection were deduced from the
measures in right ascension and declination independently.
These were in close agreement.

%% -----File: 130.png---Folio 120-------

A diagram showing the relative positions of the stars is given
in \Figref{18}.

The square shows the limits of the plates used at Principe,
and the oblique rectangle the limits with the 4-inch lens at
Sobral. The centre of the sun moved from $S$ to~$P$ in the $2\frac{1}{4}$~hours
%[Illustration: \textsc{Fig}.~18.]
\begin{figure*}[hbt]
\begin{center}
\Graphic[18]{5in}{130a}
\end{center}
\end{figure*}%
interval between totality at the two stations; the sun is
here represented for a time about midway between. The stars
measured on the Principe plates were Nos.~3, 4, 5, 6, 10, 11; those
at Sobral were 11, 10, 6, 5, 4, 2, 3 (in the order of the dots
from left to right in \Figref{17}). None of these were fainter than
$6\Magnitude.0$, the brightest $\kappa^1$~Tauri (No.~4) being $4\Magnitude.5$.

It has been objected that although the observations establish
%% -----File: 131.png---Folio 121-------
a deflection of light in passing the sun equal to that predicted
by Einstein, it is not immediately obvious that this deflection
must necessarily be attributed to the sun's gravitational field.
It is suggested that it may not be an essential effect of the sun
as a massive body, but an accidental effect owing to the circumstance
that the sun is surrounded by a corona which acts as
a refracting atmosphere.
\index{Corona, refraction by}%
\index{Refraction of light in corona}%
It would be a strange coincidence if
this atmosphere imitated the theoretical law in the exact
quantitative way shown in \Figref{17}; and the suggestion appears
to us far-fetched. However the objection can be met in a more
direct way. We have already shown that the gravitational
effect on light is equivalent to that produced by a refracting
medium round the sun and have calculated the necessary
refractive index. At a height of $400,000$ miles above the surface
the refractive index required is~$1.0000021$. This corresponds to
air at $\frac{1}{140}$ atmosphere, hydrogen at $\frac{1}{70}$ atmosphere, helium at
$\frac{1}{20}$ atmospheric pressure. It seems obvious that there can be no
material of this order of density at such a distance from the sun.
The pressure on the sun's surface of the columns of material
involved would be of the order $10,000$ atmospheres; and we
know from spectroscopic evidence that there is no pressure of
this order. If it is urged that the mass could perhaps be supported
by electrical forces, the argument from absorption is
even more cogent. The light from the stars photographed during
the eclipse has passed through a depth of at least a million miles
of material of this order of density---or say the equivalent of
$10,000$ miles of air at atmospheric density. We know to our
cost what absorption the earth's $5$~miles of homogeneous
atmosphere can effect. And yet at the eclipse the stars appeared
on the photographs with their normal brightness. If the irrepressible
critic insists that the material round the sun may be
composed of some new element with properties unlike any
material known to us, we may reply that the mechanism of
refraction and of absorption is the same, and there is a limit to
the possibility of refraction without appreciable absorption.
Finally it would be necessary to arrange that the density of the
material falls off inversely as the distance from the sun's centre
in order to give the required variation of refractive index.

Several comets have been known to approach the sun within
%% -----File: 132.png---Folio 122-------
the limits of distance here considered. If they had to pass
through an atmosphere of the density required to account for
the displacement, they would have suffered enormous resistance.
Dr~Crommelin has shown that a study of these comets sets an
upper limit to the density of the corona, which makes the
refractive effect quite negligible.%
\index{Comets!motion through coronal medium}%
\index{Crommelin}%

Those who regard Einstein's law of gravitation as a natural
deduction from a theory based on the minimum of hypotheses
will be satisfied to find that his remarkable prediction is quantitatively
confirmed by observation, and that no unforeseen cause
has appeared to invalidate the test.
%% -----File: 133.png---Folio 123-------


\Chapter{VIII}{Other Tests of the Theory}

\Quote[break]{\textit{Love's Labour's Lost.}}
{The words of Mercury are harsh after the songs of Apollo.}

\First{We} have seen that the swift-moving light-waves possess great
advantages as a means of exploring the non-Euclidean property
of space. But there is an old fable about the hare and the
tortoise. The slow-moving planets have qualities which must
not be overlooked. The light-wave traverses the region in a few
minutes and makes its report; the planet plods on and on for
centuries going over the same ground again and again. Each
time it goes round it reveals a little about the space, and the
knowledge slowly accumulates.

According to Newton's law a planet moves round the sun in
an ellipse, and if there are no other planets disturbing it, the
ellipse remains the same for ever. According to Einstein's law
the path is very nearly an ellipse, but it does not quite close up;
and in the next revolution the path has advanced slightly in the
same direction as that in which the planet was moving. The
orbit is thus an ellipse which very slowly revolves\footnote%
{Appendix, \Noteref{9}.}.%
\Pagelabel{note9}%
\index{Mercury, perihelion of}%
\index{Orbits under Einstein's law}%
\index{Perihelia of planets, motions of}%

The exact prediction of Einstein's law is that in one revolution
of the planet the orbit will advance through a fraction of a
revolution equal to $3v^2/C^2$, where $v$ is the speed of the planet
and $C$ the speed of light. The earth has $1/10,000$ of the speed of
light; thus in one revolution (one year) the point where the
earth is at greatest distance from the sun will move on
$3/100,000,000$ of a revolution, or $0''.038$. We could not detect
this difference in a year, but we can let it add up for a century
at least. It would then be observable but for one thing---the
earth's orbit is very blunt, very nearly circular, and so we
cannot tell accurately enough which way it is pointing and how
its sharpest apses move. We can choose a planet with higher
speed so that the effect is increased, not only because $v^2$ is
increased, but because the revolutions take less time; but, what
%% -----File: 134.png---Folio 124-------
is perhaps more important, we need a planet with a sharp
elliptical orbit, so that it is easy to observe how its apses move
round. Both these conditions are fulfilled in the case of Mercury.
It is the fastest of the planets, and the predicted advance of the
orbit amounts to $43''$ per century; further the eccentricity of
its orbit is far greater than that of any of the other seven
planets.

Now an unexplained advance of the orbit of Mercury had
long been known. It had occupied the attention of Le~Verrier,
\index{Le Verrier}%
who, having successfully predicted the planet Neptune from the
disturbances of Uranus, thought that the anomalous motion of
Mercury might be due to an interior planet, which was called
Vulcan in anticipation. But, though thoroughly sought for,
Vulcan has never turned up. Shortly before Einstein arrived
at his law of gravitation, the accepted figures were as follows.
The actual observed advance of the orbit was $574''$ per century;
the calculated perturbations produced by all the known planets
amounted to $532''$ per century. The excess of $42''$ per century
remained to be explained. Although the amount could scarcely
be relied on to a second of arc, it was at least thirty times as
great as the probable accidental error.

The big discrepancy from the Newtonian gravitational theory
is thus in agreement with Einstein's prediction of an advance
of $43''$ per century.

The derivation of this prediction from Einstein's law can only
be followed by mathematical analysis; but it may be remarked
that any slight deviation from the inverse square law is likely
to cause an advance or recession of the apse of the orbit. That
a particle, if it does not move in a circle, should oscillate between
two extreme distances is natural enough; it could scarcely do
anything else unless it had sufficient speed to break away
altogether. But the interval between the two extremes will not
in general be half a revolution. It is only under the exact
adjustment of the inverse square law that this happens, so that
the orbit closes up and the next revolution starts at the same
point. I do not think that any ``simple explanation'' of this
property of the inverse-square law has been given; and it seems
fair to remind those, who complain of the difficulty of understanding
Einstein's prediction of the advance of the perihelion,
\index{Gravitational field of Sun!motion of perihelion} % [** PP: Index entry reads p. 122]
%% -----File: 135.png---Folio 125-------
that the real trouble is that they have not yet succeeded in
making clear to the uninitiated this recondite result of the
Newtonian theory. The slight modifications introduced by
Einstein's law of gravitation upset this fine adjustment, so that
the oscillation between the extremes occupies slightly more than
a revolution. A simple example of this effect of a small deviation
from the inverse-square law was actually given by Newton.

It had already been recognised that the change of mass with
velocity may cause an advance of perihelion; but owing to the
ambiguity of Newton's law of gravitation the discussion was
unsatisfactory. It was, however, clear that the effect was too
small to account for the motion of perihelion of Mercury, the
prediction being $\frac{1}{2} v^2/C^2$, or at most $v^2/C^2$. Einstein's theory is
the only one which gives the full amount $3v^2/C^2$.%
\index{Mercury, perihelion of}%

It was suggested by Lodge that, % [** PP: Retaining awkward commas]
\index{Lodge}%
this variation of mass with
velocity might account for the whole motion of the orbit of
Mercury, if account were taken of the sun's unknown absolute
motion through the aether, combining sometimes additively and
sometimes negatively with the orbital motion. In a discussion
between him and the writer, it appeared that, if the absolute
motion were sufficient to produce this effect on Mercury, it
must give observable effects for Venus and the Earth; and these
do not exist. Indeed from the close accordance of Venus and
the Earth with observation, it is possible to conclude that, either
the sun's motion through the aether is improbably small, or
gravitation must conform to relativity, in the sense of the
restricted principle (\Pageref{20}), and conceal the effects of the
increase of mass with speed so far as an additive uniform motion
is concerned.%
\index{Gravitation!relativity for uniform motion}%

Unfortunately it is not possible to obtain any further test of
Einstein's law of gravitation from the remaining planets. We
have to pass over Venus and the Earth, whose orbits are too
nearly circular to show the advance of the apses observationally.
Coming next to Mars with a moderately eccentric orbit, the
speed is very much smaller, and the predicted advance is only
$1''.3$ per century. Now the accepted figures show an observed
advance (additional to that produced by known causes) of $5''$
per century, so that Einstein's correction improves the accordance
of observation with theory; but, since the result for Mars
%% -----File: 136.png---Folio 126-------
is in any case scarcely trustworthy to $5''$ owing to the inevitable
errors of observation, the improvement is not very important.
The main conclusion is that Einstein's theory brings Mercury
into line, without upsetting the existing good accordance of all
the other planets.

We have tested Einstein's law of gravitation for fast movement
(light) and for moderately slow movement (Mercury).
For very slow movement it agrees with Newton's law, and the
general accordance of the latter with observation can be transferred
to Einstein's law. These tests appear to be sufficient to
establish the law firmly. We can express it in this way.

Every particle or light-pulse moves so that the quantity $s$
measured along its track between two points has the maximum
possible value, where
\[
ds^2 = - (1-2m/r)^{-1}\, dr^2 -r^2\, d\theta^2 + (1-2m/r)\, dt^2.
\]
And the accuracy of the experimental test is sufficient to verify
the coefficients as far as terms of order $m/r$ in the coefficient of~$dr^2$,
and as far as terms of order $m^2/r^2$ in the coefficient of~$dt^2$\footnote{Appendix, \Noteref{10}.}.
\Pagelabel{note10}

In this form the law appears to be firmly based on experiment,
and the revision or even the complete abandonment of the
general ideas of Einstein's theory would scarcely affect it.

These experimental proofs, that space in the gravitational
field of the sun is non-Euclidean or curved, have appeared
puzzling to those unfamiliar with the theory. It is pointed out
that the experiments show that physical objects or loci are
``warped'' in the sun's field; but it is suggested that there is
nothing to show that the space in which they exist is warped.
\index{Gravitational field of Sun!result of observational verification}%
The answer is that it does not seem possible to draw any distinction
between the warping of physical space and the warping
of physical objects which define space.
\index{Warping of space}%
If our purpose were
merely to call attention to these phenomena of the gravitational
field as curiosities, it would, no doubt, be preferable to avoid
using words which are liable to be misconstrued. But if we wish
to arrive at an understanding of the conditions of the gravitational
field, we cannot throw over the vocabulary appropriate
for that purpose, merely because there may be some who insist
on investing the words with a metaphysical meaning which is
clearly inappropriate to the discussion.

%% -----File: 137.png---Folio 127-------

We come now to another kind of test. In the statement of
the law of gravitation just given, a quantity $s$ is mentioned;
and, so far as that statement goes, $s$ is merely an intermediary
quantity defined mathematically. But in our theory we have
been identifying $s$ with interval-length, measured with an
apparatus of scales and clocks; and it is very desirable to test
whether this identification can be confirmed---whether the
geometry of scales and clocks is the same as the geometry of
moving particles and light-pulses.

The question has been mooted whether we may not divide
the present theory into two parts. Can we not accept the law
of gravitation in the form suggested above as a self-contained
result proved by observation, leaving the further possibility
that $s$ is to be identified with interval-length open to debate?
The motive is partly a desire to consolidate our gains, freeing
them from the least taint of speculation; but perhaps also it is
inspired by the wish to leave an opening by which clock-scale
geometry, i.e.\ the space and time of ordinary perception, may
remain Euclidean. Disregarding the connection of $s$ with
interval-length, there is no object in attributing any significance
of length to it;
\index{Interval-length!geometrical significance essential}%
it can be regarded as a dynamical quantity like
Action, and the new law of gravitation can be expressed after
the traditional manner without dragging in strange theories of
space and time. Thus interpreted, the law perhaps loses its
theoretical inevitability; but it remains strongly grounded on
observation. Unfortunately for this proposal, it is impossible
to make a clean division of the theory at the point suggested.
Without some geometrical interpretation of $s$ our conclusions as
to the courses of planets and light-waves cannot be connected
with the astronomical measurements which verify them. The
track of a light-wave in terms of the coordinates $r$, $\theta$, $t$ cannot
be tested directly; the coordinates afford only a temporary
resting-place; and the measurement of the displacement of the
star-image on the photographic plate involves a reconversion
from the coordinates to~$s$, which here appears in its significance
as the interval in clock-scale geometry.

Thus even from the experimental standpoint, a rough correspondence
of the quantity $s$ occurring in the law of gravitation
with the clock-scale interval is an essential feature. We have
%% -----File: 138.png---Folio 128-------
now to examine whether experimental evidence can be found
as to the exactness of this correspondence.

It seems reasonable to suppose that a vibrating atom is an
ideal type of clock. The beginning and end of a single vibration
constitute two events, and the interval $ds$ between two events
is an absolute quantity independent of any mesh-system. This
interval must be determined by the nature of the atom; and
hence atoms which are absolutely similar will measure by their
vibrations equal values of the absolute interval $ds$. Let us
adopt the usual mesh-system $(r, \theta, t)$ for the solar system, so
that
\[
ds^2 = - \gamma^{-1}\, dr^2 -r^2\, d\theta^2 + \gamma\, dt^2.
\]
Consider an atom momentarily at rest at some point in the solar
system; we say \textit{momentarily}, because it must undergo the
acceleration of the gravitational field where it is. If $ds$ corresponds
to one vibration, then, since the atom has not moved,
the corresponding $dr$ and $d\theta$ will be zero, and we have
\[
ds^2 = \gamma\, dt^2.
\]
The \textit{time} of vibration $dt$ is thus $1/\surd\gamma$ times the \textit{interval} of
vibration~$ds$.

Accordingly, if we have two similar atoms at rest at different
points in the system, the interval of vibration will be the same
for both; but the time of vibration will be proportional to the
inverse square-root of~$\gamma$, which differs for the two atoms. Since
\begin{align*}
        \gamma &= 1 - \frac{2m}{r} \\
1/\surd \gamma &= 1 + \,\frac{m}{r}, \quad\text{very approximately.}
\end{align*}

Take an atom on the surface of the sun, and a similar atom
in a terrestrial laboratory.
\index{Atom, vibrating on sun}%
\index{Clock!on sun}%
For the first, $1 + m/r = 1.00000212$,
and for the second $1 + m/r$ is practically~$1$. The time of vibration
of the solar atom is thus longer in the ratio $1.00000212$, and it
might be possible to test this by spectroscopic examination.

There is one important point to consider. The spectroscopic
examination must take place in the terrestrial laboratory; and
we have to test the period of the solar atom by the period of
the waves emanating from it when they reach the earth. Will
they carry the period to us unchanged? Clearly they must.
%% -----File: 139.png---Folio 129-------
The first and second pulse have to travel the same distance ($r$),
and they travel with the same velocity ($dr/dt$); for the velocity
of light in the mesh-system used is $1- 2m/r$, and though this
velocity depends on~$r$, it does not depend on~$t$. Hence the difference
$dt$ at one end of the waves is the same as that at the other
end.

Thus in the laboratory the light from a solar source should
be of greater period and greater wave-length (i.e.\ redder) than
that from a corresponding terrestrial source. Taking blue light
of wave-length $4000$~�, the solar lines should be displaced
$4000 � .00000212$, or $0.008$~� towards the red end of the
spectrum.%
\index{Displacement of spectral lines}%
\index{Gravitational field of Sun!displacement of spectral lines}%

The properties of a gravitational field of force are similar to
those of a centrifugal field of force; and it may be of interest
to see how a corresponding shift of the spectral lines occurs for
an atom in a field of centrifugal force.
\index{Retardation of time!in centrifugal field}%
Suppose that, as we rotate
with the earth, we observe a very remote atom momentarily at
rest relative to our rotating axes. The case is just similar to
that of the solar atom; both are at rest relative to the respective
mesh-systems; the solar atom is in a field of gravitational force,
and the other is in a field of centrifugal force. The direction of
the force is in both cases the same---from the earth towards the
atom observed. Hence the atom in the centrifugal field ought
also to vibrate more slowly, and show a displacement to the red
in its spectral lines.
\index{Centrifugal Force!vibrating atom in field of}%
It does, if the theory hitherto given is
right. We can abolish the centrifugal force by choosing non-rotating
axes. But the distant atom was at rest relative to the
rotating axes, that is to say, it was whizzing round with them.
Thus from the ordinary standpoint the atom has a large velocity
relative to the observer, and, in accordance with \Chapref{I}, its
vibrations slow down just as the aviator's watch did. The shift
of spectral lines due to a field of centrifugal force is only another
aspect of a phenomenon already discussed.

The expected shift of the spectral lines on the sun, compared
with the corresponding terrestrial lines, has been looked for;
but it has not been found.

In estimating the importance of this observational result in
regard to the relativity theory, we must distinguish between
a failure of the test and a definite conclusion that the lines are
%% -----File: 140.png---Folio 130-------
undisplaced. The chief investigators St~John, Schwarzschild,
Evershed, and Grebe and Bachem, seem to be agreed that the
observed displacement is at any rate less than that predicted
by the theory. The theory can therefore in no case claim support
from the present evidence. But something more must be
established, if the observations are to be regarded as in the
slightest degree adverse to the theory. If for instance the mean
deflection is found to be $.004$ instead of $.008$ Angstr�m units,
the only possible conclusion is that there are certain causes of
displacement of the lines, acting in the solar atmosphere and not
yet identified. No one could be much surprised if this were the
case; and it would, of course, render the test nugatory. The
case is not much altered if the observed displacement is $.002$
units, provided the latter quantity is above the accidental error
of measurement; if we have to postulate some unexplained disturbance,
it may just as well produce a displacement $-.006$ as
$+.002$. For this reason Evershed's evidence is by no means
adverse to the theory, since he finds unexplained displacements
in any case.
\index{Evershed}%
\index{St John}%
One set of lines measured by St~John gave a mean
displacement of $.0036$~units; and this also shows that the test
has failed. The only evidence \textit{adverse} to the theory, and not
merely neutral, is a series of measures by St~John on $17$ cyanogen
lines, which he regarded as most dependable. These gave a mean
shift of exactly~$.000$. If this stood alone we should certainly be
disposed to infer that the test had gone against Einstein's
theory, and that nothing had intervened to cast doubt on the
validity of the test. The writer is unqualified to criticise these
mutually contradictory spectroscopic conclusions; but he has
formed the impression that the last-mentioned result obtained
by St~John has the greatest weight of any investigations up to
the present\footnotemark.
  \footnotetext{A further paper by Grebe and Bachem (\textit{Zeitschrift f�r Physik}, 1920, p.~51),
  \index{Grebe and Bachem}%
  received whilst this is passing through the press, makes out a case strongly
  favourable for the Einstein displacement, and reconciles the discordant results
  found by most of the investigators. But it may still be the best counsel to
  ``wait and see,'' and I have made no alteration in the discussion here given.}

It seems that judgment must be reserved; but it may be well
to examine how the present theory would stand if the verdict
of this third crucial experiment finally went against it.

It has become apparent that there is something illogical in
%% -----File: 141.png---Folio 131-------
the sequence we have followed in developing the theory, owing
to the necessity of proceeding from the common ideas of space
and time to the more fundamental properties of the absolute
world.
\index{Clock-scale geometry, not fundamental}%
We started with a definition of the interval by measurements
made with clocks and scales, and afterwards connected
it with the tracks of moving particles. Clearly this is an inversion
of the logical order. The simplest kind of clock is an elaborate
mechanism, and a material scale is a very complex piece of
apparatus. The best course then is to discover $ds$ by exploration
of space and time with a moving particle or light-pulse, rather
than by measures with scales and clocks. On this basis by
astronomical observation alone the formula for $ds$ in the gravitational
field of the sun has already been established. To proceed
from this to determine exactly what is measured by a scale and
a clock, it would at first seem necessary to have a detailed theory
of the mechanisms involved in a scale and clock. But there is
a short-cut which seems legitimate. This short-cut is in fact
the Principle of Equivalence.
\index{Equivalence!Principle of}%
\index{Principle of Equivalence}%
Whatever the mechanism of the
clock, whether it is a good clock or a bad clock, the intervals it
is beating must be something absolute; the clock cannot know
what mesh-system the observer is using, and therefore its
absolute rate cannot be altered by position or motion which is
relative merely to a mesh-system. Thus wherever it is placed,
and however it moves, provided it is not constrained by impacts
or electrical forces, it must always beat equal intervals as we
have previously assumed. Thus a clock may fairly be used to
measure intervals, even when the interval is defined in the new
manner; any other result seems to postulate that it pays heed
to some particular mesh-system\footnotemark.
  \footnotetext{Of course, there is always the possibility that this might be the case,
  though it seems unlikely. The essential point of the relativity theory is that
  (contrary to the common opinion) no experiments yet made have revealed any
  mesh-system of an absolute character, not that experiments never will reveal
  such a system.}

Three modes of escape from this conclusion seem to be left
open. A clock cannot pay any heed to the mesh-system used;
but it may be affected by the kind of space-time around it\footnote%
  {Appendix, \Noteref{11}.}.
\Pagelabel{note11}%
The terrestrial atom is in a field of gravitation so weak that the
space-time may be considered practically flat; but the space-%
%% -----File: 142.png---Folio 132-------
time round the solar atom is not flat. It may happen that the
two atoms actually detect this absolute difference in the world
around them and do not vibrate with the same interval $ds$---contrary
to our assumption above. Then the prediction of the
shift of the lines in the solar spectrum is invalidated. Now it is
very doubtful if an atom can detect the curving of the region it
occupies, because curvature is only apparent when an extended
region is considered; still an atom has some extension, and it is
not impossible that its equations of motion involve the quantities
$B^{\rho}_{\mu\nu\sigma}$ which distinguish gravitational from flat space-time. An
apparently insuperable objection to this explanation is that the
effect of curvature on the period would almost certainly be
represented by terms of the form $m^{2}/r^{2}$, whereas to account for
a negative result for the shift of the spectral lines terms of much
greater order of magnitude $m/r$ are needed.

The second possibility depends on the question whether it is
possible for an atom at rest on the sun to be precisely similar to
one on the earth. If an atom fell from the earth to the sun it
would acquire a velocity of $610$~km.\ per sec., and could only be
brought to rest by a systematic hammering by other atoms.
May not this have made a permanent alteration in its time-keeping
properties? It is true that every atom is continually
undergoing collisions, but it is just possible that the average
solar atom has a different period from the average terrestrial
atom owing to this systematic difference in its history.

What are the two events which mark the beginning and end
of an atomic vibration? This question suggests a third possibility.
If they are two absolute events, like the explosions of
two detonators, then the interval between them will be a definite
quantity, and our argument applies. But if, for example, an
atomic vibration is determined by the revolution of an electron
around a nucleus, it is not marked by any definite events. A
revolution means a return to the same position as before; but
we cannot define what is the same position as before without
reference to some mesh-system. Hence it is not clear that there
is any absolute interval corresponding to the vibration of an
atom; an absolute interval only exists between two events
absolutely defined.

It is unlikely that any of these three possibilities can negative %[** PP: OK]
%% -----File: 143.png---Folio 133-------
the expected shift of the spectral lines. The uncertainties introduced
by them are, so far as we can judge, of a much smaller
order of magnitude. But it will be realised that this third test
of Einstein's theory involves rather more complicated considerations
than the two simple tests with light-waves and the moving
planet. I think that a shift of the Fraunhofer lines is a highly
probable prediction from the theory and I anticipate that
experiment will ultimately confirm the prediction; but it is not
entirely free from guess-work. These theoretical uncertainties
are apart altogether from the great practical difficulties of the
test, including the exact allowance for the unfamiliar circumstances
of an absorbing atom in the sun's atmosphere.

Outside the three leading tests, there appears to be little
chance of checking the theory unless our present methods of
measurement are greatly improved. It is not practicable to
measure the deflection of light by any body other than the sun.
The apparent displacement of a star just grazing the limb of
Jupiter should be~$0''.017$.
\index{Jupiter, deflection of light by}%
A hundredth of a second of arc is
just about within reach of the most refined measurements with
the largest telescopes. If the observation could be conducted
under the same conditions as the best parallax measurements,
the displacement could be detected but not measured with any
accuracy. The glare from the light of the planet ruins any chance
of success.

% [** PP: Retaining commas]
Most astronomers, who look into the subject, are entrapped
sooner or later by a fallacy in connection with double stars.
\index{Double stars and Einstein effect}%
It is thought that when one component passes behind the other
it will appear displaced from its true position, like a star passing
behind the sun; if the size of the occulting star is comparable
with that of the sun, the displacement should be of the same
order, $1''.7$. This would cause a very conspicuous irregularity in
the apparent orbit of a double star. But reference to \Pageref{113}
shows that an essential point in the argument was the enormous
ratio of the distance $QP$ of the star from the sun to the distance
$EF$ of the sun from the earth. It is only in these conditions that
the apparent displacement of the object is equal to the deflection
undergone by its light. It is easy to see that where this ratio is
reversed, as in the case of the double star, the apparent displacement
is an extremely small fraction of the deflection of the light.
It would be quite imperceptible to observation.

%% -----File: 144.png---Folio 134-------

If two independent stars are seen in the same line of vision
within about~$1''$, one being a great distance behind the other,
the conditions seem at first more favourable. I do not know if
any such pairs exist. It would seem that we ought to see the
more distant star not only by the direct ray, which would be
practically undisturbed, but also by a ray passing round the
other side of the nearer star and bent by it to the necessary
extent. The second image would, of course, be indistinguishable
from that of the nearer star; but it would give it additional
brightness, which would disappear in time when the two stars
receded. But consider a pencil of light coming past the nearer
star; the inner edge will be bent more than the outer edge, so
that the divergence is increased. The increase is very small;
but then the whole divergence of a pencil from a source some
hundred billion miles away is very minute. It is easily calculated
that the increased divergence would so weaken the light as to
make it impossible to detect it when it reached us\footnote%
  {Appendix, \Noteref{12}.}.
\Pagelabel{note12}

If two unconnected stars approached the line of sight still
more closely, so that one almost occulted the other, observable
effects might be perceived. When the proximity was such that
the direct ray from the more distant star passed within about
$100$~million kilometres of the nearer star, it would begin to fade
appreciably. The course of the ray would not yet be appreciably
deflected, but the divergence of the pencil would be rapidly
increased, and less light from the star would enter our telescopes.
The test is scarcely likely to be an important one, since a
sufficiently close approach is not likely to occur; and in any
case it would be difficult to feel confident that the fading was
not due to a nebulous atmosphere around the nearer star.

The theory gives small corrections to the motion of the moon
which have been investigated by de~Sitter.
\index{Moon, motion of}%
\index{de Sitter}%
Both the axis of
the orbit and its line of intersection with the ecliptic should
advance about $2''$ per century more than the Newtonian theory
indicates. Neither observation nor Newtonian theory are as yet
pushed to sufficient accuracy to test this; but a comparatively
small increase in accuracy would make a comparison possible.

Since certain stars are perhaps ten times more massive than
the sun, without the radius being unduly increased, they should
show a greater shift of the spectral lines and might be more
%% -----File: 145.png---Folio 135-------
favourable for the third crucial test. Unfortunately the predicted
shift is indistinguishable from that caused by a velocity
of the star in the line-of-sight on Doppler's principle. Thus the
expected shift on the sun is equivalent to that caused by a receding
velocity of $0.634$ kilometres per second. In the case of the
sun we know by other evidence exactly what the line-of-sight
velocity should be; but we have not this knowledge for other
stars. The only indication that could be obtained would be the
detection of an \textit{average} motion of recession of the more massive
stars. It seems rather unlikely that there should be a real
preponderance of receding motions among stars taken indiscriminately
from all parts of the sky; and the apparent effect
might then be attributed to the Einstein shift. Actually the
most massive stars (those of spectral type~$B$) have been found
to show an average velocity of recession of about $4.5$~km.\ per
sec., which would be explained if the values of $m/r$ for these
stars are about seven times greater than the value for the sun---a
quite reasonable hypothesis.
\index{Displacement of spectral lines!in stars}%
\index{Receding velocities of B@Receding velocities!of $B$-type stars}%
This phenomenon was well-known
to astrophysicists some years before Einstein's theory
was published. But there are so many possible interpretations
that no stress should be placed on this evidence. Moreover the
very diffuse ``giant'' stars of type~$M$ have also a considerable
systematic velocity of recession, and for these $m/r$ must be much
less than for the sun.
%% -----File: 146.png---Folio 136-------


\Chapter{IX}{Momentum and Energy}

\null\hspace{0.75in}
\begin{minipage}{4.25in} % [** PP: Hard-coded width; ~2.6pt overfull]
\Quote{Newman, \textit{Dream of Gerontius}.}
{\hspace*{-\QIndent}For spirits and men by different standards mete \\
The less and greater in the flow of time. \\
By sun and moon, primeval ordinances--- \\
By stars which rise and set harmoniously--- \\
By the recurring seasons, and the swing \\
This way and that of the suspended rod \\
Precise and punctual, men divide the hours, \\
Equal, continuous, for their common use. \\
Not so with us in the immaterial world; \\
But intervals in their succession \\
Are measured by the living thought alone \\
And grow or wane with its intensity. \\
And time is not a common property; \\
But what is long is short, and swift is slow \\
And near is distant, as received and grasped \\
By this mind and by that.}
\end{minipage}%
\index{Motion!Newton's first law}%
\index{Newton!law of motion}%

\First{One} of the most important consequences of the relativity theory
is the unification of inertia and gravitation.

The beginner in mechanics does not accept Newton's first law
of motion without a feeling of hesitation. He readily agrees that
a body at rest will remain at rest unless something causes it to
move; but he is not satisfied that a body in motion will remain
in uniform motion so long as it is not interfered with. It is
quite natural to think that motion is an impulse which will
exhaust itself, and that the body will finally come to a stop.
The teacher easily disposes of the arguments urged in support
of this view, pointing out the friction which has to be overcome
when a train or a bicycle is kept moving uniformly. He shows
that if the friction is diminished, as when a stone is projected
across ice, the motion lasts for a longer time, so that if all interference
by friction were removed uniform motion might continue
indefinitely. But he glosses over the point that if there
were no interference with the motion---if the ice were abolished
altogether---the motion would be by no means uniform, but like
that of a falling body. The teacher probably insists that the
continuance of uniform motion does not require anything that
%% -----File: 147.png---Folio 137-------
can properly be called a \textit{cause}. The property is given a name
\textit{inertia}; but it is thought of as an innate tendency in contrast
to \textit{force} which is an active cause.
\index{Force!compared with inertia}%
\index{Inertia!compared with force}%
So long as forces are confined
to the thrusts and tensions of elementary mechanics, where there
is supposed to be direct contact of material, there is good ground
for this distinction; we can visualise the active hammering of
the molecules on the body, causing it to change its motion. But
when force is extended to include the gravitational field the
distinction is not so clear.

For our part we deny the distinction in this last case. Gravitational
force is not an active agent working against the passive
tendency of inertia. Gravitation and inertia are one. The
uniform straight track is only relative to some mesh-system,
which is assigned by arbitrary convention. We cannot imagine
that a body looks round to see who is observing it and then feels
an innate tendency to move in that observer's straight line---probably
at the same time feeling an active force compelling
it to move some other way. If there is anything that can be
called an innate tendency it is the tendency to follow what we
have called the natural track---the longest track between two
points. We might restate the first law of motion in the form
``Every body tends to move in the track in which it actually
does move, except in so far as it is compelled by material impacts
to follow some other track than that in which it would otherwise
move.'' Probably no one will dispute this profound statement!

Whether the natural track is straight or curved, whether the
motion is uniform or changing, a cause is in any case required.
This cause is in all cases the combined inertia-gravitation.
\index{Inertia-gravitation}%
To
have given it a name does not excuse us from attempting an
explanation of it in due time. Meanwhile this identification of
inertia and gravitation as arbitrary components of one property
explains why weight is always proportional to inertia.
\index{Weight!proportional to inertia}%
This
experimental fact verified to a very high degree of accuracy
would otherwise have to be regarded as a remarkable law of
nature.

We have learnt that the natural track is the longest track
between two points; and since this is the only definable track
having an absolute significance in nature, we seem to have a
sufficient explanation of why an undisturbed particle must
%% -----File: 148.png---Folio 138-------
follow it. That is satisfactory, so far as it goes, but still we should
naturally wish for a clearer picture of the cause---inertia-gravitation---which
propels it in this track.

It has been seen that the gravitational field round a body
involves a kind of curvature of space-time, and accordingly
round each particle there is a minute pucker. Now at each
successive instant a particle is displaced continuously in time if
not in space; and so in our four-dimensional representation
which gives a bird's-eye-view of all time, the pucker has the
form of a long groove along the track of the particle. Now such
a groove or pleat in a continuum cannot take an arbitrary
course---as every dress-maker knows. Einstein's law of gravitation
gives the rule according to which the curvatures at any
point of space-time link on to those at surrounding points; so
that when a groove is started in any direction the rest of its
course can be forecasted. We have hitherto thought of the law
of gravitation as showing how the pucker spreads out in space,
cf.\ Newton's statement that the corresponding force weakens as
the inverse square of the distance. But the law of Einstein
equally shows how the gravitational field spreads out in time,
since there is no absolute distinction of time and space. It can
be deduced mathematically from Einstein's law that a pucker
of the form corresponding to a particle necessarily runs along
the track of greatest interval-length between two points.

The track of a particle of matter is thus determined by the
interaction of the minute gravitational field, which surrounds
and, so far as we know, constitutes it, with the general space-time
of the region. The various forms which it can take, find
their explanation in the new law of gravitation. The straight
tracks of the stars and the curved tracks of the planets are
placed on the same level, and receive the same kind of explanation.
The one universal law, that the space-time continuum
can be curved only in the first degree, is sufficient to prescribe
the forms of all possible grooves crossing it.%
\index{Geodesic!motion of particles in}%

The application of Einstein's law to trace the gravitational
field not only through space but through time leads to a great
unification of mechanics. If we have given for a start a narrow
slice of space-time representing the state of the universe for a few
seconds, with all the little puckers belonging to particles of matter
%% -----File: 149.png---Folio 139-------
properly described, then step by step all space-time can be linked
on and the positions of the puckers shown at all subsequent
times (electrical forces being excluded). Nothing is needed for
this except the law of gravitation---that the curvature is only
of the first degree---and there can thus be nothing in the predictions
of mechanics which is not comprised in the law of
gravitation. The conservation of mass, of energy, and of
momentum must all be contained implicitly in Einstein's law.%
\index{Conservation!of energy and momentum}%
\index{Energy!conservation of}%
\IndexExtra{Conservation!of mass}%

It may seem strange that Einstein's law of gravitation should
take over responsibility for the whole of mechanics; because in
many mechanical problems gravitation in the ordinary sense
can be neglected. But inertia and gravitation are unified; the
law is also the law of inertia, and inertia or mass appears in all
mechanical problems. When, as in many problems, we say that
gravitation is negligible, we mean only that the interaction of
the minute puckers with one another can be neglected; we do
not mean that the interaction of the pucker of a particle with
the general character of the space-time in which it lies can be
neglected, because this constitutes the inertia of the particle.%
\index{Inertia!relativity theory of}%

The conservation of energy and the conservation of momentum
in three independent directions, constitute together four laws
or equations which are fundamental in all branches of mechanics.
Although they apply when gravitation in the ordinary sense is
not acting, they must be deducible like everything else in
mechanics from the law of gravitation. It is a great triumph for
Einstein's theory that his law gives correctly these experimental
principles, which have generally been regarded as unconnected
with gravitation. We cannot enter into the mathematical
deduction of these equations; but we shall examine generally
how they are arrived at.

It has already been explained that although the values of
$G_{\mu\nu}$ are strictly zero everywhere in space-time, yet if we take
average values through a small region containing a large number
of particles of matter their average or ``macroscopic'' values
will not be zero\footnotemark.
  \footnotetext{It is the $g$'s which are first averaged, then the $G_{\mu\nu}$ are calculated by the
  formulae in \Noteref{5}.}%
\index{Macroscopic!equations}%
Expressions for these macroscopic values can
be found in terms of the number, masses and motions of the
particles. Since we have averaged the $G_{\mu\nu}$, we should also
%% -----File: 150.png---Folio 140-------
average the particles; that is to say, we replace them by a
distribution of continuous matter having equivalent properties.
We thus obtain macroscopic equations of the form
\index{Continuous matter}%
\[
G_{\mu\nu} = K_{\mu\nu},
\]
where on the one side we have the somewhat abstruse quantities
describing the kind of space-time, and on the other side we have
well-known physical quantities describing the density, momentum,
energy and internal stresses of the matter present. These
macroscopic equations are obtained solely from the law of
gravitation by the process of averaging.

By an exactly similar process we pass from Laplace's equation
$\nabla^2\phi = 0$ to Poisson's equation for continuous matter $\nabla^2\phi = -4\pi\rho$,
in the Newtonian theory of gravitation.%
\index{Laplace's equation}%

When continuous matter is admitted, \textit{any} kind of space-time
becomes possible. The law of gravitation instead of denying the
possibility of certain kinds, states what values of $K_{\mu\nu}$, i.e.\ what
distribution and motion of continuous matter in the region, are
a necessary accompaniment. This is no contradiction with the
original statement of the law, since that referred to the case in
which continuous matter is denied or excluded. Any set of
values of the potentials is now possible; we have only to calculate
by the formulae the corresponding values of $G_{\mu\nu}$, and we at
once obtain ten equations giving the $K_{\mu\nu}$ which define the
conditions of the matter necessary to produce these potentials.
But suppose the necessary distribution of matter through space
and time is an impossible one, violating the laws of mechanics!
No, there is only one law of mechanics, the law of gravitation;
we have specified the distribution of matter so as to satisfy
$G_{\mu\nu} = K_{\mu\nu}$, and there can be no other condition for it to fulfil.
The distribution must be mechanically possible; it might, however,
be unrealisable in practice, involving inordinately high or
even negative density of matter.%
\index{Gravitation, Einstein's law of!macroscopic equations}%

In connection with the law for empty space, $G_{\mu\nu} = 0$, it was
noticed that whereas this apparently forms a set of ten equations,
only six of them can be independent. This was because ten
equations would suffice to determine the ten potentials precisely,
and so fix not only the kind of space-time but the mesh-system.
It is clear that we must preserve the right to draw the mesh-system
as we please; it is fixed by arbitrary choice not by a law
%% -----File: 151.png---Folio 141-------
of nature. To allow for the four-fold arbitrariness of choice,
there must be four relations always satisfied by the $G_{\mu\nu}$, so that
when six of the equations are given the remaining four become
tautological.

These relations must be identities implied in the mathematical
definition of $G_{\mu\nu}$; that is to say, when the $G_{\mu\nu}$ have been written
out in full according to their definition, and the operations
indicated by the identities carried out, all the terms will cancel,
leaving only $0 = 0$. The essential point is that the four relations
follow from the mode of formation of the $G_{\mu\nu}$ from their simpler
constituents ($g_{\mu\nu}$ and their differential coefficients) and apply
universally. These four identical relations have actually been
discovered\footnote{Appendix, \Noteref{13}.}.%
\Pagelabel{note13}%
\index{Identities@Identities connecting $G_{\mu \nu}$}%

When in continuous matter $G_{\mu\nu} = K_{\mu\nu}$ clearly the same four
relations must exist between the $K_{\mu\nu}$, not now as identities,
but as consequences of the law of gravitation, viz.\ the equality
of $G_{\mu\nu}$ and $K_{\mu\nu}$.%
\index{Matter!gravitational equations in}%

Thus the four dimensions of the world bring about a four-fold
arbitrariness of choice of mesh-system; this in turn necessitates
four identical relations between the $G_{\mu\nu}$; and finally, in consequence
of the law of gravitation, these identities reveal four new
facts or laws relating to the density, energy, momentum or stress
of matter, summarised in the expressions $K_{\mu\nu}$.

These four laws turn out to be the laws of conservation of
momentum and energy.

The argument is so general that we can even assert that
corresponding to any \textit{absolute} property of a volume of a world
of four dimensions (in this case, \textit{curvature}), there must be four
\textit{relative} properties which are conserved. This might be made the
starting-point of a general inquiry into the necessary qualities
of a permanent perceptual world, i.e.\ a world whose substance
is conserved.%
\index{Conservation!of mass}%
\index{Mass!conservation of}%
\index{Momentum!conservation of}%
\index{Permanent perceptual world}%
\IndexExtra{Conservation!of energy and momentum}%

There is another law of physics which was formerly regarded
as funda\-mental---the conservation of mass. Modern progress
has somewhat altered our position with regard to it; not that
its validity is denied, but it has been reinterpreted, and has
finally become merged in the conservation of energy. It will be
desirable to consider this in detail.

%% -----File: 152.png---Folio 142-------

%[Illustration: \textsc{Fig}. 19.]
% [** PP: Moved to top of paragraph]
\begin{figure*}[hbt]
\begin{center}
\Graphic[19]{3in}{152a}
\end{center}
\end{figure*}%
It was formerly supposed that the mass of a particle was a
number attached to the particle, expressing an intrinsic property,
which remained unaltered in all its vicissitudes. If $m$ is this
number, and $u$ the velocity of the particle, the momentum is~$mu$;
and it is through this relation, coupled with the law of conservation
of momentum that the mass~$m$ was defined. Let us take
for example two particles of masses $m_1 = 2$ and $m_2 = 3$, moving
in the same straight line. In the space-time diagram for an
observer $S$ the velocity of the first particle will be represented
by a direction~$OA$ (\Figref{19}). The first particle moves through
a space $MA$ in unit time, so that $MA$ is equal to its velocity
referred to the observer~$S$. Prolonging the line $OA$ to meet the
second time-partition, $NB$ is equal to the velocity multiplied
by the mass~$2$; thus the horizontal distance $NB$ represents the
momentum. Similarly, starting from $B$ and drawing $BC$ in the
direction of the velocity of~$m_2$, prolonged through three
time-partitions, the horizontal progress from $B$ represents the
momentum of the second particle. The length $PC$ then represents
the total momentum of the system of two particles.

Suppose that some change of their velocities occurs, not
involving any transference of momentum from outside, e.g.\ a
collision. Since the total momentum $PC$ is unaltered, a similar
%% -----File: 153.png---Folio 143-------
construction made with the new velocities must again bring us
to~$C$; that is to say, the new velocities are represented by the
directions $OB'$, $B'C$, where $B'$ is some other point on the line~$NB$.

%[Illustration: \textsc{Fig}.~20.]
% [** PP: Moved to top of paragraph]
\begin{figure*}[hbt]
\begin{center}
\Graphic[20]{3in}{153a}
\end{center}
\end{figure*}%
Now examine how this will appear to some other observer $S_1$
in uniform motion relative to $S$. His transformation of space
and time has been described in \Chapref{III} and is represented in
\Figref{20}, which shows how his time-partitions run as compared
with those of~$S$. The same actual motion is, of course, represented
by parallel directions in the two diagrams; but the
interpretation as a velocity $MA$ is different in the two cases.
Carrying the velocity of $m_1$ through two time-partitions, and of
$m_2$ through three time-partitions, as before, we find that the total
momentum for the observer $S_1$ is represented by~$PC$ (\Figref{20});
but making a similar construction with the velocities after
collision, we arrive at a different point~$C'$. Thus whilst momentum
is conserved for the observer~$S$, it has altered from $PC$ to
$PC'$ for the observer~$S_1$.

The discrepancy arises because in the construction the lines
are prolonged to meet partitions which are different for the two
%% -----File: 154.png---Folio 144-------
observers. The rule for determining momentum ought to be
such that both observers make the same construction, independent
of their partitions, so that both arrive by the two routes
at the same point~$C$. Then it will not matter if, through their
different measures of time, one observer measures momentum
by horizontal progress and the other by oblique progress; both
will agree that the momentum has not been altered by the
collision. To describe such a construction, we must use the
interval which is alike for both observers; make the interval-length
of $OB$ equal to $2$~units, and that of $BC$ equal to $3$~units,
disregarding the mesh-system altogether. Then both observers
will make the same diagram and arrive at the same point~$C$
(different from $C$ or~$C'$ in the previous diagrams). Then if
momentum is conserved for one observer, it will be conserved
for the other.

This involves a modified definition of momentum. Momentum
must now be the mass multiplied by the change of position $\delta x$
per lapse of interval $\delta s$, instead of per lapse of time $\delta t$. Thus
\index{Momentum!redefinition of}%
\begin{align*}
           \text{momentum} &= m \frac{\delta x}{\delta s} \\
\text{instead of momentum} &= m \frac{\delta x}{\delta t},
\end{align*}
and the mass~$m$ still preserves its character as an invariant
number associated with the particle.

Whether the momentum as now defined is actually conserved
or not, is a matter for experiment, or for theoretical deduction
from the law of gravitation. The point is that with the original
definition general conservation is impossible, because if it held
good for one observer it could not hold for another. The new
definition makes general conservation possible. Actually this
form of the momentum is the one deduced from the law of
gravitation through the identities already described. With
regard to experimental confirmation it is sufficient at present
to state that in all ordinary cases the interval and the time are
so nearly equal that such experimental foundation as existed
for the law of conservation of the old momentum is just as
applicable to the new momentum.

Thus in the theory of relativity momentum appears as an
%% -----File: 155.png---Folio 145-------
invariant mass multiplied by a modified velocity $\delta x/\delta s$. The
physicist, however, prefers for practical purposes to keep to the
old definition of momentum as mass multiplied by the velocity
$\delta x/\delta t$. We have
\index{Invariant mass}%
\index{Mass!invariant}%
\[
m \frac{\delta x}{\delta s}
  =  m \frac{\delta t}{\delta s} � \frac{\delta x}{\delta t},
\]
accordingly the momentum is separated into two factors, the
velocity $\delta x/\delta t$, and a mass $M = m \delta t/\delta s$, which is no longer an
invariant for the particle but depends on its motion relative to
the observer's space and time. In accordance with the usual
practice of physicists the mass (unless otherwise qualified) is
taken to mean the quantity~$M$.

Using unaccelerated rectangular axes, we have by definition
of~$s$
\[
\delta s^2 = \delta t^{2} - \delta x^{2} - \delta y^{2} - \delta z^{2},
\]
so that
\begin{align*}
\left(\frac{\delta s}{\delta t}\right)^2
  &= 1 - \left(\frac{\delta x}{\delta t}\right)^2
       - \left(\frac{\delta y}{\delta t}\right)^2
       - \left(\frac{\delta z}{\delta t}\right)^2, \\
  &= 1-u^2,
\end{align*}
where $u$ is the resultant velocity of the particle (the velocity of
light being unity). Hence
\index{Mass!variation with velocity}%
\[
M = \frac{m}{\surd(1-u^2)}.
\]
Thus the mass increases as the velocity increases, the factor
being the same as that which determines the FitzGerald contraction.

The increase of mass with velocity is a property which challenges
experimental test. For success it is necessary to be able
to experiment with high velocities and to apply a known force
large enough to produce appreciable deflection in the fast-moving
particle. These conditions are conveniently fulfilled by
the small negatively charged particles emitted by radio-active % [** PP: Regularized radioactive]
substances, known as $\beta$~particles, or the similar particles which
constitute cathode rays.
\index{Beta particles}%
They attain speeds up to 0.8 of the
velocity of light, for which the increase of mass is in the ratio
1.66; and the negative charge enables a large electric or magnetic
force to be applied. Modern experiments fully confirm the
theoretical increase of mass, and show that the factor $1/\surd(1-u^2)$
%% -----File: 156.png---Folio 146-------
is at least approximately correct. The experiment was originally
performed by Kaufmann; but much greater accuracy has been
obtained by recent modified methods.%
\index{Electron!Kaufmann's experiment on}%
\index{Kaufmann's experiment}%

Unless the velocity is very great the mass~$M$ may be written
\[
m/\surd(1-u^2) = m + \tfrac{1}{2} mu^2.
\]
Thus it consists of two parts, the mass when at rest, together
with the second term which is simply the energy of the motion.
If we can say that the term $m$ represents a kind of potential
energy concealed in the matter, mass can be identified with
energy. The increase of mass with velocity simply means that
the energy of motion has been added on.
\index{Energy!identified with mass}%
\index{Mass!identified with energy}%
We are emboldened
to do this because in the case of an electrical charge the electrical
mass is simply the energy of the static field. Similarly the mass
of light is simply the electromagnetic energy of the light.

In our ordinary units the velocity of light is not unity, and
a rather artificial distinction between mass and energy is introduced.
They are measured by different units, and energy~$E$ has
a mass $E/C^2$ where $C$ is the velocity of light in the units used.
But it seems very probable that mass and energy are two ways
of measuring what is essentially the same thing, in the same
sense that the parallax and distance of a star are two ways of
expressing the same property of location. If it is objected that
they ought not to be confused inasmuch as they are distinct
properties, it must be pointed out that they are not sense-properties,
but mathematical terms expressing the dividend
and product of more immediately apprehensible properties, viz.\
momentum and velocity. They are essentially mathematical
compositions, and are at the disposal of the mathematician.

This proof of the variation of mass with velocity is much more
general than that based on the electrical theory of inertia.
\index{Energy!inertia of}%
It
applies immediately to matter in bulk. The masses $m_1$ and~$m_2$
need not be particles; they can be bodies of any size or composition.
On the electrical theory alone, there is no means of
deducing the variation of mass of a planet from that of an
electron.

It has to be remarked that, although the inertial mass of a
particle only comes under physical measurement in connection
with a change of its motion, it is just when the motion is changing
that the conception of its mass is least definite; because it is at
%% -----File: 157.png---Folio 147-------
that time that the kinetic energy, which forms part of the mass,
is being passed on to another particle or radiated into the
surrounding field; and it is scarcely possible to define the
moment at which this energy ceases to be associated with the
particle and must be reckoned as broken loose. The amount of
energy or mass in a given region is always a definite quantity;
but the amount attributable to a particle is only definite when
the motion is uniform. In rigorous work it is generally necessary
to consider the mass not of a particle but of a region.

The motion of matter from one place to another causes an
alteration of the gravitational field in the surrounding space.
If the motion is uniform, the field is simply convected; but if
the motion is accelerated, something of the nature of a gravitational
wave is propagated outwards. The velocity of propagation
is the velocity of light.
\index{Gravitation!propagation with velocity of light}%
\index{Propagation of Gravitation}%
\index{Velocity of gravitation}%
The exact laws are not very simple
because we have seen that the gravitational field modifies the
velocity of light; and so the disturbance itself modifies the
velocity with which it is propagated. In the same way the
exact laws of propagation of sound are highly complicated,
because the disturbance of the air by sound modifies the speed
with which it is propagated. But the approximate laws of
propagation of gravitation are quite simple and are the same as
those of electromagnetic disturbances.

After mass and energy there is one physical quantity which
plays a very fundamental part in modern physics, known as
\textit{Action}.
\index{Action}%
\textit{Action} here is a very technical term, and is not to be
confused with Newton's ``Action and Reaction.'' In the relativity
theory in particular this seems in many respects to be the most
fundamental thing of all. The reason is not difficult to see. If
we wish to speak of the continuous matter present \textit{at} any particular
point of space and time, we must use the term \textit{density}.
Density multiplied by volume in space gives us \textit{mass} or, what
appears to be the same thing, \textit{energy}. But from our space-time
point of view, a far more important thing is density multiplied
by a four-dimensional volume of space and time; this is \textit{action}.
The multiplication by three dimensions gives mass or energy;
and the fourth multiplication gives mass or energy multiplied
by time. Action is thus mass multiplied by time, or energy
multiplied by time, and is more fundamental than either.

%% -----File: 158.png---Folio 148-------

Action is the curvature of the world.
\index{Curvature!identified with action}%
It is scarcely possible
to visualise this statement, because our notion of curvature is
derived from surfaces of two dimensions in a three-dimensional
space, and this gives too limited an idea of the possibilities of a
four-dimensional surface in space of five or more dimensions.
In two dimensions there is just one total curvature, and if that
vanishes the surface is flat or at least can be unrolled into a
plane. In four dimensions there are many coefficients of
curvature; but there is one curvature \textit{par excellence}, which is,
of course, an invariant independent of our mesh-system. It is
the quantity we have denoted by~$G$. It does not follow that if
the curvature vanishes space-time is flat; we have seen in fact
that in a natural gravitational field space-time is not flat
although there may be no mass or energy and therefore no action
or curvature.

Wherever there is matter\footnote%
  {It is rather curious that there is no action in space containing only light.
  Light has mass ($M$) of the ordinary kind; but the invariant mass ($m$) vanishes.}
\index{Invariant mass!of light}%
\index{Light!mass of}%
\index{Mass of light}%
there is action and therefore
curvature; and it is interesting to notice that in ordinary matter
the curvature of the space-time world is by no means insignificant.
For example, in water of ordinary density the curvature is the
same as that of space in the form of a sphere of radius $570,000,000$
kilometres. The result is even more surprising if expressed in
time units; the radius is about half-an-hour.

It is difficult to picture quite what this means; but at least
we can predict that a globe of water of $570,000,000$~km.\ radius
would have extraordinary properties.
\index{Curvature!of a globe of water}%
\index{Globe of water, limit to size of}%
Presumably there must
be an upper limit to the possible size of a globe of water. So
far as I can make out a homogeneous mass of water of about
this size (and no larger) could exist. It would have no centre,
and no boundary, every point of it being in the same position
with respect to the whole mass as every other point of it---like
points on the \textit{surface} of a sphere with respect to the surface.
Any ray of light after travelling for an hour or two would come
back to the starting point. Nothing could enter or leave the
mass, because there is no boundary to enter or leave by; in
fact, it is coextensive with space. There could not be any other
world anywhere else, because there isn't an ``anywhere else.''

The mass of this volume of water is not so great as the most
%% -----File: 159.png---Folio 149-------
moderate estimates of the mass of the stellar system. Some
physicists have predicted a distant future when all energy will
be degraded, and the stellar universe will gradually fall together
into one mass. Perhaps then these strange conditions will be
realised!

The law of gravitation, the laws of mechanics, and the laws
of the electromagnetic field have all been summed up in a single
Principle of Least Action.
\index{Action, Principle of Least}%
\index{Principle of Least Action}%
For the most part this unification
was accomplished before the advent of the relativity theory,
and it is only the addition of gravitation to the scheme which is
novel. We can see now that if action is something absolute,
a configuration giving minimum action is capable of absolute
definition; and accordingly we should expect that the laws of
the world would be expressible in some such form. The argument
is similar to that by which we first identified the natural
tracks of particles with the tracks of greatest interval-length.
The fact that some such form of law is inevitable, rather discourages
us from seeking in it any clue to the structural details
of our world.

Action is one of the two terms in pre-relativity physics which
survive unmodified in a description of the absolute world. The
only other survival is entropy. The coming theory of relativity
had cast its shadow before; and physics was already converging
to two great generalisations, the principle of least action and
the second law of thermodynamics or principle of maximum
entropy.%
\index{Entropy}%

We are about to pass on to recent and more shadowy developments
of this subject; and this is an appropriate place to glance
back on the chief results that have emerged. The following
summary will recall some of the salient points.

\Paragraph{1.} The order of events in the external world is a four-dimensional
order.

\Paragraph{2.} The observer either intuitively or deliberately constructs
a system of meshes (space and time partitions) and locates the
events with respect to these.

\Paragraph{3.} Although it seems to be theoretically possible to describe
phenomena without reference to any mesh-system (by a catalogue
of coincidences), such a description would be cumbersome. In
%% -----File: 160.png---Folio 150-------
practice, physics describes the relations of the events to our
mesh-system; and all the terms of elementary physics and of
daily life refer to this relative aspect of the world.

\Paragraph{4.} Quantities like length, duration, mass, force, etc.\ have no
absolute significance; their values will depend on the mesh-system
to which they are referred. When this fact is realised,
the results of modern experiments relating to changes of length
of rigid bodies are no longer paradoxical.

\Paragraph{5.} There is no fundamental mesh-system. In particular
problems, and more particularly in restricted regions, it may
be possible to choose a mesh-system which follows more or less
closely the lines of absolute structure in the world, and so
simplify the phenomena which are related to it. But the world-structure
is not of a kind which can be traced in an exact way
by mesh-systems, and in any large region the mesh-system
drawn must be considered arbitrary. In any case the systems
used in current physics are arbitrary.

\Paragraph{6.} The study of the absolute structure of the world is based
on the ``interval'' between two events close together, which is
an absolute attribute of the events independent of any mesh-system.
A world-geometry is constructed by adopting the
interval as the analogue of distance in ordinary geometry.%
\index{Interval}%

\Paragraph{7.} This world-geometry has a property unlike that of
Euclidean geometry in that the interval between two real
events may be real or imaginary. The necessity for a physical
distinction, corresponding to the mathematical distinction between
real and imaginary intervals, introduces us to the separation
of the four-dimensional order into time and space. But this
separation is not unique, and the separation commonly adopted
depends on the observer's track through the four-dimensional
world.%
\index{Imaginary intervals}%

\Paragraph{8.} The geodesic, or track of maximum or minimum interval-length
between two distant events, has an absolute significance.
And since no other kind of track can be defined absolutely, it
is concluded that the tracks of freely moving particles are
geodesics.%
\index{Geodesic!absolute significance of}%
\index{Geodesic!motion of particles in}%
\index{Interval-length!tracks of maximum}%

\Paragraph{9.} In Euclidean geometry the geodesics are straight lines. It
is evidently impossible to choose space and time-reckoning so
that all free particles in the solar system move in straight lines.
%% -----File: 161.png---Folio 151-------
Hence the geometry must be non-Euclidean in a field of gravitation.

\Paragraph{10.} Since the tracks of particles in a gravitational field are
evidently governed by some law, the possible geometries must
be limited to certain types.

\Paragraph{11.} The limitation concerns the absolute structure of the
world, and must be independent of the choice of mesh-system.
This narrows down the possible discriminating characters.
Practically the only reasonable suggestion is that the world
must (in empty space) be ``curved no higher than the first
degree''; and this is taken as the law of gravitation.

\Paragraph{12.} The simplest type of hummock with this limited curvature
has been investigated. It has a kind of infinite chimney at the
summit, which we must suppose cut out and filled up with a
region where this law is not obeyed, i.e.\ with a particle of matter.

\Paragraph{13.} The tracks of the geodesics on the hummock are such as
to give a very close accordance with the tracks computed by
Newton's law of gravitation. The slight differences from the
Newtonian law have been experimentally verified by the motion
of Mercury and the deflection of light.

\Paragraph{14.} The hummock might more properly be described as a
ridge extending linearly. Since the interval-length along it is
real or time-like, the ridge can be taken as a time-direction.
Matter has thus a continued existence in time. Further, in
order to conform with the law, a small ridge must always follow
a geodesic in the general field of space-time, confirming the conclusion
arrived at under~(8).

\Paragraph{15.} The laws of conservation of energy and momentum in
mechanics can be deduced from this law of world-curvature.

\Paragraph{16.} Certain phenomena such as the FitzGerald contraction
and the variation of mass with velocity, which were formerly
thought to depend on the behaviour of electrical forces concerned,
are now seen to be general consequences of the relativity
of knowledge. That is to say, length and mass being the relations
of some absolute thing to the observer's mesh-system, we can
foretell how these relations will be altered when referred to
another mesh-system.
%% -----File: 162.png---Folio 152-------


\Chapter{X}{Towards Infinity}

\Quote{W.~K. Clifford (1873).}
{The geometer of to-day knows nothing about the nature of actually existing
space at an infinite distance; he knows nothing about the properties of this
present space in a past or a future eternity. He knows, indeed, that the laws
assumed by Euclid are true with an accuracy that no direct experiment can
approach, not only in this place where we are, but in places at a distance from
us that no astronomer has conceived; but he knows this as of Here and Now;
beyond his range is a There and Then of which he knows nothing at present,
but may ultimately come to know more.}%
\index{Clifford}%


\First{The} great stumbling-block for a philosophy which denies
absolute space is the experimental detection of absolute rotation.
\index{Absolute rotation}%
\index{Relativity of rotation}%
\index{Rotation, absolute}%
The belief that the earth rotates on its axis was suggested by
the diurnal motions of the heavenly bodies; this observation is
essentially one of relative rotation, and, if the matter rested
there, no difficulty would be felt. But we can detect the same
rotation, or a rotation very closely equal to it, by methods
which do not seem to bring the heavenly bodies into consideration;
and such a rotation is apparently absolute. The planet
Jupiter is covered with cloud, so that an inhabitant would
probably be unaware of the existence of bodies outside; yet he
could quite well measure the rotation of Jupiter. By the gyrocompass
he would fix two points on the planet---the north and
south poles. Then by Foucault's experiment on the change of
the plane of motion of a freely suspended pendulum, he would
determine an angular velocity about the poles.
\index{Foucault's pendulum}%
\index{Gyrocompass} % [** PP: Removed hyphen for consistency]
Thus there is
certainly a definite physical constant, an angular velocity about
an axis, which has a fundamental importance for the inhabitants
of Jupiter; the only question is whether we are right in giving
it the name absolute rotation.

Contrast this with absolute translation. Here it is not a
question of giving the right name to a physical constant; the
inhabitants of Jupiter would find no constant to name. We see
at once that a relativity theory of translation is on a different
footing from a relativity theory of rotation. The duty of the
%% -----File: 163.png---Folio 153-------
former is to explain facts; the duty of the latter is to explain
away facts.

Our present theory seems to make a start at tackling this
problem, but gives it up. It permits the observer, if he wishes,
to consider the earth as non-rotating, but surrounded by a field
of centrifugal force; all the other bodies in the universe are then
revolving round the earth in orbits mainly controlled by this
field of centrifugal force. Astronomy on this basis is a little
cumbersome; but all the phenomena are explained perfectly.
The centrifugal force is part of the gravitational field, and obeys
Einstein's law of gravitation, so that the laws of nature are
completely satisfied by this representation. One awkward
question remains, What causes the centrifugal force? Certainly
not the earth which is here represented as non-rotating. As we
go further into space to look for a cause, the centrifugal force
becomes greater and greater, so that the more we defer the debt
the heavier the payment demanded in the end. Our present
theory is like the debtor who does not mind how big an obligation
accumulates satisfied that he can always put off the payment.
It chases the cause away to infinity, content that the laws of
nature---the relations between contiguous parts of the world---are
satisfied all the way.

One suggested loophole must be explored. Our new law of
gravitation admits that a rapid motion of the attracting body
will affect the field of force. If the earth is non-rotating, the
stars must be going round it with terrific speed. May they not
in virtue of their high velocities produce gravitationally a
sensible field of force on the earth, which we recognise as the
centrifugal field? This would be a genuine elimination of
absolute rotation, attributing all effects indifferently to the
rotation of the earth the stars being at rest, or to the revolution
of the stars the earth being at rest; nothing matters except the
relative rotation. I doubt whether anyone will persuade himself
that the stars have anything to do with the phenomenon. We
do not believe that if the heavenly bodies were all annihilated
it would upset the gyrocompass. In any case, precise calculation
shows that the centrifugal force could not be produced by the
motion of the stars, so far as they are known.%
\index{Centrifugal Force!not caused by stars}%

We are therefore forced to give up the idea that the signs of
%% -----File: 164.png---Folio 154-------
the earth's rotation---the protuberance of its equator, the
phenomena of the gyrocompass, etc.---are due to a rotation
relative to any matter we can recognise. The philosopher who
persists that a rotation which is not relative to matter is unthinkable,
will no doubt reply that the rotation must then be
relative to some matter which we have not yet recognised. We
have hitherto been greatly indebted to the suggestions of
philosophy in evolving this theory, because the suggestions
related to the things we know about; and, as it turned out, they
were confirmed by experiment. But as physicists we cannot
take the same interest in the new demand; we do not necessarily
challenge it, but it is outside our concern. Physics demands of
its scheme of nature something else besides truth, namely a
certain quality that we may call convergence. The law of
conservation of energy is only strictly true when the whole
universe is taken into account; but its value in physics lies in
the fact that it is \textit{approximately} true for a very limited system.
Physics is an exact science because the chief essentials of a
problem are limited to a few conditions; and it draws near to
the truth with ever-increasing approximation as it widens its
purview. The approximations of physics form a convergent
series.
\index{Convergence of physical approximations}%
History, on the other hand, is very often like a divergent
series; no approximation to its course is reached until the last
term of the infinite series has been included in the data of
prediction. Physics, if it wishes to retain its advantage, must
take its own course, formulating those laws which are approximately
true for the limited data of sense, and extending them
into the unknown. The relativity of rotation is not approximately
true for the data of sense, although it may possibly be
true when the unknown as well as the known are included.

The same considerations that apply to rotation apply to
acceleration, although the difficulty is less striking. We can if
we like attribute to the sun some arbitrary acceleration, balancing
it by introducing a uniform gravitational field. Owing to this
field the rest of the stars will move with the same acceleration
and no phenomena will be altered. But then it seems necessary
to find a cause for this field. It is not produced by the gravitation
of the stars. Our only course is to pursue the cause further and
further towards infinity; the further we put it away, the greater
%% -----File: 165.png---Folio 155-------
the mass of attracting matter needed to produce it. On the
other hand, the earth's absolute acceleration does not intrude
on our attention in the way that its absolute rotation does\footnotemark.
  \footnotetext{To determine even roughly the earth's absolute acceleration we should
  need a fairly full knowledge of the disturbing effects of all the matter in the
  universe. A similar knowledge would be required to determine the absolute
  rotation \textit{accurately}; but all the matter likely to exist would have so small an
  effect, that we can at once assume that the absolute rotation is very nearly
  the same as the experimentally determined rotation.}%
\index{Absolute acceleration}% [** PP: Index reads p. 154]


We are vaguely conscious of a difficulty in these results; but
if we examine it closely, the difficulty does not seem to be a
very serious one. The theory of relativity, as we have understood
it, asserts that our partitions of space and time are introduced
by the observer and are irrelevant to the laws of nature; and
therefore the current quantities of physics, length, duration,
mass, force, etc., which are relative to these partitions, are not
things having an absolute significance in nature. But we have
never denied that there are features of the world having an
absolute significance; in fact, we have spent much time in finding
such features. The geodesics or natural tracks have been shown
to have an absolute significance;
\index{Geodesic structure!absolute character of}%
\index{Structure, geodesic!absolute character of}%
and it is possible in a limited
region of the world to choose space and time partitions such
that all geodesics become approximately straight lines. We may
call this a ``natural'' frame for that region, although it is not
as a rule the space and time adopted in practice;
\index{Natural frame}%
it is for example
the space and time of the observers in the falling projectile, not
of Newton's super-observer. It is capable of absolute definition,
except that it is ambiguous in regard to uniform motion. Now
the rotation of the earth determined by Foucault's pendulum
experiment is the rotation referred to this natural frame. But
we must have misunderstood our own theory of relativity
altogether, if we think there is anything inadmissible in an
absolute rotation of such a kind.%
\index{Relativity of rotation}%

Material particles and geodesics are both features of the
absolute structure of the world; and a rotation relative to
geodesic structure does not seem to be on any different footing
from a velocity relative to matter. There is, however, the
striking feature that rotation seems to be relative not merely
to the local geodesic structure but to a generally accepted
universal frame; whereas it is necessary to specify precisely
%% -----File: 166.png---Folio 156-------
what matter a velocity is measured with respect to. This is
largely a question of how much accuracy is needed in specifying
velocities and rotations, respectively. If in stating the speed of
a $\beta$~particle we do not mind an error of $10,000$ kilometres a
second, we need not specify precisely what star or planet its
velocity is referred to. The moon's (local) angular velocity is
sometimes given to fourteen significant figures; I doubt if
any universal frame is well-defined enough for this accuracy.
There is no doubt much greater continuity in the geodesic
structure in different parts of the world than in the material
structure; but the difference is in degree rather than in
principle.

It is probable that here we part company from many of the
continental relativists, who give prominent place to a principle
known as the \textit{law of causality}---that only those things are to be
regarded as being in causal connection which are capable of
being actually observed.
\index{Causality, law of}%
This seems to be interpreted as placing
matter on a plane above geodesic structure in regard to the
formulation of physical laws, though it is not easy to see in
what sense a distribution of matter can be regarded as more
observable than the field of influence in surrounding space
which makes us aware of its existence. The principle itself is
debateable; that which is observable to us is determined by
the accident of our own structure, and the law of causality
seems to impose our own limitations on the free interplay of
entities in the world outside us. In this book the tradition of
Faraday and Maxwell still rules our outlook; and for us matter
and electricity are but incidental points of complexity, the
activity of nature being primarily in the so-called empty spaces
between.

The vague universal frame to which rotation is referred is
called the \textit{inertial frame}.
\index{Frame, inertial}%
\index{Inertial frame}%
It is definite in the flat space-time far
away from all matter. In the undulating country corresponding
to the stellar universe it is not a precise conception; it is rather
a rude outline, arbitrary within reasonable limits, but with the
general course indicated. The reason for the term inertial frame
is of interest. We can quite freely use a mesh-system deviating
widely from the inertial frame (e.g.\ rotating axes); but we have
seen that there is a postponed debt to pay in the shape of an
%% -----File: 167.png---Folio 157-------
apparently uncaused field of force. But is there no debt to pay,
even when the inertial frame is used? In that case there is no
gravitational or centrifugal force at infinity;
\index{Centrifugal Force!debt at infinity}%
\index{Geodesic!in regions at infinity}%
\index{Infinity, conditions at}%
\index{Structure, geodesic!behaviour at infinity}%
but there is still
inertia, which is of the same nature. The distinction between
force as requiring a cause and inertia as requiring no cause
cannot be sustained. We shall not become any more solvent by
commuting our debt into pure inertia. The debt is inevitable
whatever mesh-system is used; we are only allowed to choose
the form it shall take.

The debt after all is a very harmless one. At infinity we have
the absolute geodesics in space-time, and we have our own
arbitrarily drawn mesh-system. The relation of the geodesics
to the mesh-system decides whether our axes shall be termed
rotating or non-rotating; and ideally it is this relation that is
determined when a so-called absolute rotation is measured.
No one could reasonably expect that there would be no determinable
relation. On the other hand uniform translation does
not affect the relation of the geodesics to the mesh-system---if
they were straight lines originally, they remain straight lines---thus
uniform translation cannot be measured except relative to
matter.

We have been supposing that the conditions found in the
remotest parts of space accessible to observation can be extrapolated
to infinity; and that there are still definite natural
tracks in space-time far beyond the influence of matter. Feelings
of objection to this view arise in certain minds. It is urged that
as matter influences the course of geodesics it may well be
responsible for them altogether; so that a region outside the
field of action of matter could have no geodesics, and consequently
no intervals. All the potentials would then necessarily
be zero. Various modified forms of this objection arise; but the
main feeling seems to be that it is unsatisfactory to have certain
conditions prevailing in the world, which can be traced away to
infinity and so have, as it were, their source at infinity; and
there is a desire to find some explanation of the inertial frame
as built up through conditions at a finite distance.%
\index{Inertia!in regions at infinity}%

Now if all intervals vanished space-time would shrink to a
point. Then there would be no space, no time, no inertia, no
anything. Thus a cause which creates intervals and geodesics
%% -----File: 168.png---Folio 158-------
must, so to speak, extend the world. We can imagine the world
stretched out like a plane sheet; but then the stretching cause---the
cause of the intervals---is relegated beyond the bounds of
space and time, i.e.\ to infinity. This is the view objected to,
though the writer does not consider that the objection has
much force. An alternative way is to inflate the world from
inside, as a balloon is blown out. In this case the stretching
force is not relegated to infinity, and ruled outside the scope of
experiment; it is acting at every point of space and time, curving
the world to a sphere. We thus get the idea that space-time
may have an essential curvature on a great scale independent
of the small hummocks due to recognised matter.

It is not necessary to speculate whether the curvature is
produced (as in the balloon) by some pressure applied from a
fifth dimension. For us it will appear as an innate tendency of
four-dimensional space-time to curve. It may be asked, what
have we gained by substituting a natural curvature of space-time
for a natural stretched condition corresponding to the
inertial frame?
\index{Curvature!of space and time}%
As an explanation, nothing. But there is this
difference, that the theory of the inertial frame can now be
included in the differential law of gravitation instead of remaining
outside and additional to the law.

It will be remembered that one clue by which we previously
reached the law of gravitation was that flat space-time must be
compatible with it. But if space-time is to have a small natural
curvature independent of matter this condition is now altered.
It is not difficult to find the necessary alteration of the law\footnote{Appendix, \Noteref{14}.}.
\Pagelabel{note14}%
It will contain an additional, and at present unknown, constant,
which determines the size of the world.

Spherical space is not very easy to imagine. We have to
think of the properties of the surface of a sphere---the two-dimensional
case---and try to conceive something similar applied
to three-dimensional space. Stationing ourselves at a point let
us draw a series of spheres of successively greater radii. The
surface of a sphere of radius~$r$ should be proportional to~$r^{2}$; but
in spherical space the areas of the more distant spheres begin
to fall below the proper proportion. There is not so much room
out there as we expected to find. Ultimately we reach a sphere
%% -----File: 169.png---Folio 159-------
of biggest possible area, and beyond it the areas begin to decrease\footnotemark.
  \footnotetext{The area is, of course, to be determined by measurement of some kind.}%
The last sphere of all shrinks to a point---our antipodes.
Is there nothing beyond this? Is there a kind of boundary
there? There is nothing beyond and yet there is no boundary.
On the earth's surface there is nothing beyond our own antipodes
but there is no boundary there.

The difficulty is that we try to realise this spherical world by
imagining how it would appear to us and to our measurements.
There has been nothing in our experience to compare it with,
and it seems fantastic. But if we could get rid of the personal
point of view, and regard the sphericity of the world as a statement
of the type of order of events outside us, we should think
that it was a simple and natural order which is as likely as any
other to occur in the world.

In such a world there is no difficulty about accumulated debt
at the boundary. There is no boundary. The centrifugal force
increases until we reach the sphere of greatest area, and then,
still obeying the law of gravitation, diminishes to zero at the
antipodes. The debt has paid itself automatically.

We must not exaggerate what has been accomplished by this
modification of the theory. A new constant has been introduced
into the law of gravitation which gives the world a definite
extension. Previously there was nothing to fix the scale of the
world; it was simply given \textit{a~priori} that it was infinite. Granted
extension, so that the intervals are not invariably zero, we can
determine geodesics everywhere, and hence mark out the inertial
frame.

Spherical space-time, that is to say a four-dimensional continuum
of space and imaginary time forming the surface of a
sphere in five dimensions, has been investigated by Prof.\ de~Sitter.
\index{de Sitter}%
\index{Spherical space-time}%
If real time is used the world is spherical in its space
dimensions, but open towards plus and minus infinity in its
time dimension, like an hyperboloid. This happily relieves us
of the necessity of supposing that as we progress in time we
shall ultimately come back to the instant we started from!
\index{Retardation of time!in spherical world} % [** PP: Index reads p. 160]
History never repeats itself. But in the space dimensions we
should, if we went on, ultimately come back to the starting
point. This would have interesting physical results, and we
%% -----File: 170.png---Folio 160-------
shall see presently that Einstein has a theory of the world in
which the return can actually happen; but in de~Sitter's theory
it is rather an abstraction, because, as he says, ``all the paradoxical
phenomena can only happen after the end or before the
beginning of eternity.''

The reason is this. Owing to curvature in the time dimension,
as we examine the condition of things further and further from
our starting point, our time begins to run faster and faster, or
to put it another way natural phenomena and natural clocks
slow down. The condition becomes like that described in
Mr~H.~G. Wells's story ``The new accelerator.''

When we reach half-way to the antipodal point, time stands
still. Like the Mad Hatter's tea party, it is always 6~o'clock;
and nothing whatever can happen however long we wait. There
is no possibility of getting any further, because everything
including light has come to rest here. All that lies beyond is
for ever cut off from us by this barrier of time; and light can
never complete its voyage round the world.

That is what happens when the world is viewed from one
station; but if attracted by such a delightful prospect, we proceeded
to visit this scene of repose, we should be disappointed.
We should find nature there as active as ever. We thought time
was standing still, but it was really proceeding there at the
usual rate, as if in a fifth dimension of which we had no
cognisance. Casting an eye back on our old home we should see
that time apparently had stopped still there. Time in the two
places is proceeding in directions at right angles, so that the
progress of time at one point has no relation to the perception
of time at the other point. The reader will easily see that a being
confined to the surface of a sphere and not cognisant of a third
dimension, will, so to speak, lose one of his dimensions altogether
when he watches things occurring at a point $90�$ away. He
regains it if he visits the spot and so adapts himself to the two
dimensions which prevail there.

It might seem that this kind of fantastic world-building can
have little to do with practical problems. But that is not quite
certain. May we not be able actually to observe the slowing
down of natural phenomena at great distances from us? The
most remote objects known are the spiral nebulae, whose
%% -----File: 171.png---Folio 161-------
distances may perhaps be of the order a million light years.
If natural phenomena are slowed down there, the vibrations of
an atom are slower, and its characteristic spectral lines will
appear displaced to the red.
\index{Displacement of spectral lines!in nebulae}%
\index{Nebulae, atomic vibrations in}%
We should generally interpret this
as a Doppler effect, implying that the nebula is receding. The
motions in the line-of-sight of a number of nebulae have been
determined, chiefly by Prof.\ Slipher. The data are not so ample
as we should like; but there is no doubt that large receding
motions greatly preponderate.
\index{Receding velocities!of spiral nebulae}%
This may be a genuine phenomenon
in the evolution of the material universe; but it is also
possible that the interpretation of spectral displacement as a
receding velocity is erroneous; and the effect is really the slowing
of atomic vibrations predicted by de~Sitter's theory.

Prof.\ Einstein himself prefers a different theory of curved
space-time. His world is cylindrical---curved in the three space
dimensions and straight in the time dimension.
\index{Cylindrical world, Einstein's}%
Since time is no
longer curved, the slowing of phenomena at great distances
from the observer disappears, and with it the slight experimental
support given to the theory by the observations of spiral nebulae.
There is no longer a barrier of eternal rest, and a ray of light is
able to go round the world.

In various ways crude estimates of the size of the world both
on de~Sitter's and Einstein's hypotheses have been made; and
in both cases the radius is thought to be of the order $10^{13}$ times
the distance of the earth from the sun. A ray of light from the
sun would thus take about $1000$~million years to go round the
world; and after the journey the rays would converge again at
the starting point, and then diverge for the next circuit.
\index{Light!voyage round the world}%
The
convergent would have all the characteristics of a real sun so
far as light and heat are concerned, only there would be no
substantial body present. Thus corresponding to the sun we
might see a series of ghosts occupying the positions where the
sun was $1000$, $2000$, $3000$, etc., million years ago, if (as seems
probable) the sun has been luminous for so long.%
\index{Ghosts of stars}%

It is rather a pleasing speculation that records of the previous
states of the sidereal universe may be automatically reforming
themselves on the original sites. Perhaps one or more of the
many spiral nebulae are really phantoms of our own stellar
system. Or it may be that only a proportion of the stars are
%% -----File: 172.png---Folio 162-------
substantial bodies; the remainder are optical ghosts revisiting
their old haunts. It is, however, unlikely that the light rays
after their long journey would converge with the accuracy which
this theory would require. The minute deflections by the various
gravitational fields encountered on the way would turn them
aside, and the focus would be blurred. Moreover there is a
likelihood that the light would gradually be absorbed or
scattered by matter diffused in space, which is encountered on
the long journey.

It is sometimes suggested that the return of the light-wave
to its starting point can most easily be regarded as due to the
force of gravitation, there being sufficient mass distributed
through the universe to control its path in a closed orbit. We
should have no objection in principle to this way of looking at
it; but we doubt whether it is correct in fact. It is quite possible
for light to return to its starting point in a world without
gravitation. We can roll flat space-time into a cylinder and join
the edges; its geometry will still be Euclidean and there will be
no gravitation; but a ray of light can go right round the cylinder
and return to the starting point in space. Similarly in Einstein's
more complex type of cylinder (three dimensions curved and
one dimension linear), it seems likely that the return of the
light is due as much to the connectivity of his space, as to
the non-Euclidean properties which express the gravitational
field.

For Einstein's cylindrical world it is necessary to postulate
the existence of vast quantities of matter (not needed on de~Sitter's
theory) far in excess of what has been revealed by our
telescopes. This additional material may either be in the form
of distant stars and galaxies beyond our limits of vision, or it
may be uniformly spread through space and escape notice by
its low density. There is a definite relation between the average
density of matter and the radius of the world; the greater the
radius the smaller must be the average density.

Two objections to this theory may be urged. In the first
place, absolute space and time are restored for phenomena on
a cosmical scale. The ghost of a star appears at the spot where
the star was a certain number of million years ago; and from
the ghost to the present position of the star is a definite distance%
%% -----File: 173.png---Folio 163-------
---the absolute motion of the star in the meantime\footnotemark.
  \footnotetext{The ghost is not formed where the star is now. If two stars were near
  together when the light left them their ghosts must be near together, although
  the stars may now be widely separated.}%
The world
taken as a whole has one direction in which it is not curved;
that direction gives a kind of absolute time distinct from space.
\index{Absolute time, in cylindrical world}%
Relativity is reduced to a local phenomenon; and although this
is quite sufficient for the theory hitherto described, we are
inclined to look on the limitation rather grudgingly. But we
have already urged that the relativity theory is not concerned
to deny the possibility of an absolute time, but to deny that it
is concerned in any experimental knowledge yet found; and it
need not perturb us if the conception of absolute time turns up
in a new form in a theory of phenomena on a cosmical scale,
as to which no experimental knowledge is yet available. Just
as each limited observer has his own particular separation of
space and time, so a being coextensive with the world might
well have a special separation of space and time natural to him.
It is the time for this being that is here dignified by the title
``absolute.''%
\index{Time!absolute}%

Secondly, the revised law of gravitation involves a new
constant which depends on the total amount of matter in the
world; or conversely the total amount of matter in the world
is determined by the law of gravitation. This seems very hard
to accept---at any rate without some plausible explanation of
how the adjustment is brought about. We can see that, the
constant in the law of gravitation being fixed, there may be
some upper limit to the amount of matter possible; as more
and more matter is added in the distant parts, space curves
round and ultimately closes; the process of adding more matter
must stop, because there is no more space, and we can only
return to the region already dealt with. But there seems nothing
to prevent a defect of matter, leaving space unclosed. Some
mechanism seems to be needed, whereby either gravitation
creates matter, or all the matter in the universe conspires to
define a law of gravitation.

Although this appears to the writer rather bewildering, it is
welcomed by those philosophers who follow the lead of Mach.
For it leads to the result that the extension of space and time
%% -----File: 174.png---Folio 164-------
depends on the amount of matter in the world---partly by its
direct effect on the curvature and partly by its influence on the
constant of the law of gravitation. The more matter there is,
the more space is created to contain it, and if there were no
matter the world would shrink to a point.

In the philosophy of Mach a world without \textit{matter} is unthinkable.
Matter in Mach's philosophy is not merely required as
a test body to display properties of something already there,
which have no physical meaning except in relation to matter;
it is an essential factor in causing those properties which it is
able to display. Inertia, for example, would not appear by the
insertion of one test body in the world; in some way the presence
of other matter is a necessary condition. It will be seen how
welcome to such a philosophy is the theory that space and the
inertial frame come into being with matter, and grow as it grows.
Since the laws of inertia are part of the law of gravitation,
Mach's philosophy was summed up---perhaps unconsciously---in
the profound saying ``If there were no matter in the universe,
the law of gravitation would fall to the ground.''%
\index{Inertia!Mach's views}%
\index{Mach's philosophy}%

No doubt a world without matter, in which nothing could
ever happen, would be very uninteresting; and some might deny
its claim to be regarded as a world at all. But a world uniformly
filled with matter would be equally dull and unprofitable; so
there seems to be little object in denying the possibility of the
former and leaving the latter possible.

The position can be summed up as follows:---in a space
without absolute features, an absolute rotation would be as
meaningless as an absolute translation;
\index{Absolute rotation}%
\index{Rotation, absolute}%
accordingly, the existence
of an experimentally determined quantity generally
identified with absolute rotation requires explanation. It was
remarked on \Pageref{41} that it would be difficult to devise a plan
of the world according to which uniform motion has no significance
but non-uniform motion is significant; but such a world has
been arrived at---a plenum, of which the absolute features are
intervals and geodesics.
\index{Aether!a plenum with geodesic structure}%
\index{Geodesic structure!absolute character of}%
\index{Structure, geodesic!absolute character of}%
In a limited region this plenum gives
a natural frame with respect to which an acceleration or rotation
(but not a velocity) capable of absolute definition can be
measured. In the case of rotation the local distortions of the
frame are of comparatively little account; and this explains
%% -----File: 175.png---Folio 165-------
why in practice rotation appears to have reference to some world-wide % [** PP: Hyphenated across a line in original]
inertial frame.

Thus absolute rotation does not indicate any logical flaw in
the theory hitherto developed; and there is no need to accept
any modification of our views. Possibly there may be a still
wider relativity theory, in which our supposed plenum is to be
regarded as itself an abstraction of the relations of the matter
distributed throughout the world, and not existent apart from
such matter. This seems to exalt matter rather unnecessarily.
It may be true; but we feel no necessity for it, unless experiment
points that way. It is with some such underlying idea that
Einstein's cylindrical space-time was suggested, since this
cannot exist without matter to keep it stretched. Now we freely
admit that our assumption of perfect flatness in the remote
parts of space was arbitrary, and there is no justification for
insisting on it. A small curvature is possible both conceptually
and experimentally. The arguments on both sides have hitherto
been little more than prejudices, which would be dissipated by
any experimental or theoretical lead in one direction. Weyl's
theory of the electromagnetic field, discussed in the next
chapter, assigns a definite function to the curvature of space;
and this considerably alters the aspect of the question. We are
scarcely sufficiently advanced to offer a final opinion; but the
conception of cylindrical space-time seems to be favoured by
this new development of the theory.

Some may be inclined to challenge the right of the Einstein
theory, at least as interpreted in this book, to be called a
relativity theory. Perhaps it has not all the characteristics
which have at one time or another been associated with that
name; but the reader, who has followed us so far, will see how
our search for an absolute world has been guided by a recognition
of the relativity of the measurements of physics. It may be
urged that our geodesics ought not to be regarded as fundamental;
a geodesic has no meaning in itself; what we are really concerned
with is the relation of a particle following a geodesic to all the
other matter of the world and the geodesic cannot be thought of
apart from such other matter. We would reply, ``Your particle
of matter is not fundamental; it has no meaning in itself; what
you are really concerned with is its `field'---the relation of the
%% -----File: 176.png---Folio 166-------
geodesics about it to the other geodesics in the world---and
matter cannot be thought of apart from its field.'' It is all
a tangle of relations; physical theory starts with the simplest
constituents, philosophical theory with the most familiar constituents.
They may reach the same goal; but their methods
are often incompatible.
%% -----File: 177.png---Folio 167-------


\Chapter{XI}{Electricity and Gravitation}

% [** PP: Explicit formatting containg hard-coded dimensions]
\noindent\begin{minipage}{\textwidth}
\small
Thou shalt not have in thy bag divers weights, a great and a small.

Thou shalt not have in thine house divers measures, a great and a small.

\hangindent 2em
But thou shalt have a perfect and just weight, a perfect and just measure shalt thou have.\hfill\hbox{\textit{Book of Deuteronomy.}\hspace*{\QIndent}}
\end{minipage}%
\index{Electricity and gravitation}%


\First{The} relativity theory deduces from geometrical principles the
existence of gravitation and the laws of mechanics of matter.
Mechanics is derived from geometry, not by \textit{adding} arbitrary
hypotheses, but by \textit{removing} unnecessary assumptions, so that
a geometer like Riemann might almost have foreseen the more
important features of the actual world. But nature has in
reserve one great surprise---electricity.%
\index{Riemann}%

Electrical phenomena are not in any way a misfit in the
relativity theory, and historically it is through them that it has
been developed. Yet we cannot rest satisfied until a deeper
unity between the gravitational and electrical properties of the
world is apparent. The electron, which seems to be the smallest
particle of matter, is a singularity in the gravitational field and
also a singularity in the electrical field.
\index{Electron!singularity in field}%
How can these two facts
be connected? The gravitational field is the expression of some
state of the world, which also manifests itself in the natural
geometry determined with measuring appliances; the electric
field must also express some state of the world, but we have not
as yet connected it with natural geometry. May there not still
be unnecessary assumptions to be removed, so that a yet more
comprehensive geometry can be found, in which gravitational
and electrical fields both have their place?

There \textit{is} an arbitrary assumption in our geometry up to this
point, which it is desirable now to point out. We have based
everything on the ``interval,'' which, it has been said, is something
which all observers, whatever their motion or whatever
their mesh-system, can measure absolutely, agreeing on the
result. This assumes that they are provided with identical
standards of measurement---scales and clocks. But if $A$ is in
%% -----File: 178.png---Folio 168-------
motion relative to $B$ and wishes to hand his standards to $B$ to
check his measures, he must stop their motion; this means in
practice that he must bombard his standards with material
molecules until they come to rest. Is it fair to assume that no
alteration of the standard is caused by this process? Or if $A$
measures time by the vibrations of a hydrogen atom, and space
by the wave-length of the vibration, still it is necessary to stop
the atom by a collision in which electrical forces are involved? % [** PP: Changed . to ?]

The standard of length in physics is the length in the year
1799 of a bar deposited at Paris.
\index{Standard metre, comparison with}%
Obviously no interval is ever
compared directly with that length; there must be a continuous
chain of intermediate steps extending like a geodetic triangulation
through space and time, first along the past history of the
scale actually used, then through intermediate standards, and
finally along the history of the Paris metre itself. It may be
that these intermediate steps are of no importance---that the
same result is reached by whatever route we approach the
standard; but clearly we ought not to make that assumption
without due consideration. We ought to construct our geometry
in such a way as to show that there are intermediate steps, and
that the comparison of the interval with the ultimate standard
is not a kind of action at a distance.

To compare intervals in different directions at a point in
space and time does not require this comparison with a distant
standard. The physicist's method of describing phenomena
near a point~$P$ is to lay down for comparison (1)~a mesh-system,
(2)~a unit of length (some kind of material standard), which can
also be used for measuring time, the velocity of light being unity.
With this system of reference he can measure in terms of his
unit small intervals $PP'$ running in any direction from~$P$,
summarising the results in the fundamental formula
\[
ds^{2} = g_{11}\, dx_{1}^{2}
       + g_{22}\, dx_{2}^{2} + \dotsb
      + 2g_{12}\, dx_{1}dx_{2} + \dotsb .
\]
If now he wishes to measure intervals near a distant point~$Q$, he
must lay down a mesh-system and a unit of measure there. He
naturally tries to simplify matters by using what he would call
the \textit{same} unit of measure at $P$ and~$Q$, either by transporting a
material rod or some equivalent device. If it is immaterial by
what route the unit is carried from $P$ to~$Q$, and replicas of the
%% -----File: 179.png---Folio 169-------
unit carried by different routes all agree on arrival at~$Q$, this
method is at any rate explicit. The question whether the unit
at~$Q$ defined in this way is \textit{really} the same as that at $P$ is mere
metaphysics. But if the units carried by different routes disagree,
there is no unambiguous means of identifying a unit at
$Q$ with the unit at~$P$. Suppose $P$ is an event at Cambridge on
March~1, and $Q$ at London on May~1; we are contemplating the
possibility that there will be a difference in the results of measures
made with our standard in London on May~1, according as the
standard is taken up to London on March~1 and remains there,
or is left at Cambridge and taken up on May~1. This seems at
first very improbable; but our reasons for allowing for this
possibility will appear presently. If there is this ambiguity the
only possible course is to lay down (1)~a mesh-system filling all
the space and time considered, (2)~a definite unit of interval, or
gauge, \textit{at every point of space and time}.
\index{Gauge-system}%
The geometry of the
world referred to such a system will be more complicated than
that of Riemann hitherto used; and we shall see that it is
necessary to specify not only the 10~$g$'s, but four other functions
of position, which will be found to have an important physical
meaning.%
\index{Geometry!non-Riemannian}%
\index{Non-Riemannian geometry}%

The observer will naturally simplify things by making the
units of gauge at different points as nearly as possible equal,
judged by ordinary comparisons. But the fact remains that,
when the comparison depends on the route taken, exact equality
is not definable; and we have therefore to admit that the \textit{exact}
standards are laid down at every point independently.

It is the same problem over again as occurs in regard to
mesh-systems. We lay down particular rectangular axes near
a point~$P$; presently we make some observations near a distant
point~$Q$. To what coordinates shall the latter be referred? The
natural answer is that we must use the same coordinates as we
were using at~$P$. But, except in the particular case of flat space,
there is no means of defining exactly what coordinates at $Q$ are
the \textit{same} as those at~$P$. In many cases the ambiguity may be
too trifling to trouble us; but in exact work the only course is
to lay down a definite mesh-system extending throughout space,
the precise route of the partitions being necessarily arbitrary.
We now find that we have to add to this by placing in each
%% -----File: 180.png---Folio 170-------
mesh a gauge whose precise length must be arbitrary. Having
done this the next step is to make measurements of intervals
(using our gauges). This connects the absolute properties of the
world with our arbitrarily drawn mesh-system and gauge-system.
And so by measurement we determine the $g$'s and the
new additional quantities, which determine the geometry of our
chosen system of reference, and at the same time contain within
themselves the absolute geometry of the world---the kind of
space-time which exists in the field of our experiments.

Having laid down a unit-gauge at every point, we can speak
quite definitely of the change in interval-length of a measuring-rod
moved from point to point, meaning, of course, the change
compared with the unit-gauges. Let us take a rod of interval-length
$l$ at~$P$, and move it successively through the displacements
$dx_{1}$, $dx_{2}$, $dx_{3}$, $dx_{4}$; and let the result be to increase its length
in terms of the gauges by the amount~$\lambda l$. The change depends
as much on the difference of the gauges at the two points as
on the behaviour of the rod; but there is no possibility of
separating the two factors. It is clear that $\lambda$ will not depend
on~$l$, because the change of length must be proportional to
the original length---unless indeed our whole idea of measurement
by comparison with a gauge is wrong\footnotemark.
  \footnotetext{We refuse to contemplate the idea that when the metre rod changes its
  length to two metres, each centimetre of it changes to three centimetres.}%
Further it will
not depend on the direction of the rod either in its initial or
final positions because the interval-length is independent of
direction. (Of course, the space-length would change, but that
is already taken care of by the~$g$'s.) $\lambda$~can thus only depend on
the displacements $dx_{1}$, $dx_{2}$, $dx_{3}$, $dx_{4}$, and we may write it
\[
\lambda
  = \kappa_{1}\, dx_{1} + \kappa_{2}\, dx_{2}
  + \kappa_{3}\, dx_{3} + \kappa_{4}\, dx_{4},
\]
so long as the displacements are small. The coefficients $\kappa_{1}$, $\kappa_{2}$,
$\kappa_{3}$, $\kappa_{4}$ apply to the neighbourhood of~$P$, and will in general be
different in different parts of space.

This indeed assumes that the result is independent of the
order of the displacements $dx_{1}$, $dx_{2}$, $dx_{3}$, $dx_{4}$---that is to say
that the ambiguity of the comparison by different routes disappears
in the limit when the whole route is sufficiently small.
It is parallel with our previous implicit assumption that although
the length of the track from a point~$P$ to a distant point~$Q$
%% -----File: 181.png---Folio 171-------
depends on the route, and no definite meaning can be attached
to the interval between them without specifying a route, yet in
the limit there is a definite small interval between $P$ and~$Q$ when
they are sufficiently close together.

To understand the meaning of these new coefficients $\kappa$ let us
briefly recapitulate what we understand by the~$g$'s. Primarily
they are quantities derived from experimental measurements of
intervals, and describe the geometry of the space and time
partitions which the observer has chosen. As consequential
properties they describe the field of force, gravitational, centrifugal,
etc., with which he perceives himself surrounded. They
relate to the particular mesh-system of the observer; and by
altering his mesh-system, he can alter their values, though not
entirely at will. From their values can be deduced intrinsic
properties of the world---the \textit{kind} of space-time in which the
phenomena occur. Further they satisfy a definite condition---the
law of gravitation---so that not all mathematically possible
space-times and not all arbitrary values of the $g$'s are such as
can occur in nature.%
\index{Real world of physics}%

All this applies equally to the~$\kappa$'s, if we substitute gauge-system
for mesh-system, and some at present unknown force
for gravitation. They can theoretically be determined by
interval-measurement; but they will be more conspicuously
manifested to the observer through their consequential property
of describing some kind of field of force surrounding him. The
$\kappa$'s refer to the arbitrary gauge-system of the observer; but he
cannot by altering his gauge-system alter their values entirely
at will. Intrinsic properties of the world are contained in their
values, unaffected by any change of gauge-system. Further we
may expect that they will have to satisfy some law corresponding
to the law of gravitation, so that not all arbitrary values of the
$\kappa$'s are such as can occur in nature.

It is evident that the $\kappa$'s must refer to some type of phenomenon
which has not hitherto appeared in our discussion; and
the obvious suggestion is that they refer to the electromagnetic
field.
\index{Fields of force!electromagnetic}%
This hypothesis is strengthened when we recall that the
electromagnetic field is, in fact, specified at every point by the
values of four quantities, viz.\ the three components of electromagnetic
vector potential, and the scalar potential of electrostatics.
%% -----File: 182.png---Folio 172-------
Surely it is more than a coincidence that the physicist
needs just four more quantities to specify the state of the world
at a point in space, and four more quantities are provided by
removing a rather illogical restriction on our system of geometry
of natural measures.

[The general reader will perhaps pardon a few words addressed
especially to the mathematical physicist. Taking the ordinary
unaccelerated rectangular coordinates $x$, $y$, $z$, $t$, let us write
$F$, $G$, $H$, $-\Phi$ for $\kappa_{1}$, $\kappa_{2}$, $\kappa_{3}$, $\kappa_{4}$, then
\[
\frac{dl}{l} = \lambda = F\, dx + G\, dy + H\, dx - \Phi\, dt.
\]
From which, by integration,
\[
\log l + \text{const.} = \int(F\, dx + G\, dy + H\, dz - \Phi\, dt).
\]

The length $l$ will be independent of the route taken if
\[
F\, dx + G\, dy + H\, dz - \Phi\, dt
\]
is a perfect differential. The condition for this is
\index{Electromagnetic potentials and forces}%
\index{Force!electromagnetic}%
\index{Potentials, electromagnetic}%
\begin{align*}
\frac{\partial H}{\partial y} - \frac{\partial G}{\partial z} &= 0, &
\frac{\partial F}{\partial z} - \frac{\partial H}{\partial x} &= 0, &
\frac{\partial G}{\partial x} - \frac{\partial F}{\partial y} &= 0, \\
%
-\frac{\partial \Phi}{\partial x} - \frac{\partial F}{\partial t} &= 0, &
-\frac{\partial \Phi}{\partial y} - \frac{\partial G}{\partial t} &= 0, &
-\frac{\partial \Phi}{\partial z} - \frac{\partial H}{\partial t} &= 0.
\end{align*}
If $F$, $G$, $H$, $\Phi$ are the potentials of electromagnetic theory, these
are precisely the expressions for the three components of
magnetic force and the three components of electric force, given
in the text-books. Thus the condition that distant intervals can
be compared directly without specifying a particular route of
comparison is that the electric and magnetic forces are zero in
the intervening space and time.

It may be noted that, even when the coordinate system has
been defined, the electromagnetic potentials are not unique in
value; but arbitrary additions can be made provided these
additions form a perfect differential. It is just this flexibility
which in our geometrical theory appears in the form of the
arbitrary choice of gauge-system. The electromagnetic \textit{forces}
on the other hand are independent of the gauge-system, which
is eliminated by ``curling.'']

It thus appears that the four new quantities appearing in our
extended geometry may actually be the four potentials of
%% -----File: 183.png---Folio 173-------
electromagnetic theory; and further, when there is no electromagnetic
field our previous geometry is valid. But in the more
general case we have to adopt the more general geometry in
which there appear fourteen coefficients, ten describing the
gravitational and four the electrical conditions of the world.

We ought now to seek the law of the electromagnetic field
on the same lines as we sought for the law of gravitation, laying
down the condition that it must be independent of mesh-system
and gauge-system since it seeks to limit the possible kinds of
world which can exist in nature. Happily this presents no
difficulty, because the law expressed by Maxwell's equations,
and universally adopted, fulfils the conditions. There is no
need to modify it fundamentally as we modified the law of
gravitation. We do, however, generalise it so that it applies
when a gravitational field is present at the same time---not
merely, as given by Maxwell, for flat space-time. The deflection
of electromagnetic waves (light) by a gravitational field is duly
contained in this generalised law.

Strictly speaking the laws of gravitation and of the electromagnetic
field are not two laws but one law, as the geometry
of the $g$'s and the $\kappa$'s is one geometry. Although it is often
convenient to separate them, they are really parts of the general
condition limiting the possible kinds of metric that can occur in
empty space.

It will be remembered that the four-fold arbitrariness of our
mesh-system involved four identities, which were found to
express the conservation of energy and momentum. In the new
geometry there is a fifth arbitrariness, namely that of the selected
gauge-system. This must also give rise to an identity; and it is
found that the new identity expresses the law of conservation of
electric charge.%
\index{Conservation!of electric charge}%

A grasp of the new geometry may perhaps be assisted by a
further comparison. Suppose an observer has laid down a line
of a certain length and in a certain direction at a point~$P$, and
he wishes to lay down an exactly similar line at a distant point~$Q$.
\index{Integrability of length and direction} % [** PP: Index reads p. 174]
If he is in flat space there will be no difficulty; he will have
to proceed by steps, a kind of triangulation, but the route chosen
is of no importance. We know definitely that there is just one
direction at $Q$ parallel to the original direction at~$P$; and it is
%% -----File: 184.png---Folio 174-------
in ordinary geometry supposed that the length is equally
determinate. But if space is not flat the case is different.
Imagine a two-dimensional observer confined to the curved
surface of the earth trying to perform this task. As he does not
appreciate the third dimension he will not immediately perceive
the impossibility; but he will find that the direction which he has
transferred to $Q$ differs according to the route chosen. Or if he
went round a complete circuit he would find on arriving back
at $P$ that the direction he had so carefully tried to preserve on
the journey did not agree with that originally drawn\footnotemark.
  \footnotetext{It might be thought that if the observer preserved mentally the original
  direction in three-dimensional space, and obtained the direction at any point
  in the two-dimensional space by projecting it, there would be no ambiguity.
  But the three-dimensional space in which a curved two-dimensional space is
  conceived to exist is quite arbitrary. A two-dimensional observer cannot
  ascertain by any observation whether he is on a plane or a cylinder, a sphere
  or any other convex surface of the same total curvature.}%
We
describe this by saying that in curved space, direction is not
integrable; and it is this non-integrability of direction which
characterises the gravitational field. In the case considered the
length would be preserved throughout the circuit; but it is
possible to conceive a more general kind of space in which the
length which it was attempted to preserve throughout the
circuit, as well as the direction, disagreed on return to the starting
point with that originally drawn. In that case length is not
integrable; and the non-integrability of length characterises the
electromagnetic field. Length associated with direction is called
a vector; and the combined gravitational and electric field
describe that influence of the world on our measurements by
which a vector carried by physical measurement round a closed
circuit changes insensibly into a different vector.

The welding together of electricity and gravitation into one
geometry is the work of Prof.\ H.~Weyl, first published in 1918\footnote%
{Appendix, \Noteref{15}.}.
\Pagelabel{note15}%
\index{Vector, non-integrable on Weyl's theory}%
\index{Weyl}%
It appears to the writer to carry conviction, although up to the
present no experimental test has been proposed. It need scarcely
be said that the inconsistency of length for an ordinary circuit
would be extremely minute\footnote{I do not think that any numerical estimate has been made.}, and the ordinary manifestations
of the electromagnetic field are the consequential results of
%% -----File: 185.png---Folio 175-------
changes which would be imperceptible to direct measurement.
It will be remembered that the gravitational field is likewise
perceived by the consequential effects, and not by direct interval-measurement.

But the theory does appear to require that, for example, the
time of vibration of an atom is not quite independent of its
previous history. It may be assumed that the previous histories
of terrestrial atoms are so much alike that there are no significant
differences in their periods. The possibility that the systematic
difference of history of solar and terrestrial atoms may have an
effect on the expected shift of the spectral lines on the sun has
already been alluded to. It seems doubtful, however, whether
the effect could attain the necessary magnitude.

It may seem difficult to identify these abstract geometrical
qualities of the world with the physical forces of electricity and
magnetism. How, for instance, can the change in the length of
a rod taken round a circuit in space and time be responsible for
the sensations of an electric shock? The geometrical potentials
($\kappa$) obey the recognised laws of electromagnetic potentials, and
each entity in the physical theory---charge, electric force,
magnetic element, light, etc.---has its exact analogue in the
geometrical theory; but is this formal correspondence a sufficient
ground for identification? The doubt which arises in our minds
is due to a failure to recognise the formalism of all physical
knowledge.
\index{Formalism of knowledge}%
The suggestion ``This is not the thing I am speaking
of, though it behaves exactly like it in all respects'' carries no
physical meaning. Anything which behaves exactly like
electricity must manifest itself to us as electricity. Distinction
of form is the only distinction that physics can recognise; and
distinction of individuality, if it has any meaning at all, has no
bearing on physical manifestations.

We can only explore the world with apparatus, which is itself
part of the world. Our idealised apparatus is reduced to a few
simple types---a neutral particle, a charged particle, a rigid
scale, etc. The absolute constituents of the world are related in
various ways, which we have studied, to the indications of these
test-bodies. The main features of the absolute world are so
simple that there is a redundancy of apparatus at our disposal;
and probably all that there is to be known could theoretically
%% -----File: 186.png---Folio 176-------
be found out by exploration with an uncharged particle. Actually
we prefer to look at the world as revealed by exploration with
scales and clocks---the former for measuring so-called imaginary
intervals, and the latter for real intervals; this gives us a unified
geometrical conception of the world.
\index{Geometrical conception of the world}%
Presumably, we could obtain
a unified mechanical conception by taking the moving uncharged
particle as standard indicator; or a unified electrical conception
by taking the charged particle. For particular purposes one
test-body is generally better adapted than others. The gravitational
field is more sensitively explored with a moving particle
than a scale. Although the electrical field can theoretically be
explored by the change of length of a scale taken round a circuit,
a far more sensitive way is to use a little bit of the scale---an
electron. And in general for practical efficiency, we do not use
any simple type of apparatus, but a complicated construction
built up with a view to a particular experiment. The reason for
emphasising the theoretical interchangeability of test-bodies is
that it brings out the unity and simplicity of the world; and for
that reason there is an importance in characterising the electromagnetic
condition of the world by reference to the indications
of a scale and clock, however inappropriate they may be as
practical test-bodies.

Weyl's theory opens up interesting avenues for development.
The details of the further steps involve difficult mathematics;
but a general outline is possible. As on Einstein's more limited
theory there is at any point an important property of the world
called the curvature; but on the new theory it is not an absolute
quantity in the strictest sense of the word.
\index{Curvature!on Weyl's theory}%
It is independent of
the observer's mesh-system, but it depends on his gauge. It is
obvious that the number expressing the radius of curvature of
the world at a point must depend on the unit of length; so we
cannot say that the curvatures at two points are absolutely
equal, because they depend on the gauges assigned at the two
points. Conversely the radius of curvature of the world provides
a natural and absolute gauge at every point;
\index{Natural gauge}%
and it will presumably
introduce the greatest possible symmetry into our laws
if the observer chooses this, or some definite fraction of it, as
his gauge. He, so to speak, forces the world to be spherical by
adopting at every point a unit of length which will make it so.
%% -----File: 187.png---Folio 177-------
Actual rods as they are moved about change their lengths compared
with this absolute unit according to the route taken, and
the differences correspond to the electromagnetic field. Einstein's
curved space appears in a perfectly natural manner in this
theory; no part of space-time is flat, even in the absence of
ordinary matter, for that would mean infinite radius of curvature,
and there would be no natural gauge to determine, for
example, the dimensions of an electron---the electron could not
know how large it ought to be, unless it had something to
measure itself against.%
\index{Electron!dimensions of}%
\index{Gauge!provided by radius of space}%

The connection between the form of the law of gravitation
and the total amount of matter in the world now appears less
mysterious. The curvature of space indirectly provides the
gauge which we use for measuring the amount of matter in the
world.

Since the curvature is not independent of the gauge, Weyl
does not identify it with the most fundamental quantity in
nature. There is, however, a slightly more complicated invariant
which is a pure number, and this is taken to be Action\footnotemark.
\footnotetext{Appendix, \Noteref{16}.}%
\Pagelabel{note16}%
\index{Action!on Weyl's theory}%
\index{Atomicity!of Action}%
We
can thus mark out a definite volume of space and time, and
say that the action within it is~$5$, without troubling to define
coordinates or the unit of measure! It might be expected that
the action represented by the number~$1$ would have specially
interesting properties; it might, for instance, be an atom of
action and indivisible. Experiment has isolated what are believed
to be units of action, which at least in many phenomena
behave as indivisible atoms called quanta;
\index{Quanta}%
but the theory, as
at present developed, does not permit us to represent the
quantum of action by the number~$1$. The quantum is a very
minute fraction of the absolute unit.

When we come across a pure number having some absolute
significance in the world it is natural to speculate on its possible
interpretation. It might represent a number of discrete entities;
but in that case it must necessarily be an integer, and it seems
clear that action can have fractional values. An angle is commonly
represented as a pure number, but it has not really this
character; an angle can only be measured in terms of a unit of
angle, just as a length is measured in terms of a unit of length.
%% -----File: 188.png---Folio 178-------
I can only think of one interpretation of a fractional number
which can have an absolute significance, though doubtless there
are others. The number may represent the \textit{probability} of something,
or some function of a probability.
\index{Probability, a pure number}%
The precise function
is easily found. We combine probabilities by multiplying, but
we combine the actions in two regions by adding; hence the
logarithm of a probability is indicated. Further, since the
logarithm of a probability is necessarily negative, we may
identify action provisionally with minus the logarithm of
the statistical probability of the state of the world which
exists.

The suggestion is particularly attractive because the Principle
of Least Action now becomes the Principle of Greatest Probability.
\index{Action, Principle of Least}%
\index{Principle of Least Action}%
The law of nature is that the actual state of the world is
that which is statistically most probable.

Weyl's theory also shows that the mass of a portion of matter
is necessarily positive; on the original theory no adequate reason
is given why negative matter should not exist. It is further
claimed that the theory shows to some extent why the world
is four-dimensional. To the mathematician it seems so easy to
generalise geometry to $n$~dimensions, that we naturally expect
a world of four dimensions to have an analogue in five dimensions.
Apparently this is not the case, and there are some essential
properties, without which it could scarcely be a world, which
exist only for four dimensions. Perhaps this may be compared
with the well-known difficulty of generalising the idea of a knot;
a knot can exist in space of any odd number of dimensions, but
not in space of an even number.

Finally the theory suggests a mode of attacking the problem
of how the electric charge of an electron is held together; at
least it gives an explanation of why the gravitational force is so
extremely weak compared with the electric force. It will be
remembered that associated with the mass of the sun is a certain
length, called the gravitational mass, which is equal to $1.5$~kilometres.
In the same way the gravitational mass or radius of an
electron is $7�10^{-56}$~cms.
\index{Electron, gravitational mass of}%
Its electrical properties are similarly
associated with a length $2�10^{-13}$~cms., which is called the electrical
radius. The latter is generally supposed to correspond to the
electron's actual dimensions. The theory suggests that the ratio
%% -----File: 189.png---Folio 179-------
of the gravitational to the electrical radius, $3�10^{42}$, ought to be
of the same order as the ratio of the latter to the radius of
curvature of the world. This would require the radius of space
to be of the order $6�10^{29}$~cms., or $2�10^{11}$ parsecs., which though
somewhat larger than the provisional estimates made by de~Sitter,
is within the realm of possibility.%
\index{de Sitter}%

%% -----File: 190.png---Folio 180-------


\Chapter{XII}{On the Nature of Things}

% [** PP: Explicit formatting containing hard-coded dimensions]]
{\small%
\settowidth{\TmpLen}{\textit{Hippolyta}:\ }
\hspace*{\QIndent}\makebox[\TmpLen][l]{\textit{Hippolyta}.} This is the silliest stuff that ever I heard. \\
\hspace*{\QIndent}\makebox[\TmpLen][l]{\textit{Theseus}.}
\begin{minipage}[t]{3.75in}
\hangindent 2em The best in this kind are but shadows; and the worst are
no worse, if imagination amend them.
\end{minipage} \\[1ex]
\null\hfill\textit{A Midsummer-Night's Dream.}}% End of \small
\medskip

\First{The} constructive results of the theory of relativity are based
on two principles which have been enunciated---the restricted
principle of relativity, and the principle of equivalence. These
may be summed up in the statement that uniform motion and
fields of force are purely relative. In their more formal enunciations
they are experimental generalisations, which can be
admitted or denied; if admitted, all the observational results
obtained by us can be deduced mathematically without any
reference to the views of space, time, or force, described in this
book. In many respects this is the most attractive aspect of
Einstein's work; it deduces a great number of remarkable
phenomena solely from two general principles, aided by a
mathematical calculus of great power; and it leaves aside as
irrelevant all questions of mechanism. But this mode of development
of the theory cannot be described in a non-technical book.

To avoid mathematical analysis we have had to resort to
geometrical illustrations, which run parallel with the mathematical
development and enable its processes to be understood
to some extent. The question arises, are these merely illustrations
of the mathematical argument, or illustrations of the actual
processes of nature. No doubt the safest course is to avoid the
thorny questions raised by the latter suggestion, and to say
that it is quite sufficient that the illustrations should correctly
replace the mathematical argument. But I think that this
would give a misleading view of what the theory of relativity
has accomplished in science.

The physicist, so long as he thinks as a physicist, has a definite
belief in a real world outside him. For instance, he believes that
atoms and molecules really exist; they are not mere inventions
%% -----File: 191.png---Folio 181-------
that enable him to grasp certain laws of chemical combination.
That suggestion might have sufficed in the early days of the
atomic theory; but now the existence of atoms as entities in the
real world of physics is fully demonstrated. This confident
assertion is not inconsistent with philosophic doubts as to the
meaning of ultimate reality.

When therefore we are asked whether the four-dimensional
world may not be regarded merely as an illustration of mathematical
processes, we must bear in mind that our questioner has
probably an ulterior motive.
\index{Four-dimensional space-time!reality of}%
He has already a belief in a real
world of three Euclidean dimensions, and he hopes to be allowed
to continue in this belief undisturbed. In that case our answer
must be definite; the real three-dimensional world is obsolete,
and must be replaced by the four-dimensional space-time with
non-Euclidean properties. In this book we have sometimes
employed illustrations which certainly do not correspond to any
physical reality---imaginary time, and an unperceived fifth
dimension.
\index{Imaginary time}%
But the four-dimensional world is no mere illustration;
it is the real world of physics, arrived at in the recognised
way by which physics has always (rightly or wrongly) sought for
reality.

I hold a certain object before me, and see an outline of the
figure of Britannia; another observer on the other side sees a
picture of a monarch; a third observer sees only a thin rectangle.
Am I to say that the figure of Britannia is the real object; and
that the crude impressions of the other observers must be
corrected to make allowance for their positions? All the appearances
can be accounted for if we are all looking at a three-dimensional
object---a penny---and no reasonable person can
doubt that the penny is the corresponding physical reality.
Similarly, an observer on the earth sees and measures an oblong
block; an observer on another star contemplating the same
object finds it to be a cube. Shall we say that the oblong block
is the real thing, and that the other observer must correct his
measures to make allowance for his motion? All the appearances
are accounted for if the real object is four-dimensional, and the
observers are merely measuring different three-dimensional
appearances or sections; and it seems impossible to doubt that
this is the true explanation. He who doubts the reality of the
%% -----File: 192.png---Folio 182-------
four-dimensional world (for logical, as distinct from experimental,
reasons) can only be compared to a man who doubts the
reality of the penny, and prefers to regard one of its innumerable
appearances as the real object.

Physical reality is the synthesis of all possible physical aspects
of nature.
\index{Synthesis of appearances}%
An illustration may be taken from the phenomena of
radiant-energy, or light. In a very large number of phenomena
the light coming from an atom appears to be a series of spreading
waves, extending so as to be capable of filling the largest
telescope yet made. In many other phenomena the light coming
from an atom appears to remain a minute bundle of energy, all
of which can enter and blow up a single atom.
\index{Quanta}%
There may be
some illusion in these experimental deductions; but if not, it
must be admitted that the physical reality corresponding to
light must be some synthesis comprehending both these appearances.
How to make this synthesis has hitherto baffled conception.
But the lesson is that a vast number of appearances
may be combined into one consistent whole---perhaps all
appearances that are directly perceived by terrestrial observers---and
yet the result may still be only an appearance. Reality
is only obtained when all conceivable points of view have been
combined.

That is why it has been necessary to give up the reality of
the everyday world of three dimensions. Until recently it comprised
all the possible appearances that had been considered.
But now it has been discovered that there are new points of
view with new appearances; and the reality must contain them
all. It is by bringing in all these new points of view that we
have been able to learn the nature of the real world of
physics.

Let us briefly recapitulate the steps of our synthesis. We
found one step already accomplished. The immediate perception
of the world with one eye is a two-dimensional appearance. But
we have two eyes, and these combine the appearances of the
world as seen from two positions; in some mysterious way the
brain makes the synthesis by suggesting solid relief, and we
obtain the familiar appearance of a three-dimensional world.
This suffices for all possible positions of the observer within the
parts of space hitherto explored. The next step was to combine
%% -----File: 193.png---Folio 183-------
the appearances for all possible states of uniform motion of the
observer. The result was to add another dimension to the world,
making it four-dimensional. Next the synthesis was extended
to include all possible variable motions of the observer. The
process of adding dimensions stopped, but the world became
non-Euclidean; a new geometry called Riemannian geometry
was adopted. Finally the points of view of observers varying
in size in any way were added; and the result was to replace
the Riemannian geometry by a still more general geometry
described in the last chapter.

The search for physical reality is not necessarily utilitarian,
but it has been by no means profitless. As the geometry became
more complex, the physics became simpler; until finally it
almost appears that the physics has been absorbed into the
geometry. We did not consciously set out to construct a
geometrical theory of the world; we were seeking physical
reality by approved methods, and this is what has happened.%
\index{Geometrical conception of the world}%

Is the point now reached the ultimate goal? Have the points
of view of all conceivable observers now been absorbed? We do
not assert that they have. But it seems as though a definite
task has been rounded off, and a natural halting-place reached.
So far as we know, the different possible impersonal points of
view have been exhausted---those for which the observer can be
regarded as a mechanical automaton, and can be replaced by
scientific measuring-appliances. A variety of more personal
points of view may indeed be needed for an ultimate reality;
but they can scarcely be incorporated in a real world of physics.
There is thus justification for stopping at this point but not for
stopping earlier.

It may be asked whether it is necessary to take into account
all conceivable observers, many of whom, we suspect, have no
existence. Is not the \textit{real} world that which comprehends the
appearances to all \textit{real} observers? Whether or not it is a tenable
hypothesis that that which no one observes does not exist,
science uncompromisingly rejects it. If we deny the rights of
extra-terrestrial observers, we must take the side of the Inquisition
against Galileo. And if extra-terrestrial observers are
admitted, the other observers, whose results are here combined,
cannot be excluded.

%% -----File: 194.png---Folio 184-------

Our inquiry into the nature of things is subject to certain
limitations which it is important to realise. The best comparison
I can offer is with a future antiquarian investigation, which may
be dated about the year 5000~\textsc{a.d.} An interesting find has been
made relating to a vanished civilisation which flourished about
the twentieth century, namely a volume containing a large
number of games of chess, written out in the obscure symbolism
usually adopted for that purpose.
\index{Chess, analogy of}%
The antiquarians, to whom
the game was hitherto unknown, manage to discover certain
uniformities; and by long research they at last succeed in
establishing beyond doubt the nature of the moves and rules of
the game. But it is obvious that no amount of study of the
volume will reveal the true nature either of the participants in
the game---the chessmen---or the field of the game---the chess-board.
With regard to the former, all that is possible is to give
arbitrary names distinguishing the chessmen according to their
properties; but with regard to the chess-board something more
can be stated. The material of the board is unknown, so too
are the shapes of the meshes---whether squares or diamonds;
but it is ascertainable that the different points of the field are
connected with one another by relations of two-dimensional
order, and a large number of hypothetical types of chess-board
satisfying these relations of order can be constructed. In
spite of these gaps in their knowledge, our antiquarians may
fairly claim that they thoroughly understand the game of
chess.%
\index{Ordering of events in external world}%

The application of this analogy is as follows. The recorded
games are our physical experiments. The rules of the game,
ascertained by study of them, are the laws of physics. The
hypothetical chess-board of 64~squares is the space and time of
some particular observer or player; whilst the more general
relations of two-fold order, are the absolute relations of order
in space-time which we have been studying. The chessmen are
the entities of physics---electrons, particles, or point-events; and
the range of movement may perhaps be compared to the fields
of relation radiating from them---electric and gravitational
fields, or intervals. By no amount of study of the experiments
can the absolute nature or appearance of these participants be
deduced; nor is this knowledge relevant, for without it we may
%% -----File: 195.png---Folio 185-------
yet learn ``the game'' in all its intricacy. Our knowledge of the
nature of things must be like the antiquarians' knowledge of
the nature of chessmen, viz.\ their nature as pawns and pieces
in the game, not as carved shapes of wood. In the latter aspect
they may have relations and significance transcending anything
dreamt of in physics.

It is believed that the familiar things of experience are very
complex; and the scientific method is to analyse them into
simpler elements. Theories and laws of behaviour of these
simpler constituents are studied; and from these it becomes
possible to predict and explain phenomena. It seems a natural
procedure to explain the complex in terms of the simple, but
it carries with it the necessity of explaining the familiar in terms
of the unfamiliar.

There are thus two reasons why the ultimate constituents of
the real world must be of an unfamiliar nature. Firstly, all
familiar objects are of a much too complex character. Secondly,
familiar objects belong not to the real world of physics, but to
a much earlier stage in the synthesis of appearances. The
ultimate elements in a theory of the world must be of a nature
impossible to define in terms recognisable to the mind.

The fact that he has to deal with entities of unknown nature
presents no difficulty to the mathematician. As the mathematician
in the Prologue explained, he is never so happy as
when he does not know what he is talking about. But we ourselves
cannot take any interest in the chain of reasoning he is
producing, unless we can give it some meaning---a meaning,
which we find by experiment, it will bear. We have to be in
a position to make a sort of running comment on his work.
At first his symbols bring no picture of anything before our
eyes, and we watch in silence. Presently we can say ``Now he
is talking about a particle of matter''\ldots ``Now he is talking
about another particle''\ldots ``Now he is saying where they will
be at a certain time of day''\ldots ``Now he says that they will be
in the same spot at a certain time.'' We watch to see.---``Yes.
The two particles have collided. For once he is speaking about
something familiar, and speaking the truth, although, of course,
he does not know it.'' Evidently his chain of symbols can be
interpreted as describing what occurs in the world; we need not,
%% -----File: 196.png---Folio 186-------
and do not, form any idea of the meaning of each individual
symbol; it is only certain elaborate combinations of them that
we recognise.

Thus, although the elementary concepts of the theory are of
undefined nature, at some later stage we must link the derivative
concepts to the familiar objects of experience.

We shall now collect the results arrived at in the previous
chapters by successive steps, and set the theory out in more
logical order. The extension in \Chapref{XI} will not be considered
here, partly because it would increase the difficulty of grasping
the main ideas, partly because it is less certainly established.

In the relativity theory of nature the most elementary concept
is the \textit{point-event}.
\index{Event, definition of}%
\index{Point-event}%
In ordinary language a point-event is an
instant of time at a point of space; but this is only one aspect
of the point-event, and it must not be taken as a definition.
Time and space---the familiar terms---are derived concepts to
be introduced much later in our theory. The first simple concepts
are necessarily undefinable, and their nature is beyond
human understanding. The aggregate of all the point-events is
called the \textit{world}.
\index{World}%
It is postulated that the world is four-dimensional,
which means that a particular point-event has to be
specified by the values of four variables or coordinates, though
there is entire freedom as to the way in which these four identifying
numbers are to be assigned.

The meaning of the statement that the world is four-dimensional
is not so clear as it appears at first. An aggregate of a
large number of things has in itself no particular number of
dimensions. Consider, for example, the words on this page. To
a casual glance they form a two-dimensional distribution; but
they were written in the hope that the reader would regard
them as a one-dimensional distribution. In order to define the
number of dimensions we have to postulate some ordering
relation; and the result depends entirely on what this ordering
relation is---whether the words are ordered according to sense
or to position on the page. Thus the statement that the world
is four-dimensional contains an implicit reference to some ordering
relation.
\index{Four-dimensional order}%
\index{Order and dimensions}%
This relation appears to be the \textit{interval}, though I am
not sure whether that alone suffices without some relation
corresponding to \textit{proximity}. It must be remembered that if the
%% -----File: 197.png---Folio 187-------
interval $s$ between two events is small, the events are not
necessarily near together in the ordinary sense.%
\index{Time!``standing still''}%

Between any two neighbouring point-events there exists a
certain relation known as the \textit{interval} between them.
\index{Interval}%
The
relation is a quantitative one which can be measured on a
definite scale of numerical values\footnotemark.
\index{Space-like intervals}%
\index{Time-like intervals}%
  \footnotetext{There is also a qualitative distinction into two kinds, ultimately identified
  as time-like and space-like, which for mathematical treatment are distinguished
  by real and imaginary numbers.}%
But the term ``interval''
is not to be taken as a guide to the real nature of the relation,
which is altogether beyond our conception.
\index{Imaginary intervals}%
Its geometrical
properties, which we have dwelt on so often in the previous
chapters, can only represent one aspect of the relation. It may
have other aspects associated with features of the world outside
the scope of physics. But in physics we are concerned not with
the nature of the relation but with the number assigned to
express its intensity; and this suggests a graphical representation,
leading to a geometrical theory of the world of physics.%
\index{World}%

What we have here called the \textit{world} might perhaps have been
legitimately called the \textit{aether}; at least it is the universal substratum
of things which the relativity theory gives us in place
of the aether.%
\index{Aether!identified with the ``world''}%

We have seen that the number expressing the intensity of
the interval-relation can be measured practically with scales and
clocks. Now, I think it is improbable that our coarse measures
can really get hold of the individual intervals of point-events;
our measures are not sufficiently microscopic for that. The
interval which has appeared in our analysis must be a \textit{macroscopic}
value; and the potentials and kinds of space deduced from
it are averaged properties of regions, perhaps small in comparison
even with the electron, but containing vast numbers of the
primitive intervals. We shall therefore pass at once to the
consideration of the macroscopic interval; but we shall not
forestall later results by assuming that it is measurable with
a scale and clock. That property must be introduced in its
logical order.%
\index{Macroscopic!interval}%

Consider a small portion of the world. It consists of a large
(possibly infinite) number of point-events between every two of
which an interval exists. If we are given the intervals between
%% -----File: 198.png---Folio 188-------
a point $A$ and a sufficient number of other points, and also
between $B$ and the same points, can we calculate what will be
the interval between $A$ and~$B$? In ordinary geometry this
would be possible; but, since in the present case we know nothing
of the relation signified by the word interval, it is impossible to
predict any law \textit{a~priori}. But we have found in our previous
work that there is such a rule, expressed by the formula
\[
ds^{2}
  = g_{11}\, dx_{1}^{2} + g_{22}\, dx_{2}^{2} + \dotsb
 + 2g_{12}\, dx_{1} dx_{2} + \dotsb.
\]
This means that, having assigned our identification numbers
% [** PP: overfull at 5in; set as individual entries]
$(x_{1}$, $x_{2}$, $x_{3}$, $x_{4})$ to the point-events, we have only to measure
ten different intervals to enable us to determine the ten coefficients,
$g_{11}$, etc., which in a small region may be considered to be
constants; then all other intervals in this region can be predicted
from the formula. For any other region we must make fresh
measures, and determine the coefficients for a new formula.

I think it is unlikely that the \textit{individual} interval-relations of
point-events follow any such definite rule. A microscopic
examination would probably show them as quite arbitrary, the
relations of so-called intermediate points being not necessarily
intermediate. Perhaps even the primitive interval is not
quantitative, but simply $1$ for certain pairs of point-events and
$0$ for others. The formula given is just an average summary
which suffices for our coarse methods of investigation, and holds
true only statistically. Just as statistical averages of one community
may differ from those of another, so may this statistical
formula for one region of the world differ from that of another.
This is the starting point of the infinite variety of nature.

Perhaps an example may make this clearer. Compare the
point-events to persons, and the intervals to the degree of
acquaintance between them. There is no means of forecasting
the degree of acquaintance between $A$ and $B$ from a knowledge
of the familiarity of both with $C$, $D$, $E$,~etc. But a statistician
may compute in any community a kind of average rule. In
most cases if $A$ and $B$ both know~$C$, it slightly increases the
probability of their knowing one another. A community in
which this correlation was very high would be described as
\textit{cliquish}.
\index{Cliquishness}%
There may be differences among communities in this
respect, corresponding to their degree of cliquishness; and so
%% -----File: 199.png---Folio 189-------
the statistical laws may be the means of expressing intrinsic
differences in communities.

Now comes the difficulty which is by this time familiar to us.
The ten $g$'s are concerned, not only with intrinsic properties of
the world, but with our arbitrary system of identification-numbers
for the point-events; or, as we have previously expressed
it, they describe not only the kind of space-time, but
the nature of the arbitrary mesh-system that is used. Mathematics
shows the way of steering through this difficulty by fixing
attention on expressions called tensors, of which $B^{\rho}_{\mu\nu\sigma}$ and $G_{\mu\nu}$
are examples.%
\index{Tensors}%

A tensor does not express explicitly the measure of an intrinsic
quality of the world, for some kind of mesh-system is essential
to the idea of measurement of a property, except in certain very
special cases where the property is expressed by a single number
termed an invariant, e.g.\ the interval, or the total curvature.
But to state that a tensor vanishes, or that it is equal to another
tensor in the same region, is a statement of intrinsic property,
quite independent of the mesh-system chosen. Thus by keeping
entirely to tensors, we contrive that there shall be behind our
formulae an undercurrent of information having reference to the
intrinsic state of the world.

In this way we have found two absolute formulae, which
appear to be fully confirmed by observation, namely
\begin{DPalign*}
\lintertext{in empty space,}
G_{\mu\nu} &= 0, \\
\lintertext{in space containing matter,}
G_{\mu\nu} &= K_{\mu\nu},
\end{DPalign*}
where $K_{\mu\nu}$ contains only physical quantities which are perfectly
familiar to us, viz.\ the density and state of motion of the matter
in the region.

I think the usual view of these equations would be that the
first expresses some law existing in the world, so that the point-events
by natural necessity tend to arrange their relations in
conformity with this equation. But when matter intrudes it
causes a disturbance or strain of the natural linkages; and a
rearrangement takes place to the extent indicated by the second
equation.

But let us examine more closely what the equation $G_{\mu\nu} = 0$
tells us. We have been giving the mathematician a free hand
%% -----File: 200.png---Folio 190-------
with his indefinable intervals and point-events. He has arrived
at the quantity~$G_{\mu\nu}$; but as yet this means to us---absolutely
nothing. The pure mathematician left to himself never ``deviates
into sense.'' His work can never relate to the familiar things
around us, unless we boldly lay hold of some of his symbols and
\textit{give} them an intelligible meaning---tentatively at first, and then
definitely as we find that they satisfy all experimental knowledge.
We have decided that in empty space $G_{\mu\nu}$~vanishes. Here
is our opportunity. In default of any other suggestion as to
what the vanishing of $G_{\mu\nu}$ might mean, let us say that the
vanishing of $G_{\mu\nu}$ \textit{means} emptiness; so that $G_{\mu\nu}$, if it does not
vanish, is a condition of the world which distinguishes space
said to be occupied from space said to be empty. Hitherto $G_{\mu\nu}$
was merely a formal outline to be filled with some undefined
contents; we are as far as ever from being able to explain what
those contents are; but we have now given a recognisable
meaning to the completed picture, so that we shall know it
when we come across it in the familiar world of experience.

The two equations are accordingly merely definitions---definitions
of the way in which certain states of the world
(described in terms of the indefinables) impress themselves on
our perceptions. When we perceive that a certain region of the
world is empty, that is merely the mode in which our senses
recognise that it is curved no higher than the first degree.
\index{Curvature!perception of}%
When we perceive that a region contains matter we are recognising
the intrinsic curvature of the world; and when we believe we
are measuring the mass and momentum of the matter (relative
to some axes of reference) we are measuring certain components
of world-curvature (referred to those axes). The statistical
averages of something unknown, which have been used to
describe the state of the world, vary from point to point; and
it is out of these that the mind has constructed the familiar
notions of matter and emptiness.%
\index{Emptiness, perception of}%
\index{Matter!perception of}%

The law of gravitation is not a law in the sense that it restricts
the possible behaviour of the substratum of the world; it is
merely the definition of a vacuum.
\index{Vacuum, defined by law of gravitation}%
We need not regard matter
as a foreign entity causing a disturbance in the gravitational
field; the disturbance is matter. In the same way we do not
regard light as an intruder in the electromagnetic field, causing
%% -----File: 201.png---Folio 191-------
the electromagnetic force to oscillate along its path; the oscillations
constitute the light. Nor is heat a fluid causing agitation
of the molecules of a body; the agitation is heat.

This view, that matter is a symptom and not a cause, seems
so natural that it is surprising that it should be obscured in the
usual presentation of the theory. The reason is that the connection
of mathematical analysis with the things of experience
is usually made, not by determining what matter is, but by
what certain combinations of matter do. Hence the interval is
at once identified with something familiar to experience, namely
the thing that a scale and a clock measure. However advantageous
that may be for the sake of bringing the theory into
touch with experiment at the outset, we can scarcely hope to
build up a theory of the nature of things if we take a scale and
clock as the simplest unanalysable concepts. The result of this
logical inversion is that by the time the equation $G_{\mu\nu} = K_{\mu\nu}$ is
encountered, both sides of the equation are well-defined
quantities. Their \textit{necessary} identity is overlooked, and the
equation is regarded as a new law of nature. This is the fault
of introducing the scale and clock prematurely. For our part
we prefer first to define what matter is in terms of the elementary
concepts of the theory; then we can introduce any kind of
scientific apparatus; and finally determine what property of the
world that apparatus will measure.%
\index{Clock-scale geometry, not fundamental}%

Matter defined in this way obeys all the laws of mechanics,
including conservation of energy and momentum. Proceeding
with a similar development of Weyl's more general theory of
the combined gravitational and electrical fields, we should find
that it has the familiar electrical and optical properties. It is
purely gratuitous to suppose that there is anything else present,
controlling but not to be identified with the relations of the
fourteen potentials ($g$'s and $\kappa$'s). % [** PP: Changed k to \kappa]

There is only one further requirement that can be demanded
from matter. Our brains are constituted of matter, and they
feel and think---or at least feeling and thinking are closely
associated with motions or changes of the matter of the brain.
\index{Brain, constitution of}%
It would be difficult to say that any hypothesis as to the nature
of matter makes this process less or more easily understood;
and a brain constituted out of differential coefficients of $g$'s can
%% -----File: 202.png---Folio 192-------
scarcely be said to be less adapted to the purposes of thought
than one made, say, out of tiny billiard balls! But I think we
may even go a little beyond this negative justification. The
primary interval relation is of an undefined nature, and the
$g$'s contain this undefinable element. The expression $G_{\mu\nu}$ is
therefore of defined \textit{form}, but of undefined \textit{content}.
\index{Content contrasted with structural form}%
\index{Form contrasted with content}%
By its form
alone it is fitted to account for all the physical properties of
matter; and physical investigation can never penetrate beneath
the form. The matter of the brain in its physical aspects is
merely the form; but the reality of the brain includes the
content. We cannot expect the form to explain the activities of
the content, any more than we can expect the number~4 to
explain the activities of the Big Four at Versailles.

Some of these views of matter were anticipated with marvellous
foresight by W.~K. Clifford forty years ago.
\index{Clifford}%
Whilst other English
physicists were distracted by vortex-atoms and other will-o'-the-wisps,
Clifford was convinced that matter and the motion
of matter were aspects of space-curvature \textit{and nothing more}.
And he was no less convinced that these geometrical notions
were only partial aspects of the relations of what he calls
``elements of feeling.''---%
\index{Feeling, elements of}%
\index{Matter!physical and psychological aspects}%
``The reality corresponding to our perception
of the motion of matter is an element of the complex
thing we call feeling. What we might perceive as a plexus of
nerve-disturbances is really in itself a feeling; and the succession
of feelings which constitutes a man's consciousness is the reality
which produces in our minds the perception of the motions of
his brain. These elements of feeling have relations of \textit{nextness}
or contiguity in space, which are exemplified by the sight-perceptions
of contiguous points; and relations of succession in
time which are exemplified by all perceptions. Out of these two
relations the future theorist has to build up the world as best
he may. Two things may perhaps help him. There are many
lines of mathematical thought which indicate that distance or
quantity may come to be expressed in terms of \textit{position} in the
wide sense of the \textit{analysis situs}. And the theory of space-curvature
hints at a possibility of describing matter and motion
in terms of extension only.'' (\textit{Fortnightly Review}, 1875.)

The equation $G_{\mu\nu} = K_{\mu\nu}$ is a kind of dictionary explaining
what the different components of world-curvature mean in
%% -----File: 203.png---Folio 193-------
terms ordinarily used in mechanics. If we write it in the slightly
modified, but equivalent, form
\[
G_{\mu\nu} - \tfrac{1}{2} g_{\mu\nu} G = - 8\pi T_{\mu\nu},
\]
we have the following scheme of interpretation
\index{Gravitation, Einstein's law of!macroscopic equations}%
\[
\begin{matrix}
T_{11}, & T_{12}, & T_{13}, & T_{14} \\
        & T_{22}, & T_{23}, & T_{24} \\
        &         & T_{33}, & T_{34} \\
        &         &         & T_{44}
\end{matrix}
=
\begin{matrix}
p_{11} + \rho u^{2}, & p_{12} + \rho uv,  & p_{13} + \rho uw,  & - \rho u, \\
                     & p_{22} + \rho v^2, & p_{23} + \rho vw,  & - \rho v, \\
                     &                    & p_{33} + \rho w^2, & - \rho w, \\
                     &                    &                    &   \rho.
\end{matrix}
\]
Here we are using the partitions of space and time adopted in
ordinary mechanics; $\rho$ is the density of the matter, $u$, $v$, $w$ its
component velocities, and $p_{11}$, $p_{12}$, $\dotsc p_{33}$, the components of
the internal stresses which are believed to be analysable into
molecular movements.%
\index{Stresses in continuous matter}%

Now the question arises, is it legitimate to make identifications
on such a wholesale scale? Having identified $T_{44}$ as density,
can we go on to identify another quantity $T_{34}$ as density
multiplied by velocity? It is as though we identified one ``thing''
as \textit{air}, and a quite different ``thing'' as \textit{wind}. Yes, it is legitimate,
because we have not hitherto explained what is to be the
counterpart of velocity in our scheme of the world; and this is
the way we choose to introduce it. All identifications are at
this stage provisional, being subject to subsequent test by
observation.

A definition of the velocity of matter in some such terms as
``\textit{wind} divided by \textit{air},'' does not correspond to the way in
which motion primarily manifests itself in our experience.
\index{Velocity, definition of}%
Motion is generally recognised by the disappearance of a particle
at one point of space and the appearance of an apparently
identical particle at a neighbouring point. This manifestation
of motion can be deduced mathematically from the identifying
definition here adopted. Remembering that in physical theory
it is necessary to proceed from the simple to the complex,
which is often opposed to the instinctive desire to proceed from
the familiar to the unfamiliar, this inversion of the order in
which the manifestations of motion appear need occasion no
surprise. Permanent identity of particles of matter (without
which the ordinary notion of velocity fails) is a very familiar
idea, but it appears to be a very complex feature of the world.%
\index{Identity, permanent} % [** PP: Added comma]
\index{Permanent identity}%

%% -----File: 204.png---Folio 194-------

A simple instance may be given where the familiar kinematical
conception of motion is insufficient.
\index{Motion!insufficiency of kinematical conception}%
Suppose a perfectly homogeneous
continuous ring is rotating like a wheel, what meaning
can we attach to its motion?
\index{Rotation of a continuous ring}%
The kinematical conception of
motion implies change---disappearance at one point and reappearance
at another point---but no change is detectable. The
state at any one moment is the same as at a previous moment,
and the matter occupying one position now is indistinguishable
from the matter in the same position a moment ago. At the
most it can only differ in a mysterious non-physical quality---that
of identity; but if, as most physicists are willing to believe,
matter is some state in the aether, what can we mean by saying
that two states are exactly alike, but are not identical? Is the
hotness of the room equal to, but not identical with, its hotness
yesterday? Considered kinematically, the rotation of the ring
appears to have no meaning; yet the revolving ring differs
mechanically from a stationary ring. For example, it has
gyrostatic properties. The fact that in nature a ring has atomic
and not continuous structure is scarcely relevant. A conception
of motion which affords a distinction between a rotating and
non-rotating continuous ring must be possible; otherwise this
would amount to an \textit{a~priori} proof that matter is atomic.
According to the conception now proposed, velocity of matter
is as much a static quality as density.
\index{Velocity!static character}%
Generally velocity is
accompanied by changes in the physical state of the world,
which afford the usual means of recognising its existence; but
the foregoing illustration shows that these symptoms do not
always occur.

This definition of velocity enables us to understand why
velocity except in reference to matter is meaningless, whereas
acceleration and rotation have a meaning.
\index{Absolute acceleration}%
\index{Absolute rotation}%
\index{Rotation, absolute}%
The philosophical
argument, that velocity through space is meaningless, ceases to
apply as soon as we admit any kind of structure or aether in
empty regions; consequently the problem is by no means so
simple as is often supposed. But our definition of velocity is
dynamical, not kinematical. Velocity is the ratio of certain
components of~$T_{\mu\nu}$, and only exists when $T_{44}$ is not zero. Thus
matter (or electromagnetic energy) is the only thing that can
have a velocity relative to the frame of reference. The velocity
%% -----File: 205.png---Folio 195-------
of the world-structure or aether, where the $T_{\mu\nu}$ vanish, is always
of the indeterminate form $0�0$. On the other hand acceleration
and rotation are defined by means of the $g_{\mu\nu}$ and exist wherever
these exist\footnotemark;
  \footnotetext{Even in Newtonian mechanics we speak of the ``field of acceleration,'' and
  think of it as existing even when there is no test body to display the acceleration.
  In the present theory this field of acceleration is described by the $g_{\mu\nu}$.
  There is no such thing as a ``field of velocity'' in empty space;
  \index{Field of velocity}%
  but there is in
  a material ocean.}%
so that the acceleration and rotation of the world-structure
or aether relative to the frame of reference are determinate.
Notice that acceleration is not defined as change of
velocity; it is an independent entity, much simpler and more
universal than velocity. It is from a comparison of these two
entities that we ultimately obtain the definition of time.

This finally resolves the difficulty encountered in \Chapref{X}---the
apparent difference in the Principle of Relativity as
applied to uniform and non-uniform motion. Fundamentally
velocity and acceleration are both static qualities of a region
of the world (referred to some mesh-system). Acceleration is a
comparatively simple quality present wherever there is geodesic
structure, that is to say everywhere.
\index{Acceleration!a simpler quality than velocity}%
\index{Geodesic structure!acceleration of}%
\index{Structure, geodesic!acceleration of}%
Velocity is a highly complex
quality existing only where the structure is itself more
than ordinarily complicated, viz.\ in matter. Both these qualities
commonly give physical manifestations, to which the terms
acceleration and velocity are more particularly applied; but it
is by examining their more fundamental meaning that we can
understand the universality of the one and the localisation of
the other.

It has been shown that there are four identical relations
between the ten qualities of a piece of matter here identified,
which depend solely on the way the $G_{\mu\nu}$ were by definition
constructed out of simpler elements. These four relations state
that, \textit{provided the mesh-system is drawn in one of a certain number
of ways}, mass (or energy) and momentum will be conserved.
The conservation of mass is of great importance;
\index{Mass!conservation of}%
\IndexExtra{Conservation of mass} %
matter will
be permanent, and for every particle disappearing at any point
a corresponding mass will appear at a neighbouring point; the
change consists in the displacement of matter, not its creation
or destruction. This gives matter the right to be regarded, not
as a mere assemblage of symbols, but as the substance of a
%% -----File: 206.png---Folio 196-------
permanent world. But the permanent world so found demands
the partitioning of space-time in one of a certain number of
ways, viz.\ those discussed in \Chapref{III}\footnotemark;
  \footnotetext{When the kind of space-time is such that a strict partition of this kind is
  impossible, strict conservation does not exist; but we retain the principle as
  formally satisfied by attributing energy and momentum to the gravitational
  field.}%
from these a particular
space and time are selected, because the observer wishes to
consider himself, or some arbitrary body, at rest. This gives
the space and time used for ordinary descriptions of experience.
In this way we are able to introduce perceptual space and time
into the four-dimensional world, as derived concepts depending
on our desire that the new-found matter should be permanent.

I think it is now possible to discern something of the reason
why the world must of necessity be as we have described it.
When the eye surveys the tossing waters of the ocean, the
eddying particles of water leave little impression; it is the waves
that strike the attention, because they have a certain degree
of permanence. The motion particularly noticed is the motion of
the wave-form, which is not a motion of the water at all. So
the mind surveying the world of point-events looks for the
permanent things. The simpler relations, the intervals and
potentials, are transient, and are not the stuff out of which
mind can build a habitation for itself. But the thing that has
been identified with matter is permanent, and because of its
permanence it must be for mind the substance of the world.
Practically no other choice was possible.

It must be recognised that the conservation of mass is not
exactly equivalent to the permanence of matter.
\index{Conservation!of mass}%
\index{Permanence of matter}%
If a loaf of
bread suddenly transforms into a cabbage, our surprise is not
diminished by the fact that there may have been no change of
weight. It is not very easy to define this extra element of
permanence required, because we accept as quite natural
apparently similar transformations---an egg into an omelette,
or radium into lead. But at least it seems clear that some degree
of permanence of one quality, mass, would be the primary
property looked for in matter, and this gives sufficient reason
for the particular choice.

We see now that the choice of a permanent substance for the
%% -----File: 207.png---Folio 197-------
world of perception necessarily carries with it the law of gravitation,
all the laws of mechanics, and the introduction of the
ordinary space and time of experience. Our whole theory has
really been a discussion of the most general way in which
permanent substance can be built up out of relations; and it is
the mind which, by insisting on regarding only the things that
are permanent, has actually imposed these laws on an indifferent
world. Nature has had very little to do with the matter; she
had to provide a basis---point-events; but practically anything
would do for that purpose if the relations were of a reasonable
degree of complexity. The relativity theory of physics reduces
everything to relations; that is to say, it is structure, not
material, which counts. The structure cannot be built up without
material; but the nature of the material is of no importance.
We may quote a passage from Bertrand Russell's \textit{Introduction
to Mathematical Philosophy}.%
\index{Mathematics, Russell's description of}%
\index{Russell}%
\index{Structure opposed to content}%

``There has been a great deal of speculation in traditional
philosophy which might have been avoided if the importance
of structure, and the difficulty of getting behind it, had been
realised. For example it is often said that space and time are
subjective, but they have objective counterparts; or that
phenomena are subjective, but are caused by things in themselves,
which must have differences \textit{inter~se} corresponding with
the differences in the phenomena to which they give rise. Where
such hypotheses are made, it is generally supposed that we can
know very little about the objective counterparts. In actual
fact, however, if the hypotheses as stated were correct, the
objective counterparts would form a world having the same
structure as the phenomenal world\ldots. In short, every proposition
having a communicable significance must be true of both worlds
or of neither: the only difference must lie in just that essence
of individuality which always eludes words and baffles description,
but which for that very reason is irrelevant to science.''

This is how our theory now stands.---We have a world of
point-events with their primary interval-relations. Out of these
an unlimited number of more complicated relations and qualities
can be built up mathematically, describing various features of
the state of the world. These exist in nature in the same sense
as an unlimited number of walks exist on an open moor. But
%% -----File: 208.png---Folio 198-------
the existence is, as it were, latent unless someone gives a significance
to the walk by following it; and in the same way the
existence of any one of these qualities of the world only acquires
significance above its fellows, if a mind singles it out for
recognition. Mind filters out matter from the meaningless
jumble of qualities, as the prism filters out the colours of the
rainbow from the chaotic pulsations of white light. Mind exalts
the permanent and ignores the transitory; and it appears from
the mathematical study of relations that the only way in which
mind can achieve her object is by picking out one particular
quality as the permanent substance of the perceptual world,
partitioning a perceptual time and space for it to be permanent
in, and, as a necessary consequence of this Hobson's choice, the
laws of gravitation and mechanics and geometry have to be
obeyed.
\index{Permanent perceptual world}%
Is it too much to say that mind's search for permanence
has created the world of physics? So that the world we
perceive around us could scarcely have been other than it is\footnotemark?
  \footnotetext{This summary is intended to indicate the direction in which the views
  suggested by the relativity theory appear to me to be tending, rather than to
  be a precise statement of what has been established. I am aware that there
  are at present many gaps in the argument. Indeed the whole of this part of
  the discussion should be regarded as suggestive rather than dogmatic.}

The last sentence possibly goes too far, but it illustrates the
direction in which these views are tending. With Weyl's more
general theory of interval-relations, the laws of electrodynamics
appear in like manner to depend merely on the identification
of another permanent thing---electric charge. In this case the
identification is due, not to the rudimentary instinct of the
savage or the animal, but the more developed reasoning-power
of the scientist. But the conclusion is that the whole of those
laws of nature which have been woven into a unified scheme---mechanics,
gravitation, electrodynamics and optics---have their
origin, not in any special mechanism of nature, but in the
workings of the mind.

``Give me matter and motion,'' said Descartes, ``and I will
construct the universe.'' The mind reverses this. ``Give me a
world---a world in which there are relations---and I will construct
matter and motion.''

Are there then no genuine laws in the external world? Laws
inherent in the substratum of events, which break through into
%% -----File: 209.png---Folio 199-------
the phenomena otherwise regulated by the despotism of the
mind? We cannot foretell what the final answer will be; but,
at present, we have to admit that there are laws which appear
to have their seat in external nature. The most important of
these, if not the only law, is a law of atomicity.
\index{Atomicity!law of}%
Why does that
quality of the world which distinguishes matter from emptiness
exist only in certain lumps called atoms or electrons, all of
comparable mass? Whence arises this discontinuity? At
present, there seems no ground for believing that discontinuity
is a law due to the mind; indeed the mind seems rather to take
pains to smooth the discontinuities of nature into continuous
perception. We can only suppose that there is something in
the nature of things that causes this aggregation into atoms.
Probably our analysis into point-events is not final; and if it
could be pushed further to reach something still more fundamental,
then atomicity and the remaining laws of physics would
be seen as identities. This indeed is the only kind of explanation
that a physicist could accept as ultimate. But this more ultimate
analysis stands on a different plane from that by which the
point-events were reached. The world \textit{may} be so constituted
that the laws of atomicity must necessarily hold; but, so far as
the mind is concerned, there seems no reason why it should
have been constituted in that way. We can conceive a world
constituted otherwise. But our argument hitherto has been
that, however the world is constituted, the necessary combinations
of things can be found which obey the laws of mechanics,
gravitation and electrodynamics, and these combinations are
ready to play the part of the world of perception for any mind
that is tuned to appreciate them; and further, any world of
perception of a different character would be rejected by the
mind as unsubstantial.

If atomicity depends on laws inherent in nature, it seems at
first difficult to understand why it should relate to matter
especially; since matter is not of any great account in the
analytical scheme, and owes its importance to irrelevant considerations
introduced by the mind. It has appeared, however,
that atomicity is by no means confined to matter and electricity;
the quantum, which plays so great a part in recent physics, is
apparently an atom of action.
\index{Action!atomicity of}%
So nature cannot be accused of
%% -----File: 210.png---Folio 200-------
connivance with mind in singling out matter for special distinction.
Action is generally regarded as the most fundamental
thing in the real world of physics, although the mind passes it
over because of its lack of permanence; and it is vaguely believed
that the atomicity of action is the general law, and the appearance
of electrons is in some way dependent on this. But the
precise formulation of the theory of quanta of action has hitherto
baffled physicists.%
\index{Quanta}%

There is a striking contrast between the triumph of the
scientific mind in formulating the great general scheme of
natural laws, nowadays summed up in the principle of least
action, and its present defeat by the newly discovered but equally
general phenomena depending on the laws of atomicity of
quanta. It is too early to cry failure in the latter case; but
possibly the contrast is significant. It is one thing for the human
mind to extract from the phenomena of nature the laws which
it has itself put into them; it may be a far harder thing to
extract laws over which it has had no control. It is even possible
that laws which have not their origin in the mind may be
irrational, and we can never succeed in formulating them. This
is, however, only a remote possibility; probably if they were
really irrational it would not have been possible to make the
limited progress that has been achieved. But if the laws of
quanta do indeed differentiate the actual world from other
worlds possible to the mind, we may expect the task of formulating
them to be far harder than anything yet accomplished
by physics.

The theory of relativity has passed in review the whole subject-matter
of physics. It has unified the great laws, which by the
precision of their formulation and the exactness of their application
have won the proud place in human knowledge which
physical science holds to-day. And yet, in regard to the nature
of things, this knowledge is only an empty shell---a form of
symbols. It is knowledge of structural form, and not knowledge
of content.
\index{Content contrasted with structural form}%
\index{Form contrasted with content}%
\index{Structure opposed to content}%
All through the physical world runs that unknown
content, which must surely be the stuff of our consciousness.
Here is a hint of aspects deep within the world of physics, and
yet unattainable by the methods of physics. And, moreover,
we have found that where science has progressed the farthest,
%% -----File: 211.png---Folio 201-------
the mind has but regained from nature that which the mind has
put into nature.

We have found a strange foot-print on the shores of the
unknown. We have devised profound theories, one after
another, to account for its origin. At last, we have succeeded
in reconstructing the creature that made the foot-print. And
Lo! it is our own.
%% -----File: 212.png---Folio 202-------


\Appendix

\First{The} references marked ``Report'' are to the writer's ``Report
on the Relativity Theory of Gravitation'' for the Physical
Society of London (Fleetway Press), where fuller mathematical
details are given.

Probably the most complete treatise on the mathematical
theory of the subject is H.~Weyl's \textit{Raum, Zeit, Materie} (Julius
Springer, Berlin).

\AppNote{1}{(\Pageref{note1}).} % p.~20

\Indent It is not possible to predict the contraction rigorously from
the universally accepted electromagnetic equations, because
these do not cover the whole ground. There must be other forces
or conditions which govern the form and size of an electron;
under electromagnetic forces alone it would expand indefinitely.
The old electrodynamics is entirely vague as to these forces.

The theory of Larmor and Lorentz shows that if any system
at rest in the aether is in equilibrium, a similar system in
uniform motion through the aether, but with all lengths in the
direction of motion diminished in FitzGerald's ratio, will also
be in equilibrium so far as the differential equations of the
electromagnetic field are concerned. There is thus a general
theoretical agreement with the observed contraction, provided
the boundary conditions at the surface of an electron behave in
the same way. The latter suggestion is confirmed by experiments
on isolated electrons in rapid motion (Kaufmann's experiment).
\IndexExtra{Electron!Kaufmann's experiment on} %
It turns out that this requires an electron to suffer the same
kind of contraction as a material rod; and thus, although the
theory throws light on the adjustments involved in material
contraction, it can scarcely be said to give an explanation of the
occurrence of contraction generally.
%% -----File: 213.png---Folio 203-------


\AppNote{2}{(\Pageref{note2}).} % p.~47

\Indent Suppose a particle moves from $(x_{1}, y_{1}, z_{1}, t_{1})$ to $(x_{2}, y_{2}, z_{2}, t_{2})$,
its velocity~$u$ is given by
\[
u^{2} = \frac{(x_{2}-x_{1})^{2} + (y_{2}-y_{1})^{2} + (z_{2}-z_{1})^{2}}
             {(t_{2}-t_{1})^{2}}.
\]
Hence from the formula for $s^{2}$
\[
s = (t_{2}-t_{1}) \surd(1-u^{2}).
\]
(We omit a $\sqrt{-1}$, as the sign of $s^{2}$ is changed later in the
chapter.)

If we take $t_{1}$ and $t_{2}$ to be the start and finish of the aviator's
cigar (\Chapref{I}), then as judged by a terrestrial observer,
$t_{2}-t_{1} = 60~\text{minutes}$, $\surd(1- u^{2}) = \text{FitzGerald contraction} = \frac{1}{2}$.

As judged by the aviator,
\[
t_{2} - t_{1} = 30~\text{minutes}, \quad \surd(1-u^{2}) = 1.
\]

Thus for both observers $s = 30$ minutes, verifying that it is
an absolute quantity independent of the observer.

\AppNote{3}{(\Pageref{note3}).} % p.~48

\Indent The formulae of transformation to axes with a different
orientation are
\[
   x = x' \cos\theta - \tau' \sin\theta, \quad
   y = y', \quad
   z = z', \quad
\tau = x' \sin\theta + \tau' \cos\theta,
\]
where $\theta$ is the angle turned through in the plane~$x\tau$.

Let $u = i \tan \theta$, so that $\cos\theta = (1-u^{2})^{-\frac{1}{2}} = \beta$, say. The
formulae become
\[
   x = \beta (x'- iu\tau'), \quad
   y = y', \quad
   z = z', \quad
\tau = \beta (\tau' + iux'),
\]
or, reverting to real time by setting $i\tau = t$,
\[
x = \beta (x' - ut'), \quad
y = y', \quad
z = z', \quad
t = \beta (t' - ux'),
\]
which gives the relation between the estimates of space and
time by two different observers.

The factor $\beta$ gives in the first equation the FitzGerald contraction,
and in the fourth equation the retardation of time.
The terms $ut'$ and~$ux'$ correspond to the changed conventions
as to \textit{rest} and \textit{simultaneity}.

A point at rest, $x = \text{const.}$, for the first observer corresponds
to a point moving with velocity~$u$, $x'-ut' = \text{const.}$, for the second
observer. Hence their relative velocity is~$u$.
%% -----File: 214.png---Folio 204-------


\AppNote{4}{(\Pageref{note4}).} % p.~81

\Indent The condition for flat space in two dimensions is
\begin{multline*}
\frac{\partial}{\partial x_1}
  \left( \frac{g_{12}}{g_{11}
         \surd(g_{11} g_{22} - g_{12}^2)}\,
         \frac{\partial g_{11}}{\partial x_2}
  - \frac{1}{\surd(g_{11} g_{22} - g_{12}^2)}\,
         \frac{\partial g_{22}}{\partial x_1}\right) \\ % [** PP: Changed \delta to \partial]
%
+ \frac{\partial}{\partial x_2}
  \left( \frac{2}{\surd(g_{11} g_{22} - g_{12}^2)}\,
         \frac{\partial g_{12}}{\partial x_1}
  - \frac{1}{\surd(g_{11} g_{22} - g_{12}^2)}\,
  \frac{\partial g_{11}}{\partial x_2} \right. \\
%
- \left.\frac{g_{12}}{g_{11} \surd(g_{11} g_{22} - g_{12}^2)}\,
  \frac{\partial g_{11}}{\partial x_1}\right) = 0.
\end{multline*}


\AppNote{5}{(\Pageref{note5}).} % p.~89

\Indent Let~$g$ be the determinant of four rows and columns formed
with the elements~$g_{\mu\nu}$.

Let $g^{\mu\nu}$ be the minor of~$g_{\mu\nu}$, divided by~$g$.

Let the ``$3$-index symbol'' $\{\mu\nu, \lambda\}$ denote
\[
\tfrac{1}{2} g^{\lambda\alpha}
  \left( \frac{\partial g_{\mu\alpha}}{\partial x_\nu}
       + \frac{\partial g_{\nu\alpha}}{\partial x_\mu}
       - \frac{\partial g_{\mu\nu}}{\partial x_\alpha} \right)
\]
summed for values of~$\alpha$ from~$1$ to~$4$. There will be $40$ different
$3$-index symbols.

Then the Riemann-Christoffel tensor is
\[
B^\rho_{\mu\nu\sigma}
  = \{\mu\sigma, \epsilon\} \{\epsilon\nu, \rho\}
  - \{\mu\nu, \epsilon\} \{\epsilon\sigma, \rho\}
  + \frac{\partial}{\partial x_\nu} \{\mu\sigma, \rho\}
  - \frac{\partial}{\partial x_\sigma} \{\mu\nu, \rho\},
\]
the terms containing~$\epsilon$ being summed for values of~$\epsilon$ from~$1$ to~$4$.

The ``contracted'' Riemann-Christoffel tensor~$G_{\mu\nu}$ can be
reduced to
\begin{multline*}
G_{\mu\nu}
  = - \frac{\partial}{\partial x_\alpha} \{\mu\nu, \alpha\}
    + \{\mu\alpha, \beta\} \{\nu\beta, \alpha\} \\
    + \frac{\partial^2}{\partial x_\mu \partial x_\nu} \log \sqrt{-g}
    - \{\mu\nu, \alpha\} \frac{\partial}{\partial x_\alpha} \log \sqrt{-g},
\end{multline*}
where in accordance with a general convention in this subject,
each term containing a suffix twice over ($\alpha$ and~$\beta$) must be
summed for the values $1$, $2$, $3$, $4$ of that suffix.

The curvature $G = g^{\mu\nu} G_{\mu\nu}$, summed in accordance with the
foregoing convention.


\AppNote{6}{(\Pageref{note6}).} % p.~94

\Indent The electric potential due to a charge~$e$ is
\[
\phi =  \frac{e}{\bigl[r(1-v_r/C)\bigr]},
\]
%% -----File: 215.png---Folio 205-------
where $v_r$ is the velocity of the charge in the direction of~$r$, $C$ the
velocity of light, and the square bracket signifies antedated
values. To the first order of $v_r/C$, the denominator is equal to
the \textit{present} distance~$r$, so the expression reduces to~$e/r$ in spite
of the time of propagation. The foregoing formula for the
potential was found by Li�nard and Wiechert.


\AppNote{7}{(\Pageref{note7}).} % p.~97

\Indent It is found that the following scheme of potentials rigorously
satisfies the equations $G_{\mu\nu} = 0$, according to the values of $G_{\mu\nu}$
in \Noteref{5},
\[
\begin{array}{cccc}
-1/\gamma & 0          & 0  &   0 \\
          & -x_{1}{}^{2} & 0  &   0 \\
          &            & -x_{1}{}^{2}\sin^{2} x_{2}{}^{2} &  0 \\
          &            &    & \gamma
\end{array}
\]
where $\gamma = 1-\kappa/x_{1}$ and $\kappa$ is any constant (see Report, �~28).
Hence these potentials describe a kind of space-time which can
occur in nature referred to a possible mesh-system. If $\kappa = 0$,
the potentials reduce to those for flat space-time referred to
polar coordinates; and, since in the applications required $\kappa$ will
always be extremely small, our coordinates can scarcely be
distinguished from polar coordinates. We can therefore use the
familiar symbols $r$, $\theta$, $\phi$, $t$, instead of $x_{1}$, $x_{2}$, $x_{3}$, $x_{4}$. It must,
however, be remembered that the identification with polar
coordinates is only approximate; and, for example, an equally
good approximation is obtained if we write $x_{1} = r + \frac{1}{2} \kappa$, a substitution
often used instead of $x_{1} = r$ since it has the advantage
of making the coordinate-velocity of light more symmetrical.

We next work out analytically all the mechanical and optical
properties of this kind of space-time, and find that they agree
observationally with those existing round a particle at rest at
the origin with gravitational mass~$\frac{1}{2} \kappa$. The conclusion is that
the gravitational field here described is produced by a particle
of mass $\frac{1}{2} \kappa$---or, if preferred, a particle of matter at rest is
produced by the kind of space-time here described.


\AppNote{8}{(\Pageref{note8}).} % p.~98

\Indent Setting the gravitational constant equal to unity, we have for
a circular orbit
\begin{DPalign*}
m/r^{2} &= v^{2}/r, \\
\lintertext{so that}
      m &= v^{2}r.
\end{DPalign*}

%% -----File: 216.png---Folio 206-------

The earth's speed,~$v$, is approximately $30$~km.\ per sec., or
$\frac{1}{10000}$ in terms of the velocity of light. The radius of its orbit,~$r$,
is about $1.5 � 10^8$~km. Hence,~$m$, the gravitational mass of the
sun is approximately $1.5$~km. % [** PP: Changing scientific notation]

The radius of the sun is $697,000$~kms., so that the quantity
$2m/r$ occurring in the formulae is, for the sun's surface, $.00000424$
or~$0''.87$.


\AppNote{9}{(\Pageref{note9}).} % p.~123

\index{Gravitational field of Sun!motion of perihelion}%
\Indent See Report, ��~29, 30. The general equations of a geodesic are
\[
\frac{d^2x_\mu}{ds^2}
  + \{\alpha\beta, \mu\}\, \frac{dx_\alpha}{ds} \frac{dx_\beta}{ds} = 0\quad
 (\mu = 1,\ 2,\ 3,\ 4).
\]

From the formula for the line-element
\[
ds^2 = -\gamma^{-1}\, dr^2 - r^2\, d\theta^2 + \gamma\, dt^2,
\tag{1}
\]
we calculate the three-index symbols and it is found that two
of the equations of the geodesic take the rather simple form
\begin{align*}
&\frac{d^2\theta}{ds^2}
  + \frac{2}{r} � \frac{dr}{ds} \frac{d\theta}{ds} = 0, \\
%
&\frac{d^2t}{ds^2}
  + \frac{d(\log\gamma)}{dr} � \frac{dr}{ds} \frac{dt}{ds} = 0,
\end{align*}
which can be integrated giving
\begin{align*}
r^2\, \frac{d\theta}{ds} &= h,
\tag{2} \\
%
\frac{dt}{ds} &= \frac{c}{\gamma},
\tag{3}
\end{align*}
where $h$ and~$c$ are constants of integration.

Eliminating $dt$ and~$ds$ from (1), (2) and~(3), we have
\[
\left(\frac{h}{r^2} \frac{dr}{d\theta}\right)^2 + \frac{h^2}{r^2}
  = c^2 - 1 + \frac{2m}{r} + \frac{2mh^2}{r^3},
\]
or writing $u = 1/r$, % [** PP: Condensed intertext in original]
\[
\left(\frac{du}{d\theta}\right)^2 + u^2
  = \frac{c^2 - 1}{h^2} + \frac{2mu}{h^2} + 2mu^3.
\]

Differentiating with respect to~$\theta$
\[
\frac{d^2u}{d\theta^2} + u = \frac{m}{h^2} + 3mu^2,
\]
%% -----File: 217.png---Folio 207-------
which gives the equation of the orbit in the usual form in particle
dynamics. It differs from the equation of the Newtonian orbit
by the small term $3mu^2$, which is easily shown to give the motion
of perihelion.

The track of a ray of light is also obtained from this formula,
since by the principle of equivalence it agrees with that of a
material particle moving with the speed of light.
\index{Bending of light!theory of}%
\index{Light, bending of}%
This case is
given by $ds = 0$, and therefore $h = \infty$. The differential equation
for the path of a light-ray is thus
\[
\frac{d^2u}{d\theta^2} + u = 3mu^2.
\]

An approximate solution is
\[
u = \frac{\cos\theta}{R} + \frac{m}{R^2}(\cos^2\theta + 2\sin^2\theta),
\]
neglecting the very small quantity~$m^2/R^2$. Converting to
Cartesian coordinates, this becomes
\[
x = R - \frac{m}{R} \frac{x^2 + 2y^2}{\surd(x^2 + y^2)}.
\]

The asymptotes of the light-track are found by taking~$y$
very large compared with~$x$, giving
\[
x = R � \frac{2m}{R} y
\]
so that the angle between them is~$4m/R$.


\AppNote{10}{(\Pageref{note10}).} % p.~126

\Indent Writing the line element in the form
\[
ds^2
  = -\left(1 + a \frac{m}{r} + \dotsb\right) dr^2
  - r^2\, d\theta^2
  + \left(1 + b \frac{m}{r} + c \frac{m^2}{r^2} + \dotsb\right) dt^2,
\]
the approximate Newtonian attraction fixes~$b$ equal to~$-2$;
then the observed deflection of light fixes~$a$ equal to~$+2$; and
with these values the observed motion of Mercury fixes~$c$ equal
to~$0$.%
\index{Deflection of light!theory of}%
\index{Gravitational field of Sun!deflection of light}%

To insert an arbitrary coefficient of $r^2\, d\theta^2$ would merely vary
the coordinate system. We cannot arrive at any intrinsically
different kind of space-time in that way. Hence, within the
limits of accuracy mentioned, the expression found by Einstein
is completely determinable by observation.

%% -----File: 218.png---Folio 208-------

It may be mentioned that the line-element
\[
ds^2 = - dr^2 - r^2\, d\theta^2 + (1-2m/r)\, dt^2,
\]
gives one-half the observed deflection of light, and one-third
the motion of perihelion of Mercury. As both these can be
obtained on older theories, taking account of the variation of
mass with velocity, the coefficient~$\gamma^{-1}$ of~$dr^2$ is the essentially
novel point in Einstein's theory.


\AppNote{11}{(\Pageref{note11}).} % p.~131

\Indent It is often supposed that by the Principle of Equivalence any
invariant property which holds outside a gravitational field also
holds in a gravitational field; but there is necessarily some
limitation on this equivalence. Consider for instance the two
invariant equations
\begin{gather*}
ds^2 = 1, \\
ds^2 (1 + k^4 B^\rho_{\mu\nu\sigma} B^{\mu\nu\sigma}_\rho) = 1,
\end{gather*}
where $k$ is some constant having the dimensions of a length.
Since $B^\rho_{\mu\nu\sigma}$ vanishes outside a gravitational field, if one of these
equations is true the other will be. But they cannot both hold
in a gravitational field, since there $B^\rho_{\mu\nu\sigma} B^{\mu\nu\sigma}_\rho$ does not vanish,
and is in fact equal to $24m^2/r^6$. (I believe that the numerical
factor~$24$ is correct; but there are 65,536~terms in the expression,
and the terms which do not vanish have to be picked out.) % [** F1: Added closing parenthesis]

This ambiguity of the Principle of Equivalence is referred to
in Report, ��~14, 27; and an enunciation is given which makes
it definite. The enunciation however is merely an explicit statement,
and not a defence, of the assumptions commonly made in
applying the principle.

So far as general reasoning goes there seems no ground for
choosing $ds^2$ rather than $ds^2\, (1 + 24k^4m^2/r^6)$, or any similar expression,
as the constant character in the vibration of an atom.


\AppNote{12}{(\Pageref{note12}).} % p.~134

\Indent Let two rays diverging from a point at a distance~$R$ pass at
distances $r$ and~$r + dr$ from a star of mass~$m$. The deflection
being~$4m/r$, their divergence will be increased by~$4m\, dr/r^2$. This
increase will be equal to the original divergence~$dr/R$ if
$r = \sqrt{4mR}$. Take for instance $4m = 10$~km., $R = 10^{15}$~km., then
$r = 10^8$~km. So that the divergence of the light will be doubled,
%% -----File: 219.png---Folio 209-------
when the actual deflection of the ray is only $10^{-7}$, or $0''.02$.
In the case of a star seen behind the sun the added divergence
has no time to take effect; but when the light has to travel a
stellar distance after the divergence is produced, it becomes
weakened by it. Generally in stellar phenomena the weakening
of the light should be more prominent than the actual deflection.


\AppNote{13}{(\Pageref{note13}).} % p.~141

\Indent The relations are (Report, �~39)
\[
G^\nu_{\mu\nu} = \tfrac{1}{2} \frac{\partial G}{\partial x_\mu}\quad
(\mu = 1,\ 2,\ 3,\ 4),
\]
where $G^\nu_{\mu\nu}$ is the (contracted) covariant derivative of~$G^\nu_\mu$,
or~$g^{\nu\alpha} G_{\mu\alpha}$.

I doubt whether anyone has performed the laborious task of
verifying these identities by straightforward algebra.


\AppNote{14}{(\Pageref{note14}).} % p.~158

\Indent The modified law for spherical space-time is in empty space
\[
G_{\mu\nu} = \lambda g_{\mu\nu}.
\]

In cylindrical space-time, matter is essential. The law in space
occupied by matter is
\index{Cylindrical world, Einstein's}%
\[
G_{\mu\nu} - \tfrac{1}{2} g_{\mu\nu} (G - 2\lambda) = -8\pi T_{\mu\nu},
\]
the term $2\lambda$ being the only modification. Spherical space-time
of radius~$R$ is given by $\lambda = 3/R^2$; cylindrical space-time by
$\lambda = 1/R^2$ provided matter of average density $\rho = 1/4 \pi R^2$ is
present. (See Report, ��~50, 51.) The total mass of matter in
the cylindrical world is $\frac{1}{2} \pi R$. This must be enormous, seeing
that the sun's mass is only $1\frac{1}{2}$~kilometres.


\AppNote{15}{(\Pageref{note15}).} % p.~174

\Indent Weyl's theory is given in \textit{Berlin.\ Sitzungsberichte}, 30~May, 1918;
\textit{Annalen der Physik}, Bd.~59 (1919), p.~101.


\AppNote{16}{(\Pageref{note16}).} % p.~177

\Indent The argument is rather more complicated than appears in
the text, where the distinction between action-density and
action in a region, curvature and total curvature in a region,
has not been elaborated. Taking a definitely marked out region
in space and time, its measured volume will be increased $16$-fold
%% -----File: 220.png---Folio 210-------
by halving the gauge. Therefore for action-density we must
take an expression which will be diminished $16$-fold by halving
the gauge. Now~$G$ is proportional to $1/R^{2}$, where~$R$ is the radius
of curvature, and so is diminished $4$-fold. The invariant $B^{\rho}_{\mu\nu\sigma} B^{\mu\nu\sigma}_{\rho}$
has the same gauge-dimensions as~$G^{2}$; and hence when integrated
through a volume gives a pure number independent of the gauge.
In Weyl's theory this is only the gravitational part of the complete
invariant
\[
(B^{\rho}_{\mu \nu \sigma} - \tfrac{1}{2} g^{\rho}_{\mu} F_{\nu \sigma})
(B^{\mu \nu \sigma}_{\rho} - \tfrac{1}{2} g^{\mu}_{\rho} F^{\nu \sigma}),
\]
which reduces to
\[
B^{\rho}_{\mu \nu \sigma}
B^{\mu \nu \sigma}_{\rho} + F_{\nu \sigma}F^{\nu \sigma}.
\]

The second term gives actually the well-known expression for
the action-density of the electromagnetic field, and this evidently
strengthens the identification of this invariant with action-density.

Einstein's theory, on the other hand, creates a difficulty here,
because although there may be action in an electromagnetic
field without electrons, the curvature is zero.

\fancyhead[CO]{\textsc{HISTORICAL NOTE}}
\phantomsection\addcontentsline{toc}{chapter}
{\texorpdfstring{\scshape Historical Note}{Historical Note}}
\section*{\centering\normalfont\textsc{\large HISTORICAL NOTE}}

\Indent Before the Michelson-Morley experiment the question had
been widely discussed whether the aether in and near the earth
was carried along by the earth in its motion, or whether it
slipped through the interstices between the atoms. Astronomical
aberration pointed decidedly to a stagnant aether;
\index{Aether!stagnant}%
but the
experiments of Arago and Fizeau on the effect of motion of
transparent media on the velocity of light in those media,
suggested a partial convection of the aether in such cases. These
experiments were first-order experiments, i.e.\ %[** PP: Add missing .]
they depended on
the ratio of the velocity of the transparent body to the velocity
of light. The Michelson-Morley experiment is the first example
of an experiment delicate enough to detect second-order effects,
depending on the square of the above ratio; the result, that no
current of aether past terrestrial objects could be detected,
appeared favourable to the view that the aether must be convected
by the earth. The difficulty of reconciling this with
astronomical aberration was recognised.

%% -----File: 221.png---Folio 211-------

An attempt was made by Stokes to reconcile mathematically
a convection of aether by the earth with the accurately verified
facts of astronomical aberration; but his theory cannot be
regarded as tenable. Lodge investigated experimentally the
question whether smaller bodies carried the aether with them
in their motion, and showed that the aether between two
whirling steel discs was undisturbed.%
\index{Lodge}%

{\stretchyspace
The controversy, stagnant \textit{versus} convected aether, had now
reached an intensely interesting stage. In 1895, Lorentz discussed
the problem from the point of view of the electrical
theory of light and matter. By his famous transformation of
the electromagnetic equations, he cleared up the difficulties
associated with the first-order effects, showing that they could
all be reconciled with a stagnant aether. In 1900, Larmor carried
the theory as far as second-order effects, and obtained an exact
theoretical foundation for FitzGerald's hypothesis of contraction,
which had been suggested in 1892 as an explanation of the
Michelson-Morley experiment. The theory of a stagnant aether
was thus reconciled with all observational results; and henceforward % [** PP: Hyphenated across a line in original]
it held the field.%
\index{Larmor}%
\index{Lorentz}%

Further second-order experiments were performed by Rayleigh
and Brace on double refraction (1902, 1904), Trouton and Noble
on a torsional effect on a charged condenser (1903), and Trouton
and Rankine on electric conductivity (1908). All showed that
the earth's motion has no effect on the phenomena. On the
theoretical side, Lorentz (1902) showed that the indifference of
the equations of the electromagnetic field to any velocity of the
axes of reference, which he had previously established to the
first order, and Larmor to the second order, was exact to all
orders. He was not, however, able to establish with the same
exactness a corresponding transformation for bodies containing
electrons.

Both Larmor and Lorentz had introduced a ``local time'' for
the moving system. It was clear that for many phenomena this
local time would replace the ``real'' time; but it was not
suggested that the observer in the moving system would be
deceived into thinking that it was the real time. Einstein, in
1905 founded the modern principle of relativity by postulating
that this local time was \textit{the time} for the moving observer; no
%% -----File: 222.png---Folio 212-------
real or absolute time existed, but only the local times, different
for different observers. He showed that absolute simultaneity
and absolute location in space are inextricably bound together,
and the denial of the latter carries with it the denial of the
former. By realising that an observer in the moving system
would measure all velocities in terms of the local space and time
of that system, Einstein removed the last discrepancies from
Lorentz's transformation.

The relation between the space and time coordinates in two
systems in relative motion was now obtained immediately from
the principles of space and time-measurement. It must hold
for all phenomena provided they do not postulate a medium
which can serve as a standard for absolute location and simultaneity.
The previous deduction of these formulae by lengthy
transformation of the electromagnetic equations now appears
as a particular case; it shows that electromagnetic phenomena
have no reference to a medium with such properties.

The combination of the local spaces and times of Einstein
into an absolute space-time of four dimensions is the work of
Minkowski~(1908).
\index{Minkowski}%
\index{Space-time!due to Minkowski}%
\Chapref{III} is largely based on his researches.
Much progress was made in the four-dimensional vector-analysis
of the world; but the whole problem was greatly
simplified when Einstein and Grossmann introduced for this
purpose the more powerful mathematical calculus of Riemann,
Ricci, and Levi-Civita.

}In 1911, Einstein put forward the Principle of Equivalence,
thus turning the subject towards gravitation for the first time.
\index{Equivalence!Principle of}%
\index{Principle of Equivalence}%
By postulating that not only mechanical but optical and
electrical phenomena in a field of gravitation and in a field
produced by acceleration of the observer were equivalent, he
deduced the displacement of the spectral lines on the sun and
the displacement of a star during a total eclipse. In the latter
case, however, he predicted only the half-deflection, since he
was still working with Newton's law of gravitation. Freundlich
\index{Freundlich}%
at once examined plates obtained at previous eclipses, but failed
to find sufficient data; he also prepared to observe the eclipse
of 1914 in Russia with this object, but was stopped by the outbreak
of war. Another attempt was made by the Lick Observatory
at the not very favourable eclipse of 1918. Only preliminary
%% -----File: 223.png---Folio 213-------
results have been published; according to the information given,
the probable accidental error of the mean result (reduced to the
sun's limb) was about $1''.6$, so that no conclusion was permissible.

The principle of equivalence opened up the possibility of a
general theory of relativity not confined to uniform motion, for
it pointed a way out of the objections which had been urged
against such an extension from the time of Newton. At first
the opening seemed a very narrow one, merely indicating that
the objections could not be considered final until the possibilities
of complications by gravitation had been more fully exhausted.
By 1913, Einstein had surmounted the main difficulties. His
theory in a complete form was published in 1915; but it was not
generally accessible in England until a year or two later. As
this theory forms the main subject-matter of the book, we may
leave our historical survey at this point.

%% -----File: 224.png---Folio 214-------
% [Blank Page]
%% -----File: 225.png---Folio 215-------
\iffalse
INDEX

Absolute, approached through the
relative, 82

Absolute acceleration, 68, 154, 194

Absolute past and future, 50

Absolute rotation, 152, 164, 194

Absolute simultaneity, 12, 51

Absolute time, in cylindrical world,
163

Acceleration, a simpler quality than
velocity, 195; modifies FitzGerald
contraction, 75

Action, 147; atomicity of, 199; on
Weyl's theory, 177

Action, Principle of Least, 149, 178

Addition of velocities, 59

Aether, a plenum with geodesic structure,
164; identified with the
``world,'' 187; non-material nature
of, 39; stagnant, 210

Artificial fields of force, 64

Atom, vibrating on sun, 128

Atomicity, law of, 199; of Action, 177

Aviator, space and time-reckoning of,
23

Bending of light, effect on star's
position, 112; observational results,
118; theory of, 107, 207

Beta particles, 59, 145

Brain, constitution of, 191

Brazil, eclipse expedition to, 117

Causality, law of, 156

Causation and free will, 51

Centrifugal Force, compared with
gravitation, 41, 65; debt at infinity,
157; not caused by stars, 153;
vibrating atom in field of, 129

Chess, analogy of, 184

Christoffel, 89

Circle in non-Euclidean space, 104

Clifford, 77, 152, 192

Cliquishness, 188

Clock, affected by velocity, 58; on sun,
74, 128; perfect, 13; recording
proper-time, 71

Clock-scale, 58

Clock-scale geometry, not fundamental,
73, 131, 191

Coincidences, 87

Comets, motion through coronal medium,
121; radiation-pressure in, 110

Conservation of electric charge, 173;
of energy and momentum, 139; of
mass, 141, 196

Content contrasted with structural
form, 192, 200

Continuous matter, 91, 140

Contraction, FitzGerald, 19, 54

Convergence of physical approximations,
154

Coordinates, 77

Coordinate velocity, 107

Corona, refraction by, 121

Cottingham, 114

Crommelin, 114, 122

Curvature, degrees of, 91; identified
with action, 148; merely illustrative,
84; of a globe of water, 148;
of space and time, 158; on Weyl's
theory, 176; perception of, 190

Cylinder and plane, indistinguishable
in two dimensions, 81

Cylindrical world, Einstein's, 161, 177

Davidson, 114

Deflection of light, effect on star's
position, 112; observational results,
118; theory of, 107, 207

Density, effect of motion on, 62

Displacement of spectral lines, 129; in
nebulae, 161; in stars, 135

Displacement of star-images, 112, 115

Double stars and Einstein effect, 133

Duration, not inherent in external
world, 34
\fi
%% -----File: 226.png---Folio 216-------
\iffalse
Eclipse, observations during, 113

Ehrenfest's paradox, 75

Electrical theory of inertia, 61

Electricity and gravitation, 167

Electromagnetic potentials and forces,
172

Electron, dimensions of, 177; geometry
inside, 91; gravitational mass of,
178; inertia of, 61; Kaufmann's
experiment on, 62, 146; singularity
in field, 167

``Elsewhere,'' 50

Emptiness, perception of, 190

Energy, conservation of, 139; identified
with mass, 146; inertia of, 61,
146; weight of radio-active, 112

Entropy, 149

E�tv�s torsion-balance, 112

Equivalence, Principle of, 76, 131, 212

Euclidean geometry, 1, 47, 73

Euclidean space of five dimensions, 84

Event, definition of, 45, 186

Evershed, 130

Extension in four dimensions, 37, 46

Feeling, elements of, 192

Fields of force, artificial, 64; due to
disturbance of observer, 69; electromagnetic,
171; relativity of, 67

Field of velocity, 195

FitzGerald contraction, 19; consequences
of, 22; modified by
acceleration, 75; relativity explanation
of, 54

Flat space in two dimensions, 80

Flat space-time, 83; at infinity, 84;
conditions for, 89

Flatfish, analogy of, 95

Flatland, 57

Force, compared with inertia, 137;
electromagnetic, 172; elementary
conception of, 63; fields of, 64;
relativity of, 43, 67, 76

Form contrasted with content, 192, 200

Formalism of knowledge, 175

Foucault's pendulum, 152

Four-dimensional order, 35, 56, 186

Four-dimensional space-time, geometry
of, 45, 82; reality of, 181

Fourth dimension, 13

Frame, inertial, 156

Frames of reference, ``right'' and
``wrong,'' 42

Freewill, 51

Freundlich, 212

Future, absolute, 50

Galilean potentials, 83

Gauge, effect on observations, 31;
provided by radius of space, 177

Gauge-system, 169

Geodesic, absolute significance of, 70,
150; definition of, 75; motion of
particles in, 138, 151; in regions at
infinity, 157

Geodesic structure, absolute character
of, 155, 164; acceleration of, 195

Geometrical conception of the world,
176, 183

Geometry, Euclidean, 1; hyperbolic,
47; Lobatchewskian, 1, 9; natural,
2; non-Euclidean, or Riemannian,
6, 73, 84, 90; non-Riemannian, 169;
semi-Euclidean, 47

Ghosts of stars, 161

Globe of water, limit to size of, 148

Gravitation, Einstein's law of, differential
formula, 90; integrated formula
for a particle, 97; macroscopic
equations, 140, 193

Gravitation, Newton's law of, ambiguity
of, 93; approximation to
Einstein's law, 103; deflection of
light, 109, 111

Gravitation, propagation with velocity
of light, 94, 147; relativity for
uniform motion, 21, 125

Gravitational field of Sun, 97; deflection
of light, 107, 118, 207; displacement
of spectral lines, 129;
motion of perihelion, 122, 206;
Newtonian attraction, 102; result
of observational verification, 126

Grebe and Bachem, 130

Greenwich, Royal Observatory, 114

Gyro-compass, 152

Hummock in space-time, 97

Hurdles, analogy of counts of, 104

Hyperbolic geometry, 47
\fi
%% -----File: 227.png---Folio 217-------
\iffalse
Identities connecting G_{\mu \nu}, 141

Identity permanent, 40, 193

Imaginary intervals, 150, 187

Imaginary time, 48, 181

Inertia, compared with force, 137;
electrical theory of, 61; in regions
at infinity, 157; infinite, 56;
Mach's views, 164; of light, 110;
relativity theory of, 139

Inertia-gravitation, 137

Inertial frame, 156

Infinity, conditions at, 157

Integrability of length and direction,
174

Interval, 46, 150, 187; general expression
for, 82; practical measurement
of, 58, 75

Interval-length, geometrical significance
essential, 127; identified with
proper time, 71; tracks of maximum,
70, 150; zero for velocity of light, 71

Invariant mass, 145; of light, 148

Jupiter, deflection of light by, 133

Kaufmann's experiment, 62, 146

Kinds of space, 81

Laplace's equation, 96, 140

Larmor, 19, 211

Length, definition of, 2; effect of
motion on, 19; relativity of, 34

Le Verrier, 124

Levi-Civita, 89

Lift, accelerated, 64

Light, bending of, 107, 112, 118, 207;
coordinate velocity of, 107; mass
of, 62, 110, 148; voyage round the
world, 161; weight of, 111

Light, velocity of, an absolute velocity,
59; importance of, 60; system
moving with, 26, 56

Lobatchewsky, 1, 9

Lodge, 32, 125, 211

Longest tracks, 70

Lorentz, 19, 211

Mach's philosophy, 163

Macroscopic equations, 92, 139; interval,
187

Map of sun's gravitational field, 99

Mass, conservation of, 141, 195;
electrical theory of, 61; gravitational,
98; identified with energy,
146; invariant, 145; of light, 62,
110, 148; variation with velocity,
145

Mathematics, Russell's description of,
14

Matter, continuous, 91; definition of a
particle, 98; extensional relations
of, 8; gravitational equations in,
141; perception of, 190; physical
and psychological aspects, 192

Mercury, perihelion of, 123, 125

Mesh-systems, 77; irrelevance to laws
of nature, 87

Michelson-Morley experiment, 18

Minkowski, 30, 212

Mirror, distortion by moving, 22

Momentum, conservation of, 141; redefinition
of, 144; of light, 111

Moon, motion of, 93, 134

Motion, insufficiency of kinematical
conception, 194; Newton's first
law, 136

Natural frame, 155

Natural gauge, 176

Natural geometry, 2

Natural tracks, 70

Nebulae, atomic vibrations in, 161

Newton, absolute rotation, 41; bending
of light, 110; law of gravitation,
93; law of motion, 136; relativity
for uniform motion, 40; super-observer,
68

Non-Euclidean geometry, 6, 73, 84, 90

Non-Riemannian geometry, 169

Observer, an unsymmetrical object, 57

Observer and observed, 30

Orbits under Einstein's law, 123

Order and dimensions, 14, 186

Ordering of events in external world,
35, 54, 184

Past, absolute, 50

Perceptions, as crude measures, 10, 15,
31
\fi
%% -----File: 228.png---Folio 218-------
\iffalse
Perihelia of planets, motions of, 123
Permanence of matter, 196
Permanent identity, 40, 193
Permanent perceptual world, 141, 198
Poincar�, 9
Point-event, 45, 186
Potentials, 80; Galilean values, 83
Potentials, electromagnetic, 172
Principe, eclipse expedition to, 114
Principle of Equivalence, 76, 131, 212
Principle of Least Action, 149, 178
Principle of Relativity (restricted), 20
Probability, a pure number, 178
Projectile, Jules Verne's, 65
Propagation of Gravitation, 94, 147
Proper-length, 11
Proper-time, 71
Pucker in space-time, 85

Quanta, 60, 177, 182, 200

Radiation-pressure, 110
Real world of physics, 37, 181
Receding velocities of B-type stars,
   135; of spiral nebulae, 161
Reflection by moving mirror, 22
Refracting medium equivalent to
   gravitational field, 109
Refraction of light in corona, 121
Relativity of force, 43, 76; of length
  and duration, 34; of motion, 38; of
  rotation, 152, 155; of size, 33
Relativity, Newtonian, 40; restricted
   Principle of, 20; standpoint of, 28
Repulsion of light proceeding radially,
   102, 108
Retardation of time, 24, 55; in centrifugal
   field, 129; in spherical world,
   160
Ricci, 89
Riemann, 2, 89, 167
Riemann-Christoffel tensor, 89
Riemannian, or non-Euclidean, geometry,
   6, 73, 84, 90
Rigid scale, definition of, 3
Rotation, absolute, 152, 164, 194
Rotation of a continuous ring, 194
Russell, 14, 197

St John, 130
Semi-Euclidean geometry, 47
Simultaneity, 12, 51
de Sitter, 134, 159, 179
Sobral, eclipse expedition to, 117
Space, conventional, 9; kinds of, 81;
   meaning of, 3, 8, 15; relativity of,
   34
Space-like intervals, 60, 187
Space-time, 45; due to Minkowski, 212;
   partitions of, 54;
Spherical space-time, 159
Standard metre, comparison with, 168
Stresses in continuous matter, 193
Structure opposed to content, 197, 200
Structure, geodesic, absolute character
   of, 155, 164; acceleration of, 195;
   behaviour at infinity, 157
Super-observer, Newton's, 68
Synthesis of appearances, 31, 182

Tensors, 89, 189
Thomson, J. J., 61
Time, absolute, 163; depends on
   observer's track, 38, 57; for moving
   observer, 24; imaginary, 48; measurement
   of, 13; past and future,
51; ``standing still,'' 26, 160
Time-like intervals, 60, 187
Tracks, natural, 70

Vacuum, defined by law of gravitation,
   190
Vector, non-integrable on Weyl's
   theory, 174
Velocity, addition-law, 59; definition
   of, 193; static character, 194
Velocity of gravitation, 94, 147
Velocity of light, importance of, 60;
   in gravitational field, 108; system
   moving with, 26, 56

Warping of space, 8, 126
Wave-front, slewing of, 108
Weight, of light, 107, 111; of radio-active
   energy, 111; proportional to
   inertia, 137; vanishes inside free
   projectile, 65
Weyl, 174
World, 186, 187
World-line, 87
\fi

\cleardoublepage

\fancyhead[C]{\textsc{INDEX}}
{\small
\printindex

}

%%%%%%%%%%%%%%%%%%%%%%%%% GUTENBERG LICENSE %%%%%%%%%%%%%%%%%%%%%%%%%%

\cleardoublepage

\phantomsection
\pdfbookmark[-1]{Back Matter}{Back Matter}
\pdfbookmark[0]{PG License}{Project Gutenberg License}
\fancyhead[C]{\textsc{LICENSING}}

\begin{PGtext}
End of Project Gutenberg's Space, Time and Gravitation, by A. S. Eddington

*** END OF THIS PROJECT GUTENBERG EBOOK SPACE, TIME AND GRAVITATION ***

***** This file should be named 29782-pdf.pdf or 29782-pdf.zip *****
This and all associated files of various formats will be found in:
        http://www.gutenberg.org/2/9/7/8/29782/

Produced by David Clarke, Andrew D. Hwang and the Online
Distributed Proofreading Team at http://www.pgdp.net (This
file was produced from images generously made available
by The Internet Archive/American Libraries.)


Updated editions will replace the previous one--the old editions
will be renamed.

Creating the works from public domain print editions means that no
one owns a United States copyright in these works, so the Foundation
(and you!) can copy and distribute it in the United States without
permission and without paying copyright royalties.  Special rules,
set forth in the General Terms of Use part of this license, apply to
copying and distributing Project Gutenberg-tm electronic works to
protect the PROJECT GUTENBERG-tm concept and trademark.  Project
Gutenberg is a registered trademark, and may not be used if you
charge for the eBooks, unless you receive specific permission.  If you
do not charge anything for copies of this eBook, complying with the
rules is very easy.  You may use this eBook for nearly any purpose
such as creation of derivative works, reports, performances and
research.  They may be modified and printed and given away--you may do
practically ANYTHING with public domain eBooks.  Redistribution is
subject to the trademark license, especially commercial
redistribution.



*** START: FULL LICENSE ***

THE FULL PROJECT GUTENBERG LICENSE
PLEASE READ THIS BEFORE YOU DISTRIBUTE OR USE THIS WORK

To protect the Project Gutenberg-tm mission of promoting the free
distribution of electronic works, by using or distributing this work
(or any other work associated in any way with the phrase "Project
Gutenberg"), you agree to comply with all the terms of the Full Project
Gutenberg-tm License (available with this file or online at
http://gutenberg.org/license).


Section 1.  General Terms of Use and Redistributing Project Gutenberg-tm
electronic works

1.A.  By reading or using any part of this Project Gutenberg-tm
electronic work, you indicate that you have read, understand, agree to
and accept all the terms of this license and intellectual property
(trademark/copyright) agreement.  If you do not agree to abide by all
the terms of this agreement, you must cease using and return or destroy
all copies of Project Gutenberg-tm electronic works in your possession.
If you paid a fee for obtaining a copy of or access to a Project
Gutenberg-tm electronic work and you do not agree to be bound by the
terms of this agreement, you may obtain a refund from the person or
entity to whom you paid the fee as set forth in paragraph 1.E.8.

1.B.  "Project Gutenberg" is a registered trademark.  It may only be
used on or associated in any way with an electronic work by people who
agree to be bound by the terms of this agreement.  There are a few
things that you can do with most Project Gutenberg-tm electronic works
even without complying with the full terms of this agreement.  See
paragraph 1.C below.  There are a lot of things you can do with Project
Gutenberg-tm electronic works if you follow the terms of this agreement
and help preserve free future access to Project Gutenberg-tm electronic
works.  See paragraph 1.E below.

1.C.  The Project Gutenberg Literary Archive Foundation ("the Foundation"
or PGLAF), owns a compilation copyright in the collection of Project
Gutenberg-tm electronic works.  Nearly all the individual works in the
collection are in the public domain in the United States.  If an
individual work is in the public domain in the United States and you are
located in the United States, we do not claim a right to prevent you from
copying, distributing, performing, displaying or creating derivative
works based on the work as long as all references to Project Gutenberg
are removed.  Of course, we hope that you will support the Project
Gutenberg-tm mission of promoting free access to electronic works by
freely sharing Project Gutenberg-tm works in compliance with the terms of
this agreement for keeping the Project Gutenberg-tm name associated with
the work.  You can easily comply with the terms of this agreement by
keeping this work in the same format with its attached full Project
Gutenberg-tm License when you share it without charge with others.

1.D.  The copyright laws of the place where you are located also govern
what you can do with this work.  Copyright laws in most countries are in
a constant state of change.  If you are outside the United States, check
the laws of your country in addition to the terms of this agreement
before downloading, copying, displaying, performing, distributing or
creating derivative works based on this work or any other Project
Gutenberg-tm work.  The Foundation makes no representations concerning
the copyright status of any work in any country outside the United
States.

1.E.  Unless you have removed all references to Project Gutenberg:

1.E.1.  The following sentence, with active links to, or other immediate
access to, the full Project Gutenberg-tm License must appear prominently
whenever any copy of a Project Gutenberg-tm work (any work on which the
phrase "Project Gutenberg" appears, or with which the phrase "Project
Gutenberg" is associated) is accessed, displayed, performed, viewed,
copied or distributed:

This eBook is for the use of anyone anywhere at no cost and with
almost no restrictions whatsoever.  You may copy it, give it away or
re-use it under the terms of the Project Gutenberg License included
with this eBook or online at www.gutenberg.org

1.E.2.  If an individual Project Gutenberg-tm electronic work is derived
from the public domain (does not contain a notice indicating that it is
posted with permission of the copyright holder), the work can be copied
and distributed to anyone in the United States without paying any fees
or charges.  If you are redistributing or providing access to a work
with the phrase "Project Gutenberg" associated with or appearing on the
work, you must comply either with the requirements of paragraphs 1.E.1
through 1.E.7 or obtain permission for the use of the work and the
Project Gutenberg-tm trademark as set forth in paragraphs 1.E.8 or
1.E.9.

1.E.3.  If an individual Project Gutenberg-tm electronic work is posted
with the permission of the copyright holder, your use and distribution
must comply with both paragraphs 1.E.1 through 1.E.7 and any additional
terms imposed by the copyright holder.  Additional terms will be linked
to the Project Gutenberg-tm License for all works posted with the
permission of the copyright holder found at the beginning of this work.

1.E.4.  Do not unlink or detach or remove the full Project Gutenberg-tm
License terms from this work, or any files containing a part of this
work or any other work associated with Project Gutenberg-tm.

1.E.5.  Do not copy, display, perform, distribute or redistribute this
electronic work, or any part of this electronic work, without
prominently displaying the sentence set forth in paragraph 1.E.1 with
active links or immediate access to the full terms of the Project
Gutenberg-tm License.

1.E.6.  You may convert to and distribute this work in any binary,
compressed, marked up, nonproprietary or proprietary form, including any
word processing or hypertext form.  However, if you provide access to or
distribute copies of a Project Gutenberg-tm work in a format other than
"Plain Vanilla ASCII" or other format used in the official version
posted on the official Project Gutenberg-tm web site (www.gutenberg.org),
you must, at no additional cost, fee or expense to the user, provide a
copy, a means of exporting a copy, or a means of obtaining a copy upon
request, of the work in its original "Plain Vanilla ASCII" or other
form.  Any alternate format must include the full Project Gutenberg-tm
License as specified in paragraph 1.E.1.

1.E.7.  Do not charge a fee for access to, viewing, displaying,
performing, copying or distributing any Project Gutenberg-tm works
unless you comply with paragraph 1.E.8 or 1.E.9.

1.E.8.  You may charge a reasonable fee for copies of or providing
access to or distributing Project Gutenberg-tm electronic works provided
that

- You pay a royalty fee of 20% of the gross profits you derive from
     the use of Project Gutenberg-tm works calculated using the method
     you already use to calculate your applicable taxes.  The fee is
     owed to the owner of the Project Gutenberg-tm trademark, but he
     has agreed to donate royalties under this paragraph to the
     Project Gutenberg Literary Archive Foundation.  Royalty payments
     must be paid within 60 days following each date on which you
     prepare (or are legally required to prepare) your periodic tax
     returns.  Royalty payments should be clearly marked as such and
     sent to the Project Gutenberg Literary Archive Foundation at the
     address specified in Section 4, "Information about donations to
     the Project Gutenberg Literary Archive Foundation."

- You provide a full refund of any money paid by a user who notifies
     you in writing (or by e-mail) within 30 days of receipt that s/he
     does not agree to the terms of the full Project Gutenberg-tm
     License.  You must require such a user to return or
     destroy all copies of the works possessed in a physical medium
     and discontinue all use of and all access to other copies of
     Project Gutenberg-tm works.

- You provide, in accordance with paragraph 1.F.3, a full refund of any
     money paid for a work or a replacement copy, if a defect in the
     electronic work is discovered and reported to you within 90 days
     of receipt of the work.

- You comply with all other terms of this agreement for free
     distribution of Project Gutenberg-tm works.

1.E.9.  If you wish to charge a fee or distribute a Project Gutenberg-tm
electronic work or group of works on different terms than are set
forth in this agreement, you must obtain permission in writing from
both the Project Gutenberg Literary Archive Foundation and Michael
Hart, the owner of the Project Gutenberg-tm trademark.  Contact the
Foundation as set forth in Section 3 below.

1.F.

1.F.1.  Project Gutenberg volunteers and employees expend considerable
effort to identify, do copyright research on, transcribe and proofread
public domain works in creating the Project Gutenberg-tm
collection.  Despite these efforts, Project Gutenberg-tm electronic
works, and the medium on which they may be stored, may contain
"Defects," such as, but not limited to, incomplete, inaccurate or
corrupt data, transcription errors, a copyright or other intellectual
property infringement, a defective or damaged disk or other medium, a
computer virus, or computer codes that damage or cannot be read by
your equipment.

1.F.2.  LIMITED WARRANTY, DISCLAIMER OF DAMAGES - Except for the "Right
of Replacement or Refund" described in paragraph 1.F.3, the Project
Gutenberg Literary Archive Foundation, the owner of the Project
Gutenberg-tm trademark, and any other party distributing a Project
Gutenberg-tm electronic work under this agreement, disclaim all
liability to you for damages, costs and expenses, including legal
fees.  YOU AGREE THAT YOU HAVE NO REMEDIES FOR NEGLIGENCE, STRICT
LIABILITY, BREACH OF WARRANTY OR BREACH OF CONTRACT EXCEPT THOSE
PROVIDED IN PARAGRAPH F3.  YOU AGREE THAT THE FOUNDATION, THE
TRADEMARK OWNER, AND ANY DISTRIBUTOR UNDER THIS AGREEMENT WILL NOT BE
LIABLE TO YOU FOR ACTUAL, DIRECT, INDIRECT, CONSEQUENTIAL, PUNITIVE OR
INCIDENTAL DAMAGES EVEN IF YOU GIVE NOTICE OF THE POSSIBILITY OF SUCH
DAMAGE.

1.F.3.  LIMITED RIGHT OF REPLACEMENT OR REFUND - If you discover a
defect in this electronic work within 90 days of receiving it, you can
receive a refund of the money (if any) you paid for it by sending a
written explanation to the person you received the work from.  If you
received the work on a physical medium, you must return the medium with
your written explanation.  The person or entity that provided you with
the defective work may elect to provide a replacement copy in lieu of a
refund.  If you received the work electronically, the person or entity
providing it to you may choose to give you a second opportunity to
receive the work electronically in lieu of a refund.  If the second copy
is also defective, you may demand a refund in writing without further
opportunities to fix the problem.

1.F.4.  Except for the limited right of replacement or refund set forth
in paragraph 1.F.3, this work is provided to you 'AS-IS' WITH NO OTHER
WARRANTIES OF ANY KIND, EXPRESS OR IMPLIED, INCLUDING BUT NOT LIMITED TO
WARRANTIES OF MERCHANTIBILITY OR FITNESS FOR ANY PURPOSE.

1.F.5.  Some states do not allow disclaimers of certain implied
warranties or the exclusion or limitation of certain types of damages.
If any disclaimer or limitation set forth in this agreement violates the
law of the state applicable to this agreement, the agreement shall be
interpreted to make the maximum disclaimer or limitation permitted by
the applicable state law.  The invalidity or unenforceability of any
provision of this agreement shall not void the remaining provisions.

1.F.6.  INDEMNITY - You agree to indemnify and hold the Foundation, the
trademark owner, any agent or employee of the Foundation, anyone
providing copies of Project Gutenberg-tm electronic works in accordance
with this agreement, and any volunteers associated with the production,
promotion and distribution of Project Gutenberg-tm electronic works,
harmless from all liability, costs and expenses, including legal fees,
that arise directly or indirectly from any of the following which you do
or cause to occur: (a) distribution of this or any Project Gutenberg-tm
work, (b) alteration, modification, or additions or deletions to any
Project Gutenberg-tm work, and (c) any Defect you cause.


Section  2.  Information about the Mission of Project Gutenberg-tm

Project Gutenberg-tm is synonymous with the free distribution of
electronic works in formats readable by the widest variety of computers
including obsolete, old, middle-aged and new computers.  It exists
because of the efforts of hundreds of volunteers and donations from
people in all walks of life.

Volunteers and financial support to provide volunteers with the
assistance they need, are critical to reaching Project Gutenberg-tm's
goals and ensuring that the Project Gutenberg-tm collection will
remain freely available for generations to come.  In 2001, the Project
Gutenberg Literary Archive Foundation was created to provide a secure
and permanent future for Project Gutenberg-tm and future generations.
To learn more about the Project Gutenberg Literary Archive Foundation
and how your efforts and donations can help, see Sections 3 and 4
and the Foundation web page at http://www.pglaf.org.


Section 3.  Information about the Project Gutenberg Literary Archive
Foundation

The Project Gutenberg Literary Archive Foundation is a non profit
501(c)(3) educational corporation organized under the laws of the
state of Mississippi and granted tax exempt status by the Internal
Revenue Service.  The Foundation's EIN or federal tax identification
number is 64-6221541.  Its 501(c)(3) letter is posted at
http://pglaf.org/fundraising.  Contributions to the Project Gutenberg
Literary Archive Foundation are tax deductible to the full extent
permitted by U.S. federal laws and your state's laws.

The Foundation's principal office is located at 4557 Melan Dr. S.
Fairbanks, AK, 99712., but its volunteers and employees are scattered
throughout numerous locations.  Its business office is located at
809 North 1500 West, Salt Lake City, UT 84116, (801) 596-1887, email
business@pglaf.org.  Email contact links and up to date contact
information can be found at the Foundation's web site and official
page at http://pglaf.org

For additional contact information:
     Dr. Gregory B. Newby
     Chief Executive and Director
     gbnewby@pglaf.org


Section 4.  Information about Donations to the Project Gutenberg
Literary Archive Foundation

Project Gutenberg-tm depends upon and cannot survive without wide
spread public support and donations to carry out its mission of
increasing the number of public domain and licensed works that can be
freely distributed in machine readable form accessible by the widest
array of equipment including outdated equipment.  Many small donations
($1 to $5,000) are particularly important to maintaining tax exempt
status with the IRS.

The Foundation is committed to complying with the laws regulating
charities and charitable donations in all 50 states of the United
States.  Compliance requirements are not uniform and it takes a
considerable effort, much paperwork and many fees to meet and keep up
with these requirements.  We do not solicit donations in locations
where we have not received written confirmation of compliance.  To
SEND DONATIONS or determine the status of compliance for any
particular state visit http://pglaf.org

While we cannot and do not solicit contributions from states where we
have not met the solicitation requirements, we know of no prohibition
against accepting unsolicited donations from donors in such states who
approach us with offers to donate.

International donations are gratefully accepted, but we cannot make
any statements concerning tax treatment of donations received from
outside the United States.  U.S. laws alone swamp our small staff.

Please check the Project Gutenberg Web pages for current donation
methods and addresses.  Donations are accepted in a number of other
ways including checks, online payments and credit card donations.
To donate, please visit: http://pglaf.org/donate


Section 5.  General Information About Project Gutenberg-tm electronic
works.

Professor Michael S. Hart is the originator of the Project Gutenberg-tm
concept of a library of electronic works that could be freely shared
with anyone.  For thirty years, he produced and distributed Project
Gutenberg-tm eBooks with only a loose network of volunteer support.


Project Gutenberg-tm eBooks are often created from several printed
editions, all of which are confirmed as Public Domain in the U.S.
unless a copyright notice is included.  Thus, we do not necessarily
keep eBooks in compliance with any particular paper edition.


Most people start at our Web site which has the main PG search facility:

     http://www.gutenberg.org

This Web site includes information about Project Gutenberg-tm,
including how to make donations to the Project Gutenberg Literary
Archive Foundation, how to help produce our new eBooks, and how to
subscribe to our email newsletter to hear about new eBooks.
\end{PGtext}

% %%%%%%%%%%%%%%%%%%%%%%%%%%%%%%%%%%%%%%%%%%%%%%%%%%%%%%%%%%%%%%%%%%%%%%% %
%                                                                         %
% End of Project Gutenberg's Space, Time and Gravitation, by A. S. Eddington
%                                                                         %
% *** END OF THIS PROJECT GUTENBERG EBOOK SPACE, TIME AND GRAVITATION *** %
%                                                                         %
% ***** This file should be named 29782-t.tex or 29782-t.zip *****        %
% This and all associated files of various formats will be found in:      %
%         http://www.gutenberg.org/2/9/7/8/29782/                         %
%                                                                         %
% %%%%%%%%%%%%%%%%%%%%%%%%%%%%%%%%%%%%%%%%%%%%%%%%%%%%%%%%%%%%%%%%%%%%%%% %

\end{document}
###
@ControlwordReplace = (
  ['\\Appendix', 'Appendix. '],
  ['\\Prologue', 'Prologue. ']
  );

@MathEnvironments = (
  ['\\begin{DPalign*}','\\end{DPalign*}','<DPALIGN>'],
  ['\\begin{DPgather*}','\\end{DPgather*}','<DPGATHER>']
  );

@ControlwordArguments = (
  ['\\hyperref', 0, 0, '', ''],
  ['\\rotatebox', 0, 0, '', '', 1, 0, '', ''],
  ['\\AppNote', 1, 1, 'Note ', '. '],
  ['\\Chapter', 0, 0, '', '', 1, 1, 'Chapter ', '. ', 1, 1, '', '. '],
  ['\\IndexExtra', 1, 0, '', ''],
  ['\\Pagelabel', 1, 0, '', ''],
  ['\\Publine', 1, 1, '', '  ', 1, 1, '', ' '],
  ['\\Quote', 0, 0, '', '', 1, 1, 'Quote: ', ' '],
  ['\\Chapref', 1, 1, 'Chapter ', ''],
  ['\\Figref', 1, 1, 'Fig. ', ''],
  ['\\Eqref', 1, 1, '', ' ', 1, 1, '(', ')'],
  ['\\Noteref', 1, 1, 'Note ', ''],
  ['\\Pageref', 1, 1, 'page ', ''],
  ['\\Paragraph', 1, 1, '', ''],
  ['\\Signature', 1, 1, ' ', ' '],
  ['\\Graphic', 0, 0, '', '', 1, 0, '[Illustration]', '', 1, 0, '', ''],
  ['\\hangindent', 1, 0, '', '', 1, 0, '', '', 1, 0, '', '']
  );
###
This is pdfTeXk, Version 3.141592-1.40.3 (Web2C 7.5.6) (format=pdflatex 2009.6.29)  24 AUG 2009 07:14
entering extended mode
 %&-line parsing enabled.
**29782-t.tex
(./29782-t.tex
LaTeX2e <2005/12/01>
Babel <v3.8h> and hyphenation patterns for english, usenglishmax, dumylang, noh
yphenation, loaded.
(/usr/share/texmf-texlive/tex/latex/base/book.cls
Document Class: book 2005/09/16 v1.4f Standard LaTeX document class
(/usr/share/texmf-texlive/tex/latex/base/bk12.clo
File: bk12.clo 2005/09/16 v1.4f Standard LaTeX file (size option)
)
\c@part=\count79
\c@chapter=\count80
\c@section=\count81
\c@subsection=\count82
\c@subsubsection=\count83
\c@paragraph=\count84
\c@subparagraph=\count85
\c@figure=\count86
\c@table=\count87
\abovecaptionskip=\skip41
\belowcaptionskip=\skip42
\bibindent=\dimen102
) (/usr/share/texmf-texlive/tex/latex/base/inputenc.sty
Package: inputenc 2006/05/05 v1.1b Input encoding file
\inpenc@prehook=\toks14
\inpenc@posthook=\toks15
(/usr/share/texmf-texlive/tex/latex/base/latin1.def
File: latin1.def 2006/05/05 v1.1b Input encoding file
)) (/usr/share/texmf-texlive/tex/latex/base/textcomp.sty
Package: textcomp 2005/09/27 v1.99g Standard LaTeX package
Package textcomp Info: Sub-encoding information:
(textcomp)               5 = only ISO-Adobe without \textcurrency
(textcomp)               4 = 5 + \texteuro
(textcomp)               3 = 4 + \textohm
(textcomp)               2 = 3 + \textestimated + \textcurrency
(textcomp)               1 = TS1 - \textcircled - \t
(textcomp)               0 = TS1 (full)
(textcomp)             Font families with sub-encoding setting implement
(textcomp)             only a restricted character set as indicated.
(textcomp)             Family '?' is the default used for unknown fonts.
(textcomp)             See the documentation for details.
Package textcomp Info: Setting ? sub-encoding to TS1/1 on input line 71.
(/usr/share/texmf-texlive/tex/latex/base/ts1enc.def
File: ts1enc.def 2001/06/05 v3.0e (jk/car/fm) Standard LaTeX file
)
LaTeX Info: Redefining \oldstylenums on input line 266.
Package textcomp Info: Setting cmr sub-encoding to TS1/0 on input line 281.
Package textcomp Info: Setting cmss sub-encoding to TS1/0 on input line 282.
Package textcomp Info: Setting cmtt sub-encoding to TS1/0 on input line 283.
Package textcomp Info: Setting cmvtt sub-encoding to TS1/0 on input line 284.
Package textcomp Info: Setting cmbr sub-encoding to TS1/0 on input line 285.
Package textcomp Info: Setting cmtl sub-encoding to TS1/0 on input line 286.
Package textcomp Info: Setting ccr sub-encoding to TS1/0 on input line 287.
Package textcomp Info: Setting ptm sub-encoding to TS1/4 on input line 288.
Package textcomp Info: Setting pcr sub-encoding to TS1/4 on input line 289.
Package textcomp Info: Setting phv sub-encoding to TS1/4 on input line 290.
Package textcomp Info: Setting ppl sub-encoding to TS1/3 on input line 291.
Package textcomp Info: Setting pag sub-encoding to TS1/4 on input line 292.
Package textcomp Info: Setting pbk sub-encoding to TS1/4 on input line 293.
Package textcomp Info: Setting pnc sub-encoding to TS1/4 on input line 294.
Package textcomp Info: Setting pzc sub-encoding to TS1/4 on input line 295.
Package textcomp Info: Setting bch sub-encoding to TS1/4 on input line 296.
Package textcomp Info: Setting put sub-encoding to TS1/5 on input line 297.
Package textcomp Info: Setting uag sub-encoding to TS1/5 on input line 298.
Package textcomp Info: Setting ugq sub-encoding to TS1/5 on input line 299.
Package textcomp Info: Setting ul8 sub-encoding to TS1/4 on input line 300.
Package textcomp Info: Setting ul9 sub-encoding to TS1/4 on input line 301.
Package textcomp Info: Setting augie sub-encoding to TS1/5 on input line 302.
Package textcomp Info: Setting dayrom sub-encoding to TS1/3 on input line 303.
Package textcomp Info: Setting dayroms sub-encoding to TS1/3 on input line 304.

Package textcomp Info: Setting pxr sub-encoding to TS1/0 on input line 305.
Package textcomp Info: Setting pxss sub-encoding to TS1/0 on input line 306.
Package textcomp Info: Setting pxtt sub-encoding to TS1/0 on input line 307.
Package textcomp Info: Setting txr sub-encoding to TS1/0 on input line 308.
Package textcomp Info: Setting txss sub-encoding to TS1/0 on input line 309.
Package textcomp Info: Setting txtt sub-encoding to TS1/0 on input line 310.
Package textcomp Info: Setting futs sub-encoding to TS1/4 on input line 311.
Package textcomp Info: Setting futx sub-encoding to TS1/4 on input line 312.
Package textcomp Info: Setting futj sub-encoding to TS1/4 on input line 313.
Package textcomp Info: Setting hlh sub-encoding to TS1/3 on input line 314.
Package textcomp Info: Setting hls sub-encoding to TS1/3 on input line 315.
Package textcomp Info: Setting hlst sub-encoding to TS1/3 on input line 316.
Package textcomp Info: Setting hlct sub-encoding to TS1/5 on input line 317.
Package textcomp Info: Setting hlx sub-encoding to TS1/5 on input line 318.
Package textcomp Info: Setting hlce sub-encoding to TS1/5 on input line 319.
Package textcomp Info: Setting hlcn sub-encoding to TS1/5 on input line 320.
Package textcomp Info: Setting hlcw sub-encoding to TS1/5 on input line 321.
Package textcomp Info: Setting hlcf sub-encoding to TS1/5 on input line 322.
Package textcomp Info: Setting pplx sub-encoding to TS1/3 on input line 323.
Package textcomp Info: Setting pplj sub-encoding to TS1/3 on input line 324.
Package textcomp Info: Setting ptmx sub-encoding to TS1/4 on input line 325.
Package textcomp Info: Setting ptmj sub-encoding to TS1/4 on input line 326.
)
\MySkip=\skip43
(/usr/share/texmf-texlive/tex/latex/base/fix-cm.sty
Package: fix-cm 2006/03/24 v1.1n fixes to LaTeX
(/usr/share/texmf-texlive/tex/latex/base/ts1enc.def
File: ts1enc.def 2001/06/05 v3.0e (jk/car/fm) Standard LaTeX file
LaTeX Font Info:    Redeclaring font encoding TS1 on input line 42.
)) (/usr/share/texmf-texlive/tex/latex/base/ifthen.sty
Package: ifthen 2001/05/26 v1.1c Standard LaTeX ifthen package (DPC)
) (/usr/share/texmf-texlive/tex/latex/amsmath/amsmath.sty
Package: amsmath 2000/07/18 v2.13 AMS math features
\@mathmargin=\skip44
For additional information on amsmath, use the `?' option.
(/usr/share/texmf-texlive/tex/latex/amsmath/amstext.sty
Package: amstext 2000/06/29 v2.01
(/usr/share/texmf-texlive/tex/latex/amsmath/amsgen.sty
File: amsgen.sty 1999/11/30 v2.0
\@emptytoks=\toks16
\ex@=\dimen103
)) (/usr/share/texmf-texlive/tex/latex/amsmath/amsbsy.sty
Package: amsbsy 1999/11/29 v1.2d
\pmbraise@=\dimen104
) (/usr/share/texmf-texlive/tex/latex/amsmath/amsopn.sty
Package: amsopn 1999/12/14 v2.01 operator names
)
\inf@bad=\count88
LaTeX Info: Redefining \frac on input line 211.
\uproot@=\count89
\leftroot@=\count90
LaTeX Info: Redefining \overline on input line 307.
\classnum@=\count91
\DOTSCASE@=\count92
LaTeX Info: Redefining \ldots on input line 379.
LaTeX Info: Redefining \dots on input line 382.
LaTeX Info: Redefining \cdots on input line 467.
\Mathstrutbox@=\box26
\strutbox@=\box27
\big@size=\dimen105
LaTeX Font Info:    Redeclaring font encoding OML on input line 567.
LaTeX Font Info:    Redeclaring font encoding OMS on input line 568.
\macc@depth=\count93
\c@MaxMatrixCols=\count94
\dotsspace@=\muskip10
\c@parentequation=\count95
\dspbrk@lvl=\count96
\tag@help=\toks17
\row@=\count97
\column@=\count98
\maxfields@=\count99
\andhelp@=\toks18
\eqnshift@=\dimen106
\alignsep@=\dimen107
\tagshift@=\dimen108
\tagwidth@=\dimen109
\totwidth@=\dimen110
\lineht@=\dimen111
\@envbody=\toks19
\multlinegap=\skip45
\multlinetaggap=\skip46
\mathdisplay@stack=\toks20
LaTeX Info: Redefining \[ on input line 2666.
LaTeX Info: Redefining \] on input line 2667.
) (/usr/share/texmf-texlive/tex/latex/amsfonts/amssymb.sty
Package: amssymb 2002/01/22 v2.2d
(/usr/share/texmf-texlive/tex/latex/amsfonts/amsfonts.sty
Package: amsfonts 2001/10/25 v2.2f
\symAMSa=\mathgroup4
\symAMSb=\mathgroup5
LaTeX Font Info:    Overwriting math alphabet `\mathfrak' in version `bold'
(Font)                  U/euf/m/n --> U/euf/b/n on input line 132.
)) (/usr/share/texmf-texlive/tex/latex/base/alltt.sty
Package: alltt 1997/06/16 v2.0g defines alltt environment
) (/usr/share/texmf-texlive/tex/latex/tools/array.sty
Package: array 2005/08/23 v2.4b Tabular extension package (FMi)
\col@sep=\dimen112
\extrarowheight=\dimen113
\NC@list=\toks21
\extratabsurround=\skip47
\backup@length=\skip48
) (/usr/share/texmf-texlive/tex/latex/footmisc/footmisc.sty
Package: footmisc 2005/03/17 v5.3d a miscellany of footnote facilities
\FN@temptoken=\toks22
\footnotemargin=\dimen114
\c@pp@next@reset=\count100
\c@@fnserial=\count101
Package footmisc Info: Declaring symbol style bringhurst on input line 817.
Package footmisc Info: Declaring symbol style chicago on input line 818.
Package footmisc Info: Declaring symbol style wiley on input line 819.
Package footmisc Info: Declaring symbol style lamport-robust on input line 823.

Package footmisc Info: Declaring symbol style lamport* on input line 831.
Package footmisc Info: Declaring symbol style lamport*-robust on input line 840
.
) (/usr/share/texmf-texlive/tex/latex/tools/multicol.sty
Package: multicol 2006/05/18 v1.6g multicolumn formatting (FMi)
\c@tracingmulticols=\count102
\mult@box=\box28
\multicol@leftmargin=\dimen115
\c@unbalance=\count103
\c@collectmore=\count104
\doublecol@number=\count105
\multicoltolerance=\count106
\multicolpretolerance=\count107
\full@width=\dimen116
\page@free=\dimen117
\premulticols=\dimen118
\postmulticols=\dimen119
\multicolsep=\skip49
\multicolbaselineskip=\skip50
\partial@page=\box29
\last@line=\box30
\mult@rightbox=\box31
\mult@grightbox=\box32
\mult@gfirstbox=\box33
\mult@firstbox=\box34
\@tempa=\box35
\@tempa=\box36
\@tempa=\box37
\@tempa=\box38
\@tempa=\box39
\@tempa=\box40
\@tempa=\box41
\@tempa=\box42
\@tempa=\box43
\@tempa=\box44
\@tempa=\box45
\@tempa=\box46
\@tempa=\box47
\@tempa=\box48
\@tempa=\box49
\@tempa=\box50
\@tempa=\box51
\c@columnbadness=\count108
\c@finalcolumnbadness=\count109
\last@try=\dimen120
\multicolovershoot=\dimen121
\multicolundershoot=\dimen122
\mult@nat@firstbox=\box52
\colbreak@box=\box53
) (/usr/share/texmf-texlive/tex/latex/base/makeidx.sty
Package: makeidx 2000/03/29 v1.0m Standard LaTeX package
) (/usr/share/texmf-texlive/tex/latex/graphics/graphicx.sty
Package: graphicx 1999/02/16 v1.0f Enhanced LaTeX Graphics (DPC,SPQR)
(/usr/share/texmf-texlive/tex/latex/graphics/keyval.sty
Package: keyval 1999/03/16 v1.13 key=value parser (DPC)
\KV@toks@=\toks23
) (/usr/share/texmf-texlive/tex/latex/graphics/graphics.sty
Package: graphics 2006/02/20 v1.0o Standard LaTeX Graphics (DPC,SPQR)
(/usr/share/texmf-texlive/tex/latex/graphics/trig.sty
Package: trig 1999/03/16 v1.09 sin cos tan (DPC)
) (/etc/texmf/tex/latex/config/graphics.cfg
File: graphics.cfg 2007/01/18 v1.5 graphics configuration of teTeX/TeXLive
)
Package graphics Info: Driver file: pdftex.def on input line 90.
(/usr/share/texmf-texlive/tex/latex/pdftex-def/pdftex.def
File: pdftex.def 2007/01/08 v0.04d Graphics/color for pdfTeX
\Gread@gobject=\count110
))
\Gin@req@height=\dimen123
\Gin@req@width=\dimen124
) (/usr/share/texmf-texlive/tex/latex/wrapfig/wrapfig.sty
\wrapoverhang=\dimen125
\WF@size=\dimen126
\c@WF@wrappedlines=\count111
\WF@box=\box54
\WF@everypar=\toks24
Package: wrapfig 2003/01/31  v 3.6
) (/usr/share/texmf-texlive/tex/latex/fancyhdr/fancyhdr.sty
\fancy@headwidth=\skip51
\f@ncyO@elh=\skip52
\f@ncyO@erh=\skip53
\f@ncyO@olh=\skip54
\f@ncyO@orh=\skip55
\f@ncyO@elf=\skip56
\f@ncyO@erf=\skip57
\f@ncyO@olf=\skip58
\f@ncyO@orf=\skip59
) (/usr/share/texmf-texlive/tex/latex/geometry/geometry.sty
Package: geometry 2002/07/08 v3.2 Page Geometry
\Gm@cnth=\count112
\Gm@cntv=\count113
\c@Gm@tempcnt=\count114
\Gm@bindingoffset=\dimen127
\Gm@wd@mp=\dimen128
\Gm@odd@mp=\dimen129
\Gm@even@mp=\dimen130
\Gm@dimlist=\toks25
) (/usr/share/texmf-texlive/tex/latex/hyperref/hyperref.sty
Package: hyperref 2007/02/07 v6.75r Hypertext links for LaTeX
\@linkdim=\dimen131
\Hy@linkcounter=\count115
\Hy@pagecounter=\count116
(/usr/share/texmf-texlive/tex/latex/hyperref/pd1enc.def
File: pd1enc.def 2007/02/07 v6.75r Hyperref: PDFDocEncoding definition (HO)
) (/etc/texmf/tex/latex/config/hyperref.cfg
File: hyperref.cfg 2002/06/06 v1.2 hyperref configuration of TeXLive
) (/usr/share/texmf-texlive/tex/latex/oberdiek/kvoptions.sty
Package: kvoptions 2006/08/22 v2.4 Connects package keyval with LaTeX options (
HO)
)
Package hyperref Info: Option `hyperfootnotes' set `false' on input line 2238.
Package hyperref Info: Option `bookmarks' set `true' on input line 2238.
Package hyperref Info: Option `linktocpage' set `false' on input line 2238.
Package hyperref Info: Option `pdfdisplaydoctitle' set `true' on input line 223
8.
Package hyperref Info: Option `pdfpagelabels' set `true' on input line 2238.
Package hyperref Info: Option `bookmarksopen' set `true' on input line 2238.
Package hyperref Info: Option `colorlinks' set `true' on input line 2238.
Package hyperref Info: Hyper figures OFF on input line 2288.
Package hyperref Info: Link nesting OFF on input line 2293.
Package hyperref Info: Hyper index ON on input line 2296.
Package hyperref Info: Plain pages OFF on input line 2303.
Package hyperref Info: Backreferencing OFF on input line 2308.
Implicit mode ON; LaTeX internals redefined
Package hyperref Info: Bookmarks ON on input line 2444.
(/usr/share/texmf-texlive/tex/latex/ltxmisc/url.sty
\Urlmuskip=\muskip11
Package: url 2005/06/27  ver 3.2  Verb mode for urls, etc.
)
LaTeX Info: Redefining \url on input line 2599.
\Fld@menulength=\count117
\Field@Width=\dimen132
\Fld@charsize=\dimen133
\Choice@toks=\toks26
\Field@toks=\toks27
Package hyperref Info: Hyper figures OFF on input line 3102.
Package hyperref Info: Link nesting OFF on input line 3107.
Package hyperref Info: Hyper index ON on input line 3110.
Package hyperref Info: backreferencing OFF on input line 3117.
Package hyperref Info: Link coloring ON on input line 3120.
\Hy@abspage=\count118
\c@Item=\count119
)
*hyperref using driver hpdftex*
(/usr/share/texmf-texlive/tex/latex/hyperref/hpdftex.def
File: hpdftex.def 2007/02/07 v6.75r Hyperref driver for pdfTeX
\Fld@listcount=\count120
)
\TmpLen=\skip60
\ToCBox=\skip61
\QIndent=\skip62
\@indexfile=\write3
\openout3 = `29782-t.idx'.

Writing index file 29782-t.idx
\DP@lign@no=\count121
\DP@lignb@dy=\toks28
(./29782-t.aux)
\openout1 = `29782-t.aux'.

LaTeX Font Info:    Checking defaults for OML/cmm/m/it on input line 742.
LaTeX Font Info:    ... okay on input line 742.
LaTeX Font Info:    Checking defaults for T1/cmr/m/n on input line 742.
LaTeX Font Info:    ... okay on input line 742.
LaTeX Font Info:    Checking defaults for OT1/cmr/m/n on input line 742.
LaTeX Font Info:    ... okay on input line 742.
LaTeX Font Info:    Checking defaults for OMS/cmsy/m/n on input line 742.
LaTeX Font Info:    ... okay on input line 742.
LaTeX Font Info:    Checking defaults for OMX/cmex/m/n on input line 742.
LaTeX Font Info:    ... okay on input line 742.
LaTeX Font Info:    Checking defaults for U/cmr/m/n on input line 742.
LaTeX Font Info:    ... okay on input line 742.
LaTeX Font Info:    Checking defaults for TS1/cmr/m/n on input line 742.
LaTeX Font Info:    ... okay on input line 742.
LaTeX Font Info:    Checking defaults for PD1/pdf/m/n on input line 742.
LaTeX Font Info:    ... okay on input line 742.
-------------------- Geometry parameters
paper: class default
landscape: --
twocolumn: --
twoside: true
asymmetric: --
h-parts: 95.39648pt, 375.80377pt, 143.09474pt
v-parts: 83.11047pt, 587.19374pt, 124.66577pt
hmarginratio: 2:3
vmarginratio: 2:3
lines: --
heightrounded: --
bindingoffset: 0.0pt
truedimen: --
includehead: --
includefoot: --
includemp: --
driver: pdftex
-------------------- Page layout dimensions and switches
\paperwidth  614.295pt
\paperheight 794.96999pt
\textwidth  375.80377pt
\textheight 587.19374pt
\oddsidemargin  23.1265pt
\evensidemargin 70.82475pt
\topmargin  -21.03331pt
\headheight 12.0pt
\headsep    19.8738pt
\footskip   30.0pt
\marginparwidth 98.0pt
\marginparsep   7.0pt
\columnsep  10.0pt
\skip\footins  10.8pt plus 4.0pt minus 2.0pt
\hoffset 0.0pt
\voffset 0.0pt
\mag 1000
\@twosidetrue \@mparswitchtrue 
(1in=72.27pt, 1cm=28.45pt)
-----------------------
(/usr/share/texmf-texlive/tex/latex/graphics/color.sty
Package: color 2005/11/14 v1.0j Standard LaTeX Color (DPC)
(/etc/texmf/tex/latex/config/color.cfg
File: color.cfg 2007/01/18 v1.5 color configuration of teTeX/TeXLive
)
Package color Info: Driver file: pdftex.def on input line 130.
)
Package hyperref Info: Link coloring ON on input line 742.
(/usr/share/texmf-texlive/tex/latex/hyperref/nameref.sty
Package: nameref 2006/12/27 v2.28 Cross-referencing by name of section
(/usr/share/texmf-texlive/tex/latex/oberdiek/refcount.sty
Package: refcount 2006/02/20 v3.0 Data extraction from references (HO)
)
\c@section@level=\count122
)
LaTeX Info: Redefining \ref on input line 742.
LaTeX Info: Redefining \pageref on input line 742.
(./29782-t.out) (./29782-t.out)
\@outlinefile=\write4
\openout4 = `29782-t.out'.

LaTeX Font Info:    Try loading font information for U+msa on input line 779.
(/usr/share/texmf-texlive/tex/latex/amsfonts/umsa.fd
File: umsa.fd 2002/01/19 v2.2g AMS font definitions
)
LaTeX Font Info:    Try loading font information for U+msb on input line 779.
(/usr/share/texmf-texlive/tex/latex/amsfonts/umsb.fd
File: umsb.fd 2002/01/19 v2.2g AMS font definitions
) [1

{/var/lib/texmf/fonts/map/pdftex/updmap/pdftex.map}] [2

] [1


] <./images/device.pdf, id=129, 469.755pt x 561.09625pt>
File: ./images/device.pdf Graphic file (type pdf)
<use ./images/device.pdf> [2

 <./images/device.pdf>] [3

] <./images/frontis_bw.jpg, id=142, 97.5645pt x 59.9841pt>
File: ./images/frontis_bw.jpg Graphic file (type jpg)
<use ./images/frontis_bw.jpg> [4 <./images/frontis_bw.jpg>] [5

] [6

] [7

] [8] (./29782-t.toc
\FPlen=\skip63
[9


])
\tf@toc=\write5
\openout5 = `29782-t.toc'.

[10] [1


] [2] [3] [4] [5] [6] [7] [8] [9] [10] [11] [12] [13] [14] <./images/027a.pdf, 
id=263, 124.465pt x 165.61874pt>
File: ./images/027a.pdf Graphic file (type pdf)
<use ./images/027a.pdf> [15

 <./images/027a.pdf>] [16] [17] [18] [19] [20] [21]
Overfull \hbox (0.6925pt too wide) in paragraph at lines 2288--2288
 []\OT1/cmr/m/n/10 Stationary 
 []

[22] [23] [24] [25] [26

] [27] [28] [29] [30] [31] [32] [33] [34] [35] [36] [37] [38] [39] [40

] [41] <./images/056a.pdf, id=398, 144.54pt x 162.6075pt>
File: ./images/056a.pdf Graphic file (type pdf)
<use ./images/056a.pdf> [42 <./images/056a.pdf>] [43] <./images/059a.pdf, id=42
6, 265.99374pt x 277.035pt>
File: ./images/059a.pdf Graphic file (type pdf)
<use ./images/059a.pdf> [44] [45 <./images/059a.pdf>] [46] <./images/062a.pdf, 
id=456, 259.97125pt x 282.05376pt>
File: ./images/062a.pdf Graphic file (type pdf)
<use ./images/062a.pdf> [47 <./images/062a.pdf>] [48] <./images/064a.pdf, id=48
5, 358.33875pt x 220.825pt>
File: ./images/064a.pdf Graphic file (type pdf)
<use ./images/064a.pdf> <./images/065a.pdf, id=487, 372.39125pt x 296.10625pt>
File: ./images/065a.pdf Graphic file (type pdf)
<use ./images/065a.pdf> [49 <./images/064a.pdf>] <./images/065b.pdf, id=505, 21
8.8175pt x 169.63374pt>
File: ./images/065b.pdf Graphic file (type pdf)
<use ./images/065b.pdf> [50 <./images/065a.pdf> <./images/065b.pdf>] [51] <./im
ages/068a.pdf, id=542, 360.34625pt x 104.39pt>
File: ./images/068a.pdf Graphic file (type pdf)
<use ./images/068a.pdf> [52] [53 <./images/068a.pdf>] [54] [55] [56] [57

] [58] [59] [60] [61] [62] [63] [64] [65] [66] [67] [68] [69

] <./images/088a.pdf, id=653, 351.3125pt x 114.4275pt>
File: ./images/088a.pdf Graphic file (type pdf)
<use ./images/088a.pdf> [70 <./images/088a.pdf>] [71] [72] [73] [74] [75] [76] 
[77] [78] [79] [80] [81] [82] [83] [84

] [85] <./images/104a.pdf, id=761, 265.99374pt x 274.02374pt>
File: ./images/104a.pdf Graphic file (type pdf)
<use ./images/104a.pdf> [86 <./images/104a.pdf>] [87] [88] [89] [90] <./images/
110a.pdf, id=815, 203.76125pt x 283.0575pt>
File: ./images/110a.pdf Graphic file (type pdf)
<use ./images/110a.pdf> [91] <./images/111a.pdf, id=826, 219.82124pt x 236.885p
t>
File: ./images/111a.pdf Graphic file (type pdf)
<use ./images/111a.pdf> [92 <./images/110a.pdf>] [93 <./images/111a.pdf>]
Underfull \vbox (badness 10000) has occurred while \output is active []

[94] [95] [96] [97] [98] [99] [100] [101

] <./images/122a.pdf, id=896, 333.245pt x 73.27374pt>
File: ./images/122a.pdf Graphic file (type pdf)
<use ./images/122a.pdf> [102] [103 <./images/122a.pdf>] [104] [105] [106] [107]
<./images/129a.pdf, id=937, 341.275pt x 335.2525pt>
File: ./images/129a.pdf Graphic file (type pdf)
<use ./images/129a.pdf> [108] <./images/130a.pdf, id=944, 417.56pt x 439.6425pt
>
File: ./images/130a.pdf Graphic file (type pdf)
<use ./images/130a.pdf> [109 <./images/129a.pdf>] [110 <./images/130a.pdf>] [11
1] [112

] [113] [114] [115] [116]
Underfull \vbox (badness 1389) has occurred while \output is active []

[117] [118] [119] [120] [121] [122] [123] [124] [125

] [126] [127] [128] [129] <./images/152a.pdf, id=1079, 199.74625pt x 240.9pt>
File: ./images/152a.pdf Graphic file (type pdf)
<use ./images/152a.pdf> [130 <./images/152a.pdf>] <./images/153a.pdf, id=1097, 
213.79875pt x 283.0575pt>
File: ./images/153a.pdf Graphic file (type pdf)
<use ./images/153a.pdf> [131 <./images/153a.pdf>] [132] [133] [134] [135] [136]
[137] [138] [139] [140

] [141] [142] [143] [144] [145] [146] [147] [148] [149] [150] [151] [152] [153

] [154] [155] [156] [157] [158] [159] [160] [161] [162] [163] [164

] [165] [166] [167] [168] [169] [170] [171] [172] [173] [174]
Underfull \vbox (badness 6542) has occurred while \output is active []

[175] [176] [177] [178] [179] [180] [181] [182] [183

] [184] [185] [186] [187] [188] [189] [190] [191] [192] [193] [194] (./29782-t.
ind [195

] [196] [197] [198] [199] [200])
Underfull \vbox (badness 10000) has occurred while \output is active []

[201



]
Underfull \vbox (badness 10000) has occurred while \output is active []

[202]
Underfull \vbox (badness 10000) has occurred while \output is active []

[203]
Underfull \vbox (badness 10000) has occurred while \output is active []

[204]
Underfull \vbox (badness 10000) has occurred while \output is active []

[205]
Underfull \vbox (badness 10000) has occurred while \output is active []

[206] [207] (./29782-t.aux)

 *File List*
    book.cls    2005/09/16 v1.4f Standard LaTeX document class
    bk12.clo    2005/09/16 v1.4f Standard LaTeX file (size option)
inputenc.sty    2006/05/05 v1.1b Input encoding file
  latin1.def    2006/05/05 v1.1b Input encoding file
textcomp.sty    2005/09/27 v1.99g Standard LaTeX package
  ts1enc.def    2001/06/05 v3.0e (jk/car/fm) Standard LaTeX file
  fix-cm.sty    2006/03/24 v1.1n fixes to LaTeX
  ts1enc.def    2001/06/05 v3.0e (jk/car/fm) Standard LaTeX file
  ifthen.sty    2001/05/26 v1.1c Standard LaTeX ifthen package (DPC)
 amsmath.sty    2000/07/18 v2.13 AMS math features
 amstext.sty    2000/06/29 v2.01
  amsgen.sty    1999/11/30 v2.0
  amsbsy.sty    1999/11/29 v1.2d
  amsopn.sty    1999/12/14 v2.01 operator names
 amssymb.sty    2002/01/22 v2.2d
amsfonts.sty    2001/10/25 v2.2f
   alltt.sty    1997/06/16 v2.0g defines alltt environment
   array.sty    2005/08/23 v2.4b Tabular extension package (FMi)
footmisc.sty    2005/03/17 v5.3d a miscellany of footnote facilities
multicol.sty    2006/05/18 v1.6g multicolumn formatting (FMi)
 makeidx.sty    2000/03/29 v1.0m Standard LaTeX package
graphicx.sty    1999/02/16 v1.0f Enhanced LaTeX Graphics (DPC,SPQR)
  keyval.sty    1999/03/16 v1.13 key=value parser (DPC)
graphics.sty    2006/02/20 v1.0o Standard LaTeX Graphics (DPC,SPQR)
    trig.sty    1999/03/16 v1.09 sin cos tan (DPC)
graphics.cfg    2007/01/18 v1.5 graphics configuration of teTeX/TeXLive
  pdftex.def    2007/01/08 v0.04d Graphics/color for pdfTeX
 wrapfig.sty    2003/01/31  v 3.6
fancyhdr.sty    
geometry.sty    2002/07/08 v3.2 Page Geometry
hyperref.sty    2007/02/07 v6.75r Hypertext links for LaTeX
  pd1enc.def    2007/02/07 v6.75r Hyperref: PDFDocEncoding definition (HO)
hyperref.cfg    2002/06/06 v1.2 hyperref configuration of TeXLive
kvoptions.sty    2006/08/22 v2.4 Connects package keyval with LaTeX options (HO
)
     url.sty    2005/06/27  ver 3.2  Verb mode for urls, etc.
 hpdftex.def    2007/02/07 v6.75r Hyperref driver for pdfTeX
   color.sty    2005/11/14 v1.0j Standard LaTeX Color (DPC)
   color.cfg    2007/01/18 v1.5 color configuration of teTeX/TeXLive
 nameref.sty    2006/12/27 v2.28 Cross-referencing by name of section
refcount.sty    2006/02/20 v3.0 Data extraction from references (HO)
 29782-t.out
 29782-t.out
    umsa.fd    2002/01/19 v2.2g AMS font definitions
    umsb.fd    2002/01/19 v2.2g AMS font definitions
./images/device.pdf
./images/frontis_bw.jpg
./images/027a.pdf
./images/056a.pdf
./images/059a.pdf
./images/062a.pdf
./images/064a.pdf
./images/065a.pdf
./images/065b.pdf
./images/068a.pdf
./images/088a.pdf
./images/104a.pdf
./images/110a.pdf
./images/111a.pdf
./images/122a.pdf
./images/129a.pdf
./images/130a.pdf
./images/152a.pdf
./images/153a.pdf
 29782-t.ind
 ***********

 ) 
Here is how much of TeX's memory you used:
 5552 strings out of 95086
 75691 string characters out of 1183255
 136656 words of memory out of 1500000
 8282 multiletter control sequences out of 10000+50000
 19382 words of font info for 73 fonts, out of 1200000 for 2000
 28 hyphenation exceptions out of 8191
 27i,14n,43p,258b,805s stack positions out of 5000i,500n,6000p,200000b,5000s
 </home/widger/.texmf-var/fonts/pk/ljfour/jknappen/ec/tcrm1200.600pk> </home/
widger/.texmf-var/fonts/pk/ljfour/jknappen/ec/tctt1000.600pk></usr/share/texmf-
texlive/fonts/type1/bluesky/cm/cmbx12.pfb></usr/share/texmf-texlive/fonts/type1
/bluesky/cm/cmcsc10.pfb></usr/share/texmf-texlive/fonts/type1/bluesky/cm/cmex10
.pfb></usr/share/texmf-texlive/fonts/type1/bluesky/cm/cmmi10.pfb></usr/share/te
xmf-texlive/fonts/type1/bluesky/cm/cmmi12.pfb></usr/share/texmf-texlive/fonts/t
ype1/bluesky/cm/cmmi7.pfb></usr/share/texmf-texlive/fonts/type1/bluesky/cm/cmmi
8.pfb></usr/share/texmf-texlive/fonts/type1/bluesky/cm/cmr10.pfb></usr/share/te
xmf-texlive/fonts/type1/bluesky/cm/cmr12.pfb></usr/share/texmf-texlive/fonts/ty
pe1/bluesky/cm/cmr17.pfb></usr/share/texmf-texlive/fonts/type1/bluesky/cm/cmr6.
pfb></usr/share/texmf-texlive/fonts/type1/bluesky/cm/cmr7.pfb></usr/share/texmf
-texlive/fonts/type1/bluesky/cm/cmr8.pfb></usr/share/texmf-texlive/fonts/type1/
bluesky/cm/cmsy10.pfb></usr/share/texmf-texlive/fonts/type1/bluesky/cm/cmsy7.pf
b></usr/share/texmf-texlive/fonts/type1/bluesky/cm/cmsy8.pfb></usr/share/texmf-
texlive/fonts/type1/bluesky/cm/cmti10.pfb></usr/share/texmf-texlive/fonts/type1
/bluesky/cm/cmti12.pfb></usr/share/texmf-texlive/fonts/type1/bluesky/cm/cmti8.p
fb></usr/share/texmf-texlive/fonts/type1/bluesky/cm/cmtt9.pfb>
Output written on 29782-t.pdf (219 pages, 1134557 bytes).
PDF statistics:
 2142 PDF objects out of 2487 (max. 8388607)
 334 named destinations out of 1000 (max. 131072)
 312 words of extra memory for PDF output out of 10000 (max. 10000000)

